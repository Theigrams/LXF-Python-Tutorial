\hypertarget{ux8c03ux8bd5}{%
\subsection{调试}\label{ux8c03ux8bd5}}

程序能一次写完并正常运行的概率很小,基本不超过 1\%。总会有各种各样的 bug
需要修正。有的 bug 很简单,看看错误信息就知道,有的 bug
很复杂,我们需要知道出错时,哪些变量的值是正确的,哪些变量的值是错误的,因此,需要一整套调试程序的手段来修复
bug。

第一种方法简单直接粗暴有效,就是用\texttt{print()}把可能有问题的变量打印出来看看:

\begin{pythoncode}
def foo(s):
    n = int(s)
    print('>>> n = %d' % n)
    return 10 / n

def main():
    foo('0')

main()
\end{pythoncode}

执行后在输出中查找打印的变量值:

\begin{pythoncode}
$ python err.py
>>> n = 0
Traceback (most recent call last):
  ...
ZeroDivisionError: integer division or modulo by zero
\end{pythoncode}

用\texttt{print()}最大的坏处是将来还得删掉它,想想程序里到处都是\texttt{print()},运行结果也会包含很多垃圾信息。所以,我们又有第二种方法。

\hypertarget{ux65adux8a00}{%
\subsubsection{断言}\label{ux65adux8a00}}

凡是用\texttt{print()}来辅助查看的地方,都可以用断言(assert)来替代:

\begin{pythoncode}
def foo(s):
    n = int(s)
    assert n != 0, 'n is zero!'
    return 10 / n

def main():
    foo('0')
\end{pythoncode}

\texttt{assert}的意思是,表达式\texttt{n\ !=\ 0}应该是\texttt{True},否则,根据程序运行的逻辑,后面的代码肯定会出错。

如果断言失败,\texttt{assert}语句本身就会抛出\texttt{AssertionError}:

\begin{pythoncode}
$ python err.py
Traceback (most recent call last):
  ...
AssertionError: n is zero!
\end{pythoncode}

程序中如果到处充斥着\texttt{assert},和\texttt{print()}相比也好不到哪去。不过,启动
Python 解释器时可以用\texttt{-O}参数来关闭\texttt{assert}:

\begin{pythoncode}
$ python -O err.py
Traceback (most recent call last):
  ...
ZeroDivisionError: division by zero
\end{pythoncode}

\begin{quote}
注意:断言的开关 ``-O'' 是英文大写字母 O,不是数字 0。
\end{quote}

关闭后,你可以把所有的\texttt{assert}语句当成\texttt{pass}来看。

\hypertarget{logging}{%
\subsubsection{logging}\label{logging}}

把\texttt{print()}替换为\texttt{logging}是第 3
种方式,和\texttt{assert}比,\texttt{logging}不会抛出错误,而且可以输出到文件:

\begin{pythoncode}
import logging

s = '0'
n = int(s)
logging.info('n = %d' % n)
print(10 / n)
\end{pythoncode}

\texttt{logging.info()}就可以输出一段文本。运行,发现除了\texttt{ZeroDivisionError},没有任何信息。怎么回事?

别急,在\texttt{import\ logging}之后添加一行配置再试试:

\begin{pythoncode}
import logging
logging.basicConfig(level=logging.INFO)
\end{pythoncode}

看到输出了:

\begin{pythoncode}
$ python err.py
INFO:root:n = 0
Traceback (most recent call last):
  File "err.py", line 8, in <module>
    print(10 / n)
ZeroDivisionError: division by zero
\end{pythoncode}

这就是\texttt{logging}的好处,它允许你指定记录信息的级别,有\texttt{debug},\texttt{info},\texttt{warning},\texttt{error}等几个级别,当我们指定\texttt{level=INFO}时,\texttt{logging.debug}就不起作用了。同理,指定\texttt{level=WARNING}后,\texttt{debug}和\texttt{info}就不起作用了。这样一来,你可以放心地输出不同级别的信息,也不用删除,最后统一控制输出哪个级别的信息。

\texttt{logging}的另一个好处是通过简单的配置,一条语句可以同时输出到不同的地方,比如
console 和文件。

\hypertarget{pdb}{%
\subsubsection{pdb}\label{pdb}}

第 4 种方式是启动 Python 的调试器
pdb,让程序以单步方式运行,可以随时查看运行状态。我们先准备好程序:

\begin{pythoncode}
s = '0'
n = int(s)
print(10 / n)
\end{pythoncode}

然后启动:

\begin{pythoncode}
$ python -m pdb err.py
> /Users/michael/Github/learn-python3/samples/debug/err.py(2)<module>()
-> s = '0'
\end{pythoncode}

以参数\texttt{-m\ pdb}启动后,pdb
定位到下一步要执行的代码\texttt{-\textgreater{}\ s\ =\ \textquotesingle{}0\textquotesingle{}}。输入命令\texttt{l}来查看代码:

\begin{pythoncode}
(Pdb) l
  1     
  2  -> s = '0'
  3     n = int(s)
  4     print(10 / n)
\end{pythoncode}

输入命令\texttt{n}可以单步执行代码:

\begin{pythoncode}
(Pdb) n
> /Users/michael/Github/learn-python3/samples/debug/err.py(3)<module>()
-> n = int(s)
(Pdb) n
> /Users/michael/Github/learn-python3/samples/debug/err.py(4)<module>()
-> print(10 / n)
\end{pythoncode}

任何时候都可以输入命令\texttt{p\ 变量名}来查看变量:

\begin{pythoncode}
(Pdb) p s
'0'
(Pdb) p n
0
\end{pythoncode}

输入命令\texttt{q}结束调试,退出程序:

\begin{pythoncode}
(Pdb) q
\end{pythoncode}

这种通过 pdb
在命令行调试的方法理论上是万能的,但实在是太麻烦了,如果有一千行代码,要运行到第
999 行得敲多少命令啊。还好,我们还有另一种调试方法。

\hypertarget{pdb.set_trace}{%
\subsubsection{pdb.set\_trace()}\label{pdb.set_trace}}

这个方法也是用
pdb,但是不需要单步执行,我们只需要\texttt{import\ pdb},然后,在可能出错的地方放一个\texttt{pdb.set\_trace()},就可以设置一个断点:

\begin{pythoncode}
import pdb

s = '0'
n = int(s)
pdb.set_trace() 
print(10 / n)
\end{pythoncode}

运行代码,程序会自动在\texttt{pdb.set\_trace()}暂停并进入 pdb
调试环境,可以用命令\texttt{p}查看变量,或者用命令\texttt{c}继续运行:

\begin{pythoncode}
$ python err.py 
> /Users/michael/Github/learn-python3/samples/debug/err.py(7)<module>()
-> print(10 / n)
(Pdb) p n
0
(Pdb) c
Traceback (most recent call last):
  File "err.py", line 7, in <module>
    print(10 / n)
ZeroDivisionError: division by zero
\end{pythoncode}

这个方式比直接启动 pdb 单步调试效率要高很多,但也高不到哪去。

\hypertarget{ide}{%
\subsubsection{IDE}\label{ide}}

如果要比较爽地设置断点、单步执行,就需要一个支持调试功能的
IDE。目前比较好的 Python IDE 有:

Visual Studio Code:\url{https://code.visualstudio.com/},需要安装
Python 插件。

PyCharm:\url{http://www.jetbrains.com/pycharm/}

另外,\href{http://www.eclipse.org/}{Eclipse} 加上
\href{http://pydev.org/}{pydev} 插件也可以调试 Python 程序。

\hypertarget{ux5c0fux7ed3}{%
\subsubsection{小结}\label{ux5c0fux7ed3}}

写程序最痛苦的事情莫过于调试,程序往往会以你意想不到的流程来运行,你期待执行的语句其实根本没有执行,这时候,就需要调试了。

虽然用 IDE 调试起来比较方便,但是最后你会发现,logging 才是终极武器。

\hypertarget{ux53c2ux8003ux6e90ux7801}{%
\subsubsection{参考源码}\label{ux53c2ux8003ux6e90ux7801}}

\href{https://github.com/michaelliao/learn-python3/blob/master/samples/debug/do_assert.py}{do\_assert.py}

\href{https://github.com/michaelliao/learn-python3/blob/master/samples/debug/do_logging.py}{do\_logging.py}

\href{https://github.com/michaelliao/learn-python3/blob/master/samples/debug/do_pdb.py}{do\_pdb.py}

