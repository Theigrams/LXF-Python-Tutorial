\hypertarget{itertools}{%
\subsection{itertools}\label{itertools}}

Python
的内建模块\texttt{itertools}提供了非常有用的用于操作迭代对象的函数。

首先,我们看看\texttt{itertools}提供的几个 ``无限'' 迭代器:

\begin{pythoncode}
>>> import itertools
>>> natuals = itertools.count(1)
>>> for n in natuals:
...     print(n)
...
1
2
3
...
\end{pythoncode}

因为\texttt{count()}会创建一个无限的迭代器,所以上述代码会打印出自然数序列,根本停不下来,只能按\texttt{Ctrl+C}退出。

\texttt{cycle()}会把传入的一个序列无限重复下去:

\begin{pythoncode}
>>> import itertools
>>> cs = itertools.cycle('ABC') 
>>> for c in cs:
...     print(c)
...
'A'
'B'
'C'
'A'
'B'
'C'
...
\end{pythoncode}

同样停不下来。

\texttt{repeat()}负责把一个元素无限重复下去,不过如果提供第二个参数就可以限定重复次数:

\begin{pythoncode}
>>> ns = itertools.repeat('A', 3)
>>> for n in ns:
...     print(n)
...
A
A
A
\end{pythoncode}

无限序列只有在\texttt{for}迭代时才会无限地迭代下去,如果只是创建了一个迭代对象,它不会事先把无限个元素生成出来,事实上也不可能在内存中创建无限多个元素。

无限序列虽然可以无限迭代下去,但是通常我们会通过\texttt{takewhile()}等函数根据条件判断来截取出一个有限的序列:

\begin{pythoncode}
>>> natuals = itertools.count(1)
>>> ns = itertools.takewhile(lambda x: x <= 10, natuals)
>>> list(ns)
[1, 2, 3, 4, 5, 6, 7, 8, 9, 10]
\end{pythoncode}

\texttt{itertools}提供的几个迭代器操作函数更加有用:

\hypertarget{chain}{%
\subsubsection{chain()}\label{chain}}

\texttt{chain()}可以把一组迭代对象串联起来,形成一个更大的迭代器:

\begin{pythoncode}
>>> for c in itertools.chain('ABC', 'XYZ'):
...     print(c)
\end{pythoncode}

\hypertarget{groupby}{%
\subsubsection{groupby()}\label{groupby}}

\texttt{groupby()}把迭代器中相邻的重复元素挑出来放在一起:

\begin{pythoncode}
>>> for key, group in itertools.groupby('AAABBBCCAAA'):
...     print(key, list(group))
...
A ['A', 'A', 'A']
B ['B', 'B', 'B']
C ['C', 'C']
A ['A', 'A', 'A']
\end{pythoncode}

实际上挑选规则是通过函数完成的,只要作用于函数的两个元素返回的值相等,这两个元素就被认为是在一组的,而函数返回值作为组的
key。如果我们要忽略大小写分组,就可以让元素\texttt{\textquotesingle{}A\textquotesingle{}}和\texttt{\textquotesingle{}a\textquotesingle{}}都返回相同的
key:

\begin{pythoncode}
>>> for key, group in itertools.groupby('AaaBBbcCAAa', lambda c: c.upper()):
...     print(key, list(group))
...
A ['A', 'a', 'a']
B ['B', 'B', 'b']
C ['c', 'C']
A ['A', 'A', 'a']
\end{pythoncode}

\hypertarget{ux7ec3ux4e60}{%
\subsubsection{练习}\label{ux7ec3ux4e60}}

计算圆周率可以根据公式:

利用 Python 提供的 itertools 模块,我们来计算这个序列的前 N 项和:

\begin{pythoncode}
# -*- coding: utf-8 -*-
import itertools
\end{pythoncode}

\begin{pythoncode}
# 测试:
print(pi(10))
print(pi(100))
print(pi(1000))
print(pi(10000))
assert 3.04 < pi(10) < 3.05
assert 3.13 < pi(100) < 3.14
assert 3.140 < pi(1000) < 3.141
assert 3.1414 < pi(10000) < 3.1415
print('ok')
\end{pythoncode}

\hypertarget{ux5c0fux7ed3}{%
\subsubsection{小结}\label{ux5c0fux7ed3}}

\texttt{itertools}模块提供的全部是处理迭代功能的函数,它们的返回值不是
list,而是\texttt{Iterator},只有用\texttt{for}循环迭代的时候才真正计算。

\hypertarget{ux53c2ux8003ux6e90ux7801}{%
\subsubsection{参考源码}\label{ux53c2ux8003ux6e90ux7801}}

\href{https://github.com/michaelliao/learn-python3/blob/master/samples/commonlib/use_itertools.py}{use\_itertools.py}

