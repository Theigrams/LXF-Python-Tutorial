\hypertarget{psutil}{%
\subsection{psutil}\label{psutil}}

用 Python 来编写脚本简化日常的运维工作是 Python 的一个重要用途。在 Linux
下,有许多系统命令可以让我们时刻监控系统运行的状态,如\texttt{ps},\texttt{top},\texttt{free}等等。要获取这些系统信息,Python
可以通过\texttt{subprocess}模块调用并获取结果。但这样做显得很麻烦,尤其是要写很多解析代码。

在 Python
中获取系统信息的另一个好办法是使用\texttt{psutil}这个第三方模块。顾名思义,psutil
= process and system
utilities,它不仅可以通过一两行代码实现系统监控,还可以跨平台使用,支持
Linux/UNIX/OSX/Windows
等,是系统管理员和运维小伙伴不可或缺的必备模块。

\hypertarget{ux5b89ux88c5-psutil}{%
\subsubsection{安装 psutil}\label{ux5b89ux88c5-psutil}}

如果安装了 Anaconda,psutil 就已经可用了。否则,需要在命令行下通过 pip
安装:

\begin{pythoncode}
$ pip install psutil
\end{pythoncode}

如果遇到 Permission denied 安装失败,请加上 sudo 重试。

\hypertarget{ux83b7ux53d6-cpu-ux4fe1ux606f}{%
\subsubsection{获取 CPU 信息}\label{ux83b7ux53d6-cpu-ux4fe1ux606f}}

我们先来获取 CPU 的信息:

\begin{pythoncode}
>>> import psutil
>>> psutil.cpu_count() 
4
>>> psutil.cpu_count(logical=False) 
2
\end{pythoncode}

统计 CPU 的用户/系统/空闲时间:

\begin{pythoncode}
>>> psutil.cpu_times()
scputimes(user=10963.31, nice=0.0, system=5138.67, idle=356102.45)
\end{pythoncode}

再实现类似\texttt{top}命令的 CPU 使用率,每秒刷新一次,累计 10 次:

\begin{pythoncode}
>>> for x in range(10):
...     print(psutil.cpu_percent(interval=1, percpu=True))
... 
[14.0, 4.0, 4.0, 4.0]
[12.0, 3.0, 4.0, 3.0]
[8.0, 4.0, 3.0, 4.0]
[12.0, 3.0, 3.0, 3.0]
[18.8, 5.1, 5.9, 5.0]
[10.9, 5.0, 4.0, 3.0]
[12.0, 5.0, 4.0, 5.0]
[15.0, 5.0, 4.0, 4.0]
[19.0, 5.0, 5.0, 4.0]
[9.0, 3.0, 2.0, 3.0]
\end{pythoncode}

\hypertarget{ux83b7ux53d6ux5185ux5b58ux4fe1ux606f}{%
\subsubsection{获取内存信息}\label{ux83b7ux53d6ux5185ux5b58ux4fe1ux606f}}

使用 psutil 获取物理内存和交换内存信息,分别使用:

\begin{pythoncode}
>>> psutil.virtual_memory()
svmem(total=8589934592, available=2866520064, percent=66.6, used=7201386496, free=216178688, active=3342192640, inactive=2650341376, wired=1208852480)
>>> psutil.swap_memory()
sswap(total=1073741824, used=150732800, free=923009024, percent=14.0, sin=10705981440, sout=40353792)
\end{pythoncode}

返回的是字节为单位的整数,可以看到,总内存大小是 8589934592 = 8 GB,已用
7201386496 = 6.7 GB,使用了 66.6\%。

而交换区大小是 1073741824 = 1 GB。

\hypertarget{ux83b7ux53d6ux78c1ux76d8ux4fe1ux606f}{%
\subsubsection{获取磁盘信息}\label{ux83b7ux53d6ux78c1ux76d8ux4fe1ux606f}}

可以通过 psutil 获取磁盘分区、磁盘使用率和磁盘 IO 信息:

\begin{pythoncode}
>>> psutil.disk_partitions() 
[sdiskpart(device='/dev/disk1', mountpoint='/', fstype='hfs', opts='rw,local,rootfs,dovolfs,journaled,multilabel')]
>>> psutil.disk_usage('/') 
sdiskusage(total=998982549504, used=390880133120, free=607840272384, percent=39.1)
>>> psutil.disk_io_counters() 
sdiskio(read_count=988513, write_count=274457, read_bytes=14856830464, write_bytes=17509420032, read_time=2228966, write_time=1618405)
\end{pythoncode}

可以看到,磁盘\texttt{\textquotesingle{}/\textquotesingle{}}的总容量是
998982549504 = 930 GB,使用了 39.1\%。文件格式是
HFS,\texttt{opts}中包含\texttt{rw}表示可读写,\texttt{journaled}表示支持日志。

\hypertarget{ux83b7ux53d6ux7f51ux7edcux4fe1ux606f}{%
\subsubsection{获取网络信息}\label{ux83b7ux53d6ux7f51ux7edcux4fe1ux606f}}

psutil 可以获取网络接口和网络连接信息:

\begin{pythoncode}
>>> psutil.net_io_counters() # 获取网络读写字节/包的个数
snetio(bytes_sent=3885744870, bytes_recv=10357676702, packets_sent=10613069, packets_recv=10423357, errin=0, errout=0, dropin=0, dropout=0)
>>> psutil.net_if_addrs() # 获取网络接口信息
{
  'lo0': [snic(family=<AddressFamily.AF_INET: 2>, address='127.0.0.1', netmask='255.0.0.0'), ...],
  'en1': [snic(family=<AddressFamily.AF_INET: 2>, address='10.0.1.80', netmask='255.255.255.0'), ...],
  'en0': [...],
  'en2': [...],
  'bridge0': [...]
}
>>> psutil.net_if_stats() # 获取网络接口状态
{
  'lo0': snicstats(isup=True, duplex=<NicDuplex.NIC_DUPLEX_UNKNOWN: 0>, speed=0, mtu=16384),
  'en0': snicstats(isup=True, duplex=<NicDuplex.NIC_DUPLEX_UNKNOWN: 0>, speed=0, mtu=1500),
  'en1': snicstats(...),
  'en2': snicstats(...),
  'bridge0': snicstats(...)
}
\end{pythoncode}

要获取当前网络连接信息,使用\texttt{net\_connections()}:

\begin{pythoncode}
>>> psutil.net_connections()
Traceback (most recent call last):
  ...
PermissionError: [Errno 1] Operation not permitted

During handling of the above exception, another exception occurred:

Traceback (most recent call last):
  ...
psutil.AccessDenied: psutil.AccessDenied (pid=3847)
\end{pythoncode}

你可能会得到一个\texttt{AccessDenied}错误,原因是 psutil
获取信息也是要走系统接口,而获取网络连接信息需要 root
权限,这种情况下,可以退出 Python 交互环境,用\texttt{sudo}重新启动:

\begin{pythoncode}
$ sudo python3
Password: ******
Python 3.8 ... on darwin
Type "help", ... for more information.
>>> import psutil
>>> psutil.net_connections()
[
    sconn(fd=83, family=<AddressFamily.AF_INET6: 30>, type=1, laddr=addr(ip='::127.0.0.1', port=62911), raddr=addr(ip='::127.0.0.1', port=3306), status='ESTABLISHED', pid=3725),
    sconn(fd=84, family=<AddressFamily.AF_INET6: 30>, type=1, laddr=addr(ip='::127.0.0.1', port=62905), raddr=addr(ip='::127.0.0.1', port=3306), status='ESTABLISHED', pid=3725),
    sconn(fd=93, family=<AddressFamily.AF_INET6: 30>, type=1, laddr=addr(ip='::', port=8080), raddr=(), status='LISTEN', pid=3725),
    sconn(fd=103, family=<AddressFamily.AF_INET6: 30>, type=1, laddr=addr(ip='::127.0.0.1', port=62918), raddr=addr(ip='::127.0.0.1', port=3306), status='ESTABLISHED', pid=3725),
    sconn(fd=105, family=<AddressFamily.AF_INET6: 30>, type=1, ..., pid=3725),
    sconn(fd=106, family=<AddressFamily.AF_INET6: 30>, type=1, ..., pid=3725),
    sconn(fd=107, family=<AddressFamily.AF_INET6: 30>, type=1, ..., pid=3725),
    ...
    sconn(fd=27, family=<AddressFamily.AF_INET: 2>, type=2, ..., pid=1)
]
\end{pythoncode}

\hypertarget{ux83b7ux53d6ux8fdbux7a0bux4fe1ux606f}{%
\subsubsection{获取进程信息}\label{ux83b7ux53d6ux8fdbux7a0bux4fe1ux606f}}

通过 psutil 可以获取到所有进程的详细信息:

\begin{pythoncode}
>>> psutil.pids() 
[3865, 3864, 3863, 3856, 3855, 3853, 3776, ..., 45, 44, 1, 0]
>>> p = psutil.Process(3776) 
>>> p.name() 
'python3.6'
>>> p.exe() 
'/Users/michael/anaconda3/bin/python3.6'
>>> p.cwd() 
'/Users/michael'
>>> p.cmdline() 
['python3']
>>> p.ppid() 
3765
>>> p.parent() 
<psutil.Process(pid=3765, name='bash') at 4503144040>
>>> p.children() 
[]
>>> p.status() 
'running'
>>> p.username() 
'michael'
>>> p.create_time() 
1511052731.120333
>>> p.terminal() 
'/dev/ttys002'
>>> p.cpu_times() 
pcputimes(user=0.081150144, system=0.053269812, children_user=0.0, children_system=0.0)
>>> p.memory_info() 
pmem(rss=8310784, vms=2481725440, pfaults=3207, pageins=18)
>>> p.open_files() 
[]
>>> p.connections() 
[]
>>> p.num_threads() 
1
>>> p.threads() 
[pthread(id=1, user_time=0.090318, system_time=0.062736)]
>>> p.environ() 
{'SHELL': '/bin/bash', 'PATH': '/usr/local/bin:/usr/bin:/bin:/usr/sbin:/sbin:...', 'PWD': '/Users/michael', 'LANG': 'zh_CN.UTF-8', ...}
>>> p.terminate() 
Terminated: 15 <-- 自己把自己结束了
\end{pythoncode}

和获取网络连接类似,获取一个 root 用户的进程需要 root 权限,启动 Python
交互环境或者\texttt{.py}文件时,需要\texttt{sudo}权限。

psutil
还提供了一个\texttt{test()}函数,可以模拟出\texttt{ps}命令的效果:

\begin{pythoncode}
$ sudo python3
Password: ******
Python 3.6.3 ... on darwin
Type "help", ... for more information.
>>> import psutil
>>> psutil.test()
USER         PID %MEM     VSZ     RSS TTY           START    TIME  COMMAND
root           0 24.0 74270628 2016380 ?             Nov18   40:51  kernel_task
root           1  0.1 2494140    9484 ?             Nov18   01:39  launchd
root          44  0.4 2519872   36404 ?             Nov18   02:02  UserEventAgent
root          45    ? 2474032    1516 ?             Nov18   00:14  syslogd
root          47  0.1 2504768    8912 ?             Nov18   00:03  kextd
root          48  0.1 2505544    4720 ?             Nov18   00:19  fseventsd
_appleeven    52  0.1 2499748    5024 ?             Nov18   00:00  appleeventsd
root          53  0.1 2500592    6132 ?             Nov18   00:02  configd
...
\end{pythoncode}

\hypertarget{ux5c0fux7ed3}{%
\subsubsection{小结}\label{ux5c0fux7ed3}}

psutil 使得 Python 程序获取系统信息变得易如反掌。

psutil 还可以获取用户信息、Windows 服务等很多有用的系统信息,具体请参考
psutil 的官网:\url{https://github.com/giampaolo/psutil}

