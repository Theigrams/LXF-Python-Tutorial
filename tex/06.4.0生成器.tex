\hypertarget{ux751fux6210ux5668}{%
\subsection{生成器}\label{ux751fux6210ux5668}}

通过列表生成式,我们可以直接创建一个列表。但是,受到内存限制,列表容量肯定是有限的。而且,创建一个包含
100
万个元素的列表,不仅占用很大的存储空间,如果我们仅仅需要访问前面几个元素,那后面绝大多数元素占用的空间都白白浪费了。

所以,如果列表元素可以按照某种算法推算出来,那我们是否可以在循环的过程中不断推算出后续的元素呢?这样就不必创建完整的
list,从而节省大量的空间。在 Python
中,这种一边循环一边计算的机制,称为生成器:generator。

要创建一个
generator,有很多种方法。第一种方法很简单,只要把一个列表生成式的\texttt{{[}{]}}改成\texttt{()},就创建了一个
generator:

\begin{pythoncode}
>>> L = [x * x for x in range(10)]
>>> L
[0, 1, 4, 9, 16, 25, 36, 49, 64, 81]
>>> g = (x * x for x in range(10))
>>> g
<generator object <genexpr> at 0x1022ef630>
\end{pythoncode}

创建\texttt{L}和\texttt{g}的区别仅在于最外层的\texttt{{[}{]}}和\texttt{()},\texttt{L}是一个
list,而\texttt{g}是一个 generator。

我们可以直接打印出 list 的每一个元素,但我们怎么打印出 generator
的每一个元素呢?

如果要一个一个打印出来,可以通过\texttt{next()}函数获得 generator
的下一个返回值:

\begin{pythoncode}
>>> next(g)
0
>>> next(g)
1
>>> next(g)
4
>>> next(g)
9
>>> next(g)
16
>>> next(g)
25
>>> next(g)
36
>>> next(g)
49
>>> next(g)
64
>>> next(g)
81
>>> next(g)
Traceback (most recent call last):
  File "<stdin>", line 1, in <module>
StopIteration
\end{pythoncode}

我们讲过,generator
保存的是算法,每次调用\texttt{next(g)},就计算出\texttt{g}的下一个元素的值,直到计算到最后一个元素,没有更多的元素时,抛出\texttt{StopIteration}的错误。

当然,上面这种不断调用\texttt{next(g)}实在是太变态了,正确的方法是使用\texttt{for}循环,因为
generator 也是可迭代对象:

\begin{pythoncode}
>>> g = (x * x for x in range(10))
>>> for n in g:
...     print(n)
... 
0
1
4
9
16
25
36
49
64
81
\end{pythoncode}

所以,我们创建了一个 generator
后,基本上永远不会调用\texttt{next()},而是通过\texttt{for}循环来迭代它,并且不需要关心\texttt{StopIteration}的错误。

generator
非常强大。如果推算的算法比较复杂,用类似列表生成式的\texttt{for}循环无法实现的时候,还可以用函数来实现。

比如,著名的斐波拉契数列(Fibonacci),除第一个和第二个数外,任意一个数都可由前两个数相加得到:

1, 1, 2, 3, 5, 8, 13, 21, 34, \ldots{}

斐波拉契数列用列表生成式写不出来,但是,用函数把它打印出来却很容易:

\begin{pythoncode}
def fib(max):
    n, a, b = 0, 0, 1
    while n < max:
        print(b)
        a, b = b, a + b
        n = n + 1
    return 'done'
\end{pythoncode}

\emph{注意},赋值语句:

\begin{pythoncode}
a, b = b, a + b
\end{pythoncode}

相当于:

\begin{pythoncode}
t = (b, a + b) # t是一个tuple
a = t[0]
b = t[1]
\end{pythoncode}

但不必显式写出临时变量 t 就可以赋值。

上面的函数可以输出斐波那契数列的前 N 个数:

\begin{pythoncode}
>>> fib(6)
1
1
2
3
5
8
'done'
\end{pythoncode}

仔细观察,可以看出,\texttt{fib}函数实际上是定义了斐波拉契数列的推算规则,可以从第一个元素开始,推算出后续任意的元素,这种逻辑其实非常类似
generator。

也就是说,上面的函数和 generator 仅一步之遥。要把\texttt{fib}函数变成
generator,只需要把\texttt{print(b)}改为\texttt{yield\ b}就可以了:

\begin{pythoncode}
def fib(max):
    n, a, b = 0, 0, 1
    while n < max:
        yield b
        a, b = b, a + b
        n = n + 1
    return 'done'
\end{pythoncode}

这就是定义 generator
的另一种方法。如果一个函数定义中包含\texttt{yield}关键字,那么这个函数就不再是一个普通函数,而是一个
generator:

\begin{pythoncode}
>>> f = fib(6)
>>> f
<generator object fib at 0x104feaaa0>
\end{pythoncode}

这里,最难理解的就是 generator
和函数的执行流程不一样。函数是顺序执行,遇到\texttt{return}语句或者最后一行函数语句就返回。而变成
generator
的函数,在每次调用\texttt{next()}的时候执行,遇到\texttt{yield}语句返回,再次执行时从上次返回的\texttt{yield}语句处继续执行。

举个简单的例子,定义一个 generator,依次返回数字 1,3,5:

\begin{pythoncode}
def odd():
    print('step 1')
    yield 1
    print('step 2')
    yield(3)
    print('step 3')
    yield(5)
\end{pythoncode}

调用该 generator 时,首先要生成一个 generator
对象,然后用\texttt{next()}函数不断获得下一个返回值:

\begin{pythoncode}
>>> o = odd()
>>> next(o)
step 1
1
>>> next(o)
step 2
3
>>> next(o)
step 3
5
>>> next(o)
Traceback (most recent call last):
  File "<stdin>", line 1, in <module>
StopIteration
\end{pythoncode}

可以看到,\texttt{odd}不是普通函数,而是
generator,在执行过程中,遇到\texttt{yield}就中断,下次又继续执行。执行
3 次\texttt{yield}后,已经没有\texttt{yield}可以执行了,所以,第 4
次调用\texttt{next(o)}就报错。

回到\texttt{fib}的例子,我们在循环过程中不断调用\texttt{yield},就会不断中断。当然要给循环设置一个条件来退出循环,不然就会产生一个无限数列出来。

同样的,把函数改成 generator
后,我们基本上从来不会用\texttt{next()}来获取下一个返回值,而是直接使用\texttt{for}循环来迭代:

\begin{pythoncode}
>>> for n in fib(6):
...     print(n)
...
1
1
2
3
5
8
\end{pythoncode}

但是用\texttt{for}循环调用 generator 时,发现拿不到 generator
的\texttt{return}语句的返回值。如果想要拿到返回值,必须捕获\texttt{StopIteration}错误,返回值包含在\texttt{StopIteration}的\texttt{value}中:

\begin{pythoncode}
>>> g = fib(6)
>>> while True:
...     try:
...         x = next(g)
...         print('g:', x)
...     except StopIteration as e:
...         print('Generator return value:', e.value)
...         break
...
g: 1
g: 1
g: 2
g: 3
g: 5
g: 8
Generator return value: done
\end{pythoncode}

关于如何捕获错误,后面的错误处理还会详细讲解。

\hypertarget{ux7ec3ux4e60}{%
\subsubsection{练习}\label{ux7ec3ux4e60}}

\href{http://baike.baidu.com/view/7804.htm}{杨辉三角}定义如下:

\begin{pythoncode}
          1
         / \
        1   1
       / \ / \
      1   2   1
     / \ / \ / \
    1   3   3   1
   / \ / \ / \ / \
  1   4   6   4   1
 / \ / \ / \ / \ / \
1   5   10  10  5   1
\end{pythoncode}

把每一行看做一个 list,试写一个 generator,不断输出下一行的 list:

\begin{pythoncode}
# -*- coding: utf-8 -*-

def triangles():
\end{pythoncode}

\begin{pythoncode}
# 期待输出:
# [1]
# [1, 1]
# [1, 2, 1]
# [1, 3, 3, 1]
# [1, 4, 6, 4, 1]
# [1, 5, 10, 10, 5, 1]
# [1, 6, 15, 20, 15, 6, 1]
# [1, 7, 21, 35, 35, 21, 7, 1]
# [1, 8, 28, 56, 70, 56, 28, 8, 1]
# [1, 9, 36, 84, 126, 126, 84, 36, 9, 1]
n = 0
results = []
for t in triangles():
    results.append(t)
    n = n + 1
    if n == 10:
        break

for t in results:
    print(t)

if results == [
    [1],
    [1, 1],
    [1, 2, 1],
    [1, 3, 3, 1],
    [1, 4, 6, 4, 1],
    [1, 5, 10, 10, 5, 1],
    [1, 6, 15, 20, 15, 6, 1],
    [1, 7, 21, 35, 35, 21, 7, 1],
    [1, 8, 28, 56, 70, 56, 28, 8, 1],
    [1, 9, 36, 84, 126, 126, 84, 36, 9, 1]
]:
    print('测试通过!')
else:
    print('测试失败!')
\end{pythoncode}

\hypertarget{ux5c0fux7ed3}{%
\subsubsection{小结}\label{ux5c0fux7ed3}}

generator 是非常强大的工具,在 Python 中,可以简单地把列表生成式改成
generator,也可以通过函数实现复杂逻辑的 generator。

要理解 generator
的工作原理,它是在\texttt{for}循环的过程中不断计算出下一个元素,并在适当的条件结束\texttt{for}循环。对于函数改成的
generator
来说,遇到\texttt{return}语句或者执行到函数体最后一行语句,就是结束
generator 的指令,\texttt{for}循环随之结束。

请注意区分普通函数和 generator 函数,普通函数调用直接返回结果:

\begin{pythoncode}
>>> r = abs(6)
>>> r
6
\end{pythoncode}

generator 函数的 ``调用'' 实际返回一个 generator 对象:

\begin{pythoncode}
>>> g = fib(6)
>>> g
<generator object fib at 0x1022ef948>
\end{pythoncode}

\hypertarget{ux53c2ux8003ux6e90ux7801}{%
\subsubsection{参考源码}\label{ux53c2ux8003ux6e90ux7801}}

\href{https://github.com/michaelliao/learn-python3/blob/master/samples/advance/do_generator.py}{do\_generator.py}

