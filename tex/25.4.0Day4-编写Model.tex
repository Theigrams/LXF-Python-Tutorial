\hypertarget{day-4---ux7f16ux5199-model}{%
\subsection{Day 4 - 编写 Model}\label{day-4---ux7f16ux5199-model}}

有了 ORM,我们就可以把 Web App 需要的 3 个表用\texttt{Model}表示出来:

\begin{pythoncode}
import time, uuid

from orm import Model, StringField, BooleanField, FloatField, TextField

def next_id():
    return '%015d%s000' % (int(time.time() * 1000), uuid.uuid4().hex)

class User(Model):
    __table__ = 'users'

    id = StringField(primary_key=True, default=next_id, ddl='varchar(50)')
    email = StringField(ddl='varchar(50)')
    passwd = StringField(ddl='varchar(50)')
    admin = BooleanField()
    name = StringField(ddl='varchar(50)')
    image = StringField(ddl='varchar(500)')
    created_at = FloatField(default=time.time)

class Blog(Model):
    __table__ = 'blogs'

    id = StringField(primary_key=True, default=next_id, ddl='varchar(50)')
    user_id = StringField(ddl='varchar(50)')
    user_name = StringField(ddl='varchar(50)')
    user_image = StringField(ddl='varchar(500)')
    name = StringField(ddl='varchar(50)')
    summary = StringField(ddl='varchar(200)')
    content = TextField()
    created_at = FloatField(default=time.time)

class Comment(Model):
    __table__ = 'comments'

    id = StringField(primary_key=True, default=next_id, ddl='varchar(50)')
    blog_id = StringField(ddl='varchar(50)')
    user_id = StringField(ddl='varchar(50)')
    user_name = StringField(ddl='varchar(50)')
    user_image = StringField(ddl='varchar(500)')
    content = TextField()
    created_at = FloatField(default=time.time)
\end{pythoncode}

在编写 ORM 时,给一个 Field 增加一个\texttt{default}参数可以让 ORM
自己填入缺省值,非常方便。并且,缺省值可以作为函数对象传入,在调用\texttt{save()}时自动计算。

例如,主键\texttt{id}的缺省值是函数\texttt{next\_id},创建时间\texttt{created\_at}的缺省值是函数\texttt{time.time},可以自动设置当前日期和时间。

日期和时间用\texttt{float}类型存储在数据库中,而不是\texttt{datetime}类型,这么做的好处是不必关心数据库的时区以及时区转换问题,排序非常简单,显示的时候,只需要做一个\texttt{float}到\texttt{str}的转换,也非常容易。

\hypertarget{ux521dux59cbux5316ux6570ux636eux5e93ux8868}{%
\subsubsection{初始化数据库表}\label{ux521dux59cbux5316ux6570ux636eux5e93ux8868}}

如果表的数量很少,可以手写创建表的 SQL 脚本:

\begin{pythoncode}
drop database if exists awesome;

create database awesome;

use awesome;

grant select, insert, update, delete on awesome.* to 'www-data'@'localhost' identified by 'www-data';

create table users (
    `id` varchar(50) not null,
    `email` varchar(50) not null,
    `passwd` varchar(50) not null,
    `admin` bool not null,
    `name` varchar(50) not null,
    `image` varchar(500) not null,
    `created_at` real not null,
    unique key `idx_email` (`email`),
    key `idx_created_at` (`created_at`),
    primary key (`id`)
) engine=innodb default charset=utf8;

create table blogs (
    `id` varchar(50) not null,
    `user_id` varchar(50) not null,
    `user_name` varchar(50) not null,
    `user_image` varchar(500) not null,
    `name` varchar(50) not null,
    `summary` varchar(200) not null,
    `content` mediumtext not null,
    `created_at` real not null,
    key `idx_created_at` (`created_at`),
    primary key (`id`)
) engine=innodb default charset=utf8;

create table comments (
    `id` varchar(50) not null,
    `blog_id` varchar(50) not null,
    `user_id` varchar(50) not null,
    `user_name` varchar(50) not null,
    `user_image` varchar(500) not null,
    `content` mediumtext not null,
    `created_at` real not null,
    key `idx_created_at` (`created_at`),
    primary key (`id`)
) engine=innodb default charset=utf8;
\end{pythoncode}

如果表的数量很多,可以从\texttt{Model}对象直接通过脚本自动生成 SQL
脚本,使用更简单。

把 SQL 脚本放到 MySQL 命令行里执行:

\begin{pythoncode}
$ mysql -u root -p < schema.sql
\end{pythoncode}

我们就完成了数据库表的初始化。

\hypertarget{ux7f16ux5199ux6570ux636eux8bbfux95eeux4ee3ux7801}{%
\subsubsection{编写数据访问代码}\label{ux7f16ux5199ux6570ux636eux8bbfux95eeux4ee3ux7801}}

接下来,就可以真正开始编写代码操作对象了。比如,对于\texttt{User}对象,我们就可以做如下操作:

\begin{pythoncode}
import orm
from models import User, Blog, Comment

def test():
    yield from orm.create_pool(user='www-data', password='www-data', database='awesome')

    u = User(name='Test', email='test@example.com', passwd='1234567890', image='about:blank')

    yield from u.save()

for x in test():
    pass
\end{pythoncode}

可以在 MySQL 客户端命令行查询,看看数据是不是正常存储到 MySQL 里面了。

\hypertarget{ux53c2ux8003ux6e90ux7801}{%
\subsubsection{参考源码}\label{ux53c2ux8003ux6e90ux7801}}

\href{https://github.com/michaelliao/awesome-python3-webapp/tree/day-04}{day-04}

