\hypertarget{ux6b63ux5219ux8868ux8fbeux5f0f}{%
\subsection{正则表达式}\label{ux6b63ux5219ux8868ux8fbeux5f0f}}

字符串是编程时涉及到的最多的一种数据结构,对字符串进行操作的需求几乎无处不在。比如判断一个字符串是否是合法的
Email
地址,虽然可以编程提取\texttt{@}前后的子串,再分别判断是否是单词和域名,但这样做不但麻烦,而且代码难以复用。

正则表达式是一种用来匹配字符串的强有力的武器。它的设计思想是用一种描述性的语言来给字符串定义一个规则,凡是符合规则的字符串,我们就认为它
``匹配'' 了,否则,该字符串就是不合法的。

所以我们判断一个字符串是否是合法的 Email 的方法是:

\begin{enumerate}
\def\labelenumi{\arabic{enumi}.}
\item
  创建一个匹配 Email 的正则表达式;
\item
  用该正则表达式去匹配用户的输入来判断是否合法。
\end{enumerate}

因为正则表达式也是用字符串表示的,所以,我们要首先了解如何用字符来描述字符。

在正则表达式中,如果直接给出字符,就是精确匹配。用\texttt{\textbackslash{}d}可以匹配一个数字,\texttt{\textbackslash{}w}可以匹配一个字母或数字,所以:

\begin{itemize}
\item
  \texttt{\textquotesingle{}00\textbackslash{}d\textquotesingle{}}可以匹配\texttt{\textquotesingle{}007\textquotesingle{}},但无法匹配\texttt{\textquotesingle{}00A\textquotesingle{}};
\item
  \texttt{\textquotesingle{}\textbackslash{}d\textbackslash{}d\textbackslash{}d\textquotesingle{}}可以匹配\texttt{\textquotesingle{}010\textquotesingle{}};
\item
  \texttt{\textquotesingle{}\textbackslash{}w\textbackslash{}w\textbackslash{}d\textquotesingle{}}可以匹配\texttt{\textquotesingle{}py3\textquotesingle{}};
\end{itemize}

\texttt{.}可以匹配任意字符,所以:

\begin{itemize}
\item
  \texttt{\textquotesingle{}py.\textquotesingle{}}可以匹配\texttt{\textquotesingle{}pyc\textquotesingle{}}、\texttt{\textquotesingle{}pyo\textquotesingle{}}、\texttt{\textquotesingle{}py!\textquotesingle{}}等等。
\end{itemize}

要匹配变长的字符,在正则表达式中,用\texttt{*}表示任意个字符(包括 0
个),用\texttt{+}表示至少一个字符,用\texttt{?}表示 0 个或 1
个字符,用\texttt{\{n\}}表示 n 个字符,用\texttt{\{n,m\}}表示 n-m
个字符:

来看一个复杂的例子:\texttt{\textbackslash{}d\{3\}\textbackslash{}s+\textbackslash{}d\{3,8\}}。

我们来从左到右解读一下:

\begin{enumerate}
\def\labelenumi{\arabic{enumi}.}
\item
  \texttt{\textbackslash{}d\{3\}}表示匹配 3
  个数字,例如\texttt{\textquotesingle{}010\textquotesingle{}};
\item
  \texttt{\textbackslash{}s}可以匹配一个空格(也包括 Tab
  等空白符),所以\texttt{\textbackslash{}s+}表示至少有一个空格,例如匹配\texttt{\textquotesingle{}\ \textquotesingle{}},\texttt{\textquotesingle{}\ \textquotesingle{}}等;
\item
  \texttt{\textbackslash{}d\{3,8\}}表示 3-8
  个数字,例如\texttt{\textquotesingle{}1234567\textquotesingle{}}。
\end{enumerate}

综合起来,上面的正则表达式可以匹配以任意个空格隔开的带区号的电话号码。

如果要匹配\texttt{\textquotesingle{}010-12345\textquotesingle{}}这样的号码呢?由于\texttt{\textquotesingle{}-\textquotesingle{}}是特殊字符,在正则表达式中,要用\texttt{\textquotesingle{}\textbackslash{}\textquotesingle{}}转义,所以,上面的正则是\texttt{\textbackslash{}d\{3\}\textbackslash{}-\textbackslash{}d\{3,8\}}。

但是,仍然无法匹配\texttt{\textquotesingle{}010\ -\ 12345\textquotesingle{}},因为带有空格。所以我们需要更复杂的匹配方式。

\hypertarget{ux8fdbux9636}{%
\subsubsection{进阶}\label{ux8fdbux9636}}

要做更精确地匹配,可以用\texttt{{[}{]}}表示范围,比如:

\begin{itemize}
\item
  \texttt{{[}0-9a-zA-Z\textbackslash{}\_{]}}可以匹配一个数字、字母或者下划线;
\item
  \texttt{{[}0-9a-zA-Z\textbackslash{}\_{]}+}可以匹配至少由一个数字、字母或者下划线组成的字符串,比如\texttt{\textquotesingle{}a100\textquotesingle{}},\texttt{\textquotesingle{}0\_Z\textquotesingle{}},\texttt{\textquotesingle{}Py3000\textquotesingle{}}等等;
\item
  \texttt{{[}a-zA-Z\textbackslash{}\_{]}{[}0-9a-zA-Z\textbackslash{}\_{]}*}可以匹配由字母或下划线开头,后接任意个由一个数字、字母或者下划线组成的字符串,也就是
  Python 合法的变量;
\item
  \texttt{{[}a-zA-Z\textbackslash{}\_{]}{[}0-9a-zA-Z\textbackslash{}\_{]}\{0,\ 19\}}更精确地限制了变量的长度是
  1-20 个字符(前面 1 个字符 + 后面最多 19 个字符)。
\end{itemize}

\texttt{A\textbar{}B}可以匹配 A 或
B,所以\texttt{(P\textbar{}p)ython}可以匹配\texttt{\textquotesingle{}Python\textquotesingle{}}或者\texttt{\textquotesingle{}python\textquotesingle{}}。

\texttt{\^{}}表示行的开头,\texttt{\^{}\textbackslash{}d}表示必须以数字开头。

\texttt{\$}表示行的结束,\texttt{\textbackslash{}d\$}表示必须以数字结束。

你可能注意到了,\texttt{py}也可以匹配\texttt{\textquotesingle{}python\textquotesingle{}},但是加上\texttt{\^{}py\$}就变成了整行匹配,就只能匹配\texttt{\textquotesingle{}py\textquotesingle{}}了。

\hypertarget{re-ux6a21ux5757}{%
\subsubsection{re 模块}\label{re-ux6a21ux5757}}

有了准备知识,我们就可以在 Python 中使用正则表达式了。Python
提供\texttt{re}模块,包含所有正则表达式的功能。由于 Python
的字符串本身也用\texttt{\textbackslash{}}转义,所以要特别注意:

\begin{pythoncode}
s = 'ABC\\-001' 

\end{pythoncode}

因此我们强烈建议使用 Python 的\texttt{r}前缀,就不用考虑转义的问题了:

\begin{pythoncode}
s = r'ABC\-001' 

\end{pythoncode}

先看看如何判断正则表达式是否匹配:

\begin{pythoncode}
>>> import re
>>> re.match(r'^\d{3}\-\d{3,8}$', '010-12345')
<_sre.SRE_Match object; span=(0, 9), match='010-12345'>
>>> re.match(r'^\d{3}\-\d{3,8}$', '010 12345')
>>>
\end{pythoncode}

\texttt{match()}方法判断是否匹配,如果匹配成功,返回一个\texttt{Match}对象,否则返回\texttt{None}。常见的判断方法就是:

\begin{pythoncode}
test = '用户输入的字符串'
if re.match(r'正则表达式', test):
    print('ok')
else:
    print('failed')
\end{pythoncode}

\hypertarget{ux5207ux5206ux5b57ux7b26ux4e32}{%
\subsubsection{切分字符串}\label{ux5207ux5206ux5b57ux7b26ux4e32}}

用正则表达式切分字符串比用固定的字符更灵活,请看正常的切分代码:

\begin{pythoncode}
>>> 'a b   c'.split(' ')
['a', 'b', '', '', 'c']
\end{pythoncode}

嗯,无法识别连续的空格,用正则表达式试试:

\begin{pythoncode}
>>> re.split(r'\s+', 'a b   c')
['a', 'b', 'c']
\end{pythoncode}

无论多少个空格都可以正常分割。加入\texttt{,}试试:

\begin{pythoncode}
>>> re.split(r'[\s\,]+', 'a,b, c  d')
['a', 'b', 'c', 'd']
\end{pythoncode}

再加入\texttt{;}试试:

\begin{pythoncode}
>>> re.split(r'[\s\,\;]+', 'a,b;; c  d')
['a', 'b', 'c', 'd']
\end{pythoncode}

如果用户输入了一组标签,下次记得用正则表达式来把不规范的输入转化成正确的数组。

\hypertarget{ux5206ux7ec4}{%
\subsubsection{分组}\label{ux5206ux7ec4}}

除了简单地判断是否匹配之外,正则表达式还有提取子串的强大功能。用\texttt{()}表示的就是要提取的分组(Group)。比如:

\texttt{\^{}(\textbackslash{}d\{3\})-(\textbackslash{}d\{3,8\})\$}分别定义了两个组,可以直接从匹配的字符串中提取出区号和本地号码:

\begin{pythoncode}
>>> m = re.match(r'^(\d{3})-(\d{3,8})$', '010-12345')
>>> m
<_sre.SRE_Match object; span=(0, 9), match='010-12345'>
>>> m.group(0)
'010-12345'
>>> m.group(1)
'010'
>>> m.group(2)
'12345'
\end{pythoncode}

如果正则表达式中定义了组,就可以在\texttt{Match}对象上用\texttt{group()}方法提取出子串来。

注意到\texttt{group(0)}永远是原始字符串,\texttt{group(1)}、\texttt{group(2)}\ldots\ldots{}
表示第 1、2、\ldots\ldots{} 个子串。

提取子串非常有用。来看一个更凶残的例子:

\begin{pythoncode}
>>> t = '19:05:30'
>>> m = re.match(r'^(0[0-9]|1[0-9]|2[0-3]|[0-9])\:(0[0-9]|1[0-9]|2[0-9]|3[0-9]|4[0-9]|5[0-9]|[0-9])\:(0[0-9]|1[0-9]|2[0-9]|3[0-9]|4[0-9]|5[0-9]|[0-9])$', t)
>>> m.groups()
('19', '05', '30')
\end{pythoncode}

这个正则表达式可以直接识别合法的时间。但是有些时候,用正则表达式也无法做到完全验证,比如识别日期:

\begin{pythoncode}
'^(0[1-9]|1[0-2]|[0-9])-(0[1-9]|1[0-9]|2[0-9]|3[0-1]|[0-9])$'
\end{pythoncode}

对于\texttt{\textquotesingle{}2-30\textquotesingle{}},\texttt{\textquotesingle{}4-31\textquotesingle{}}这样的非法日期,用正则还是识别不了,或者说写出来非常困难,这时就需要程序配合识别了。

\hypertarget{ux8d2aux5a6aux5339ux914d}{%
\subsubsection{贪婪匹配}\label{ux8d2aux5a6aux5339ux914d}}

最后需要特别指出的是,正则匹配默认是贪婪匹配,也就是匹配尽可能多的字符。举例如下,匹配出数字后面的\texttt{0}:

\begin{pythoncode}
>>> re.match(r'^(\d+)(0*)$', '102300').groups()
('102300', '')
\end{pythoncode}

由于\texttt{\textbackslash{}d+}采用贪婪匹配,直接把后面的\texttt{0}全部匹配了,结果\texttt{0*}只能匹配空字符串了。

必须让\texttt{\textbackslash{}d+}采用非贪婪匹配(也就是尽可能少匹配),才能把后面的\texttt{0}匹配出来,加个\texttt{?}就可以让\texttt{\textbackslash{}d+}采用非贪婪匹配:

\begin{pythoncode}
>>> re.match(r'^(\d+?)(0*)$', '102300').groups()
('1023', '00')
\end{pythoncode}

\hypertarget{ux7f16ux8bd1}{%
\subsubsection{编译}\label{ux7f16ux8bd1}}

当我们在 Python 中使用正则表达式时,re 模块内部会干两件事情:

\begin{enumerate}
\def\labelenumi{\arabic{enumi}.}
\item
  编译正则表达式,如果正则表达式的字符串本身不合法,会报错;
\item
  用编译后的正则表达式去匹配字符串。
\end{enumerate}

如果一个正则表达式要重复使用几千次,出于效率的考虑,我们可以预编译该正则表达式,接下来重复使用时就不需要编译这个步骤了,直接匹配:

\begin{pythoncode}
>>> import re

>>> re_telephone = re.compile(r'^(\d{3})-(\d{3,8})$')

>>> re_telephone.match('010-12345').groups()
('010', '12345')
>>> re_telephone.match('010-8086').groups()
('010', '8086')
\end{pythoncode}

编译后生成 Regular Expression
对象,由于该对象自己包含了正则表达式,所以调用对应的方法时不用给出正则字符串。

\hypertarget{ux5c0fux7ed3}{%
\subsubsection{小结}\label{ux5c0fux7ed3}}

正则表达式非常强大,要在短短的一节里讲完是不可能的。要讲清楚正则的所有内容,可以写一本厚厚的书了。如果你经常遇到正则表达式的问题,你可能需要一本正则表达式的参考书。

\hypertarget{ux7ec3ux4e60}{%
\subsubsection{练习}\label{ux7ec3ux4e60}}

请尝试写一个验证 Email 地址的正则表达式。版本一应该可以验证出类似的
Email:

\begin{itemize}
\item
  someone@gmail.com
\item
  bill.gates@microsoft.com
\end{itemize}

\begin{pythoncode}
# -*- coding: utf-8 -*-
import re
\end{pythoncode}

\begin{pythoncode}
# 测试:
assert is_valid_email('someone@gmail.com')
assert is_valid_email('bill.gates@microsoft.com')
assert not is_valid_email('bob#example.com')
assert not is_valid_email('mr-bob@example.com')
print('ok')
\end{pythoncode}

版本二可以提取出带名字的 Email 地址:

\begin{itemize}
\item
  tom@voyager.org =\textgreater{} Tom Paris
\item
  bob@example.com =\textgreater{} bob
\end{itemize}

\begin{pythoncode}
# -*- coding: utf-8 -*-
import re
\end{pythoncode}

\begin{pythoncode}
# 测试:
assert name_of_email('<Tom Paris> tom@voyager.org') == 'Tom Paris'
assert name_of_email('tom@voyager.org') == 'tom'
print('ok')
\end{pythoncode}

\hypertarget{ux53c2ux8003ux6e90ux7801}{%
\subsubsection{参考源码}\label{ux53c2ux8003ux6e90ux7801}}

\href{https://github.com/michaelliao/learn-python3/blob/master/samples/regex/regex.py}{regex.py}

