\hypertarget{collections}{%
\subsection{collections}\label{collections}}

collections 是 Python 内建的一个集合模块,提供了许多有用的集合类。

\hypertarget{namedtuple}{%
\subsubsection{namedtuple}\label{namedtuple}}

我们知道\texttt{tuple}可以表示不变集合,例如,一个点的二维坐标就可以表示成:

\begin{pythoncode}
>>> p = (1, 2)
\end{pythoncode}

但是,看到\texttt{(1,\ 2)},很难看出这个\texttt{tuple}是用来表示一个坐标的。

定义一个 class 又小题大做了,这时,\texttt{namedtuple}就派上了用场:

\begin{pythoncode}
>>> from collections import namedtuple
>>> Point = namedtuple('Point', ['x', 'y'])
>>> p = Point(1, 2)
>>> p.x
1
>>> p.y
2
\end{pythoncode}

\texttt{namedtuple}是一个函数,它用来创建一个自定义的\texttt{tuple}对象,并且规定了\texttt{tuple}元素的个数,并可以用属性而不是索引来引用\texttt{tuple}的某个元素。

这样一来,我们用\texttt{namedtuple}可以很方便地定义一种数据类型,它具备
tuple 的不变性,又可以根据属性来引用,使用十分方便。

可以验证创建的\texttt{Point}对象是\texttt{tuple}的一种子类:

\begin{pythoncode}
>>> isinstance(p, Point)
True
>>> isinstance(p, tuple)
True
\end{pythoncode}

类似的,如果要用坐标和半径表示一个圆,也可以用\texttt{namedtuple}定义:

\begin{pythoncode}
Circle = namedtuple('Circle', ['x', 'y', 'r'])
\end{pythoncode}

\hypertarget{deque}{%
\subsubsection{deque}\label{deque}}

使用\texttt{list}存储数据时,按索引访问元素很快,但是插入和删除元素就很慢了,因为\texttt{list}是线性存储,数据量大的时候,插入和删除效率很低。

deque 是为了高效实现插入和删除操作的双向列表,适合用于队列和栈:

\begin{pythoncode}
>>> from collections import deque
>>> q = deque(['a', 'b', 'c'])
>>> q.append('x')
>>> q.appendleft('y')
>>> q
deque(['y', 'a', 'b', 'c', 'x'])
\end{pythoncode}

\texttt{deque}除了实现 list
的\texttt{append()}和\texttt{pop()}外,还支持\texttt{appendleft()}和\texttt{popleft()},这样就可以非常高效地往头部添加或删除元素。

\hypertarget{defaultdict}{%
\subsubsection{defaultdict}\label{defaultdict}}

使用\texttt{dict}时,如果引用的 Key
不存在,就会抛出\texttt{KeyError}。如果希望 key
不存在时,返回一个默认值,就可以用\texttt{defaultdict}:

\begin{pythoncode}
>>> from collections import defaultdict
>>> dd = defaultdict(lambda: 'N/A')
>>> dd['key1'] = 'abc'
>>> dd['key1'] 
'abc'
>>> dd['key2'] 
'N/A'
\end{pythoncode}

注意默认值是调用函数返回的,而函数在创建\texttt{defaultdict}对象时传入。

除了在 Key
不存在时返回默认值,\texttt{defaultdict}的其他行为跟\texttt{dict}是完全一样的。

\hypertarget{ordereddict}{%
\subsubsection{OrderedDict}\label{ordereddict}}

使用\texttt{dict}时,Key
是无序的。在对\texttt{dict}做迭代时,我们无法确定 Key 的顺序。

如果要保持 Key 的顺序,可以用\texttt{OrderedDict}:

\begin{pythoncode}
>>> from collections import OrderedDict
>>> d = dict([('a', 1), ('b', 2), ('c', 3)])
>>> d 
{'a': 1, 'c': 3, 'b': 2}
>>> od = OrderedDict([('a', 1), ('b', 2), ('c', 3)])
>>> od 
OrderedDict([('a', 1), ('b', 2), ('c', 3)])
\end{pythoncode}

注意,\texttt{OrderedDict}的 Key 会按照插入的顺序排列,不是 Key
本身排序:

\begin{pythoncode}
>>> od = OrderedDict()
>>> od['z'] = 1
>>> od['y'] = 2
>>> od['x'] = 3
>>> list(od.keys()) 
['z', 'y', 'x']
\end{pythoncode}

\texttt{OrderedDict}可以实现一个 FIFO(先进先出)的
dict,当容量超出限制时,先删除最早添加的 Key:

\begin{pythoncode}
from collections import OrderedDict

class LastUpdatedOrderedDict(OrderedDict):

    def __init__(self, capacity):
        super(LastUpdatedOrderedDict, self).__init__()
        self._capacity = capacity

    def __setitem__(self, key, value):
        containsKey = 1 if key in self else 0
        if len(self) - containsKey >= self._capacity:
            last = self.popitem(last=False)
            print('remove:', last)
        if containsKey:
            del self[key]
            print('set:', (key, value))
        else:
            print('add:', (key, value))
        OrderedDict.__setitem__(self, key, value)
\end{pythoncode}

\hypertarget{chainmap}{%
\subsubsection{ChainMap}\label{chainmap}}

\texttt{ChainMap}可以把一组\texttt{dict}串起来并组成一个逻辑上的\texttt{dict}。\texttt{ChainMap}本身也是一个
dict,但是查找的时候,会按照顺序在内部的 dict 依次查找。

什么时候使用\texttt{ChainMap}最合适?举个例子:应用程序往往都需要传入参数,参数可以通过命令行传入,可以通过环境变量传入,还可以有默认参数。我们可以用\texttt{ChainMap}实现参数的优先级查找,即先查命令行参数,如果没有传入,再查环境变量,如果没有,就使用默认参数。

下面的代码演示了如何查找\texttt{user}和\texttt{color}这两个参数:

\begin{pythoncode}
from collections import ChainMap
import os, argparse
defaults = {
    'color': 'red',
    'user': 'guest'
}
parser = argparse.ArgumentParser()
parser.add_argument('-u', '--user')
parser.add_argument('-c', '--color')
namespace = parser.parse_args()
command_line_args = { k: v for k, v in vars(namespace).items() if v }
combined = ChainMap(command_line_args, os.environ, defaults)
print('color=%s' % combined['color'])
print('user=%s' % combined['user'])
\end{pythoncode}

没有任何参数时,打印出默认参数:

\begin{pythoncode}
$ python3 use_chainmap.py 
color=red
user=guest
\end{pythoncode}

当传入命令行参数时,优先使用命令行参数:

\begin{pythoncode}
$ python3 use_chainmap.py -u bob
color=red
user=bob
\end{pythoncode}

同时传入命令行参数和环境变量,命令行参数的优先级较高:

\begin{pythoncode}
$ user=admin color=green python3 use_chainmap.py -u bob
color=green
user=bob
\end{pythoncode}

\hypertarget{counter}{%
\subsubsection{Counter}\label{counter}}

\texttt{Counter}是一个简单的计数器,例如,统计字符出现的个数:

\begin{pythoncode}
>>> from collections import Counter
>>> c = Counter()
>>> for ch in 'programming':
...     c[ch] = c[ch] + 1
...
>>> c
Counter({'g': 2, 'm': 2, 'r': 2, 'a': 1, 'i': 1, 'o': 1, 'n': 1, 'p': 1})
>>> c.update('hello') 
>>> c
Counter({'r': 2, 'o': 2, 'g': 2, 'm': 2, 'l': 2, 'p': 1, 'a': 1, 'i': 1, 'n': 1, 'h': 1, 'e': 1})
\end{pythoncode}

\texttt{Counter}实际上也是\texttt{dict}的一个子类,上面的结果可以看出每个字符出现的次数。

\hypertarget{ux5c0fux7ed3}{%
\subsubsection{小结}\label{ux5c0fux7ed3}}

\texttt{collections}模块提供了一些有用的集合类,可以根据需要选用。

\hypertarget{ux53c2ux8003ux6e90ux7801}{%
\subsubsection{参考源码}\label{ux53c2ux8003ux6e90ux7801}}

\href{https://github.com/michaelliao/learn-python3/blob/master/samples/commonlib/use_collections.py}{use\_collections.py}

