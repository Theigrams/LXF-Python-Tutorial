\hypertarget{day-6---ux7f16ux5199ux914dux7f6eux6587ux4ef6}{%
\subsection{Day 6 -
编写配置文件}\label{day-6---ux7f16ux5199ux914dux7f6eux6587ux4ef6}}

有了 Web 框架和 ORM 框架,我们就可以开始装配 App 了。

通常,一个 Web App
在运行时都需要读取配置文件,比如数据库的用户名、口令等,在不同的环境中运行时,Web
App 可以通过读取不同的配置文件来获得正确的配置。

由于 Python 本身语法简单,完全可以直接用 Python
源代码来实现配置,而不需要再解析一个单独的\texttt{.properties}或者\texttt{.yaml}等配置文件。

默认的配置文件应该完全符合本地开发环境,这样,无需任何设置,就可以立刻启动服务器。

我们把默认的配置文件命名为\texttt{config\_default.py}:

\begin{pythoncode}
configs = {
    'db': {
        'host': '127.0.0.1',
        'port': 3306,
        'user': 'www-data',
        'password': 'www-data',
        'database': 'awesome'
    },
    'session': {
        'secret': 'AwEsOmE'
    }
}
\end{pythoncode}

上述配置文件简单明了。但是,如果要部署到服务器时,通常需要修改数据库的
host
等信息,直接修改\texttt{config\_default.py}不是一个好办法,更好的方法是编写一个\texttt{config\_override.py},用来覆盖某些默认设置:

\begin{pythoncode}
configs = {
    'db': {
        'host': '192.168.0.100'
    }
}
\end{pythoncode}

把\texttt{config\_default.py}作为开发环境的标准配置,把\texttt{config\_override.py}作为生产环境的标准配置,我们就可以既方便地在本地开发,又可以随时把应用部署到服务器上。

应用程序读取配置文件需要优先从\texttt{config\_override.py}读取。为了简化读取配置文件,可以把所有配置读取到统一的\texttt{config.py}中:

\begin{pythoncode}
configs = config_default.configs

try:
    import config_override
    configs = merge(configs, config_override.configs)
except ImportError:
    pass
\end{pythoncode}

这样,我们就完成了 App 的配置。

\hypertarget{ux53c2ux8003ux6e90ux7801}{%
\subsubsection{参考源码}\label{ux53c2ux8003ux6e90ux7801}}

\href{https://github.com/michaelliao/awesome-python3-webapp/tree/day-06}{day-06}

