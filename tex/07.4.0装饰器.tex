\hypertarget{ux88c5ux9970ux5668}{%
\subsection{装饰器}\label{ux88c5ux9970ux5668}}

由于函数也是一个对象,而且函数对象可以被赋值给变量,所以,通过变量也能调用该函数。

\begin{pythoncode}
>>> def now():
...     print('2015-3-25')
...
>>> f = now
>>> f()
2015-3-25
\end{pythoncode}

函数对象有一个\texttt{\_\_name\_\_}属性,可以拿到函数的名字:

\begin{pythoncode}
>>> now.__name__
'now'
>>> f.__name__
'now'
\end{pythoncode}

现在,假设我们要增强\texttt{now()}函数的功能,比如,在函数调用前后自动打印日志,但又不希望修改\texttt{now()}函数的定义,这种在代码运行期间动态增加功能的方式,称之为
``装饰器''(Decorator)。

本质上,decorator
就是一个返回函数的高阶函数。所以,我们要定义一个能打印日志的
decorator,可以定义如下:

\begin{pythoncode}
def log(func):
    def wrapper(*args, **kw):
        print('call %s():' % func.__name__)
        return func(*args, **kw)
    return wrapper
\end{pythoncode}

观察上面的\texttt{log},因为它是一个
decorator,所以接受一个函数作为参数,并返回一个函数。我们要借助 Python
的 @语法,把 decorator 置于函数的定义处:

\begin{pythoncode}
@log
def now():
    print('2015-3-25')
\end{pythoncode}

调用\texttt{now()}函数,不仅会运行\texttt{now()}函数本身,还会在运行\texttt{now()}函数前打印一行日志:

\begin{pythoncode}
>>> now()
call now():
2015-3-25
\end{pythoncode}

把\texttt{@log}放到\texttt{now()}函数的定义处,相当于执行了语句:

\begin{pythoncode}
now = log(now)
\end{pythoncode}

由于\texttt{log()}是一个
decorator,返回一个函数,所以,原来的\texttt{now()}函数仍然存在,只是现在同名的\texttt{now}变量指向了新的函数,于是调用\texttt{now()}将执行新函数,即在\texttt{log()}函数中返回的\texttt{wrapper()}函数。

\texttt{wrapper()}函数的参数定义是\texttt{(*args,\ **kw)},因此,\texttt{wrapper()}函数可以接受任意参数的调用。在\texttt{wrapper()}函数内,首先打印日志,再紧接着调用原始函数。

如果 decorator 本身需要传入参数,那就需要编写一个返回 decorator
的高阶函数,写出来会更复杂。比如,要自定义 log 的文本:

\begin{pythoncode}
def log(text):
    def decorator(func):
        def wrapper(*args, **kw):
            print('%s %s():' % (text, func.__name__))
            return func(*args, **kw)
        return wrapper
    return decorator
\end{pythoncode}

这个 3 层嵌套的 decorator 用法如下:

\begin{pythoncode}
@log('execute')
def now():
    print('2015-3-25')
\end{pythoncode}

执行结果如下:

\begin{pythoncode}
>>> now()
execute now():
2015-3-25
\end{pythoncode}

和两层嵌套的 decorator 相比,3 层嵌套的效果是这样的:

\begin{pythoncode}
>>> now = log('execute')(now)
\end{pythoncode}

我们来剖析上面的语句,首先执行\texttt{log(\textquotesingle{}execute\textquotesingle{})},返回的是\texttt{decorator}函数,再调用返回的函数,参数是\texttt{now}函数,返回值最终是\texttt{wrapper}函数。

以上两种 decorator
的定义都没有问题,但还差最后一步。因为我们讲了函数也是对象,它有\texttt{\_\_name\_\_}等属性,但你去看经过
decorator
装饰之后的函数,它们的\texttt{\_\_name\_\_}已经从原来的\texttt{\textquotesingle{}now\textquotesingle{}}变成了\texttt{\textquotesingle{}wrapper\textquotesingle{}}:

\begin{pythoncode}
>>> now.__name__
'wrapper'
\end{pythoncode}

因为返回的那个\texttt{wrapper()}函数名字就是\texttt{\textquotesingle{}wrapper\textquotesingle{}},所以,需要把原始函数的\texttt{\_\_name\_\_}等属性复制到\texttt{wrapper()}函数中,否则,有些依赖函数签名的代码执行就会出错。

不需要编写\texttt{wrapper.\_\_name\_\_\ =\ func.\_\_name\_\_}这样的代码,Python
内置的\texttt{functools.wraps}就是干这个事的,所以,一个完整的 decorator
的写法如下:

\begin{pythoncode}
import functools

def log(func):
    @functools.wraps(func)
    def wrapper(*args, **kw):
        print('call %s():' % func.__name__)
        return func(*args, **kw)
    return wrapper
\end{pythoncode}

或者针对带参数的 decorator:

\begin{pythoncode}
import functools

def log(text):
    def decorator(func):
        @functools.wraps(func)
        def wrapper(*args, **kw):
            print('%s %s():' % (text, func.__name__))
            return func(*args, **kw)
        return wrapper
    return decorator
\end{pythoncode}

\texttt{import\ functools}是导入\texttt{functools}模块。模块的概念稍候讲解。现在,只需记住在定义\texttt{wrapper()}的前面加上\texttt{@functools.wraps(func)}即可。

\hypertarget{ux7ec3ux4e60}{%
\subsubsection{练习}\label{ux7ec3ux4e60}}

请设计一个 decorator,它可作用于任何函数上,并打印该函数的执行时间:

\begin{pythoncode}
# -*- coding: utf-8 -*-
import time, functools
\end{pythoncode}

\begin{pythoncode}
# 测试
@metric
def fast(x, y):
    time.sleep(0.0012)
    return x + y;

@metric
def slow(x, y, z):
    time.sleep(0.1234)
    return x * y * z;

f = fast(11, 22)
s = slow(11, 22, 33)
if f != 33:
    print('测试失败!')
elif s != 7986:
    print('测试失败!')
\end{pythoncode}

\hypertarget{ux5c0fux7ed3}{%
\subsubsection{小结}\label{ux5c0fux7ed3}}

在面向对象(OOP)的设计模式中,decorator 被称为装饰模式。OOP
的装饰模式需要通过继承和组合来实现,而 Python 除了能支持 OOP 的
decorator 外,直接从语法层次支持 decorator。Python 的 decorator
可以用函数实现,也可以用类实现。

decorator
可以增强函数的功能,定义起来虽然有点复杂,但使用起来非常灵活和方便。

请编写一个
decorator,能在函数调用的前后打印出\texttt{\textquotesingle{}begin\ call\textquotesingle{}}和\texttt{\textquotesingle{}end\ call\textquotesingle{}}的日志。

再思考一下能否写出一个\texttt{@log}的 decorator,使它既支持:

\begin{pythoncode}
@log
def f():
    pass
\end{pythoncode}

又支持:

\begin{pythoncode}
@log('execute')
def f():
    pass
\end{pythoncode}

\hypertarget{ux53c2ux8003ux6e90ux7801}{%
\subsubsection{参考源码}\label{ux53c2ux8003ux6e90ux7801}}

\href{https://github.com/michaelliao/learn-python3/blob/master/samples/functional/decorator.py}{decorator.py}

