\hypertarget{urllib}{%
\subsection{urllib}\label{urllib}}

urllib 提供了一系列用于操作 URL 的功能。

\hypertarget{get}{%
\subsubsection{Get}\label{get}}

urllib 的\texttt{request}模块可以非常方便地抓取 URL 内容,也就是发送一个
GET 请求到指定的页面,然后返回 HTTP 的响应:

例如,对豆瓣的一个
URL\texttt{https://api.douban.com/v2/book/2129650}进行抓取,并返回响应:

\begin{pythoncode}
from urllib import request

with request.urlopen('https://api.douban.com/v2/book/2129650') as f:
    data = f.read()
    print('Status:', f.status, f.reason)
    for k, v in f.getheaders():
        print('%s: %s' % (k, v))
    print('Data:', data.decode('utf-8'))
\end{pythoncode}

可以看到 HTTP 响应的头和 JSON 数据:

\begin{pythoncode}
Status: 200 OK
Server: nginx
Date: Tue, 26 May 2015 10:02:27 GMT
Content-Type: application/json; charset=utf-8
Content-Length: 2049
Connection: close
Expires: Sun, 1 Jan 2006 01:00:00 GMT
Pragma: no-cache
Cache-Control: must-revalidate, no-cache, private
X-DAE-Node: pidl1
Data: {"rating":{"max":10,"numRaters":16,"average":"7.4","min":0},"subtitle":"","author":["廖雪峰编著"],"pubdate":"2007-6",...}
\end{pythoncode}

如果我们要想模拟浏览器发送 GET
请求,就需要使用\texttt{Request}对象,通过往\texttt{Request}对象添加
HTTP 头,我们就可以把请求伪装成浏览器。例如,模拟 iPhone 6
去请求豆瓣首页:

\begin{pythoncode}
from urllib import request

req = request.Request('http://www.douban.com/')
req.add_header('User-Agent', 'Mozilla/6.0 (iPhone; CPU iPhone OS 8_0 like Mac OS X) AppleWebKit/536.26 (KHTML, like Gecko) Version/8.0 Mobile/10A5376e Safari/8536.25')
with request.urlopen(req) as f:
    print('Status:', f.status, f.reason)
    for k, v in f.getheaders():
        print('%s: %s' % (k, v))
    print('Data:', f.read().decode('utf-8'))
\end{pythoncode}

这样豆瓣会返回适合 iPhone 的移动版网页:

\begin{pythoncode}
...
    <meta >
    <meta >
    <link rel="apple-touch-icon" sizes="57x57" href="http://img4.douban.com/pics/cardkit/launcher/57.png" />
...
\end{pythoncode}

\hypertarget{post}{%
\subsubsection{Post}\label{post}}

如果要以 POST 发送一个请求,只需要把参数\texttt{data}以 bytes 形式传入。

我们模拟一个微博登录,先读取登录的邮箱和口令,然后按照 weibo.cn
的登录页的格式以\texttt{username=xxx\&password=xxx}的编码传入:

\begin{pythoncode}
from urllib import request, parse

print('Login to weibo.cn...')
email = input('Email: ')
passwd = input('Password: ')
login_data = parse.urlencode([
    ('username', email),
    ('password', passwd),
    ('entry', 'mweibo'),
    ('client_id', ''),
    ('savestate', '1'),
    ('ec', ''),
    ('pagerefer', 'https://passport.weibo.cn/signin/welcome?entry=mweibo&r=http%3A%2F%2Fm.weibo.cn%2F')
])

req = request.Request('https://passport.weibo.cn/sso/login')
req.add_header('Origin', 'https://passport.weibo.cn')
req.add_header('User-Agent', 'Mozilla/6.0 (iPhone; CPU iPhone OS 8_0 like Mac OS X) AppleWebKit/536.26 (KHTML, like Gecko) Version/8.0 Mobile/10A5376e Safari/8536.25')
req.add_header('Referer', 'https://passport.weibo.cn/signin/login?entry=mweibo&res=wel&wm=3349&r=http%3A%2F%2Fm.weibo.cn%2F')

with request.urlopen(req, data=login_data.encode('utf-8')) as f:
    print('Status:', f.status, f.reason)
    for k, v in f.getheaders():
        print('%s: %s' % (k, v))
    print('Data:', f.read().decode('utf-8'))
\end{pythoncode}

如果登录成功,我们获得的响应如下:

\begin{pythoncode}
Status: 200 OK
Server: nginx/1.2.0
...
Set-Cookie: SSOLoginState=1432620126; path=/; domain=weibo.cn
...
Data: {"retcode":20000000,"msg":"","data":{...,"uid":"1658384301"}}
\end{pythoncode}

如果登录失败,我们获得的响应如下:

\begin{pythoncode}
...
Data: {"retcode":50011015,"msg":"\u7528\u6237\u540d\u6216\u5bc6\u7801\u9519\u8bef","data":{"username":"example@python.org","errline":536}}
\end{pythoncode}

\hypertarget{handler}{%
\subsubsection{Handler}\label{handler}}

如果还需要更复杂的控制,比如通过一个 Proxy
去访问网站,我们需要利用\texttt{ProxyHandler}来处理,示例代码如下:

\begin{pythoncode}
proxy_handler = urllib.request.ProxyHandler({'http': 'http://www.example.com:3128/'})
proxy_auth_handler = urllib.request.ProxyBasicAuthHandler()
proxy_auth_handler.add_password('realm', 'host', 'username', 'password')
opener = urllib.request.build_opener(proxy_handler, proxy_auth_handler)
with opener.open('http://www.example.com/login.html') as f:
    pass
\end{pythoncode}

\hypertarget{ux5c0fux7ed3}{%
\subsubsection{小结}\label{ux5c0fux7ed3}}

urllib 提供的功能就是利用程序去执行各种 HTTP
请求。如果要模拟浏览器完成特定功能,需要把请求伪装成浏览器。伪装的方法是先监控浏览器发出的请求,再根据浏览器的请求头来伪装,\texttt{User-Agent}头就是用来标识浏览器的。

\hypertarget{ux7ec3ux4e60}{%
\subsubsection{练习}\label{ux7ec3ux4e60}}

利用 urllib 读取 JSON,然后将 JSON 解析为 Python 对象:

\begin{pythoncode}
# -*- coding: utf-8 -*-
from urllib import request
\end{pythoncode}

\begin{pythoncode}
# 测试
URL = 'https://query.yahooapis.com/v1/public/yql?q=select%20*%20from%20weather.forecast%20where%20woeid%20%3D%202151330&format=json'
data = fetch_data(URL)
print(data)
assert data['query']['results']['channel']['location']['city'] == 'Beijing'
print('ok')
\end{pythoncode}

\hypertarget{ux53c2ux8003ux6e90ux7801}{%
\subsubsection{参考源码}\label{ux53c2ux8003ux6e90ux7801}}

\href{https://github.com/michaelliao/learn-python3/blob/master/samples/commonlib/use_urllib.py}{use\_urllib.py}

