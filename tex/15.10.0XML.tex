\hypertarget{xml}{%
\subsection{XML}\label{xml}}

XML 虽然比 JSON 复杂,在 Web
中应用也不如以前多了,不过仍有很多地方在用,所以,有必要了解如何操作
XML。

\hypertarget{dom-vs-sax}{%
\subsubsection{DOM vs SAX}\label{dom-vs-sax}}

操作 XML 有两种方法:DOM 和 SAX。DOM 会把整个 XML
读入内存,解析为树,因此占用内存大,解析慢,优点是可以任意遍历树的节点。SAX
是流模式,边读边解析,占用内存小,解析快,缺点是我们需要自己处理事件。

正常情况下,优先考虑 SAX,因为 DOM 实在太占内存。

在 Python 中使用 SAX 解析 XML
非常简洁,通常我们关心的事件是\texttt{start\_element},\texttt{end\_element}和\texttt{char\_data},准备好这
3 个函数,然后就可以解析 xml 了。

举个例子,当 SAX 解析器读到一个节点时:

\begin{pythoncode}
<a href="/">python</a>
\end{pythoncode}

会产生 3 个事件:

\begin{enumerate}
\def\labelenumi{\arabic{enumi}.}
\item
  start\_element
  事件,在读取\texttt{\textless{}a\ href="/"\textgreater{}}时;
\item
  char\_data 事件,在读取\texttt{python}时;
\item
  end\_element 事件,在读取\texttt{\textless{}/a\textgreater{}}时。
\end{enumerate}

用代码实验一下:

\begin{pythoncode}
from xml.parsers.expat import ParserCreate

class DefaultSaxHandler(object):
    def start_element(self, name, attrs):
        print('sax:start_element: %s, attrs: %s' % (name, str(attrs)))

    def end_element(self, name):
        print('sax:end_element: %s' % name)

    def char_data(self, text):
        print('sax:char_data: %s' % text)

xml = r'''<?xml version="1.0"?>
<ol>
    <li><a href="/python">Python</a></li>
    <li><a href="/ruby">Ruby</a></li>
</ol>
'''

handler = DefaultSaxHandler()
parser = ParserCreate()
parser.StartElementHandler = handler.start_element
parser.EndElementHandler = handler.end_element
parser.CharacterDataHandler = handler.char_data
parser.Parse(xml)
\end{pythoncode}

需要注意的是读取一大段字符串时,\texttt{CharacterDataHandler}可能被多次调用,所以需要自己保存起来,在\texttt{EndElementHandler}里面再合并。

除了解析 XML 外,如何生成 XML 呢?99\% 的情况下需要生成的 XML
结构都是非常简单的,因此,最简单也是最有效的生成 XML
的方法是拼接字符串:

\begin{pythoncode}
L = []
L.append(r'<?xml version="1.0"?>')
L.append(r'<root>')
L.append(encode('some & data'))
L.append(r'</root>')
return ''.join(L)
\end{pythoncode}

如果要生成复杂的 XML 呢?建议你不要用 XML,改成 JSON。

\hypertarget{ux5c0fux7ed3}{%
\subsubsection{小结}\label{ux5c0fux7ed3}}

解析 XML
时,注意找出自己感兴趣的节点,响应事件时,把节点数据保存起来。解析完毕后,就可以处理数据。

\hypertarget{ux7ec3ux4e60}{%
\subsubsection{练习}\label{ux7ec3ux4e60}}

请利用 SAX 编写程序解析 Yahoo 的 XML 格式的天气预报,获取天气预报:

https://query.yahooapis.com/v1/public/yql?q=select\%20*\%20from\%20weather.forecast\%20where\%20woeid\%20\%3D\%202151330\&format=xml

参数\texttt{woeid}是城市代码,要查询某个城市代码,可以在
\href{https://weather.yahoo.com/}{weather.yahoo.com}
搜索城市,浏览器地址栏的 URL 就包含城市代码。

\begin{pythoncode}
# -*- coding:utf-8 -*-

from xml.parsers.expat import ParserCreate
from urllib import request
\end{pythoncode}

\begin{pythoncode}
# 测试:
URL = 'https://query.yahooapis.com/v1/public/yql?q=select%20*%20from%20weather.forecast%20where%20woeid%20%3D%202151330&format=xml'

with request.urlopen(URL, timeout=4) as f:
    data = f.read()

result = parseXml(data.decode('utf-8'))
assert result['city'] == 'Beijing'
\end{pythoncode}

\hypertarget{ux53c2ux8003ux6e90ux7801}{%
\subsubsection{参考源码}\label{ux53c2ux8003ux6e90ux7801}}

\href{https://github.com/michaelliao/learn-python3/blob/master/samples/commonlib/use_sax.py}{use\_sax.py}

