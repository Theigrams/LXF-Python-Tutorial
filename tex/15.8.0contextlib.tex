\hypertarget{contextlib}{%
\subsection{contextlib}\label{contextlib}}

在 Python
中,读写文件这样的资源要特别注意,必须在使用完毕后正确关闭它们。正确关闭文件资源的一个方法是使用\texttt{try...finally}:

\begin{pythoncode}
try:
    f = open('/path/to/file', 'r')
    f.read()
finally:
    if f:
        f.close()
\end{pythoncode}

写\texttt{try...finally}非常繁琐。Python
的\texttt{with}语句允许我们非常方便地使用资源,而不必担心资源没有关闭,所以上面的代码可以简化为:

\begin{pythoncode}
with open('/path/to/file', 'r') as f:
    f.read()
\end{pythoncode}

并不是只有\texttt{open()}函数返回的 fp
对象才能使用\texttt{with}语句。实际上,任何对象,只要正确实现了上下文管理,就可以用于\texttt{with}语句。

实现上下文管理是通过\texttt{\_\_enter\_\_}和\texttt{\_\_exit\_\_}这两个方法实现的。例如,下面的
class 实现了这两个方法:

\begin{pythoncode}
class Query(object):

    def __init__(self, name):
        self.name = name

    def __enter__(self):
        print('Begin')
        return self
    
    def __exit__(self, exc_type, exc_value, traceback):
        if exc_type:
            print('Error')
        else:
            print('End')
    
    def query(self):
        print('Query info about %s...' % self.name)
\end{pythoncode}

这样我们就可以把自己写的资源对象用于\texttt{with}语句:

\begin{pythoncode}
with Query('Bob') as q:
    q.query()
\end{pythoncode}

\hypertarget{contextmanager}{%
\subsubsection{@contextmanager}\label{contextmanager}}

编写\texttt{\_\_enter\_\_}和\texttt{\_\_exit\_\_}仍然很繁琐,因此 Python
的标准库\texttt{contextlib}提供了更简单的写法,上面的代码可以改写如下:

\begin{pythoncode}
from contextlib import contextmanager

class Query(object):

    def __init__(self, name):
        self.name = name

    def query(self):
        print('Query info about %s...' % self.name)

@contextmanager
def create_query(name):
    print('Begin')
    q = Query(name)
    yield q
    print('End')
\end{pythoncode}

\texttt{@contextmanager}这个 decorator 接受一个
generator,用\texttt{yield}语句把\texttt{with\ ...\ as\ var}把变量输出出去,然后,\texttt{with}语句就可以正常地工作了:

\begin{pythoncode}
with create_query('Bob') as q:
    q.query()
\end{pythoncode}

很多时候,我们希望在某段代码执行前后自动执行特定代码,也可以用\texttt{@contextmanager}实现。例如:

\begin{pythoncode}
@contextmanager
def tag(name):
    print("<%s>" % name)
    yield
    print("</%s>" % name)

with tag("h1"):
    print("hello")
    print("world")
\end{pythoncode}

上述代码执行结果为:

\begin{pythoncode}
<h1>
hello
world
</h1>
\end{pythoncode}

代码的执行顺序是:

\begin{enumerate}
\def\labelenumi{\arabic{enumi}.}
\item
  \texttt{with}语句首先执行\texttt{yield}之前的语句,因此打印出\texttt{\textless{}h1\textgreater{}};
\item
  \texttt{yield}调用会执行\texttt{with}语句内部的所有语句,因此打印出\texttt{hello}和\texttt{world};
\item
  最后执行\texttt{yield}之后的语句,打印出\texttt{\textless{}/h1\textgreater{}}。
\end{enumerate}

因此,\texttt{@contextmanager}让我们通过编写 generator
来简化上下文管理。

\hypertarget{closing}{%
\subsubsection{@closing}\label{closing}}

如果一个对象没有实现上下文,我们就不能把它用于\texttt{with}语句。这个时候,可以用\texttt{closing()}来把该对象变为上下文对象。例如,用\texttt{with}语句使用\texttt{urlopen()}:

\begin{pythoncode}
from contextlib import closing
from urllib.request import urlopen

with closing(urlopen('https://www.python.org')) as page:
    for line in page:
        print(line)
\end{pythoncode}

\texttt{closing}也是一个经过 @contextmanager 装饰的 generator,这个
generator 编写起来其实非常简单:

\begin{pythoncode}
@contextmanager
def closing(thing):
    try:
        yield thing
    finally:
        thing.close()
\end{pythoncode}

它的作用就是把任意对象变为上下文对象,并支持\texttt{with}语句。

\texttt{@contextlib}还有一些其他 decorator,便于我们编写更简洁的代码。

