\hypertarget{tcp-ux7f16ux7a0b}{%
\subsection{TCP 编程}\label{tcp-ux7f16ux7a0b}}

Socket 是网络编程的一个抽象概念。通常我们用一个 Socket 表示
``打开了一个网络链接'',而打开一个 Socket 需要知道目标计算机的 IP
地址和端口号,再指定协议类型即可。

\hypertarget{ux5ba2ux6237ux7aef}{%
\subsubsection{客户端}\label{ux5ba2ux6237ux7aef}}

大多数连接都是可靠的 TCP 连接。创建 TCP
连接时,主动发起连接的叫客户端,被动响应连接的叫服务器。

举个例子,当我们在浏览器中访问新浪时,我们自己的计算机就是客户端,浏览器会主动向新浪的服务器发起连接。如果一切顺利,新浪的服务器接受了我们的连接,一个
TCP 连接就建立起来的,后面的通信就是发送网页内容了。

所以,我们要创建一个基于 TCP 连接的 Socket,可以这样做:

\begin{pythoncode}
import socket
s = socket.socket(socket.AF_INET, socket.SOCK_STREAM)

s.connect(('www.sina.com.cn', 80))
\end{pythoncode}

创建\texttt{Socket}时,\texttt{AF\_INET}指定使用 IPv4
协议,如果要用更先进的
IPv6,就指定为\texttt{AF\_INET6}。\texttt{SOCK\_STREAM}指定使用面向流的
TCP 协议,这样,一个\texttt{Socket}对象就创建成功,但是还没有建立连接。

客户端要主动发起 TCP 连接,必须知道服务器的 IP 地址和端口号。新浪网站的
IP 地址可以用域名\texttt{www.sina.com.cn}自动转换到 IP
地址,但是怎么知道新浪服务器的端口号呢?

答案是作为服务器,提供什么样的服务,端口号就必须固定下来。由于我们想要访问网页,因此新浪提供网页服务的服务器必须把端口号固定在\texttt{80}端口,因为\texttt{80}端口是
Web 服务的标准端口。其他服务都有对应的标准端口号,例如 SMTP
服务是\texttt{25}端口,FTP 服务是\texttt{21}端口,等等。端口号小于 1024
的是 Internet 标准服务的端口,端口号大于 1024 的,可以任意使用。

因此,我们连接新浪服务器的代码如下:

\begin{pythoncode}
s.connect(('www.sina.com.cn', 80))
\end{pythoncode}

注意参数是一个\texttt{tuple},包含地址和端口号。

建立 TCP 连接后,我们就可以向新浪服务器发送请求,要求返回首页的内容:

\begin{pythoncode}
s.send(b'GET / HTTP/1.1\r\nHost: www.sina.com.cn\r\nConnection: close\r\n\r\n')
\end{pythoncode}

TCP
连接创建的是双向通道,双方都可以同时给对方发数据。但是谁先发谁后发,怎么协调,要根据具体的协议来决定。例如,HTTP
协议规定客户端必须先发请求给服务器,服务器收到后才发数据给客户端。

发送的文本格式必须符合 HTTP
标准,如果格式没问题,接下来就可以接收新浪服务器返回的数据了:

\begin{pythoncode}
buffer = []
while True:
    
    d = s.recv(1024)
    if d:
        buffer.append(d)
    else:
        break
data = b''.join(buffer)
\end{pythoncode}

接收数据时,调用\texttt{recv(max)}方法,一次最多接收指定的字节数,因此,在一个
while
循环中反复接收,直到\texttt{recv()}返回空数据,表示接收完毕,退出循环。

当我们接收完数据后,调用\texttt{close()}方法关闭
Socket,这样,一次完整的网络通信就结束了:

\begin{pythoncode}
s.close()
\end{pythoncode}

接收到的数据包括 HTTP 头和网页本身,我们只需要把 HTTP
头和网页分离一下,把 HTTP 头打印出来,网页内容保存到文件:

\begin{pythoncode}
header, html = data.split(b'\r\n\r\n', 1)
print(header.decode('utf-8'))

with open('sina.html', 'wb') as f:
    f.write(html)
\end{pythoncode}

现在,只需要在浏览器中打开这个\texttt{sina.html}文件,就可以看到新浪的首页了。

\hypertarget{ux670dux52a1ux5668}{%
\subsubsection{服务器}\label{ux670dux52a1ux5668}}

和客户端编程相比,服务器编程就要复杂一些。

服务器进程首先要绑定一个端口并监听来自其他客户端的连接。如果某个客户端连接过来了,服务器就与该客户端建立
Socket 连接,随后的通信就靠这个 Socket 连接了。

所以,服务器会打开固定端口(比如 80)监听,每来一个客户端连接,就创建该
Socket
连接。由于服务器会有大量来自客户端的连接,所以,服务器要能够区分一个
Socket 连接是和哪个客户端绑定的。一个 Socket 依赖 4
项:服务器地址、服务器端口、客户端地址、客户端端口来唯一确定一个
Socket。

但是服务器还需要同时响应多个客户端的请求,所以,每个连接都需要一个新的进程或者新的线程来处理,否则,服务器一次就只能服务一个客户端了。

我们来编写一个简单的服务器程序,它接收客户端连接,把客户端发过来的字符串加上\texttt{Hello}再发回去。

首先,创建一个基于 IPv4 和 TCP 协议的 Socket:

\begin{pythoncode}
s = socket.socket(socket.AF_INET, socket.SOCK_STREAM)
\end{pythoncode}

然后,我们要绑定监听的地址和端口。服务器可能有多块网卡,可以绑定到某一块网卡的
IP
地址上,也可以用\texttt{0.0.0.0}绑定到所有的网络地址,还可以用\texttt{127.0.0.1}绑定到本机地址。\texttt{127.0.0.1}是一个特殊的
IP
地址,表示本机地址,如果绑定到这个地址,客户端必须同时在本机运行才能连接,也就是说,外部的计算机无法连接进来。

端口号需要预先指定。因为我们写的这个服务不是标准服务,所以用\texttt{9999}这个端口号。请注意,小于\texttt{1024}的端口号必须要有管理员权限才能绑定:

\begin{pythoncode}
s.bind(('127.0.0.1', 9999))
\end{pythoncode}

紧接着,调用\texttt{listen()}方法开始监听端口,传入的参数指定等待连接的最大数量:

\begin{pythoncode}
s.listen(5)
print('Waiting for connection...')
\end{pythoncode}

接下来,服务器程序通过一个永久循环来接受来自客户端的连接,\texttt{accept()}会等待并返回一个客户端的连接:

\begin{pythoncode}
while True:
    
    sock, addr = s.accept()
    
    t = threading.Thread(target=tcplink, args=(sock, addr))
    t.start()
\end{pythoncode}

每个连接都必须创建新线程(或进程)来处理,否则,单线程在处理连接的过程中,无法接受其他客户端的连接:

\begin{pythoncode}
def tcplink(sock, addr):
    print('Accept new connection from %s:%s...' % addr)
    sock.send(b'Welcome!')
    while True:
        data = sock.recv(1024)
        time.sleep(1)
        if not data or data.decode('utf-8') == 'exit':
            break
        sock.send(('Hello, %s!' % data.decode('utf-8')).encode('utf-8'))
    sock.close()
    print('Connection from %s:%s closed.' % addr)
\end{pythoncode}

连接建立后,服务器首先发一条欢迎消息,然后等待客户端数据,并加上\texttt{Hello}再发送给客户端。如果客户端发送了\texttt{exit}字符串,就直接关闭连接。

要测试这个服务器程序,我们还需要编写一个客户端程序:

\begin{pythoncode}
s = socket.socket(socket.AF_INET, socket.SOCK_STREAM)

s.connect(('127.0.0.1', 9999))

print(s.recv(1024).decode('utf-8'))
for data in [b'Michael', b'Tracy', b'Sarah']:
    
    s.send(data)
    print(s.recv(1024).decode('utf-8'))
s.send(b'exit')
s.close()
\end{pythoncode}

我们需要打开两个命令行窗口,一个运行服务器程序,另一个运行客户端程序,就可以看到效果了:

\begin{pythoncode}
┌────────────────────────────────────────────────────────┐
│Command Prompt                                    - □ x │
├────────────────────────────────────────────────────────┤
│$ python echo_server.py                                 │
│Waiting for connection...                               │
│Accept new connection from 127.0.0.1:64398...           │
│Connection from 127.0.0.1:64398 closed.                 │
│                                                        │
│       ┌────────────────────────────────────────────────┴───────┐
│       │Command Prompt                                    - □ x │
│       ├────────────────────────────────────────────────────────┤
│       │$ python echo_client.py                                 │
│       │Welcome!                                                │
│       │Hello, Michael!                                         │
└───────┤Hello, Tracy!                                           │
        │Hello, Sarah!                                           │
        │$                                                       │
        │                                                        │
        │                                                        │
        └────────────────────────────────────────────────────────┘
\end{pythoncode}

需要注意的是,客户端程序运行完毕就退出了,而服务器程序会永远运行下去,必须按
Ctrl+C 退出程序。

\hypertarget{ux5c0fux7ed3}{%
\subsubsection{小结}\label{ux5c0fux7ed3}}

用 TCP 协议进行 Socket 编程在 Python
中十分简单,对于客户端,要主动连接服务器的 IP
和指定端口,对于服务器,要首先监听指定端口,然后,对每一个新的连接,创建一个线程或进程来处理。通常,服务器程序会无限运行下去。

同一个端口,被一个 Socket 绑定了以后,就不能被别的 Socket 绑定了。

\hypertarget{ux53c2ux8003ux6e90ux7801}{%
\subsubsection{参考源码}\label{ux53c2ux8003ux6e90ux7801}}

\href{https://github.com/michaelliao/learn-python3/blob/master/samples/socket/do_tcp.py}{do\_tcp.py}

