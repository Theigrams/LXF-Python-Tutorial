\hypertarget{ux51fdux6570}{%
\subsection{函数}\label{ux51fdux6570}}

我们知道圆的面积计算公式为:

\[
S = πr^2
\]
当我们知道半径\texttt{r}的值时,就可以根据公式计算出面积。假设我们需要计算
3 个不同大小的圆的面积:

\begin{pythoncode}
r1 = 12.34
r2 = 9.08
r3 = 73.1
s1 = 3.14 * r1 * r1
s2 = 3.14 * r2 * r2
s3 = 3.14 * r3 * r3
\end{pythoncode}

当代码出现有规律的重复的时候,你就需要当心了,每次写\texttt{3.14\ *\ x\ *\ x}不仅很麻烦,而且,如果要把\texttt{3.14}改成\texttt{3.14159265359}的时候,得全部替换。

有了函数,我们就不再每次写\texttt{s\ =\ 3.14\ *\ x\ *\ x},而是写成更有意义的函数调用\texttt{s\ =\ area\_of\_circle(x)},而函数\texttt{area\_of\_circle}本身只需要写一次,就可以多次调用。

基本上所有的高级语言都支持函数,Python 也不例外。Python
不但能非常灵活地定义函数,而且本身内置了很多有用的函数,可以直接调用。

\hypertarget{ux62bdux8c61}{%
\subsubsection{抽象}\label{ux62bdux8c61}}

抽象是数学中非常常见的概念。举个例子:

计算数列的和,比如:\texttt{1\ +\ 2\ +\ 3\ +\ ...\ +\ 100},写起来十分不方便,于是数学家发明了求和符号∑,可以把\texttt{1\ +\ 2\ +\ 3\ +\ ...\ +\ 100}记作:

\[
\sum_{n=1}^{100}n
\] 这种抽象记法非常强大,因为我们看到 \(\sum\)
就可以理解成求和,而不是还原成低级的加法运算。

而且,这种抽象记法是可扩展的,比如:

\[
\sum_{n=1}^{100}(n^2+1)
\]

还原成加法运算就变成了: \[
(1*1+1)+(2*2+1)+(3*3+1)+\cdots+(100*100+1)
\]

可见,借助抽象,我们才能不关心底层的具体计算过程,而直接在更高的层次上思考问题。

写计算机程序也是一样,函数就是最基本的一种代码抽象的方式。

