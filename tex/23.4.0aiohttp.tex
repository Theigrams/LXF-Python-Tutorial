\hypertarget{aiohttp}{%
\subsection{aiohttp}\label{aiohttp}}

\texttt{asyncio}可以实现单线程并发 IO
操作。如果仅用在客户端,发挥的威力不大。如果把\texttt{asyncio}用在服务器端,例如
Web 服务器,由于 HTTP 连接就是 IO 操作,因此可以用单线程
+\texttt{coroutine}实现多用户的高并发支持。

\texttt{asyncio}实现了 TCP、UDP、SSL
等协议,\texttt{aiohttp}则是基于\texttt{asyncio}实现的 HTTP 框架。

我们先安装\texttt{aiohttp}:

\begin{pythoncode}
pip install aiohttp
\end{pythoncode}

然后编写一个 HTTP 服务器,分别处理以下 URL:

\begin{itemize}
\item
  \texttt{/} -
  首页返回\texttt{b\textquotesingle{}\textless{}h1\textgreater{}Index\textless{}/h1\textgreater{}\textquotesingle{}};
\item
  \texttt{/hello/\{name\}} - 根据 URL
  参数返回文本\texttt{hello,\ \%s!}。
\end{itemize}

代码如下:

\begin{pythoncode}
import asyncio

from aiohttp import web

async def index(request):
    await asyncio.sleep(0.5)
    return web.Response(body=b'<h1>Index</h1>')

async def hello(request):
    await asyncio.sleep(0.5)
    text = '<h1>hello, %s!</h1>' % request.match_info['name']
    return web.Response(body=text.encode('utf-8'))

async def init(loop):
    app = web.Application(loop=loop)
    app.router.add_route('GET', '/', index)
    app.router.add_route('GET', '/hello/{name}', hello)
    srv = await loop.create_server(app.make_handler(), '127.0.0.1', 8000)
    print('Server started at http://127.0.0.1:8000...')
    return srv

loop = asyncio.get_event_loop()
loop.run_until_complete(init(loop))
loop.run_forever()
\end{pythoncode}

注意\texttt{aiohttp}的初始化函数\texttt{init()}也是一个\texttt{coroutine},\texttt{loop.create\_server()}则利用\texttt{asyncio}创建
TCP 服务。

\hypertarget{ux53c2ux8003ux6e90ux7801}{%
\subsubsection{参考源码}\label{ux53c2ux8003ux6e90ux7801}}

\href{https://github.com/michaelliao/learn-python3/blob/master/samples/async/aio_web.py}{aio\_web.py}

