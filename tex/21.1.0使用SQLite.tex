\hypertarget{ux4f7fux7528-sqlite}{%
\subsection{使用 SQLite}\label{ux4f7fux7528-sqlite}}

SQLite 是一种嵌入式数据库,它的数据库就是一个文件。由于 SQLite 本身是 C
写的,而且体积很小,所以,经常被集成到各种应用程序中,甚至在 iOS 和
Android 的 App 中都可以集成。

Python 就内置了 SQLite3,所以,在 Python 中使用
SQLite,不需要安装任何东西,直接使用。

在使用 SQLite 前,我们先要搞清楚几个概念:

表是数据库中存放关系数据的集合,一个数据库里面通常都包含多个表,比如学生的表,班级的表,学校的表,等等。表和表之间通过外键关联。

要操作关系数据库,首先需要连接到数据库,一个数据库连接称为\texttt{Connection};

连接到数据库后,需要打开游标,称之为\texttt{Cursor},通过\texttt{Cursor}执行
SQL 语句,然后,获得执行结果。

Python 定义了一套操作数据库的 API 接口,任何数据库要连接到
Python,只需要提供符合 Python 标准的数据库驱动即可。

由于 SQLite 的驱动内置在 Python 标准库中,所以我们可以直接来操作 SQLite
数据库。

我们在 Python 交互式命令行实践一下:

\begin{pythoncode}
>>> import sqlite3

>>> conn = sqlite3.connect('test.db')

>>> cursor = conn.cursor()

>>> cursor.execute('create table user (id varchar(20) primary key, name varchar(20))')
<sqlite3.Cursor object at 0x10f8aa260>

>>> cursor.execute('insert into user (id, name) values (\'1\', \'Michael\')')
<sqlite3.Cursor object at 0x10f8aa260>

>>> cursor.rowcount
1

>>> cursor.close()

>>> conn.commit()

>>> conn.close()
\end{pythoncode}

我们再试试查询记录:

\begin{pythoncode}
>>> conn = sqlite3.connect('test.db')
>>> cursor = conn.cursor()

>>> cursor.execute('select * from user where id=?', ('1',))
<sqlite3.Cursor object at 0x10f8aa340>

>>> values = cursor.fetchall()
>>> values
[('1', 'Michael')]
>>> cursor.close()
>>> conn.close()
\end{pythoncode}

使用 Python 的 DB-API
时,只要搞清楚\texttt{Connection}和\texttt{Cursor}对象,打开后一定记得关闭,就可以放心地使用。

使用\texttt{Cursor}对象执行\texttt{insert},\texttt{update},\texttt{delete}语句时,执行结果由\texttt{rowcount}返回影响的行数,就可以拿到执行结果。

使用\texttt{Cursor}对象执行\texttt{select}语句时,通过\texttt{fetchall()}可以拿到结果集。结果集是一个\texttt{list},每个元素都是一个\texttt{tuple},对应一行记录。

如果 SQL
语句带有参数,那么需要把参数按照位置传递给\texttt{execute()}方法,有几个\texttt{?}占位符就必须对应几个参数,例如:

\begin{pythoncode}
cursor.execute('select * from user where name=? and pwd=?', ('abc', 'password'))
\end{pythoncode}

SQLite 支持常见的标准 SQL 语句以及几种常见的数据类型。具体文档请参阅
SQLite 官方网站。

\hypertarget{ux5c0fux7ed3}{%
\subsubsection{小结}\label{ux5c0fux7ed3}}

在 Python
中操作数据库时,要先导入数据库对应的驱动,然后,通过\texttt{Connection}对象和\texttt{Cursor}对象操作数据。

要确保打开的\texttt{Connection}对象和\texttt{Cursor}对象都正确地被关闭,否则,资源就会泄露。

如何才能确保出错的情况下也关闭掉\texttt{Connection}对象和\texttt{Cursor}对象呢?请回忆\texttt{try:...except:...finally:...}的用法。

\hypertarget{ux7ec3ux4e60}{%
\subsubsection{练习}\label{ux7ec3ux4e60}}

请编写函数,在 Sqlite 中根据分数段查找指定的名字:

\begin{pythoncode}
# -*- coding: utf-8 -*-

import os, sqlite3

db_file = os.path.join(os.path.dirname(__file__), 'test.db')
if os.path.isfile(db_file):
    os.remove(db_file)

# 初始数据:
conn = sqlite3.connect(db_file)
cursor = conn.cursor()
cursor.execute('create table user(id varchar(20) primary key, name varchar(20), score int)')
cursor.execute(r"insert into user values ('A-001', 'Adam', 95)")
cursor.execute(r"insert into user values ('A-002', 'Bart', 62)")
cursor.execute(r"insert into user values ('A-003', 'Lisa', 78)")
cursor.close()
conn.commit()
conn.close()

def get_score_in(low, high):
    ' 返回指定分数区间的名字,按分数从低到高排序 '
----
    pass
----
# 测试:
assert get_score_in(80, 95) == ['Adam'], get_score_in(80, 95)
assert get_score_in(60, 80) == ['Bart', 'Lisa'], get_score_in(60, 80)
assert get_score_in(60, 100) == ['Bart', 'Lisa', 'Adam'], get_score_in(60, 100)

print('Pass')
\end{pythoncode}

\hypertarget{ux53c2ux8003ux6e90ux7801}{%
\subsubsection{参考源码}\label{ux53c2ux8003ux6e90ux7801}}

\href{https://github.com/michaelliao/learn-python3/blob/master/samples/db/do_sqlite.py}{do\_sqlite.py}

