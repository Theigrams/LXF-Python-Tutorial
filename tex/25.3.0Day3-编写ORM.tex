\hypertarget{day-3---ux7f16ux5199-orm}{%
\subsection{Day 3 - 编写 ORM}\label{day-3---ux7f16ux5199-orm}}

在一个 Web App
中,所有数据,包括用户信息、发布的日志、评论等,都存储在数据库中。在
awesome-python3-webapp 中,我们选择 MySQL 作为数据库。

Web App
里面有很多地方都要访问数据库。访问数据库需要创建数据库连接、游标对象,然后执行
SQL
语句,最后处理异常,清理资源。这些访问数据库的代码如果分散到各个函数中,势必无法维护,也不利于代码复用。

所以,我们要首先把常用的 SELECT、INSERT、UPDATE 和 DELETE
操作用函数封装起来。

由于 Web 框架使用了基于 asyncio 的
aiohttp,这是基于协程的异步模型。在协程中,不能调用普通的同步 IO
操作,因为所有用户都是由一个线程服务的,协程的执行速度必须非常快,才能处理大量用户的请求。而耗时的
IO 操作不能在协程中以同步的方式调用,否则,等待一个 IO
操作时,系统无法响应任何其他用户。

这就是异步编程的一个原则:一旦决定使用异步,则系统每一层都必须是异步,``开弓没有回头箭''。

幸运的是\texttt{aiomysql}为 MySQL 数据库提供了异步 IO 的驱动。

\hypertarget{ux521bux5efaux8fdeux63a5ux6c60}{%
\subsubsection{创建连接池}\label{ux521bux5efaux8fdeux63a5ux6c60}}

我们需要创建一个全局的连接池,每个 HTTP
请求都可以从连接池中直接获取数据库连接。使用连接池的好处是不必频繁地打开和关闭数据库连接,而是能复用就尽量复用。

连接池由全局变量\texttt{\_\_pool}存储,缺省情况下将编码设置为\texttt{utf8},自动提交事务:

\begin{pythoncode}
@asyncio.coroutine
def create_pool(loop, **kw):
    logging.info('create database connection pool...')
    global __pool
    __pool = yield from aiomysql.create_pool(
        host=kw.get('host', 'localhost'),
        port=kw.get('port', 3306),
        user=kw['user'],
        password=kw['password'],
        db=kw['db'],
        charset=kw.get('charset', 'utf8'),
        autocommit=kw.get('autocommit', True),
        maxsize=kw.get('maxsize', 10),
        minsize=kw.get('minsize', 1),
        loop=loop
    )
\end{pythoncode}

\hypertarget{select}{%
\subsubsection{Select}\label{select}}

要执行 SELECT 语句,我们用\texttt{select}函数执行,需要传入 SQL 语句和
SQL 参数:

\begin{pythoncode}
@asyncio.coroutine
def select(sql, args, size=None):
    log(sql, args)
    global __pool
    with (yield from __pool) as conn:
        cur = yield from conn.cursor(aiomysql.DictCursor)
        yield from cur.execute(sql.replace('?', '%s'), args or ())
        if size:
            rs = yield from cur.fetchmany(size)
        else:
            rs = yield from cur.fetchall()
        yield from cur.close()
        logging.info('rows returned: %s' % len(rs))
        return rs
\end{pythoncode}

SQL 语句的占位符是\texttt{?},而 MySQL
的占位符是\texttt{\%s},\texttt{select()}函数在内部自动替换。注意要始终坚持使用带参数的
SQL,而不是自己拼接 SQL 字符串,这样可以防止 SQL 注入攻击。

注意到\texttt{yield\ from}将调用一个子协程(也就是在一个协程中调用另一个协程)并直接获得子协程的返回结果。

如果传入\texttt{size}参数,就通过\texttt{fetchmany()}获取最多指定数量的记录,否则,通过\texttt{fetchall()}获取所有记录。

\hypertarget{insert-update-delete}{%
\subsubsection{Insert, Update, Delete}\label{insert-update-delete}}

要执行 INSERT、UPDATE、DELETE
语句,可以定义一个通用的\texttt{execute()}函数,因为这 3 种 SQL
的执行都需要相同的参数,以及返回一个整数表示影响的行数:

\begin{pythoncode}
@asyncio.coroutine
def execute(sql, args):
    log(sql)
    with (yield from __pool) as conn:
        try:
            cur = yield from conn.cursor()
            yield from cur.execute(sql.replace('?', '%s'), args)
            affected = cur.rowcount
            yield from cur.close()
        except BaseException as e:
            raise
        return affected
\end{pythoncode}

\texttt{execute()}函数和\texttt{select()}函数所不同的是,cursor
对象不返回结果集,而是通过\texttt{rowcount}返回结果数。

\hypertarget{orm}{%
\subsubsection{ORM}\label{orm}}

有了基本的\texttt{select()}和\texttt{execute()}函数,我们就可以开始编写一个简单的
ORM 了。

设计 ORM 需要从上层调用者角度来设计。

我们先考虑如何定义一个\texttt{User}对象,然后把数据库表\texttt{users}和它关联起来。

\begin{pythoncode}
from orm import Model, StringField, IntegerField

class User(Model):
    __table__ = 'users'

    id = IntegerField(primary_key=True)
    name = StringField()
\end{pythoncode}

注意到定义在\texttt{User}类中的\texttt{\_\_table\_\_}、\texttt{id}和\texttt{name}是类的属性,不是实例的属性。所以,在类级别上定义的属性用来描述\texttt{User}对象和表的映射关系,而实例属性必须通过\texttt{\_\_init\_\_()}方法去初始化,所以两者互不干扰:

\begin{pythoncode}
# 创建实例:
user = User(id=123, name='Michael')
# 存入数据库:
user.insert()
# 查询所有User对象:
users = User.findAll()
\end{pythoncode}

\hypertarget{ux5b9aux4e49-model}{%
\subsubsection{定义 Model}\label{ux5b9aux4e49-model}}

首先要定义的是所有 ORM 映射的基类\texttt{Model}:

\begin{pythoncode}
class Model(dict, metaclass=ModelMetaclass):

    def __init__(self, **kw):
        super(Model, self).__init__(**kw)

    def __getattr__(self, key):
        try:
            return self[key]
        except KeyError:
            raise AttributeError(r"'Model' object has no attribute '%s'" % key)

    def __setattr__(self, key, value):
        self[key] = value

    def getValue(self, key):
        return getattr(self, key, None)

    def getValueOrDefault(self, key):
        value = getattr(self, key, None)
        if value is None:
            field = self.__mappings__[key]
            if field.default is not None:
                value = field.default() if callable(field.default) else field.default
                logging.debug('using default value for %s: %s' % (key, str(value)))
                setattr(self, key, value)
        return value
\end{pythoncode}

\texttt{Model}从\texttt{dict}继承,所以具备所有\texttt{dict}的功能,同时又实现了特殊方法\texttt{\_\_getattr\_\_()}和\texttt{\_\_setattr\_\_()},因此又可以像引用普通字段那样写:

\begin{pythoncode}
>>> user['id']
123
>>> user.id
123
\end{pythoncode}

以及\texttt{Field}和各种\texttt{Field}子类:

\begin{pythoncode}
class Field(object):

    def __init__(self, name, column_type, primary_key, default):
        self.name = name
        self.column_type = column_type
        self.primary_key = primary_key
        self.default = default

    def __str__(self):
        return '<%s, %s:%s>' % (self.__class__.__name__, self.column_type, self.name)
\end{pythoncode}

映射\texttt{varchar}的\texttt{StringField}:

\begin{pythoncode}
class StringField(Field):

    def __init__(self, name=None, primary_key=False, default=None, ddl='varchar(100)'):
        super().__init__(name, ddl, primary_key, default)
\end{pythoncode}

注意到\texttt{Model}只是一个基类,如何将具体的子类如\texttt{User}的映射信息读取出来呢?答案就是通过
metaclass:\texttt{ModelMetaclass}:

\begin{pythoncode}
class ModelMetaclass(type):

    def __new__(cls, name, bases, attrs):
        
        if name=='Model':
            return type.__new__(cls, name, bases, attrs)
        
        tableName = attrs.get('__table__', None) or name
        logging.info('found model: %s (table: %s)' % (name, tableName))
        
        mappings = dict()
        fields = []
        primaryKey = None
        for k, v in attrs.items():
            if isinstance(v, Field):
                logging.info('  found mapping: %s ==> %s' % (k, v))
                mappings[k] = v
                if v.primary_key:
                    
                    if primaryKey:
                        raise RuntimeError('Duplicate primary key for field: %s' % k)
                    primaryKey = k
                else:
                    fields.append(k)
        if not primaryKey:
            raise RuntimeError('Primary key not found.')
        for k in mappings.keys():
            attrs.pop(k)
        escaped_fields = list(map(lambda f: '`%s`' % f, fields))
        attrs['__mappings__'] = mappings 
        attrs['__table__'] = tableName
        attrs['__primary_key__'] = primaryKey 
        attrs['__fields__'] = fields 
        
        attrs['__select__'] = 'select `%s`, %s from `%s`' % (primaryKey, ', '.join(escaped_fields), tableName)
        attrs['__insert__'] = 'insert into `%s` (%s, `%s`) values (%s)' % (tableName, ', '.join(escaped_fields), primaryKey, create_args_string(len(escaped_fields) + 1))
        attrs['__update__'] = 'update `%s` set %s where `%s`=?' % (tableName, ', '.join(map(lambda f: '`%s`=?' % (mappings.get(f).name or f), fields)), primaryKey)
        attrs['__delete__'] = 'delete from `%s` where `%s`=?' % (tableName, primaryKey)
        return type.__new__(cls, name, bases, attrs)
\end{pythoncode}

这样,任何继承自 Model 的类(比如 User),会自动通过 ModelMetaclass
扫描映射关系,并存储到自身的类属性如\texttt{\_\_table\_\_}、\texttt{\_\_mappings\_\_}中。

然后,我们往 Model 类添加 class 方法,就可以让所有子类调用 class 方法:

\begin{pythoncode}
class Model(dict):

    ...

    @classmethod
    @asyncio.coroutine
    def find(cls, pk):
        ' find object by primary key. '
        rs = yield from select('%s where `%s`=?' % (cls.__select__, cls.__primary_key__), [pk], 1)
        if len(rs) == 0:
            return None
        return cls(**rs[0])
\end{pythoncode}

User 类现在就可以通过类方法实现主键查找:

\begin{pythoncode}
user = yield from User.find('123')
\end{pythoncode}

往 Model 类添加实例方法,就可以让所有子类调用实例方法:

\begin{pythoncode}
class Model(dict):

    ...

    @asyncio.coroutine
    def save(self):
        args = list(map(self.getValueOrDefault, self.__fields__))
        args.append(self.getValueOrDefault(self.__primary_key__))
        rows = yield from execute(self.__insert__, args)
        if rows != 1:
            logging.warn('failed to insert record: affected rows: %s' % rows)
\end{pythoncode}

这样,就可以把一个 User 实例存入数据库:

\begin{pythoncode}
user = User(id=123, name='Michael')
yield from user.save()
\end{pythoncode}

最后一步是完善 ORM,对于查找,我们可以实现以下方法:

\begin{itemize}
\item
  findAll() - 根据 WHERE 条件查找;
\item
  findNumber() - 根据 WHERE
  条件查找,但返回的是整数,适用于\texttt{select\ count(*)}类型的 SQL。
\end{itemize}

以及\texttt{update()}和\texttt{remove()}方法。

所有这些方法都必须用\texttt{@asyncio.coroutine}装饰,变成一个协程。

调用时需要特别注意:

\begin{pythoncode}
user.save()
\end{pythoncode}

没有任何效果,因为调用\texttt{save()}仅仅是创建了一个协程,并没有执行它。一定要用:

\begin{pythoncode}
yield from user.save()
\end{pythoncode}

才真正执行了 INSERT 操作。

最后看看我们实现的 ORM 模块一共多少行代码?累计不到 300 多行。用 Python
写一个 ORM 是不是很容易呢?

\hypertarget{ux53c2ux8003ux6e90ux7801}{%
\subsubsection{参考源码}\label{ux53c2ux8003ux6e90ux7801}}

\href{https://github.com/michaelliao/awesome-python3-webapp/tree/day-03}{day-03}

