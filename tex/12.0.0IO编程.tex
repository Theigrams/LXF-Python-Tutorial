\hypertarget{io-ux7f16ux7a0b}{%
\subsection{IO 编程}\label{io-ux7f16ux7a0b}}

IO 在计算机中指
Input/Output,也就是输入和输出。由于程序和运行时数据是在内存中驻留,由
CPU
这个超快的计算核心来执行,涉及到数据交换的地方,通常是磁盘、网络等,就需要
IO 接口。

比如你打开浏览器,访问新浪首页,浏览器这个程序就需要通过网络 IO
获取新浪的网页。浏览器首先会发送数据给新浪服务器,告诉它我想要首页的
HTML,这个动作是往外发数据,叫
Output,随后新浪服务器把网页发过来,这个动作是从外面接收数据,叫
Input。所以,通常,程序完成 IO 操作会有 Input 和 Output
两个数据流。当然也有只用一个的情况,比如,从磁盘读取文件到内存,就只有
Input 操作,反过来,把数据写到磁盘文件里,就只是一个 Output 操作。

IO
编程中,Stream(流)是一个很重要的概念,可以把流想象成一个水管,数据就是水管里的水,但是只能单向流动。Input
Stream 就是数据从外面(磁盘、网络)流进内存,Output Stream
就是数据从内存流到外面去。对于浏览网页来说,浏览器和新浪服务器之间至少需要建立两根水管,才可以既能发数据,又能收数据。

由于 CPU 和内存的速度远远高于外设的速度,所以,在 IO
编程中,就存在速度严重不匹配的问题。举个例子来说,比如要把 100M
的数据写入磁盘,CPU 输出 100M 的数据只需要 0.01 秒,可是磁盘要接收这
100M 数据可能需要 10 秒,怎么办呢?有两种办法:

第一种是 CPU 等着,也就是程序暂停执行后续代码,等 100M 的数据在 10
秒后写入磁盘,再接着往下执行,这种模式称为同步 IO;

另一种方法是 CPU
不等待,只是告诉磁盘,``您老慢慢写,不着急,我接着干别的事去了'',于是,后续代码可以立刻接着执行,这种模式称为异步
IO。

同步和异步的区别就在于是否等待 IO 执行的结果。好比你去麦当劳点餐,你说
``来个汉堡'',服务员告诉你,对不起,汉堡要现做,需要等 5
分钟,于是你站在收银台前面等了 5 分钟,拿到汉堡再去逛商场,这是同步 IO。

你说 ``来个汉堡'',服务员告诉你,汉堡需要等 5
分钟,你可以先去逛商场,等做好了,我们再通知你,这样你可以立刻去干别的事情(逛商场),这是异步
IO。

很明显,使用异步 IO 来编写程序性能会远远高于同步 IO,但是异步 IO
的缺点是编程模型复杂。想想看,你得知道什么时候通知你
``汉堡做好了'',而通知你的方法也各不相同。如果是服务员跑过来找到你,这是回调模式,如果服务员发短信通知你,你就得不停地检查手机,这是轮询模式。总之,异步
IO 的复杂度远远高于同步 IO。

操作 IO
的能力都是由操作系统提供的,每一种编程语言都会把操作系统提供的低级 C
接口封装起来方便使用,Python 也不例外。我们后面会详细讨论 Python 的 IO
编程接口。

注意,本章的 IO 编程都是同步模式,异步 IO
由于复杂度太高,后续涉及到服务器端程序开发时我们再讨论。

