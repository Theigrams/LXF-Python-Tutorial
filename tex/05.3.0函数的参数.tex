\hypertarget{ux51fdux6570ux7684ux53c2ux6570}{%
\subsection{函数的参数}\label{ux51fdux6570ux7684ux53c2ux6570}}

定义函数的时候,我们把参数的名字和位置确定下来,函数的接口定义就完成了。对于函数的调用者来说,只需要知道如何传递正确的参数,以及函数将返回什么样的值就够了,函数内部的复杂逻辑被封装起来,调用者无需了解。

Python
的函数定义非常简单,但灵活度却非常大。除了正常定义的必选参数外,还可以使用默认参数、可变参数和关键字参数,使得函数定义出来的接口,不但能处理复杂的参数,还可以简化调用者的代码。

\hypertarget{ux4f4dux7f6eux53c2ux6570}{%
\subsubsection{位置参数}\label{ux4f4dux7f6eux53c2ux6570}}

我们先写一个计算 x2 的函数:

\begin{pythoncode}
def power(x):
    return x * x
\end{pythoncode}

对于\texttt{power(x)}函数,参数\texttt{x}就是一个位置参数。

当我们调用\texttt{power}函数时,必须传入有且仅有的一个参数\texttt{x}:

\begin{pythoncode}
>>> power(5)
25
>>> power(15)
225
\end{pythoncode}

现在,如果我们要计算 x3
怎么办?可以再定义一个\texttt{power3}函数,但是如果要计算
x4、x5\ldots\ldots{} 怎么办?我们不可能定义无限多个函数。

你也许想到了,可以把\texttt{power(x)}修改为\texttt{power(x,\ n)},用来计算
xn,说干就干:

\begin{pythoncode}
def power(x, n):
    s = 1
    while n > 0:
        n = n - 1
        s = s * x
    return s
\end{pythoncode}

对于这个修改后的\texttt{power(x,\ n)}函数,可以计算任意 n 次方:

\begin{pythoncode}
>>> power(5, 2)
25
>>> power(5, 3)
125
\end{pythoncode}

修改后的\texttt{power(x,\ n)}函数有两个参数:\texttt{x}和\texttt{n},这两个参数都是位置参数,调用函数时,传入的两个值按照位置顺序依次赋给参数\texttt{x}和\texttt{n}。

\hypertarget{ux9ed8ux8ba4ux53c2ux6570}{%
\subsubsection{默认参数}\label{ux9ed8ux8ba4ux53c2ux6570}}

新的\texttt{power(x,\ n)}函数定义没有问题,但是,旧的调用代码失败了,原因是我们增加了一个参数,导致旧的代码因为缺少一个参数而无法正常调用:

\begin{pythoncode}
>>> power(5)
Traceback (most recent call last):
  File "<stdin>", line 1, in <module>
TypeError: power() missing 1 required positional argument: 'n'
\end{pythoncode}

Python
的错误信息很明确:调用函数\texttt{power()}缺少了一个位置参数\texttt{n}。

这个时候,默认参数就排上用场了。由于我们经常计算
x2,所以,完全可以把第二个参数 n 的默认值设定为 2:

\begin{pythoncode}
def power(x, n=2):
    s = 1
    while n > 0:
        n = n - 1
        s = s * x
    return s
\end{pythoncode}

这样,当我们调用\texttt{power(5)}时,相当于调用\texttt{power(5,\ 2)}:

\begin{pythoncode}
>>> power(5)
25
>>> power(5, 2)
25
\end{pythoncode}

而对于\texttt{n\ \textgreater{}\ 2}的其他情况,就必须明确地传入
n,比如\texttt{power(5,\ 3)}。

从上面的例子可以看出,默认参数可以简化函数的调用。设置默认参数时,有几点要注意:

一是必选参数在前,默认参数在后,否则 Python
的解释器会报错(思考一下为什么默认参数不能放在必选参数前面);

二是如何设置默认参数。

当函数有多个参数时,把变化大的参数放前面,变化小的参数放后面。变化小的参数就可以作为默认参数。

使用默认参数有什么好处?最大的好处是能降低调用函数的难度。

举个例子,我们写个一年级小学生注册的函数,需要传入\texttt{name}和\texttt{gender}两个参数:

\begin{pythoncode}
def enroll(name, gender):
    print('name:', name)
    print('gender:', gender)
\end{pythoncode}

这样,调用\texttt{enroll()}函数只需要传入两个参数:

\begin{pythoncode}
>>> enroll('Sarah', 'F')
name: Sarah
gender: F
\end{pythoncode}

如果要继续传入年龄、城市等信息怎么办?这样会使得调用函数的复杂度大大增加。

我们可以把年龄和城市设为默认参数:

\begin{pythoncode}
def enroll(name, gender, age=6, city='Beijing'):
    print('name:', name)
    print('gender:', gender)
    print('age:', age)
    print('city:', city)
\end{pythoncode}

这样,大多数学生注册时不需要提供年龄和城市,只提供必须的两个参数:

\begin{pythoncode}
>>> enroll('Sarah', 'F')
name: Sarah
gender: F
age: 6
city: Beijing
\end{pythoncode}

只有与默认参数不符的学生才需要提供额外的信息:

\begin{pythoncode}
enroll('Bob', 'M', 7)
enroll('Adam', 'M', city='Tianjin')
\end{pythoncode}

可见,默认参数降低了函数调用的难度,而一旦需要更复杂的调用时,又可以传递更多的参数来实现。无论是简单调用还是复杂调用,函数只需要定义一个。

有多个默认参数时,调用的时候,既可以按顺序提供默认参数,比如调用\texttt{enroll(\textquotesingle{}Bob\textquotesingle{},\ \textquotesingle{}M\textquotesingle{},\ 7)},意思是,除了\texttt{name},\texttt{gender}这两个参数外,最后
1
个参数应用在参数\texttt{age}上,\texttt{city}参数由于没有提供,仍然使用默认值。

也可以不按顺序提供部分默认参数。当不按顺序提供部分默认参数时,需要把参数名写上。比如调用\texttt{enroll(\textquotesingle{}Adam\textquotesingle{},\ \textquotesingle{}M\textquotesingle{},\ city=\textquotesingle{}Tianjin\textquotesingle{})},意思是,\texttt{city}参数用传进去的值,其他默认参数继续使用默认值。

默认参数很有用,但使用不当,也会掉坑里。默认参数有个最大的坑,演示如下:

先定义一个函数,传入一个 list,添加一个\texttt{END}再返回:

\begin{pythoncode}
def add_end(L=[]):
    L.append('END')
    return L
\end{pythoncode}

当你正常调用时,结果似乎不错:

\begin{pythoncode}
>>> add_end([1, 2, 3])
[1, 2, 3, 'END']
>>> add_end(['x', 'y', 'z'])
['x', 'y', 'z', 'END']
\end{pythoncode}

当你使用默认参数调用时,一开始结果也是对的:

\begin{pythoncode}
>>> add_end()
['END']
\end{pythoncode}

但是,再次调用\texttt{add\_end()}时,结果就不对了:

\begin{pythoncode}
>>> add_end()
['END', 'END']
>>> add_end()
['END', 'END', 'END']
\end{pythoncode}

很多初学者很疑惑,默认参数是\texttt{{[}{]}},但是函数似乎每次都
``记住了''
上次添加了\texttt{\textquotesingle{}END\textquotesingle{}}后的 list。

原因解释如下:

Python
函数在定义的时候,默认参数\texttt{L}的值就被计算出来了,即\texttt{{[}{]}},因为默认参数\texttt{L}也是一个变量,它指向对象\texttt{{[}{]}},每次调用该函数,如果改变了\texttt{L}的内容,则下次调用时,默认参数的内容就变了,不再是函数定义时的\texttt{{[}{]}}了。

定义默认参数要牢记一点:默认参数必须指向不变对象!

要修改上面的例子,我们可以用\texttt{None}这个不变对象来实现:

\begin{pythoncode}
def add_end(L=None):
    if L is None:
        L = []
    L.append('END')
    return L
\end{pythoncode}

现在,无论调用多少次,都不会有问题:

\begin{pythoncode}
>>> add_end()
['END']
>>> add_end()
['END']
\end{pythoncode}

为什么要设计\texttt{str}、\texttt{None}这样的不变对象呢?因为不变对象一旦创建,对象内部的数据就不能修改,这样就减少了由于修改数据导致的错误。此外,由于对象不变,多任务环境下同时读取对象不需要加锁,同时读一点问题都没有。我们在编写程序时,如果可以设计一个不变对象,那就尽量设计成不变对象。

\hypertarget{ux53efux53d8ux53c2ux6570}{%
\subsubsection{可变参数}\label{ux53efux53d8ux53c2ux6570}}

在 Python
函数中,还可以定义可变参数。顾名思义,可变参数就是传入的参数个数是可变的,可以是
1 个、2 个到任意个,还可以是 0 个。

我们以数学题为例子,给定一组数字 a,b,c\ldots\ldots,请计算 a2 + b2 +
c2 + \ldots\ldots。

要定义出这个函数,我们必须确定输入的参数。由于参数个数不确定,我们首先想到可以把
a,b,c\ldots\ldots{} 作为一个 list 或 tuple
传进来,这样,函数可以定义如下:

\begin{pythoncode}
def calc(numbers):
    sum = 0
    for n in numbers:
        sum = sum + n * n
    return sum
\end{pythoncode}

但是调用的时候,需要先组装出一个 list 或 tuple:

\begin{pythoncode}
>>> calc([1, 2, 3])
14
>>> calc((1, 3, 5, 7))
84
\end{pythoncode}

如果利用可变参数,调用函数的方式可以简化成这样:

\begin{pythoncode}
>>> calc(1, 2, 3)
14
>>> calc(1, 3, 5, 7)
84
\end{pythoncode}

所以,我们把函数的参数改为可变参数:

\begin{pythoncode}
def calc(*numbers):
    sum = 0
    for n in numbers:
        sum = sum + n * n
    return sum
\end{pythoncode}

定义可变参数和定义一个 list 或 tuple
参数相比,仅仅在参数前面加了一个\texttt{*}号。在函数内部,参数\texttt{numbers}接收到的是一个
tuple,因此,函数代码完全不变。但是,调用该函数时,可以传入任意个参数,包括
0 个参数:

\begin{pythoncode}
>>> calc(1, 2)
5
>>> calc()
0
\end{pythoncode}

如果已经有一个 list 或者 tuple,要调用一个可变参数怎么办?可以这样做:

\begin{pythoncode}
>>> nums = [1, 2, 3]
>>> calc(nums[0], nums[1], nums[2])
14
\end{pythoncode}

这种写法当然是可行的,问题是太繁琐,所以 Python 允许你在 list 或 tuple
前面加一个\texttt{*}号,把 list 或 tuple 的元素变成可变参数传进去:

\begin{pythoncode}
>>> nums = [1, 2, 3]
>>> calc(*nums)
14
\end{pythoncode}

\texttt{*nums}表示把\texttt{nums}这个 list
的所有元素作为可变参数传进去。这种写法相当有用,而且很常见。

\hypertarget{ux5173ux952eux5b57ux53c2ux6570}{%
\subsubsection{关键字参数}\label{ux5173ux952eux5b57ux53c2ux6570}}

可变参数允许你传入 0
个或任意个参数,这些可变参数在函数调用时自动组装为一个
tuple。而关键字参数允许你传入 0
个或任意个含参数名的参数,这些关键字参数在函数内部自动组装为一个
dict。请看示例:

\begin{pythoncode}
def person(name, age, **kw):
    print('name:', name, 'age:', age, 'other:', kw)
\end{pythoncode}

函数\texttt{person}除了必选参数\texttt{name}和\texttt{age}外,还接受关键字参数\texttt{kw}。在调用该函数时,可以只传入必选参数:

\begin{pythoncode}
>>> person('Michael', 30)
name: Michael age: 30 other: {}
\end{pythoncode}

也可以传入任意个数的关键字参数:

\begin{pythoncode}
>>> person('Bob', 35, city='Beijing')
name: Bob age: 35 other: {'city': 'Beijing'}
>>> person('Adam', 45, gender='M', job='Engineer')
name: Adam age: 45 other: {'gender': 'M', 'job': 'Engineer'}
\end{pythoncode}

关键字参数有什么用?它可以扩展函数的功能。比如,在\texttt{person}函数里,我们保证能接收到\texttt{name}和\texttt{age}这两个参数,但是,如果调用者愿意提供更多的参数,我们也能收到。试想你正在做一个用户注册的功能,除了用户名和年龄是必填项外,其他都是可选项,利用关键字参数来定义这个函数就能满足注册的需求。

和可变参数类似,也可以先组装出一个 dict,然后,把该 dict
转换为关键字参数传进去:

\begin{pythoncode}
>>> extra = {'city': 'Beijing', 'job': 'Engineer'}
>>> person('Jack', 24, city=extra['city'], job=extra['job'])
name: Jack age: 24 other: {'city': 'Beijing', 'job': 'Engineer'}
\end{pythoncode}

当然,上面复杂的调用可以用简化的写法:

\begin{pythoncode}
>>> extra = {'city': 'Beijing', 'job': 'Engineer'}
>>> person('Jack', 24, **extra)
name: Jack age: 24 other: {'city': 'Beijing', 'job': 'Engineer'}
\end{pythoncode}

\texttt{**extra}表示把\texttt{extra}这个 dict 的所有 key-value
用关键字参数传入到函数的\texttt{**kw}参数,\texttt{kw}将获得一个
dict,注意\texttt{kw}获得的 dict
是\texttt{extra}的一份拷贝,对\texttt{kw}的改动不会影响到函数外的\texttt{extra}。

\hypertarget{ux547dux540dux5173ux952eux5b57ux53c2ux6570}{%
\subsubsection{命名关键字参数}\label{ux547dux540dux5173ux952eux5b57ux53c2ux6570}}

对于关键字参数,函数的调用者可以传入任意不受限制的关键字参数。至于到底传入了哪些,就需要在函数内部通过\texttt{kw}检查。

仍以\texttt{person()}函数为例,我们希望检查是否有\texttt{city}和\texttt{job}参数:

\begin{pythoncode}
def person(name, age, **kw):
    if 'city' in kw:
        
        pass
    if 'job' in kw:
        
        pass
    print('name:', name, 'age:', age, 'other:', kw)
\end{pythoncode}

但是调用者仍可以传入不受限制的关键字参数:

\begin{pythoncode}
>>> person('Jack', 24, city='Beijing', addr='Chaoyang', zipcode=123456)
\end{pythoncode}

如果要限制关键字参数的名字,就可以用命名关键字参数,例如,只接收\texttt{city}和\texttt{job}作为关键字参数。这种方式定义的函数如下:

\begin{pythoncode}
def person(name, age, *, city, job):
    print(name, age, city, job)
\end{pythoncode}

和关键字参数\texttt{**kw}不同,命名关键字参数需要一个特殊分隔符\texttt{*},\texttt{*}后面的参数被视为命名关键字参数。

调用方式如下:

\begin{pythoncode}
>>> person('Jack', 24, city='Beijing', job='Engineer')
Jack 24 Beijing Engineer
\end{pythoncode}

如果函数定义中已经有了一个可变参数,后面跟着的命名关键字参数就不再需要一个特殊分隔符\texttt{*}了:

\begin{pythoncode}
def person(name, age, *args, city, job):
    print(name, age, args, city, job)
\end{pythoncode}

命名关键字参数必须传入参数名,这和位置参数不同。如果没有传入参数名,调用将报错:

\begin{pythoncode}
>>> person('Jack', 24, 'Beijing', 'Engineer')
Traceback (most recent call last):
  File "<stdin>", line 1, in <module>
TypeError: person() takes 2 positional arguments but 4 were given
\end{pythoncode}

由于调用时缺少参数名\texttt{city}和\texttt{job},Python 解释器把这 4
个参数均视为位置参数,但\texttt{person()}函数仅接受 2 个位置参数。

命名关键字参数可以有缺省值,从而简化调用:

\begin{pythoncode}
def person(name, age, *, city='Beijing', job):
    print(name, age, city, job)
\end{pythoncode}

由于命名关键字参数\texttt{city}具有默认值,调用时,可不传入\texttt{city}参数:

\begin{pythoncode}
>>> person('Jack', 24, job='Engineer')
Jack 24 Beijing Engineer
\end{pythoncode}

使用命名关键字参数时,要特别注意,如果没有可变参数,就必须加一个\texttt{*}作为特殊分隔符。如果缺少\texttt{*},Python
解释器将无法识别位置参数和命名关键字参数:

\begin{pythoncode}
def person(name, age, city, job):
    
    pass
\end{pythoncode}

\hypertarget{ux53c2ux6570ux7ec4ux5408}{%
\subsubsection{参数组合}\label{ux53c2ux6570ux7ec4ux5408}}

在 Python
中定义函数,可以用必选参数、默认参数、可变参数、关键字参数和命名关键字参数,这
5
种参数都可以组合使用。但是请注意,参数定义的顺序必须是:必选参数、默认参数、可变参数、命名关键字参数和关键字参数。

比如定义一个函数,包含上述若干种参数:

\begin{pythoncode}
def f1(a, b, c=0, *args, **kw):
    print('a =', a, 'b =', b, 'c =', c, 'args =', args, 'kw =', kw)

def f2(a, b, c=0, *, d, **kw):
    print('a =', a, 'b =', b, 'c =', c, 'd =', d, 'kw =', kw)
\end{pythoncode}

在函数调用的时候,Python
解释器自动按照参数位置和参数名把对应的参数传进去。

\begin{pythoncode}
>>> f1(1, 2)
a = 1 b = 2 c = 0 args = () kw = {}
>>> f1(1, 2, c=3)
a = 1 b = 2 c = 3 args = () kw = {}
>>> f1(1, 2, 3, 'a', 'b')
a = 1 b = 2 c = 3 args = ('a', 'b') kw = {}
>>> f1(1, 2, 3, 'a', 'b', x=99)
a = 1 b = 2 c = 3 args = ('a', 'b') kw = {'x': 99}
>>> f2(1, 2, d=99, ext=None)
a = 1 b = 2 c = 0 d = 99 kw = {'ext': None}
\end{pythoncode}

最神奇的是通过一个 tuple 和 dict,你也可以调用上述函数:

\begin{pythoncode}
>>> args = (1, 2, 3, 4)
>>> kw = {'d': 99, 'x': '#'}
>>> f1(*args, **kw)
a = 1 b = 2 c = 3 args = (4,) kw = {'d': 99, 'x': '#'}
>>> args = (1, 2, 3)
>>> kw = {'d': 88, 'x': '#'}
>>> f2(*args, **kw)
a = 1 b = 2 c = 3 d = 88 kw = {'x': '#'}
\end{pythoncode}

所以,对于任意函数,都可以通过类似\texttt{func(*args,\ **kw)}的形式调用它,无论它的参数是如何定义的。

虽然可以组合多达 5
种参数,但不要同时使用太多的组合,否则函数接口的可理解性很差。

\hypertarget{ux7ec3ux4e60}{%
\subsubsection{练习}\label{ux7ec3ux4e60}}

以下函数允许计算两个数的乘积,请稍加改造,变成可接收一个或多个数并计算乘积:

\begin{pythoncode}
# -*- coding: utf-8 -*-
\end{pythoncode}

\begin{pythoncode}
# 测试
print('product(5) =', product(5))
print('product(5, 6) =', product(5, 6))
print('product(5, 6, 7) =', product(5, 6, 7))
print('product(5, 6, 7, 9) =', product(5, 6, 7, 9))
if product(5) != 5:
    print('测试失败!')
elif product(5, 6) != 30:
    print('测试失败!')
elif product(5, 6, 7) != 210:
    print('测试失败!')
elif product(5, 6, 7, 9) != 1890:
    print('测试失败!')
else:
    try:
        product()
        print('测试失败!')
    except TypeError:
        print('测试成功!')
\end{pythoncode}

\hypertarget{ux5c0fux7ed3}{%
\subsubsection{小结}\label{ux5c0fux7ed3}}

Python
的函数具有非常灵活的参数形态,既可以实现简单的调用,又可以传入非常复杂的参数。

默认参数一定要用不可变对象,如果是可变对象,程序运行时会有逻辑错误!

要注意定义可变参数和关键字参数的语法:

\texttt{*args}是可变参数,args 接收的是一个 tuple;

\texttt{**kw}是关键字参数,kw 接收的是一个 dict。

以及调用函数时如何传入可变参数和关键字参数的语法:

可变参数既可以直接传入:\texttt{func(1,\ 2,\ 3)},又可以先组装 list 或
tuple,再通过\texttt{*args}传入:\texttt{func(*(1,\ 2,\ 3))};

关键字参数既可以直接传入:\texttt{func(a=1,\ b=2)},又可以先组装
dict,再通过\texttt{**kw}传入:\texttt{func(**\{\textquotesingle{}a\textquotesingle{}:\ 1,\ \textquotesingle{}b\textquotesingle{}:\ 2\})}。

使用\texttt{*args}和\texttt{**kw}是 Python
的习惯写法,当然也可以用其他参数名,但最好使用习惯用法。

命名的关键字参数是为了限制调用者可以传入的参数名,同时可以提供默认值。

定义命名的关键字参数在没有可变参数的情况下不要忘了写分隔符\texttt{*},否则定义的将是位置参数。

\hypertarget{ux53c2ux8003ux6e90ux7801}{%
\subsubsection{参考源码}\label{ux53c2ux8003ux6e90ux7801}}

\href{https://github.com/michaelliao/learn-python3/blob/master/samples/function/var_args.py}{var\_args.py}

\href{https://github.com/michaelliao/learn-python3/blob/master/samples/function/kw_args.py}{kw\_args.py}

