\hypertarget{chardet}{%
\subsection{chardet}\label{chardet}}

字符串编码一直是令人非常头疼的问题,尤其是我们在处理一些不规范的第三方网页的时候。虽然
Python 提供了 Unicode
表示的\texttt{str}和\texttt{bytes}两种数据类型,并且可以通过\texttt{encode()}和\texttt{decode()}方法转换,但是,在不知道编码的情况下,对\texttt{bytes}做\texttt{decode()}不好做。

对于未知编码的\texttt{bytes},要把它转换成\texttt{str},需要先 ``猜测''
编码。猜测的方式是先收集各种编码的特征字符,根据特征字符判断,就能有很大概率``猜对''。

当然,我们肯定不能从头自己写这个检测编码的功能,这样做费时费力。chardet
这个第三方库正好就派上了用场。用它来检测编码,简单易用。

\hypertarget{ux5b89ux88c5-chardet}{%
\subsubsection{安装 chardet}\label{ux5b89ux88c5-chardet}}

如果安装了 Anaconda,chardet 就已经可用了。否则,需要在命令行下通过 pip
安装:

\begin{pythoncode}
$ pip install chardet
\end{pythoncode}

如果遇到 Permission denied 安装失败,请加上 sudo 重试。

\hypertarget{ux4f7fux7528-chardet}{%
\subsubsection{使用 chardet}\label{ux4f7fux7528-chardet}}

当我们拿到一个\texttt{bytes}时,就可以对其检测编码。用 chardet
检测编码,只需要一行代码:

\begin{pythoncode}
>>> chardet.detect(b'Hello, world!')
{'encoding': 'ascii', 'confidence': 1.0, 'language': ''}
\end{pythoncode}

检测出的编码是\texttt{ascii},注意到还有个\texttt{confidence}字段,表示检测的概率是
1.0(即 100\%)。

我们来试试检测 GBK 编码的中文:

\begin{pythoncode}
>>> data = '离离原上草,一岁一枯荣'.encode('gbk')
>>> chardet.detect(data)
{'encoding': 'GB2312', 'confidence': 0.7407407407407407, 'language': 'Chinese'}
\end{pythoncode}

检测的编码是\texttt{GB2312},注意到 GBK 是 GB2312
的超集,两者是同一种编码,检测正确的概率是
74\%,\texttt{language}字段指出的语言是\texttt{\textquotesingle{}Chinese\textquotesingle{}}。

对 UTF-8 编码进行检测:

\begin{pythoncode}
>>> data = '离离原上草,一岁一枯荣'.encode('utf-8')
>>> chardet.detect(data)
{'encoding': 'utf-8', 'confidence': 0.99, 'language': ''}
\end{pythoncode}

我们再试试对日文进行检测:

\begin{pythoncode}
>>> data = '最新の主要ニュース'.encode('euc-jp')
>>> chardet.detect(data)
{'encoding': 'EUC-JP', 'confidence': 0.99, 'language': 'Japanese'}
\end{pythoncode}

可见,用 chardet
检测编码,使用简单。获取到编码后,再转换为\texttt{str},就可以方便后续处理。

chardet 支持检测的编码列表请参考官方文档
\href{https://chardet.readthedocs.io/en/latest/supported-encodings.html}{Supported
encodings}。

\hypertarget{ux5c0fux7ed3}{%
\subsubsection{小结}\label{ux5c0fux7ed3}}

使用 chardet 检测编码非常容易,chardet
支持检测中文、日文、韩文等多种语言。

