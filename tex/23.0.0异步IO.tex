\hypertarget{ux5f02ux6b65-io}{%
\subsection{异步 IO}\label{ux5f02ux6b65-io}}

在 IO 编程一节中,我们已经知道,CPU 的速度远远快于磁盘、网络等
IO。在一个线程中,CPU 执行代码的速度极快,然而,一旦遇到 IO
操作,如读写文件、发送网络数据时,就需要等待 IO
操作完成,才能继续进行下一步操作。这种情况称为同步 IO。

在 IO 操作的过程中,当前线程被挂起,而其他需要 CPU
执行的代码就无法被当前线程执行了。

因为一个 IO
操作就阻塞了当前线程,导致其他代码无法执行,所以我们必须使用多线程或者多进程来并发执行代码,为多个用户服务。每个用户都会分配一个线程,如果遇到
IO 导致线程被挂起,其他用户的线程不受影响。

多线程和多进程的模型虽然解决了并发问题,但是系统不能无上限地增加线程。由于系统切换线程的开销也很大,所以,一旦线程数量过多,CPU
的时间就花在线程切换上了,真正运行代码的时间就少了,结果导致性能严重下降。

由于我们要解决的问题是 CPU 高速执行能力和 IO
设备的龟速严重不匹配,多线程和多进程只是解决这一问题的一种方法。

另一种解决 IO 问题的方法是异步 IO。当代码需要执行一个耗时的 IO
操作时,它只发出 IO 指令,并不等待 IO
结果,然后就去执行其他代码了。一段时间后,当 IO 返回结果时,再通知 CPU
进行处理。

可以想象如果按普通顺序写出的代码实际上是没法完成异步 IO 的:

\begin{pythoncode}
do_some_code()
f = open('/path/to/file', 'r')
r = f.read() # <== 线程停在此处等待IO操作结果
# IO操作完成后线程才能继续执行:
do_some_code(r)
\end{pythoncode}

所以,同步 IO 模型的代码是无法实现异步 IO 模型的。

异步 IO 模型需要一个消息循环,在消息循环中,主线程不断地重复 ``读取消息
- 处理消息'' 这一过程:

\begin{pythoncode}
loop = get_event_loop()
while True:
    event = loop.get_event()
    process_event(event)
\end{pythoncode}

消息模型其实早在应用在桌面应用程序中了。一个 GUI
程序的主线程就负责不停地读取消息并处理消息。所有的键盘、鼠标等消息都被发送到
GUI 程序的消息队列中,然后由 GUI 程序的主线程处理。

由于 GUI
线程处理键盘、鼠标等消息的速度非常快,所以用户感觉不到延迟。某些时候,GUI
线程在一个消息处理的过程中遇到问题导致一次消息处理时间过长,此时,用户会感觉到整个
GUI
程序停止响应了,敲键盘、点鼠标都没有反应。这种情况说明在消息模型中,处理一个消息必须非常迅速,否则,主线程将无法及时处理消息队列中的其他消息,导致程序看上去停止响应。

消息模型是如何解决同步 IO 必须等待 IO 操作这一问题的呢?当遇到 IO
操作时,代码只负责发出 IO 请求,不等待 IO
结果,然后直接结束本轮消息处理,进入下一轮消息处理过程。当 IO
操作完成后,将收到一条 ``IO 完成'' 的消息,处理该消息时就可以直接获取 IO
操作结果。

在 ``发出 IO 请求'' 到收到 ``IO 完成'' 的这段时间里,同步 IO
模型下,主线程只能挂起,但异步 IO
模型下,主线程并没有休息,而是在消息循环中继续处理其他消息。这样,在异步
IO 模型下,一个线程就可以同时处理多个 IO
请求,并且没有切换线程的操作。对于大多数 IO 密集型的应用程序,使用异步
IO 将大大提升系统的多任务处理能力。

