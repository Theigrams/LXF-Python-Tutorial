\hypertarget{htmlparser}{%
\subsection{HTMLParser}\label{htmlparser}}

如果我们要编写一个搜索引擎,第一步是用爬虫把目标网站的页面抓下来,第二步就是解析该
HTML 页面,看看里面的内容到底是新闻、图片还是视频。

假设第一步已经完成了,第二步应该如何解析 HTML 呢?

HTML 本质上是 XML 的子集,但是 HTML 的语法没有 XML
那么严格,所以不能用标准的 DOM 或 SAX 来解析 HTML。

好在 Python 提供了 HTMLParser 来非常方便地解析 HTML,只需简单几行代码:

\begin{pythoncode}
from html.parser import HTMLParser
from html.entities import name2codepoint

class MyHTMLParser(HTMLParser):

    def handle_starttag(self, tag, attrs):
        print('<%s>' % tag)

    def handle_endtag(self, tag):
        print('</%s>' % tag)

    def handle_startendtag(self, tag, attrs):
        print('<%s/>' % tag)

    def handle_data(self, data):
        print(data)

    def handle_comment(self, data):
        print('<!--', data, '-->')

    def handle_entityref(self, name):
        print('&%s;' % name)

    def handle_charref(self, name):
        print('&#%s;' % name)

parser = MyHTMLParser()
parser.feed('''<html>
<head></head>
<body>
<!-- test html parser -->
    <p>Some <a href=\"#\">html</a> HTML tutorial...<br>END</p>
</body></html>''')
\end{pythoncode}

\texttt{feed()}方法可以多次调用,也就是不一定一次把整个 HTML
字符串都塞进去,可以一部分一部分塞进去。

特殊字符有两种,一种是英文表示的\texttt{\&nbsp;},一种是数字表示的\texttt{\&\#1234;},这两种字符都可以通过
Parser 解析出来。

\hypertarget{ux5c0fux7ed3}{%
\subsubsection{小结}\label{ux5c0fux7ed3}}

利用 HTMLParser,可以把网页中的文本、图像等解析出来。

\hypertarget{ux7ec3ux4e60}{%
\subsubsection{练习}\label{ux7ec3ux4e60}}

找一个网页,例如
\url{https://www.python.org/events/python-events/},用浏览器查看源码并复制,然后尝试解析一下
HTML,输出 Python 官网发布的会议时间、名称和地点。

\hypertarget{ux53c2ux8003ux6e90ux7801}{%
\subsubsection{参考源码}\label{ux53c2ux8003ux6e90ux7801}}

\href{https://github.com/michaelliao/learn-python3/blob/master/samples/commonlib/use_htmlparser.py}{use\_htmlparser.py}

