\hypertarget{ux9ad8ux9636ux51fdux6570}{%
\subsection{高阶函数}\label{ux9ad8ux9636ux51fdux6570}}

高阶函数英文叫 Higher-order
function。什么是高阶函数?我们以实际代码为例子,一步一步深入概念。

\hypertarget{ux53d8ux91cfux53efux4ee5ux6307ux5411ux51fdux6570}{%
\subsubsection{变量可以指向函数}\label{ux53d8ux91cfux53efux4ee5ux6307ux5411ux51fdux6570}}

以 Python 内置的求绝对值的函数\texttt{abs()}为例,调用该函数用以下代码:

\begin{pythoncode}
>>> abs(-10)
10
\end{pythoncode}

但是,如果只写\texttt{abs}呢?

\begin{pythoncode}
>>> abs
<built-in function abs>
\end{pythoncode}

可见,\texttt{abs(-10)}是函数调用,而\texttt{abs}是函数本身。

要获得函数调用结果,我们可以把结果赋值给变量:

\begin{pythoncode}
>>> x = abs(-10)
>>> x
10
\end{pythoncode}

但是,如果把函数本身赋值给变量呢?

\begin{pythoncode}
>>> f = abs
>>> f
<built-in function abs>
\end{pythoncode}

结论:函数本身也可以赋值给变量,即:变量可以指向函数。

如果一个变量指向了一个函数,那么,可否通过该变量来调用这个函数?用代码验证一下:

\begin{pythoncode}
>>> f = abs
>>> f(-10)
10
\end{pythoncode}

成功!说明变量\texttt{f}现在已经指向了\texttt{abs}函数本身。直接调用\texttt{abs()}函数和调用变量\texttt{f()}完全相同。

\hypertarget{ux51fdux6570ux540dux4e5fux662fux53d8ux91cf}{%
\subsubsection{函数名也是变量}\label{ux51fdux6570ux540dux4e5fux662fux53d8ux91cf}}

那么函数名是什么呢?函数名其实就是指向函数的变量!对于\texttt{abs()}这个函数,完全可以把函数名\texttt{abs}看成变量,它指向一个可以计算绝对值的函数!

如果把\texttt{abs}指向其他对象,会有什么情况发生?

\begin{pythoncode}
>>> abs = 10
>>> abs(-10)
Traceback (most recent call last):
  File "<stdin>", line 1, in <module>
TypeError: 'int' object is not callable
\end{pythoncode}

把\texttt{abs}指向\texttt{10}后,就无法通过\texttt{abs(-10)}调用该函数了!因为\texttt{abs}这个变量已经不指向求绝对值函数而是指向一个整数\texttt{10}!

当然实际代码绝对不能这么写,这里是为了说明函数名也是变量。要恢复\texttt{abs}函数,请重启
Python 交互环境。

注:由于\texttt{abs}函数实际上是定义在\texttt{import\ builtins}模块中的,所以要让修改\texttt{abs}变量的指向在其它模块也生效,要用\texttt{import\ builtins;\ builtins.abs\ =\ 10}。

\hypertarget{ux4f20ux5165ux51fdux6570}{%
\subsubsection{传入函数}\label{ux4f20ux5165ux51fdux6570}}

既然变量可以指向函数,函数的参数能接收变量,那么一个函数就可以接收另一个函数作为参数,这种函数就称之为高阶函数。

一个最简单的高阶函数:

\begin{pythoncode}
def add(x, y, f):
    return f(x) + f(y)
\end{pythoncode}

当我们调用\texttt{add(-5,\ 6,\ abs)}时,参数\texttt{x},\texttt{y}和\texttt{f}分别接收\texttt{-5},\texttt{6}和\texttt{abs},根据函数定义,我们可以推导计算过程为:

\begin{pythoncode}
x = -5
y = 6
f = abs
f(x) + f(y) ==> abs(-5) + abs(6) ==> 11
return 11
\end{pythoncode}

用代码验证一下:

\begin{pythoncode}
# -*- coding: utf-8 -*-
\end{pythoncode}

编写高阶函数,就是让函数的参数能够接收别的函数。

\hypertarget{ux5c0fux7ed3}{%
\subsubsection{小结}\label{ux5c0fux7ed3}}

把函数作为参数传入,这样的函数称为高阶函数,函数式编程就是指这种高度抽象的编程范式。

