\hypertarget{python-ux57faux7840}{%
\subsection{Python 基础}\label{python-ux57faux7840}}

Python
是一种计算机编程语言。计算机编程语言和我们日常使用的自然语言有所不同,最大的区别就是,自然语言在不同的语境下有不同的理解,而计算机要根据编程语言执行任务,就必须保证编程语言写出的程序决不能有歧义,所以,任何一种编程语言都有自己的一套语法,编译器或者解释器就是负责把符合语法的程序代码转换成
CPU 能够执行的机器码,然后执行。Python 也不例外。

Python 的语法比较简单,采用缩进方式,写出来的代码就像下面的样子:

\begin{pythoncode}
a = 100
if a >= 0:
    print(a)
else:
    print(-a)
\end{pythoncode}

以\texttt{\#}开头的语句是注释,注释是给人看的,可以是任意内容,解释器会忽略掉注释。其他每一行都是一个语句,当语句以冒号\texttt{:}结尾时,缩进的语句视为代码块。

缩进有利有弊。好处是强迫你写出格式化的代码,但没有规定缩进是几个空格还是
Tab。按照约定俗成的惯例,应该始终坚持使用 \_4 个空格\_的缩进。

缩进的另一个好处是强迫你写出缩进较少的代码,你会倾向于把一段很长的代码拆分成若干函数,从而得到缩进较少的代码。

缩进的坏处就是 ``复制-粘贴''
功能失效了,这是最坑爹的地方。当你重构代码时,粘贴过去的代码必须重新检查缩进是否正确。此外,IDE
很难像格式化 Java 代码那样格式化 Python 代码。

最后,请务必注意,Python
程序是大小写敏感的,如果写错了大小写,程序会报错。

\hypertarget{ux5c0fux7ed3}{%
\subsubsection{小结}\label{ux5c0fux7ed3}}

Python 使用缩进来组织代码块,请务必遵守约定俗成的习惯,坚持使用 4
个空格的缩进。

在文本编辑器中,需要设置把 Tab 自动转换为 4 个空格,确保不混用 Tab
和空格。

