\hypertarget{ux5206ux5e03ux5f0fux8fdbux7a0b}{%
\subsection{分布式进程}\label{ux5206ux5e03ux5f0fux8fdbux7a0b}}

在 Thread 和 Process 中,应当优选 Process,因为 Process
更稳定,而且,Process 可以分布到多台机器上,而 Thread
最多只能分布到同一台机器的多个 CPU 上。

Python
的\texttt{multiprocessing}模块不但支持多进程,其中\texttt{managers}子模块还支持把多进程分布到多台机器上。一个服务进程可以作为调度者,将任务分布到其他多个进程中,依靠网络通信。由于\texttt{managers}模块封装很好,不必了解网络通信的细节,就可以很容易地编写分布式多进程程序。

举个例子:如果我们已经有一个通过\texttt{Queue}通信的多进程程序在同一台机器上运行,现在,由于处理任务的进程任务繁重,希望把发送任务的进程和处理任务的进程分布到两台机器上。怎么用分布式进程实现?

原有的\texttt{Queue}可以继续使用,但是,通过\texttt{managers}模块把\texttt{Queue}通过网络暴露出去,就可以让其他机器的进程访问\texttt{Queue}了。

我们先看服务进程,服务进程负责启动\texttt{Queue},把\texttt{Queue}注册到网络上,然后往\texttt{Queue}里面写入任务:

\begin{pythoncode}
import random, time, queue
from multiprocessing.managers import BaseManager
task_queue = queue.Queue()

result_queue = queue.Queue()
class QueueManager(BaseManager):
    pass
QueueManager.register('get_task_queue', callable=lambda: task_queue)
QueueManager.register('get_result_queue', callable=lambda: result_queue)

manager = QueueManager(address=('', 5000), authkey=b'abc')

manager.start()

task = manager.get_task_queue()
result = manager.get_result_queue()

for i in range(10):
    n = random.randint(0, 10000)
    print('Put task %d...' % n)
    task.put(n)

print('Try get results...')
for i in range(10):
    r = result.get(timeout=10)
    print('Result: %s' % r)

manager.shutdown()
print('master exit.')
\end{pythoncode}

请注意,当我们在一台机器上写多进程程序时,创建的\texttt{Queue}可以直接拿来用,但是,在分布式多进程环境下,添加任务到\texttt{Queue}不可以直接对原始的\texttt{task\_queue}进行操作,那样就绕过了\texttt{QueueManager}的封装,必须通过\texttt{manager.get\_task\_queue()}获得的\texttt{Queue}接口添加。

然后,在另一台机器上启动任务进程(本机上启动也可以):

\begin{pythoncode}
import time, sys, queue
from multiprocessing.managers import BaseManager
class QueueManager(BaseManager):
    pass
QueueManager.register('get_task_queue')
QueueManager.register('get_result_queue')
server_addr = '127.0.0.1'
print('Connect to server %s...' % server_addr)

m = QueueManager(address=(server_addr, 5000), authkey=b'abc')

m.connect()

task = m.get_task_queue()
result = m.get_result_queue()

for i in range(10):
    try:
        n = task.get(timeout=1)
        print('run task %d * %d...' % (n, n))
        r = '%d * %d = %d' % (n, n, n*n)
        time.sleep(1)
        result.put(r)
    except Queue.Empty:
        print('task queue is empty.')

print('worker exit.')
\end{pythoncode}

任务进程要通过网络连接到服务进程,所以要指定服务进程的 IP。

现在,可以试试分布式进程的工作效果了。先启动\texttt{task\_master.py}服务进程:

\begin{pythoncode}
$ python3 task_master.py 
Put task 3411...
Put task 1605...
Put task 1398...
Put task 4729...
Put task 5300...
Put task 7471...
Put task 68...
Put task 4219...
Put task 339...
Put task 7866...
Try get results...
\end{pythoncode}

\texttt{task\_master.py}进程发送完任务后,开始等待\texttt{result}队列的结果。现在启动\texttt{task\_worker.py}进程:

\begin{pythoncode}
$ python3 task_worker.py
Connect to server 127.0.0.1...
run task 3411 * 3411...
run task 1605 * 1605...
run task 1398 * 1398...
run task 4729 * 4729...
run task 5300 * 5300...
run task 7471 * 7471...
run task 68 * 68...
run task 4219 * 4219...
run task 339 * 339...
run task 7866 * 7866...
worker exit.
\end{pythoncode}

\texttt{task\_worker.py}进程结束,在\texttt{task\_master.py}进程中会继续打印出结果:

\begin{pythoncode}
Result: 3411 * 3411 = 11634921
Result: 1605 * 1605 = 2576025
Result: 1398 * 1398 = 1954404
Result: 4729 * 4729 = 22363441
Result: 5300 * 5300 = 28090000
Result: 7471 * 7471 = 55815841
Result: 68 * 68 = 4624
Result: 4219 * 4219 = 17799961
Result: 339 * 339 = 114921
Result: 7866 * 7866 = 61873956
\end{pythoncode}

这个简单的 Master/Worker
模型有什么用?其实这就是一个简单但真正的分布式计算,把代码稍加改造,启动多个
worker,就可以把任务分布到几台甚至几十台机器上,比如把计算\texttt{n*n}的代码换成发送邮件,就实现了邮件队列的异步发送。

Queue 对象存储在哪?注意到\texttt{task\_worker.py}中根本没有创建 Queue
的代码,所以,Queue 对象存储在\texttt{task\_master.py}进程中:

\begin{pythoncode}
                                             │
┌─────────────────────────────────────────┐     ┌──────────────────────────────────────┐
│task_master.py                           │  │  │task_worker.py                        │
│                                         │     │                                      │
│  task = manager.get_task_queue()        │  │  │  task = manager.get_task_queue()     │
│  result = manager.get_result_queue()    │     │  result = manager.get_result_queue() │
│              │                          │  │  │              │                       │
│              │                          │     │              │                       │
│              ▼                          │  │  │              │                       │
│  ┌─────────────────────────────────┐    │     │              │                       │
│  │QueueManager                     │    │  │  │              │                       │
│  │ ┌────────────┐ ┌──────────────┐ │    │     │              │                       │
│  │ │ task_queue │ │ result_queue │ │<───┼──┼──┼──────────────┘                       │
│  │ └────────────┘ └──────────────┘ │    │     │                                      │
│  └─────────────────────────────────┘    │  │  │                                      │
└─────────────────────────────────────────┘     └──────────────────────────────────────┘
                                             │

                                          Network
\end{pythoncode}

而\texttt{Queue}之所以能通过网络访问,就是通过\texttt{QueueManager}实现的。由于\texttt{QueueManager}管理的不止一个\texttt{Queue},所以,要给每个\texttt{Queue}的网络调用接口起个名字,比如\texttt{get\_task\_queue}。

\texttt{authkey}有什么用?这是为了保证两台机器正常通信,不被其他机器恶意干扰。如果\texttt{task\_worker.py}的\texttt{authkey}和\texttt{task\_master.py}的\texttt{authkey}不一致,肯定连接不上。

\hypertarget{ux5c0fux7ed3}{%
\subsubsection{小结}\label{ux5c0fux7ed3}}

Python
的分布式进程接口简单,封装良好,适合需要把繁重任务分布到多台机器的环境下。

注意 Queue
的作用是用来传递任务和接收结果,每个任务的描述数据量要尽量小。比如发送一个处理日志文件的任务,就不要发送几百兆的日志文件本身,而是发送日志文件存放的完整路径,由
Worker 进程再去共享的磁盘上读取文件。

\hypertarget{ux53c2ux8003ux6e90ux7801}{%
\subsubsection{参考源码}\label{ux53c2ux8003ux6e90ux7801}}

\href{https://github.com/michaelliao/learn-python3/blob/master/samples/multitask/task_master.py}{task\_master.py}

\href{https://github.com/michaelliao/learn-python3/blob/master/samples/multitask/task_worker.py}{task\_worker.py}

