\hypertarget{ux5b9aux4e49ux51fdux6570}{%
\subsection{定义函数}\label{ux5b9aux4e49ux51fdux6570}}

在 Python
中,定义一个函数要使用\texttt{def}语句,依次写出函数名、括号、括号中的参数和冒号\texttt{:},然后,在缩进块中编写函数体,函数的返回值用\texttt{return}语句返回。

我们以自定义一个求绝对值的\texttt{my\_abs}函数为例:

\begin{pythoncode}
# -*- coding: utf-8 -*-
\end{pythoncode}

\begin{pythoncode}
print(my_abs(-99))
\end{pythoncode}

请自行测试并调用\texttt{my\_abs}看看返回结果是否正确。

请注意,函数体内部的语句在执行时,一旦执行到\texttt{return}时,函数就执行完毕,并将结果返回。因此,函数内部通过条件判断和循环可以实现非常复杂的逻辑。

如果没有\texttt{return}语句,函数执行完毕后也会返回结果,只是结果为\texttt{None}。\texttt{return\ None}可以简写为\texttt{return}。

在 Python 交互环境中定义函数时,注意 Python
会出现\texttt{...}的提示。函数定义结束后需要按两次回车重新回到\texttt{\textgreater{}\textgreater{}\textgreater{}}提示符下:

\begin{pythoncode}
┌────────────────────────────────────────────────────────┐
│Command Prompt - python                           - □ x │
├────────────────────────────────────────────────────────┤
│>>> def my_abs(x):                                      │
│...     if x >= 0:                                      │
│...         return x                                    │
│...     else:                                           │
│...         return -x                                   │
│...                                                     │
│>>> my_abs(-9)                                          │
│9                                                       │
│>>> _                                                   │
│                                                        │
│                                                        │
└────────────────────────────────────────────────────────┘
\end{pythoncode}

如果你已经把\texttt{my\_abs()}的函数定义保存为\texttt{abstest.py}文件了,那么,可以在该文件的当前目录下启动
Python
解释器,用\texttt{from\ abstest\ import\ my\_abs}来导入\texttt{my\_abs()}函数,注意\texttt{abstest}是文件名(不含\texttt{.py}扩展名):

\begin{pythoncode}
┌────────────────────────────────────────────────────────┐
│Command Prompt - python                           - □ x │
├────────────────────────────────────────────────────────┤
│>>> from abstest import my_abs                          │
│>>> my_abs(-9)                                          │
│9                                                       │
│>>> _                                                   │
│                                                        │
│                                                        │
│                                                        │
│                                                        │
│                                                        │
│                                                        │
│                                                        │
└────────────────────────────────────────────────────────┘
\end{pythoncode}

\texttt{import}的用法在后续\href{https://www.liaoxuefeng.com/wiki/1016959663602400/1017454145014176}{模块}一节中会详细介绍。

\hypertarget{ux7a7aux51fdux6570}{%
\subsubsection{空函数}\label{ux7a7aux51fdux6570}}

如果想定义一个什么事也不做的空函数,可以用\texttt{pass}语句:

\begin{pythoncode}
def nop():
    pass
\end{pythoncode}

\texttt{pass}语句什么都不做,那有什么用?实际上\texttt{pass}可以用来作为占位符,比如现在还没想好怎么写函数的代码,就可以先放一个\texttt{pass},让代码能运行起来。

\texttt{pass}还可以用在其他语句里,比如:

\begin{pythoncode}
if age >= 18:
    pass
\end{pythoncode}

缺少了\texttt{pass},代码运行就会有语法错误。

\hypertarget{ux53c2ux6570ux68c0ux67e5}{%
\subsubsection{参数检查}\label{ux53c2ux6570ux68c0ux67e5}}

调用函数时,如果参数个数不对,Python
解释器会自动检查出来,并抛出\texttt{TypeError}:

\begin{pythoncode}
>>> my_abs(1, 2)
Traceback (most recent call last):
  File "<stdin>", line 1, in <module>
TypeError: my_abs() takes 1 positional argument but 2 were given
\end{pythoncode}

但是如果参数类型不对,Python
解释器就无法帮我们检查。试试\texttt{my\_abs}和内置函数\texttt{abs}的差别:

\begin{pythoncode}
>>> my_abs('A')
Traceback (most recent call last):
  File "<stdin>", line 1, in <module>
  File "<stdin>", line 2, in my_abs
TypeError: unorderable types: str() >= int()
>>> abs('A')
Traceback (most recent call last):
  File "<stdin>", line 1, in <module>
TypeError: bad operand type for abs(): 'str'
\end{pythoncode}

当传入了不恰当的参数时,内置函数\texttt{abs}会检查出参数错误,而我们定义的\texttt{my\_abs}没有参数检查,会导致\texttt{if}语句出错,出错信息和\texttt{abs}不一样。所以,这个函数定义不够完善。

让我们修改一下\texttt{my\_abs}的定义,对参数类型做检查,只允许整数和浮点数类型的参数。数据类型检查可以用内置函数\texttt{isinstance()}实现:

\begin{pythoncode}
def my_abs(x):
    if not isinstance(x, (int, float)):
        raise TypeError('bad operand type')
    if x >= 0:
        return x
    else:
        return -x
\end{pythoncode}

添加了参数检查后,如果传入错误的参数类型,函数就可以抛出一个错误:

\begin{pythoncode}
>>> my_abs('A')
Traceback (most recent call last):
  File "<stdin>", line 1, in <module>
  File "<stdin>", line 3, in my_abs
TypeError: bad operand type
\end{pythoncode}

错误和异常处理将在后续讲到。

\hypertarget{ux8fd4ux56deux591aux4e2aux503c}{%
\subsubsection{返回多个值}\label{ux8fd4ux56deux591aux4e2aux503c}}

函数可以返回多个值吗?答案是肯定的。

比如在游戏中经常需要从一个点移动到另一个点,给出坐标、位移和角度,就可以计算出新的坐标:

\begin{pythoncode}
import math

def move(x, y, step, angle=0):
    nx = x + step * math.cos(angle)
    ny = y - step * math.sin(angle)
    return nx, ny
\end{pythoncode}

\texttt{import\ math}语句表示导入\texttt{math}包,并允许后续代码引用\texttt{math}包里的\texttt{sin}、\texttt{cos}等函数。

然后,我们就可以同时获得返回值:

\begin{pythoncode}
>>> x, y = move(100, 100, 60, math.pi / 6)
>>> print(x, y)
151.96152422706632 70.0
\end{pythoncode}

但其实这只是一种假象,Python 函数返回的仍然是单一值:

\begin{pythoncode}
>>> r = move(100, 100, 60, math.pi / 6)
>>> print(r)
(151.96152422706632, 70.0)
\end{pythoncode}

原来返回值是一个 tuple!但是,在语法上,返回一个 tuple
可以省略括号,而多个变量可以同时接收一个
tuple,按位置赋给对应的值,所以,Python 的函数返回多值其实就是返回一个
tuple,但写起来更方便。

\hypertarget{ux5c0fux7ed3}{%
\subsubsection{小结}\label{ux5c0fux7ed3}}

定义函数时,需要确定函数名和参数个数;

如果有必要,可以先对参数的数据类型做检查;

函数体内部可以用\texttt{return}随时返回函数结果;

函数执行完毕也没有\texttt{return}语句时,自动\texttt{return\ None}。

函数可以同时返回多个值,但其实就是一个 tuple。

\hypertarget{ux7ec3ux4e60}{%
\subsubsection{练习}\label{ux7ec3ux4e60}}

请定义一个函数\texttt{quadratic(a,\ b,\ c)},接收 3
个参数,返回一元二次方程 \(ax^2+bx+c=0\) 的两个解。

提示:

一元二次方程的求根公式为:

\(x=\frac{-b\pm\sqrt{b^2-4ac}}{2a}\)

计算平方根可以调用\texttt{math.sqrt()}函数:

\begin{pythoncode}
>>> import math
>>> math.sqrt(2)
1.4142135623730951
\end{pythoncode}

\begin{pythoncode}
# -*- coding: utf-8 -*-

import math

def quadratic(a, b, c):
\end{pythoncode}

\begin{pythoncode}
# 测试:
print('quadratic(2, 3, 1) =', quadratic(2, 3, 1))
print('quadratic(1, 3, -4) =', quadratic(1, 3, -4))

if quadratic(2, 3, 1) != (-0.5, -1.0):
    print('测试失败')
elif quadratic(1, 3, -4) != (1.0, -4.0):
    print('测试失败')
else:
    print('测试成功')
\end{pythoncode}

\hypertarget{ux53c2ux8003ux6e90ux7801}{%
\subsubsection{参考源码}\label{ux53c2ux8003ux6e90ux7801}}

\href{https://github.com/michaelliao/learn-python3/blob/master/samples/function/def_func.py}{def\_func.py}

