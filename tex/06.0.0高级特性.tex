\hypertarget{ux9ad8ux7ea7ux7279ux6027}{%
\subsection{高级特性}\label{ux9ad8ux7ea7ux7279ux6027}}

掌握了 Python
的数据类型、语句和函数,基本上就可以编写出很多有用的程序了。

比如构造一个\texttt{1,\ 3,\ 5,\ 7,\ ...,\ 99}的列表,可以通过循环实现:

\begin{pythoncode}
L = []
n = 1
while n <= 99:
    L.append(n)
    n = n + 2
\end{pythoncode}

取 list 的前一半的元素,也可以通过循环实现。

但是在 Python
中,代码不是越多越好,而是越少越好。代码不是越复杂越好,而是越简单越好。

基于这一思想,我们来介绍 Python 中非常有用的高级特性,1
行代码能实现的功能,决不写 5
行代码。请始终牢记,代码越少,开发效率越高。

