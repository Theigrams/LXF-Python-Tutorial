\hypertarget{stringio-ux548c-bytesio}{%
\subsection{StringIO 和 BytesIO}\label{stringio-ux548c-bytesio}}

\hypertarget{stringio}{%
\subsubsection{StringIO}\label{stringio}}

很多时候,数据读写不一定是文件,也可以在内存中读写。

StringIO 顾名思义就是在内存中读写 str。

要把 str 写入 StringIO,我们需要先创建一个
StringIO,然后,像文件一样写入即可:

\begin{pythoncode}
>>> from io import StringIO
>>> f = StringIO()
>>> f.write('hello')
5
>>> f.write(' ')
1
>>> f.write('world!')
6
>>> print(f.getvalue())
hello world!
\end{pythoncode}

\texttt{getvalue()}方法用于获得写入后的 str。

要读取 StringIO,可以用一个 str 初始化
StringIO,然后,像读文件一样读取:

\begin{pythoncode}
>>> from io import StringIO
>>> f = StringIO('Hello!\nHi!\nGoodbye!')
>>> while True:
...     s = f.readline()
...     if s == '':
...         break
...     print(s.strip())
...
Hello!
Hi!
Goodbye!
\end{pythoncode}

\hypertarget{bytesio}{%
\subsubsection{BytesIO}\label{bytesio}}

StringIO 操作的只能是 str,如果要操作二进制数据,就需要使用 BytesIO。

BytesIO 实现了在内存中读写 bytes,我们创建一个 BytesIO,然后写入一些
bytes:

\begin{pythoncode}
>>> from io import BytesIO
>>> f = BytesIO()
>>> f.write('中文'.encode('utf-8'))
6
>>> print(f.getvalue())
b'\xe4\xb8\xad\xe6\x96\x87'
\end{pythoncode}

请注意,写入的不是 str,而是经过 UTF-8 编码的 bytes。

和 StringIO 类似,可以用一个 bytes 初始化
BytesIO,然后,像读文件一样读取:

\begin{pythoncode}
>>> from io import BytesIO
>>> f = BytesIO(b'\xe4\xb8\xad\xe6\x96\x87')
>>> f.read()
b'\xe4\xb8\xad\xe6\x96\x87'
\end{pythoncode}

\hypertarget{ux5c0fux7ed3}{%
\subsubsection{小结}\label{ux5c0fux7ed3}}

StringIO 和 BytesIO 是在内存中操作 str 和 bytes
的方法,使得和读写文件具有一致的接口。

\hypertarget{ux53c2ux8003ux6e90ux7801}{%
\subsubsection{参考源码}\label{ux53c2ux8003ux6e90ux7801}}

\href{https://github.com/michaelliao/learn-python3/blob/master/samples/io/do_stringio.py}{do\_stringio.py}

\href{https://github.com/michaelliao/learn-python3/blob/master/samples/io/do_bytesio.py}{do\_bytesio.py}

