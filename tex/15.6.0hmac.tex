\hypertarget{hmac}{%
\subsection{hmac}\label{hmac}}

通过哈希算法,我们可以验证一段数据是否有效,方法就是对比该数据的哈希值,例如,判断用户口令是否正确,我们用保存在数据库中的\texttt{password\_md5}对比计算\texttt{md5(password)}的结果,如果一致,用户输入的口令就是正确的。

为了防止黑客通过彩虹表根据哈希值反推原始口令,在计算哈希的时候,不能仅针对原始输入计算,需要增加一个
salt
来使得相同的输入也能得到不同的哈希,这样,大大增加了黑客破解的难度。

如果 salt 是我们自己随机生成的,通常我们计算 MD5
时采用\texttt{md5(message\ +\ salt)}。但实际上,把 salt 看做一个
``口令'',加 salt 的哈希就是:计算一段 message
的哈希时,根据不同口令计算出不同的哈希。要验证哈希值,必须同时提供正确的口令。

这实际上就是 Hmac 算法:Keyed-Hashing for Message
Authentication。它通过一个标准算法,在计算哈希的过程中,把 key
混入计算过程中。

和我们自定义的加 salt 算法不同,Hmac 算法针对所有哈希算法都通用,无论是
MD5 还是 SHA-1。采用 Hmac 替代我们自己的 salt
算法,可以使程序算法更标准化,也更安全。

Python 自带的 hmac 模块实现了标准的 Hmac 算法。我们来看看如何使用 hmac
实现带 key 的哈希。

我们首先需要准备待计算的原始消息 message,随机 key,哈希算法,这里采用
MD5,使用 hmac 的代码如下:

\begin{pythoncode}
>>> import hmac
>>> message = b'Hello, world!'
>>> key = b'secret'
>>> h = hmac.new(key, message, digestmod='MD5')
>>> 
>>> h.hexdigest()
'fa4ee7d173f2d97ee79022d1a7355bcf'
\end{pythoncode}

可见使用 hmac 和普通 hash 算法非常类似。hmac
输出的长度和原始哈希算法的长度一致。需要注意传入的 key 和 message
都是\texttt{bytes}类型,\texttt{str}类型需要首先编码为\texttt{bytes}。

\hypertarget{ux7ec3ux4e60}{%
\subsubsection{练习}\label{ux7ec3ux4e60}}

将上一节的 salt 改为标准的 hmac 算法,验证用户口令:

\begin{pythoncode}
# -*- coding: utf-8 -*-
import hmac, random

def hmac_md5(key, s):
    return hmac.new(key.encode('utf-8'), s.encode('utf-8'), 'MD5').hexdigest()

class User(object):
    def __init__(self, username, password):
        self.username = username
        self.key = ''.join([chr(random.randint(48, 122)) for i in range(20)])
        self.password = hmac_md5(self.key, password)

db = {
    'michael': User('michael', '123456'),
    'bob': User('bob', 'abc999'),
    'alice': User('alice', 'alice2008')
}
\end{pythoncode}

\begin{pythoncode}
# 测试:
assert login('michael', '123456')
assert login('bob', 'abc999')
assert login('alice', 'alice2008')
assert not login('michael', '1234567')
assert not login('bob', '123456')
assert not login('alice', 'Alice2008')
print('ok')
\end{pythoncode}

\hypertarget{ux5c0fux7ed3}{%
\subsubsection{小结}\label{ux5c0fux7ed3}}

Python 内置的 hmac 模块实现了标准的 Hmac 算法,它利用一个 key 对 message
计算 ``杂凑'' 后的 hash,使用 hmac 算法比标准 hash
算法更安全,因为针对相同的 message,不同的 key 会产生不同的 hash。

