\hypertarget{day-14---ux5b8cux6210-web-app}{%
\subsection{Day 14 - 完成 Web App}\label{day-14---ux5b8cux6210-web-app}}

在 Web App 框架和基本流程跑通后,剩下的工作全部是体力活了:在 Debug
开发模式下完成后端所有 API、前端所有页面。我们需要做的事情包括:

把当前用户绑定到\texttt{request}上,并对
URL\texttt{/manage/}进行拦截,检查当前用户是否是管理员身份:

\begin{pythoncode}
@asyncio.coroutine
def auth_factory(app, handler):
    @asyncio.coroutine
    def auth(request):
        logging.info('check user: %s %s' % (request.method, request.path))
        request.__user__ = None
        cookie_str = request.cookies.get(COOKIE_NAME)
        if cookie_str:
            user = yield from cookie2user(cookie_str)
            if user:
                logging.info('set current user: %s' % user.email)
                request.__user__ = user
        if request.path.startswith('/manage/') and (request.__user__ is None or not request.__user__.admin):
            return web.HTTPFound('/signin')
        return (yield from handler(request))
    return auth
\end{pythoncode}

后端 API 包括:

\begin{itemize}
\item
  获取日志:GET /api/blogs
\item
  创建日志:POST /api/blogs
\item
  修改日志:POST /api/blogs/:blog\_id
\item
  删除日志:POST /api/blogs/:blog\_id/delete
\item
  获取评论:GET /api/comments
\item
  创建评论:POST /api/blogs/:blog\_id/comments
\item
  删除评论:POST /api/comments/:comment\_id/delete
\item
  创建新用户:POST /api/users
\item
  获取用户:GET /api/users
\end{itemize}

管理页面包括:

\begin{itemize}
\item
  评论列表页:GET /manage/comments
\item
  日志列表页:GET /manage/blogs
\item
  创建日志页:GET /manage/blogs/create
\item
  修改日志页:GET /manage/blogs/
\item
  用户列表页:GET /manage/users
\end{itemize}

用户浏览页面包括:

\begin{itemize}
\item
  注册页:GET /register
\item
  登录页:GET /signin
\item
  注销页:GET /signout
\item
  首页:GET /
\item
  日志详情页:GET /blog/:blog\_id
\end{itemize}

把所有的功能实现,我们第一个 Web App 就宣告完成!

\hypertarget{ux53c2ux8003ux6e90ux7801}{%
\subsubsection{参考源码}\label{ux53c2ux8003ux6e90ux7801}}

\href{https://github.com/michaelliao/awesome-python3-webapp/tree/day-14}{day-14}

