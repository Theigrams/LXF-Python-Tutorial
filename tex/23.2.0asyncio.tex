\hypertarget{asyncio}{%
\subsection{asyncio}\label{asyncio}}

\texttt{asyncio}是 Python 3.4 版本引入的标准库,直接内置了对异步 IO
的支持。

\texttt{asyncio}的编程模型就是一个消息循环。我们从\texttt{asyncio}模块中直接获取一个\texttt{EventLoop}的引用,然后把需要执行的协程扔到\texttt{EventLoop}中执行,就实现了异步
IO。

用\texttt{asyncio}实现\texttt{Hello\ world}代码如下:

\begin{pythoncode}
import asyncio

@asyncio.coroutine
def hello():
    print("Hello world!")
    
    r = yield from asyncio.sleep(1)
    print("Hello again!")
loop = asyncio.get_event_loop()

loop.run_until_complete(hello())
loop.close()
\end{pythoncode}

\texttt{@asyncio.coroutine}把一个 generator 标记为 coroutine
类型,然后,我们就把这个\texttt{coroutine}扔到\texttt{EventLoop}中执行。

\texttt{hello()}会首先打印出\texttt{Hello\ world!},然后,\texttt{yield\ from}语法可以让我们方便地调用另一个\texttt{generator}。由于\texttt{asyncio.sleep()}也是一个\texttt{coroutine},所以线程不会等待\texttt{asyncio.sleep()},而是直接中断并执行下一个消息循环。当\texttt{asyncio.sleep()}返回时,线程就可以从\texttt{yield\ from}拿到返回值(此处是\texttt{None}),然后接着执行下一行语句。

把\texttt{asyncio.sleep(1)}看成是一个耗时 1 秒的 IO
操作,在此期间,主线程并未等待,而是去执行\texttt{EventLoop}中其他可以执行的\texttt{coroutine}了,因此可以实现并发执行。

我们用 Task 封装两个\texttt{coroutine}试试:

\begin{pythoncode}
import threading
import asyncio

@asyncio.coroutine
def hello():
    print('Hello world! (%s)' % threading.currentThread())
    yield from asyncio.sleep(1)
    print('Hello again! (%s)' % threading.currentThread())

loop = asyncio.get_event_loop()
tasks = [hello(), hello()]
loop.run_until_complete(asyncio.wait(tasks))
loop.close()
\end{pythoncode}

观察执行过程:

\begin{pythoncode}
Hello world! (<_MainThread(MainThread, started 140735195337472)>)
Hello world! (<_MainThread(MainThread, started 140735195337472)>)
(暂停约1秒)
Hello again! (<_MainThread(MainThread, started 140735195337472)>)
Hello again! (<_MainThread(MainThread, started 140735195337472)>)
\end{pythoncode}

由打印的当前线程名称可以看出,两个\texttt{coroutine}是由同一个线程并发执行的。

如果把\texttt{asyncio.sleep()}换成真正的 IO
操作,则多个\texttt{coroutine}就可以由一个线程并发执行。

我们用\texttt{asyncio}的异步网络连接来获取 sina、sohu 和 163
的网站首页:

\begin{pythoncode}
import asyncio

@asyncio.coroutine
def wget(host):
    print('wget %s...' % host)
    connect = asyncio.open_connection(host, 80)
    reader, writer = yield from connect
    header = 'GET / HTTP/1.0\r\nHost: %s\r\n\r\n' % host
    writer.write(header.encode('utf-8'))
    yield from writer.drain()
    while True:
        line = yield from reader.readline()
        if line == b'\r\n':
            break
        print('%s header > %s' % (host, line.decode('utf-8').rstrip()))
    
    writer.close()

loop = asyncio.get_event_loop()
tasks = [wget(host) for host in ['www.sina.com.cn', 'www.sohu.com', 'www.163.com']]
loop.run_until_complete(asyncio.wait(tasks))
loop.close()
\end{pythoncode}

执行结果如下:

\begin{pythoncode}
wget www.sohu.com...
wget www.sina.com.cn...
wget www.163.com...
(等待一段时间)
(打印出sohu的header)
www.sohu.com header > HTTP/1.1 200 OK
www.sohu.com header > Content-Type: text/html
...
(打印出sina的header)
www.sina.com.cn header > HTTP/1.1 200 OK
www.sina.com.cn header > Date: Wed, 20 May 2015 04:56:33 GMT
...
(打印出163的header)
www.163.com header > HTTP/1.0 302 Moved Temporarily
www.163.com header > Server: Cdn Cache Server V2.0
...
\end{pythoncode}

可见 3 个连接由一个线程通过\texttt{coroutine}并发完成。

\hypertarget{ux5c0fux7ed3}{%
\subsubsection{小结}\label{ux5c0fux7ed3}}

\texttt{asyncio}提供了完善的异步 IO 支持;

异步操作需要在\texttt{coroutine}中通过\texttt{yield\ from}完成;

多个\texttt{coroutine}可以封装成一组 Task 然后并发执行。

\hypertarget{ux53c2ux8003ux6e90ux7801}{%
\subsubsection{参考源码}\label{ux53c2ux8003ux6e90ux7801}}

\href{https://github.com/michaelliao/learn-python3/blob/master/samples/async/async_hello.py}{async\_hello.py}

\href{https://github.com/michaelliao/learn-python3/blob/master/samples/async/async_wget.py}{async\_wget.py}

