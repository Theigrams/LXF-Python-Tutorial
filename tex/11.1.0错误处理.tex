\hypertarget{ux9519ux8befux5904ux7406}{%
\subsection{错误处理}\label{ux9519ux8befux5904ux7406}}

在程序运行的过程中,如果发生了错误,可以事先约定返回一个错误代码,这样,就可以知道是否有错,以及出错的原因。在操作系统提供的调用中,返回错误码非常常见。比如打开文件的函数\texttt{open()},成功时返回文件描述符(就是一个整数),出错时返回\texttt{-1}。

用错误码来表示是否出错十分不便,因为函数本身应该返回的正常结果和错误码混在一起,造成调用者必须用大量的代码来判断是否出错:

\begin{pythoncode}
def foo():
    r = some_function()
    if r==(-1):
        return (-1)
    
    return r

def bar():
    r = foo()
    if r==(-1):
        print('Error')
    else:
        pass
\end{pythoncode}

一旦出错,还要一级一级上报,直到某个函数可以处理该错误(比如,给用户输出一个错误信息)。

所以高级语言通常都内置了一套\texttt{try...except...finally...}的错误处理机制,Python
也不例外。

\hypertarget{try}{%
\subsubsection{try}\label{try}}

让我们用一个例子来看看\texttt{try}的机制:

\begin{pythoncode}
try:
    print('try...')
    r = 10 / 0
    print('result:', r)
except ZeroDivisionError as e:
    print('except:', e)
finally:
    print('finally...')
print('END')
\end{pythoncode}

当我们认为某些代码可能会出错时,就可以用\texttt{try}来运行这段代码,如果执行出错,则后续代码不会继续执行,而是直接跳转至错误处理代码,即\texttt{except}语句块,执行完\texttt{except}后,如果有\texttt{finally}语句块,则执行\texttt{finally}语句块,至此,执行完毕。

上面的代码在计算\texttt{10\ /\ 0}时会产生一个除法运算错误:

\begin{pythoncode}
try...
except: division by zero
finally...
END
\end{pythoncode}

从输出可以看到,当错误发生时,后续语句\texttt{print(\textquotesingle{}result:\textquotesingle{},\ r)}不会被执行,\texttt{except}由于捕获到\texttt{ZeroDivisionError},因此被执行。最后,\texttt{finally}语句被执行。然后,程序继续按照流程往下走。

如果把除数\texttt{0}改成\texttt{2},则执行结果如下:

\begin{pythoncode}
try...
result: 5
finally...
END
\end{pythoncode}

由于没有错误发生,所以\texttt{except}语句块不会被执行,但是\texttt{finally}如果有,则一定会被执行(可以没有\texttt{finally}语句)。

你还可以猜测,错误应该有很多种类,如果发生了不同类型的错误,应该由不同的\texttt{except}语句块处理。没错,可以有多个\texttt{except}来捕获不同类型的错误:

\begin{pythoncode}
try:
    print('try...')
    r = 10 / int('a')
    print('result:', r)
except ValueError as e:
    print('ValueError:', e)
except ZeroDivisionError as e:
    print('ZeroDivisionError:', e)
finally:
    print('finally...')
print('END')
\end{pythoncode}

\texttt{int()}函数可能会抛出\texttt{ValueError},所以我们用一个\texttt{except}捕获\texttt{ValueError},用另一个\texttt{except}捕获\texttt{ZeroDivisionError}。

此外,如果没有错误发生,可以在\texttt{except}语句块后面加一个\texttt{else},当没有错误发生时,会自动执行\texttt{else}语句:

\begin{pythoncode}
try:
    print('try...')
    r = 10 / int('2')
    print('result:', r)
except ValueError as e:
    print('ValueError:', e)
except ZeroDivisionError as e:
    print('ZeroDivisionError:', e)
else:
    print('no error!')
finally:
    print('finally...')
print('END')
\end{pythoncode}

Python 的错误其实也是
class,所有的错误类型都继承自\texttt{BaseException},所以在使用\texttt{except}时需要注意的是,它不但捕获该类型的错误,还把其子类也
``一网打尽''。比如:

\begin{pythoncode}
try:
    foo()
except ValueError as e:
    print('ValueError')
except UnicodeError as e:
    print('UnicodeError')
\end{pythoncode}

第二个\texttt{except}永远也捕获不到\texttt{UnicodeError},因为\texttt{UnicodeError}是\texttt{ValueError}的子类,如果有,也被第一个\texttt{except}给捕获了。

Python
所有的错误都是从\texttt{BaseException}类派生的,常见的错误类型和继承关系看这里:

\url{https://docs.python.org/3/library/exceptions.html\#exception-hierarchy}

使用\texttt{try...except}捕获错误还有一个巨大的好处,就是可以跨越多层调用,比如函数\texttt{main()}调用\texttt{bar()},\texttt{bar()}调用\texttt{foo()},结果\texttt{foo()}出错了,这时,只要\texttt{main()}捕获到了,就可以处理:

\begin{pythoncode}
def foo(s):
    return 10 / int(s)

def bar(s):
    return foo(s) * 2

def main():
    try:
        bar('0')
    except Exception as e:
        print('Error:', e)
    finally:
        print('finally...')
\end{pythoncode}

也就是说,不需要在每个可能出错的地方去捕获错误,只要在合适的层次去捕获错误就可以了。这样一来,就大大减少了写\texttt{try...except...finally}的麻烦。

\hypertarget{ux8c03ux7528ux6808}{%
\subsubsection{调用栈}\label{ux8c03ux7528ux6808}}

如果错误没有被捕获,它就会一直往上抛,最后被 Python
解释器捕获,打印一个错误信息,然后程序退出。来看看\texttt{err.py}:

\begin{pythoncode}
def foo(s):
    return 10 / int(s)

def bar(s):
    return foo(s) * 2

def main():
    bar('0')

main()
\end{pythoncode}

执行,结果如下:

\begin{pythoncode}
$ python3 err.py
Traceback (most recent call last):
  File "err.py", line 11, in <module>
    main()
  File "err.py", line 9, in main
    bar('0')
  File "err.py", line 6, in bar
    return foo(s) * 2
  File "err.py", line 3, in foo
    return 10 / int(s)
ZeroDivisionError: division by zero
\end{pythoncode}

出错并不可怕,可怕的是不知道哪里出错了。解读错误信息是定位错误的关键。我们从上往下可以看到整个错误的调用函数链:

错误信息第 1 行:

\begin{pythoncode}
Traceback (most recent call last):
\end{pythoncode}

告诉我们这是错误的跟踪信息。

第 2\textasciitilde3 行:

\begin{pythoncode}
  File "err.py", line 11, in <module>
    main()
\end{pythoncode}

调用\texttt{main()}出错了,在代码文件\texttt{err.py}的第 11
行代码,但原因是第 9 行:

\begin{pythoncode}
  File "err.py", line 9, in main
    bar('0')
\end{pythoncode}

调用\texttt{bar(\textquotesingle{}0\textquotesingle{})}出错了,在代码文件\texttt{err.py}的第
9 行代码,但原因是第 6 行:

\begin{pythoncode}
  File "err.py", line 6, in bar
    return foo(s) * 2
\end{pythoncode}

原因是\texttt{return\ foo(s)\ *\ 2}这个语句出错了,但这还不是最终原因,继续往下看:

\begin{pythoncode}
  File "err.py", line 3, in foo
    return 10 / int(s)
\end{pythoncode}

原因是\texttt{return\ 10\ /\ int(s)}这个语句出错了,这是错误产生的源头,因为下面打印了:

\begin{pythoncode}
ZeroDivisionError: integer division or modulo by zero
\end{pythoncode}

根据错误类型\texttt{ZeroDivisionError},我们判断,\texttt{int(s)}本身并没有出错,但是\texttt{int(s)}返回\texttt{0},在计算\texttt{10\ /\ 0}时出错,至此,找到错误源头。

出错的时候,一定要分析错误的调用栈信息,才能定位错误的位置。

 
 \begin{figure}[htp]
	\centering
	\includegraphics[width=0.6\linewidth]{fig/1183105155068736l.png}
\end{figure}


\hypertarget{ux8bb0ux5f55ux9519ux8bef}{%
\subsubsection{记录错误}\label{ux8bb0ux5f55ux9519ux8bef}}

如果不捕获错误,自然可以让 Python
解释器来打印出错误堆栈,但程序也被结束了。既然我们能捕获错误,就可以把错误堆栈打印出来,然后分析错误原因,同时,让程序继续执行下去。

Python 内置的\texttt{logging}模块可以非常容易地记录错误信息:

\begin{pythoncode}
import logging

def foo(s):
    return 10 / int(s)

def bar(s):
    return foo(s) * 2

def main():
    try:
        bar('0')
    except Exception as e:
        logging.exception(e)

main()
print('END')
\end{pythoncode}

同样是出错,但程序打印完错误信息后会继续执行,并正常退出:

\begin{pythoncode}
$ python3 err_logging.py
ERROR:root:division by zero
Traceback (most recent call last):
  File "err_logging.py", line 13, in main
    bar('0')
  File "err_logging.py", line 9, in bar
    return foo(s) * 2
  File "err_logging.py", line 6, in foo
    return 10 / int(s)
ZeroDivisionError: division by zero
END
\end{pythoncode}

通过配置,\texttt{logging}还可以把错误记录到日志文件里,方便事后排查。

\hypertarget{ux629bux51faux9519ux8bef}{%
\subsubsection{抛出错误}\label{ux629bux51faux9519ux8bef}}

因为错误是 class,捕获一个错误就是捕获到该 class
的一个实例。因此,错误并不是凭空产生的,而是有意创建并抛出的。Python
的内置函数会抛出很多类型的错误,我们自己编写的函数也可以抛出错误。

如果要抛出错误,首先根据需要,可以定义一个错误的
class,选择好继承关系,然后,用\texttt{raise}语句抛出一个错误的实例:

\begin{pythoncode}
class FooError(ValueError):
    pass

def foo(s):
    n = int(s)
    if n==0:
        raise FooError('invalid value: %s' % s)
    return 10 / n

foo('0')
\end{pythoncode}

执行,可以最后跟踪到我们自己定义的错误:

\begin{pythoncode}
$ python3 err_raise.py 
Traceback (most recent call last):
  File "err_throw.py", line 11, in <module>
    foo('0')
  File "err_throw.py", line 8, in foo
    raise FooError('invalid value: %s' % s)
__main__.FooError: invalid value: 0
\end{pythoncode}

只有在必要的时候才定义我们自己的错误类型。如果可以选择 Python
已有的内置的错误类型(比如\texttt{ValueError},\texttt{TypeError}),尽量使用
Python 内置的错误类型。

最后,我们来看另一种错误处理的方式:

\begin{pythoncode}
def foo(s):
    n = int(s)
    if n==0:
        raise ValueError('invalid value: %s' % s)
    return 10 / n

def bar():
    try:
        foo('0')
    except ValueError as e:
        print('ValueError!')
        raise

bar()
\end{pythoncode}

在\texttt{bar()}函数中,我们明明已经捕获了错误,但是,打印一个\texttt{ValueError!}后,又把错误通过\texttt{raise}语句抛出去了,这不有病么?

其实这种错误处理方式不但没病,而且相当常见。捕获错误目的只是记录一下,便于后续追踪。但是,由于当前函数不知道应该怎么处理该错误,所以,最恰当的方式是继续往上抛,让顶层调用者去处理。好比一个员工处理不了一个问题时,就把问题抛给他的老板,如果他的老板也处理不了,就一直往上抛,最终会抛给
CEO 去处理。

\texttt{raise}语句如果不带参数,就会把当前错误原样抛出。此外,在\texttt{except}中\texttt{raise}一个
Error,还可以把一种类型的错误转化成另一种类型:

\begin{pythoncode}
try:
    10 / 0
except ZeroDivisionError:
    raise ValueError('input error!')
\end{pythoncode}

只要是合理的转换逻辑就可以,但是,决不应该把一个\texttt{IOError}转换成毫不相干的\texttt{ValueError}。

\hypertarget{ux7ec3ux4e60}{%
\subsubsection{练习}\label{ux7ec3ux4e60}}

运行下面的代码,根据异常信息进行分析,定位出错误源头,并修复:

\begin{pythoncode}
# -*- coding: utf-8 -*-
\end{pythoncode}

\hypertarget{ux5c0fux7ed3}{%
\subsubsection{小结}\label{ux5c0fux7ed3}}

Python
内置的\texttt{try...except...finally}用来处理错误十分方便。出错时,会分析错误信息并定位错误发生的代码位置才是最关键的。

程序也可以主动抛出错误,让调用者来处理相应的错误。但是,应该在文档中写清楚可能会抛出哪些错误,以及错误产生的原因。

\hypertarget{ux53c2ux8003ux6e90ux7801}{%
\subsubsection{参考源码}\label{ux53c2ux8003ux6e90ux7801}}

\href{https://github.com/michaelliao/learn-python3/blob/master/samples/debug/do_try.py}{do\_try.py}

\href{https://github.com/michaelliao/learn-python3/blob/master/samples/debug/err.py}{err.py}

\href{https://github.com/michaelliao/learn-python3/blob/master/samples/debug/err_logging.py}{err\_logging.py}

\href{https://github.com/michaelliao/learn-python3/blob/master/samples/debug/err_raise.py}{err\_raise.py}

\href{https://github.com/michaelliao/learn-python3/blob/master/samples/debug/err_reraise.py}{err\_reraise.py}

