\hypertarget{ux7ee7ux627fux548cux591aux6001}{%
\subsection{继承和多态}\label{ux7ee7ux627fux548cux591aux6001}}

在 OOP 程序设计中,当我们定义一个 class 的时候,可以从某个现有的 class
继承,新的 class 称为子类(Subclass),而被继承的 class
称为基类、父类或超类(Base class、Super class)。

比如,我们已经编写了一个名为\texttt{Animal}的
class,有一个\texttt{run()}方法可以直接打印:

\begin{pythoncode}
class Animal(object):
    def run(self):
        print('Animal is running...')
\end{pythoncode}

当我们需要编写\texttt{Dog}和\texttt{Cat}类时,就可以直接从\texttt{Animal}类继承:

\begin{pythoncode}
class Dog(Animal):
    pass

class Cat(Animal):
    pass
\end{pythoncode}

对于\texttt{Dog}来说,\texttt{Animal}就是它的父类,对于\texttt{Animal}来说,\texttt{Dog}就是它的子类。\texttt{Cat}和\texttt{Dog}类似。

继承有什么好处?最大的好处是子类获得了父类的全部功能。由于\texttt{Animial}实现了\texttt{run()}方法,因此,\texttt{Dog}和\texttt{Cat}作为它的子类,什么事也没干,就自动拥有了\texttt{run()}方法:

\begin{pythoncode}
dog = Dog()
dog.run()

cat = Cat()
cat.run()
\end{pythoncode}

运行结果如下:

\begin{pythoncode}
Animal is running...
Animal is running...
\end{pythoncode}

当然,也可以对子类增加一些方法,比如 Dog 类:

\begin{pythoncode}
class Dog(Animal):

    def run(self):
        print('Dog is running...')

    def eat(self):
        print('Eating meat...')
\end{pythoncode}

继承的第二个好处需要我们对代码做一点改进。你看到了,无论是\texttt{Dog}还是\texttt{Cat},它们\texttt{run()}的时候,显示的都是\texttt{Animal\ is\ running...},符合逻辑的做法是分别显示\texttt{Dog\ is\ running...}和\texttt{Cat\ is\ running...},因此,对\texttt{Dog}和\texttt{Cat}类改进如下:

\begin{pythoncode}
class Dog(Animal):

    def run(self):
        print('Dog is running...')

class Cat(Animal):

    def run(self):
        print('Cat is running...')
\end{pythoncode}

再次运行,结果如下:

\begin{pythoncode}
Dog is running...
Cat is running...
\end{pythoncode}

当子类和父类都存在相同的\texttt{run()}方法时,我们说,子类的\texttt{run()}覆盖了父类的\texttt{run()},在代码运行的时候,总是会调用子类的\texttt{run()}。这样,我们就获得了继承的另一个好处:多态。

要理解什么是多态,我们首先要对数据类型再作一点说明。当我们定义一个 class
的时候,我们实际上就定义了一种数据类型。我们定义的数据类型和 Python
自带的数据类型,比如 str、list、dict 没什么两样:

\begin{pythoncode}
a = list() # a是list类型
b = Animal() # b是Animal类型
c = Dog() # c是Dog类型
\end{pythoncode}

判断一个变量是否是某个类型可以用\texttt{isinstance()}判断:

\begin{pythoncode}
>>> isinstance(a, list)
True
>>> isinstance(b, Animal)
True
>>> isinstance(c, Dog)
True
\end{pythoncode}

看来\texttt{a}、\texttt{b}、\texttt{c}确实对应着\texttt{list}、\texttt{Animal}、\texttt{Dog}这
3 种类型。

但是等等,试试:

\begin{pythoncode}
>>> isinstance(c, Animal)
True
\end{pythoncode}

看来\texttt{c}不仅仅是\texttt{Dog},\texttt{c}还是\texttt{Animal}!

不过仔细想想,这是有道理的,因为\texttt{Dog}是从\texttt{Animal}继承下来的,当我们创建了一个\texttt{Dog}的实例\texttt{c}时,我们认为\texttt{c}的数据类型是\texttt{Dog}没错,但\texttt{c}同时也是\texttt{Animal}也没错,\texttt{Dog}本来就是\texttt{Animal}的一种!

所以,在继承关系中,如果一个实例的数据类型是某个子类,那它的数据类型也可以被看做是父类。但是,反过来就不行:

\begin{pythoncode}
>>> b = Animal()
>>> isinstance(b, Dog)
False
\end{pythoncode}

\texttt{Dog}可以看成\texttt{Animal},但\texttt{Animal}不可以看成\texttt{Dog}。

要理解多态的好处,我们还需要再编写一个函数,这个函数接受一个\texttt{Animal}类型的变量:

\begin{pythoncode}
def run_twice(animal):
    animal.run()
    animal.run()
\end{pythoncode}

当我们传入\texttt{Animal}的实例时,\texttt{run\_twice()}就打印出:

\begin{pythoncode}
>>> run_twice(Animal())
Animal is running...
Animal is running...
\end{pythoncode}

当我们传入\texttt{Dog}的实例时,\texttt{run\_twice()}就打印出:

\begin{pythoncode}
>>> run_twice(Dog())
Dog is running...
Dog is running...
\end{pythoncode}

当我们传入\texttt{Cat}的实例时,\texttt{run\_twice()}就打印出:

\begin{pythoncode}
>>> run_twice(Cat())
Cat is running...
Cat is running...
\end{pythoncode}

看上去没啥意思,但是仔细想想,现在,如果我们再定义一个\texttt{Tortoise}类型,也从\texttt{Animal}派生:

\begin{pythoncode}
class Tortoise(Animal):
    def run(self):
        print('Tortoise is running slowly...')
\end{pythoncode}

当我们调用\texttt{run\_twice()}时,传入\texttt{Tortoise}的实例:

\begin{pythoncode}
>>> run_twice(Tortoise())
Tortoise is running slowly...
Tortoise is running slowly...
\end{pythoncode}

你会发现,新增一个\texttt{Animal}的子类,不必对\texttt{run\_twice()}做任何修改,实际上,任何依赖\texttt{Animal}作为参数的函数或者方法都可以不加修改地正常运行,原因就在于多态。

多态的好处就是,当我们需要传入\texttt{Dog}、\texttt{Cat}、\texttt{Tortoise}\ldots\ldots{}
时,我们只需要接收\texttt{Animal}类型就可以了,因为\texttt{Dog}、\texttt{Cat}、\texttt{Tortoise}\ldots\ldots{}
都是\texttt{Animal}类型,然后,按照\texttt{Animal}类型进行操作即可。由于\texttt{Animal}类型有\texttt{run()}方法,因此,传入的任意类型,只要是\texttt{Animal}类或者子类,就会自动调用实际类型的\texttt{run()}方法,这就是多态的意思:

对于一个变量,我们只需要知道它是\texttt{Animal}类型,无需确切地知道它的子类型,就可以放心地调用\texttt{run()}方法,而具体调用的\texttt{run()}方法是作用在\texttt{Animal}、\texttt{Dog}、\texttt{Cat}还是\texttt{Tortoise}对象上,由运行时该对象的确切类型决定,这就是多态真正的威力:调用方只管调用,不管细节,而当我们新增一种\texttt{Animal}的子类时,只要确保\texttt{run()}方法编写正确,不用管原来的代码是如何调用的。这就是著名的
``开闭'' 原则:

对扩展开放:允许新增\texttt{Animal}子类;

对修改封闭:不需要修改依赖\texttt{Animal}类型的\texttt{run\_twice()}等函数。

继承还可以一级一级地继承下来,就好比从爷爷到爸爸、再到儿子这样的关系。而任何类,最终都可以追溯到根类
object,这些继承关系看上去就像一颗倒着的树。比如如下的继承树:

\begin{pythoncode}
                ┌───────────────┐
                │    object     │
                └───────────────┘
                        │
           ┌────────────┴────────────┐
           │                         │
           ▼                         ▼
    ┌─────────────┐           ┌─────────────┐
    │   Animal    │           │    Plant    │
    └─────────────┘           └─────────────┘
           │                         │
     ┌─────┴──────┐            ┌─────┴──────┐
     │            │            │            │
     ▼            ▼            ▼            ▼
┌─────────┐  ┌─────────┐  ┌─────────┐  ┌─────────┐
│   Dog   │  │   Cat   │  │  Tree   │  │ Flower  │
└─────────┘  └─────────┘  └─────────┘  └─────────┘
\end{pythoncode}

\hypertarget{ux9759ux6001ux8bedux8a00-vs-ux52a8ux6001ux8bedux8a00}{%
\subsubsection{静态语言 vs
动态语言}\label{ux9759ux6001ux8bedux8a00-vs-ux52a8ux6001ux8bedux8a00}}

对于静态语言(例如
Java)来说,如果需要传入\texttt{Animal}类型,则传入的对象必须是\texttt{Animal}类型或者它的子类,否则,将无法调用\texttt{run()}方法。

对于 Python
这样的动态语言来说,则不一定需要传入\texttt{Animal}类型。我们只需要保证传入的对象有一个\texttt{run()}方法就可以了:

\begin{pythoncode}
class Timer(object):
    def run(self):
        print('Start...')
\end{pythoncode}

这就是动态语言的 ``鸭子类型'',它并不要求严格的继承体系,一个对象只要
``看起来像鸭子,走起路来像鸭子'',那它就可以被看做是鸭子。

Python 的 ``file-like
object``就是一种鸭子类型。对真正的文件对象,它有一个\texttt{read()}方法,返回其内容。但是,许多对象,只要有\texttt{read()}方法,都被视为
``file-like object``。许多函数接收的参数就是 ``file-like
object``,你不一定要传入真正的文件对象,完全可以传入任何实现了\texttt{read()}方法的对象。

\hypertarget{ux5c0fux7ed3}{%
\subsubsection{小结}\label{ux5c0fux7ed3}}

继承可以把父类的所有功能都直接拿过来,这样就不必重零做起,子类只需要新增自己特有的方法,也可以把父类不适合的方法覆盖重写。

动态语言的鸭子类型特点决定了继承不像静态语言那样是必须的。

\hypertarget{ux53c2ux8003ux6e90ux7801}{%
\subsubsection{参考源码}\label{ux53c2ux8003ux6e90ux7801}}

\href{https://github.com/michaelliao/learn-python3/blob/master/samples/oop_basic/animals.py}{animals.py}

