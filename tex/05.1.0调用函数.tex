\hypertarget{ux8c03ux7528ux51fdux6570}{%
\subsection{调用函数}\label{ux8c03ux7528ux51fdux6570}}

Python 内置了很多有用的函数,我们可以直接调用。

要调用一个函数,需要知道函数的名称和参数,比如求绝对值的函数\texttt{abs},只有一个参数。可以直接从
Python 的官方网站查看文档:

\url{http://docs.python.org/3/library/functions.html\#abs}

也可以在交互式命令行通过\texttt{help(abs)}查看\texttt{abs}函数的帮助信息。

调用\texttt{abs}函数:

\begin{pythoncode}
>>> abs(100)
100
>>> abs(-20)
20
>>> abs(12.34)
12.34
\end{pythoncode}

调用函数的时候,如果传入的参数数量不对,会报\texttt{TypeError}的错误,并且
Python 会明确地告诉你:\texttt{abs()}有且仅有 1 个参数,但给出了两个:

\begin{pythoncode}
>>> abs(1, 2)
Traceback (most recent call last):
  File "<stdin>", line 1, in <module>
TypeError: abs() takes exactly one argument (2 given)
\end{pythoncode}

如果传入的参数数量是对的,但参数类型不能被函数所接受,也会报\texttt{TypeError}的错误,并且给出错误信息:\texttt{str}是错误的参数类型:

\begin{pythoncode}
>>> abs('a')
Traceback (most recent call last):
  File "<stdin>", line 1, in <module>
TypeError: bad operand type for abs(): 'str'
\end{pythoncode}

而\texttt{max}函数\texttt{max()}可以接收任意多个参数,并返回最大的那个:

\begin{pythoncode}
>>> max(1, 2)
2
>>> max(2, 3, 1, -5)
3
\end{pythoncode}

\hypertarget{ux6570ux636eux7c7bux578bux8f6cux6362}{%
\subsubsection{数据类型转换}\label{ux6570ux636eux7c7bux578bux8f6cux6362}}

Python
内置的常用函数还包括数据类型转换函数,比如\texttt{int()}函数可以把其他数据类型转换为整数:

\begin{pythoncode}
>>> int('123')
123
>>> int(12.34)
12
>>> float('12.34')
12.34
>>> str(1.23)
'1.23'
>>> str(100)
'100'
>>> bool(1)
True
>>> bool('')
False
\end{pythoncode}

函数名其实就是指向一个函数对象的引用,完全可以把函数名赋给一个变量,相当于给这个函数起了一个
``别名'':

\begin{pythoncode}
>>> a = abs 
>>> a(-1) 
1
\end{pythoncode}

\hypertarget{ux7ec3ux4e60}{%
\subsubsection{练习}\label{ux7ec3ux4e60}}

请利用 Python
内置的\texttt{hex()}函数把一个整数转换成十六进制表示的字符串:

\begin{pythoncode}
# -*- coding: utf-8 -*-

n1 = 255
n2 = 1000
\end{pythoncode}

\hypertarget{ux5c0fux7ed3}{%
\subsubsection{小结}\label{ux5c0fux7ed3}}

调用 Python
的函数,需要根据函数定义,传入正确的参数。如果函数调用出错,一定要学会看错误信息,所以英文很重要!

\hypertarget{ux53c2ux8003ux6e90ux7801}{%
\subsubsection{参考源码}\label{ux53c2ux8003ux6e90ux7801}}

\href{https://github.com/michaelliao/learn-python3/blob/master/samples/function/call_func.py}{call\_func.py}

