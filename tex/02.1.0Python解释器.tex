\hypertarget{python-ux89e3ux91caux5668}{%
\subsection{Python 解释器}\label{python-ux89e3ux91caux5668}}

当我们编写 Python 代码时,我们得到的是一个包含 Python
代码的以\texttt{.py}为扩展名的文本文件。要运行代码,就需要 Python
解释器去执行\texttt{.py}文件。

由于整个 Python
语言从规范到解释器都是开源的,所以理论上,只要水平够高,任何人都可以编写
Python 解释器来执行 Python 代码(当然难度很大)。事实上,确实存在多种
Python 解释器。

\hypertarget{cpython}{%
\subsubsection{CPython}\label{cpython}}

当我们从 \href{https://www.python.org/}{Python 官方网站}下载并安装好
Python 3.x
后,我们就直接获得了一个官方版本的解释器:CPython。这个解释器是用 C
语言开发的,所以叫 CPython。在命令行下运行\texttt{python}就是启动
CPython 解释器。

CPython 是使用最广的 Python 解释器。教程的所有代码也都在 CPython
下执行。

\hypertarget{ipython}{%
\subsubsection{IPython}\label{ipython}}

IPython 是基于 CPython 之上的一个交互式解释器,也就是说,IPython
只是在交互方式上有所增强,但是执行 Python 代码的功能和 CPython
是完全一样的。好比很多国产浏览器虽然外观不同,但内核其实都是调用了 IE。

CPython
用\texttt{\textgreater{}\textgreater{}\textgreater{}}作为提示符,而
IPython 用\texttt{In\ {[}序号{]}:}作为提示符。

\hypertarget{pypy}{%
\subsubsection{PyPy}\label{pypy}}

PyPy 是另一个 Python 解释器,它的目标是执行速度。PyPy 采用
\href{http://en.wikipedia.org/wiki/Just-in-time_compilation}{JIT
技术},对 Python 代码进行动态编译(注意不是解释),所以可以显著提高
Python 代码的执行速度。

绝大部分 Python 代码都可以在 PyPy 下运行,但是 PyPy 和 CPython
有一些是不同的,这就导致相同的 Python
代码在两种解释器下执行可能会有不同的结果。如果你的代码要放到 PyPy
下执行,就需要了解
\href{http://pypy.readthedocs.org/en/latest/cpython_differences.html}{PyPy
和 CPython 的不同点}。

\hypertarget{jython}{%
\subsubsection{Jython}\label{jython}}

Jython 是运行在 Java 平台上的 Python 解释器,可以直接把 Python
代码编译成 Java 字节码执行。

\hypertarget{ironpython}{%
\subsubsection{IronPython}\label{ironpython}}

IronPython 和 Jython 类似,只不过 IronPython 是运行在微软. Net 平台上的
Python 解释器,可以直接把 Python 代码编译成. Net 的字节码。

\hypertarget{ux5c0fux7ed3}{%
\subsubsection{小结}\label{ux5c0fux7ed3}}

Python 的解释器很多,但使用最广泛的还是 CPython。如果要和 Java 或. Net
平台交互,最好的办法不是用 Jython 或
IronPython,而是通过网络调用来交互,确保各程序之间的独立性。

本教程的所有代码只确保在 CPython 3.x 版本下运行。请务必在本地安装
CPython(也就是从 Python 官方网站下载的安装程序)。

