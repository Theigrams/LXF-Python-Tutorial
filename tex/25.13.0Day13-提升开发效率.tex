\hypertarget{day-13---ux63d0ux5347ux5f00ux53d1ux6548ux7387}{%
\subsection{Day 13 -
提升开发效率}\label{day-13---ux63d0ux5347ux5f00ux53d1ux6548ux7387}}

现在,我们已经把一个 Web App 的框架完全搭建好了,从后端的 API 到前端的
MVVM,流程已经跑通了。

在继续工作前,注意到每次修改 Python 代码,都必须在命令行先 Ctrl-C
停止服务器,再重启,改动才能生效。

在开发阶段,每天都要修改、保存几十次代码,每次保存都手动来这么一下非常麻烦,严重地降低了我们的开发效率。有没有办法让服务器检测到代码修改后自动重新加载呢?

Django 的开发环境在 Debug
模式下就可以做到自动重新加载,如果我们编写的服务器也能实现这个功能,就能大大提升开发效率。

可惜的是,Django 没把这个功能独立出来,不用 Django 就享受不到,怎么办?

其实 Python
本身提供了重新载入模块的功能,但不是所有模块都能被重新载入。另一种思路是检测\texttt{www}目录下的代码改动,一旦有改动,就自动重启服务器。

按照这个思路,我们可以编写一个辅助程序\texttt{pymonitor.py},让它启动\texttt{wsgiapp.py},并时刻监控\texttt{www}目录下的代码改动,有改动时,先把当前\texttt{wsgiapp.py}进程杀掉,再重启,就完成了服务器进程的自动重启。

要监控目录文件的变化,我们也无需自己手动定时扫描,Python
的第三方库\texttt{watchdog}可以利用操作系统的 API
来监控目录文件的变化,并发送通知。我们先用\texttt{pip}安装:

\begin{pythoncode}
$ pip3 install watchdog
\end{pythoncode}

利用\texttt{watchdog}接收文件变化的通知,如果是\texttt{.py}文件,就自动重启\texttt{wsgiapp.py}进程。

利用 Python
自带的\texttt{subprocess}实现进程的启动和终止,并把输入输出重定向到当前进程的输入输出中:

\begin{pythoncode}
__author__ = 'Michael Liao'

import os, sys, time, subprocess

from watchdog.observers import Observer
from watchdog.events import FileSystemEventHandler

def log(s):
    print('[Monitor] %s' % s)

class MyFileSystemEventHander(FileSystemEventHandler):

    def __init__(self, fn):
        super(MyFileSystemEventHander, self).__init__()
        self.restart = fn

    def on_any_event(self, event):
        if event.src_path.endswith('.py'):
            log('Python source file changed: %s' % event.src_path)
            self.restart()

command = ['echo', 'ok']
process = None

def kill_process():
    global process
    if process:
        log('Kill process [%s]...' % process.pid)
        process.kill()
        process.wait()
        log('Process ended with code %s.' % process.returncode)
        process = None

def start_process():
    global process, command
    log('Start process %s...' % ' '.join(command))
    process = subprocess.Popen(command, stdin=sys.stdin, stdout=sys.stdout, stderr=sys.stderr)

def restart_process():
    kill_process()
    start_process()

def start_watch(path, callback):
    observer = Observer()
    observer.schedule(MyFileSystemEventHander(restart_process), path, recursive=True)
    observer.start()
    log('Watching directory %s...' % path)
    start_process()
    try:
        while True:
            time.sleep(0.5)
    except KeyboardInterrupt:
        observer.stop()
    observer.join()

if __name__ == '__main__':
    argv = sys.argv[1:]
    if not argv:
        print('Usage: ./pymonitor your-script.py')
        exit(0)
    if argv[0] != 'python3':
        argv.insert(0, 'python3')
    command = argv
    path = os.path.abspath('.')
    start_watch(path, None)
\end{pythoncode}

一共 70 行左右的代码,就实现了 Debug
模式的自动重新加载。用下面的命令启动服务器:

\begin{pythoncode}
$ python3 pymonitor.py wsgiapp.py
\end{pythoncode}

或者给\texttt{pymonitor.py}加上可执行权限,启动服务器:

\begin{pythoncode}
$ ./pymonitor.py app.py
\end{pythoncode}

在编辑器中打开一个\texttt{.py}文件,修改后保存,看看命令行输出,是不是自动重启了服务器:

\begin{pythoncode}
$ ./pymonitor.py app.py 
[Monitor] Watching directory /Users/michael/Github/awesome-python3-webapp/www...
[Monitor] Start process python app.py...
...
INFO:root:application (/Users/michael/Github/awesome-python3-webapp/www) will start at 0.0.0.0:9000...
[Monitor] Python source file changed: /Users/michael/Github/awesome-python-webapp/www/handlers.py
[Monitor] Kill process [2747]...
[Monitor] Process ended with code -9.
[Monitor] Start process python app.py...
...
INFO:root:application (/Users/michael/Github/awesome-python3-webapp/www) will start at 0.0.0.0:9000...
\end{pythoncode}

现在,只要一保存代码,就可以刷新浏览器看到效果,大大提升了开发效率。

