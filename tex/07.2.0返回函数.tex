\hypertarget{ux8fd4ux56deux51fdux6570}{%
\subsection{返回函数}\label{ux8fd4ux56deux51fdux6570}}

\hypertarget{ux51fdux6570ux4f5cux4e3aux8fd4ux56deux503c}{%
\subsubsection{函数作为返回值}\label{ux51fdux6570ux4f5cux4e3aux8fd4ux56deux503c}}

高阶函数除了可以接受函数作为参数外,还可以把函数作为结果值返回。

我们来实现一个可变参数的求和。通常情况下,求和的函数是这样定义的:

\begin{pythoncode}
def calc_sum(*args):
    ax = 0
    for n in args:
        ax = ax + n
    return ax
\end{pythoncode}

但是,如果不需要立刻求和,而是在后面的代码中,根据需要再计算怎么办?可以不返回求和的结果,而是返回求和的函数:

\begin{pythoncode}
def lazy_sum(*args):
    def sum():
        ax = 0
        for n in args:
            ax = ax + n
        return ax
    return sum
\end{pythoncode}

当我们调用\texttt{lazy\_sum()}时,返回的并不是求和结果,而是求和函数:

\begin{pythoncode}
>>> f = lazy_sum(1, 3, 5, 7, 9)
>>> f
<function lazy_sum.<locals>.sum at 0x101c6ed90>
\end{pythoncode}

调用函数\texttt{f}时,才真正计算求和的结果:

\begin{pythoncode}
>>> f()
25
\end{pythoncode}

在这个例子中,我们在函数\texttt{lazy\_sum}中又定义了函数\texttt{sum},并且,内部函数\texttt{sum}可以引用外部函数\texttt{lazy\_sum}的参数和局部变量,当\texttt{lazy\_sum}返回函数\texttt{sum}时,相关参数和变量都保存在返回的函数中,这种称为
``闭包(Closure)'' 的程序结构拥有极大的威力。

请再注意一点,当我们调用\texttt{lazy\_sum()}时,每次调用都会返回一个新的函数,即使传入相同的参数:

\begin{pythoncode}
>>> f1 = lazy_sum(1, 3, 5, 7, 9)
>>> f2 = lazy_sum(1, 3, 5, 7, 9)
>>> f1==f2
False
\end{pythoncode}

\texttt{f1()}和\texttt{f2()}的调用结果互不影响。

\hypertarget{ux95edux5305}{%
\subsubsection{闭包}\label{ux95edux5305}}

注意到返回的函数在其定义内部引用了局部变量\texttt{args},所以,当一个函数返回了一个函数后,其内部的局部变量还被新函数引用,所以,闭包用起来简单,实现起来可不容易。

另一个需要注意的问题是,返回的函数并没有立刻执行,而是直到调用了\texttt{f()}才执行。我们来看一个例子:

\begin{pythoncode}
def count():
    fs = []
    for i in range(1, 4):
        def f():
             return i*i
        fs.append(f)
    return fs

f1, f2, f3 = count()
\end{pythoncode}

在上面的例子中,每次循环,都创建了一个新的函数,然后,把创建的 3
个函数都返回了。

你可能认为调用\texttt{f1()},\texttt{f2()}和\texttt{f3()}结果应该是\texttt{1},\texttt{4},\texttt{9},但实际结果是:

\begin{pythoncode}
>>> f1()
9
>>> f2()
9
>>> f3()
9
\end{pythoncode}

全部都是\texttt{9}!原因就在于返回的函数引用了变量\texttt{i},但它并非立刻执行。等到
3
个函数都返回时,它们所引用的变量\texttt{i}已经变成了\texttt{3},因此最终结果为\texttt{9}。

返回闭包时牢记一点:返回函数不要引用任何循环变量,或者后续会发生变化的变量。

如果一定要引用循环变量怎么办?方法是再创建一个函数,用该函数的参数绑定循环变量当前的值,无论该循环变量后续如何更改,已绑定到函数参数的值不变:

\begin{pythoncode}
def count():
    def f(j):
        def g():
            return j*j
        return g
    fs = []
    for i in range(1, 4):
        fs.append(f(i)) 
    return fs
\end{pythoncode}

再看看结果:

\begin{pythoncode}
>>> f1, f2, f3 = count()
>>> f1()
1
>>> f2()
4
>>> f3()
9
\end{pythoncode}

缺点是代码较长,可利用 lambda 函数缩短代码。

\hypertarget{ux7ec3ux4e60}{%
\subsubsection{练习}\label{ux7ec3ux4e60}}

利用闭包返回一个计数器函数,每次调用它返回递增整数:

\begin{pythoncode}
# -*- coding: utf-8 -*-
\end{pythoncode}

\begin{pythoncode}
# 测试:
counterA = createCounter()
print(counterA(), counterA(), counterA(), counterA(), counterA()) # 1 2 3 4 5
counterB = createCounter()
if [counterB(), counterB(), counterB(), counterB()] == [1, 2, 3, 4]:
    print('测试通过!')
else:
    print('测试失败!')
\end{pythoncode}

\hypertarget{ux5c0fux7ed3}{%
\subsubsection{小结}\label{ux5c0fux7ed3}}

一个函数可以返回一个计算结果,也可以返回一个函数。

返回一个函数时,牢记该函数并未执行,返回函数中不要引用任何可能会变化的变量。

\hypertarget{ux53c2ux8003ux6e90ux7801}{%
\subsubsection{参考源码}\label{ux53c2ux8003ux6e90ux7801}}

\href{https://github.com/michaelliao/learn-python3/blob/master/samples/functional/return_func.py}{return\_func.py}

