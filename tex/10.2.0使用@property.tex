\hypertarget{ux4f7fux7528-property}{%
\subsection{使用 @property}\label{ux4f7fux7528-property}}

在绑定属性时,如果我们直接把属性暴露出去,虽然写起来很简单,但是,没办法检查参数,导致可以把成绩随便改:

\begin{pythoncode}
s = Student()
s.score = 9999
\end{pythoncode}

这显然不合逻辑。为了限制 score
的范围,可以通过一个\texttt{set\_score()}方法来设置成绩,再通过一个\texttt{get\_score()}来获取成绩,这样,在\texttt{set\_score()}方法里,就可以检查参数:

\begin{pythoncode}
class Student(object):

    def get_score(self):
         return self._score

    def set_score(self, value):
        if not isinstance(value, int):
            raise ValueError('score must be an integer!')
        if value < 0 or value > 100:
            raise ValueError('score must between 0 ~ 100!')
        self._score = value
\end{pythoncode}

现在,对任意的 Student 实例进行操作,就不能随心所欲地设置 score 了:

\begin{pythoncode}
>>> s = Student()
>>> s.set_score(60) 
>>> s.get_score()
60
>>> s.set_score(9999)
Traceback (most recent call last):
  ...
ValueError: score must between 0 ~ 100!
\end{pythoncode}

但是,上面的调用方法又略显复杂,没有直接用属性这么直接简单。

有没有既能检查参数,又可以用类似属性这样简单的方式来访问类的变量呢?对于追求完美的
Python 程序员来说,这是必须要做到的!

还记得装饰器(decorator)可以给函数动态加上功能吗?对于类的方法,装饰器一样起作用。Python
内置的\texttt{@property}装饰器就是负责把一个方法变成属性调用的:

\begin{pythoncode}
class Student(object):

    @property
    def score(self):
        return self._score

    @score.setter
    def score(self, value):
        if not isinstance(value, int):
            raise ValueError('score must be an integer!')
        if value < 0 or value > 100:
            raise ValueError('score must between 0 ~ 100!')
        self._score = value
\end{pythoncode}

\texttt{@property}的实现比较复杂,我们先考察如何使用。把一个 getter
方法变成属性,只需要加上\texttt{@property}就可以了,此时,\texttt{@property}本身又创建了另一个装饰器\texttt{@score.setter},负责把一个
setter 方法变成属性赋值,于是,我们就拥有一个可控的属性操作:

\begin{pythoncode}
>>> s = Student()
>>> s.score = 60 
>>> s.score 
60
>>> s.score = 9999
Traceback (most recent call last):
  ...
ValueError: score must between 0 ~ 100!
\end{pythoncode}

注意到这个神奇的\texttt{@property},我们在对实例属性操作的时候,就知道该属性很可能不是直接暴露的,而是通过
getter 和 setter 方法来实现的。

还可以定义只读属性,只定义 getter 方法,不定义 setter
方法就是一个只读属性:

\begin{pythoncode}
class Student(object):

    @property
    def birth(self):
        return self._birth

    @birth.setter
    def birth(self, value):
        self._birth = value

    @property
    def age(self):
        return 2015 - self._birth
\end{pythoncode}

上面的\texttt{birth}是可读写属性,而\texttt{age}就是一个\_只读\_属性,因为\texttt{age}可以根据\texttt{birth}和当前时间计算出来。

\hypertarget{ux5c0fux7ed3}{%
\subsubsection{小结}\label{ux5c0fux7ed3}}

\texttt{@property}广泛应用在类的定义中,可以让调用者写出简短的代码,同时保证对参数进行必要的检查,这样,程序运行时就减少了出错的可能性。

\hypertarget{ux7ec3ux4e60}{%
\subsubsection{练习}\label{ux7ec3ux4e60}}

请利用\texttt{@property}给一个\texttt{Screen}对象加上\texttt{width}和\texttt{height}属性,以及一个只读属性\texttt{resolution}:

\begin{pythoncode}
# -*- coding: utf-8 -*-
\end{pythoncode}

\begin{pythoncode}
# 测试:
s = Screen()
s.width = 1024
s.height = 768
print('resolution =', s.resolution)
if s.resolution == 786432:
    print('测试通过!')
else:
    print('测试失败!')
\end{pythoncode}

\hypertarget{ux53c2ux8003ux6e90ux7801}{%
\subsubsection{参考源码}\label{ux53c2ux8003ux6e90ux7801}}

\href{https://github.com/michaelliao/learn-python3/blob/master/samples/oop_advance/use_property.py}{use\_property.py}

