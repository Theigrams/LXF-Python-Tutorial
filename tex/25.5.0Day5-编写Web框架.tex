\hypertarget{day-5---ux7f16ux5199-web-ux6846ux67b6}{%
\subsection{Day 5 - 编写 Web
框架}\label{day-5---ux7f16ux5199-web-ux6846ux67b6}}

在正式开始 Web 开发前,我们需要编写一个 Web 框架。

\texttt{aiohttp}已经是一个 Web 框架了,为什么我们还需要自己封装一个?

原因是从使用者的角度来说,\texttt{aiohttp}相对比较底层,编写一个 URL
的处理函数需要这么几步:

第一步,编写一个用\texttt{@asyncio.coroutine}装饰的函数:

\begin{pythoncode}
@asyncio.coroutine
def handle_url_xxx(request):
    pass
\end{pythoncode}

第二步,传入的参数需要自己从\texttt{request}中获取:

\begin{pythoncode}
url_param = request.match_info['key']
query_params = parse_qs(request.query_string)
\end{pythoncode}

最后,需要自己构造\texttt{Response}对象:

\begin{pythoncode}
text = render('template', data)
return web.Response(text.encode('utf-8'))
\end{pythoncode}

这些重复的工作可以由框架完成。例如,处理带参数的
URL\texttt{/blog/\{id\}}可以这么写:

\begin{pythoncode}
@get('/blog/{id}')
def get_blog(id):
    pass
\end{pythoncode}

处理\texttt{query\_string}参数可以通过关键字参数\texttt{**kw}或者命名关键字参数接收:

\begin{pythoncode}
@get('/api/comments')
def api_comments(*, page='1'):
    pass
\end{pythoncode}

对于函数的返回值,不一定是\texttt{web.Response}对象,可以是\texttt{str}、\texttt{bytes}或\texttt{dict}。

如果希望渲染模板,我们可以这么返回一个\texttt{dict}:

\begin{pythoncode}
return {
    '__template__': 'index.html',
    'data': '...'
}
\end{pythoncode}

因此,Web
框架的设计是完全从使用者出发,目的是让使用者编写尽可能少的代码。

编写简单的函数而非引入\texttt{request}和\texttt{web.Response}还有一个额外的好处,就是可以单独测试,否则,需要模拟一个\texttt{request}才能测试。

\hypertarget{get-ux548c-post}{%
\subsubsection{@get 和 @post}\label{get-ux548c-post}}

要把一个函数映射为一个 URL 处理函数,我们先定义\texttt{@get()}:

\begin{pythoncode}
def get(path):
    '''
    Define decorator @get('/path')
    '''
    def decorator(func):
        @functools.wraps(func)
        def wrapper(*args, **kw):
            return func(*args, **kw)
        wrapper.__method__ = 'GET'
        wrapper.__route__ = path
        return wrapper
    return decorator
\end{pythoncode}

这样,一个函数通过\texttt{@get()}的装饰就附带了 URL 信息。

\texttt{@post}与\texttt{@get}定义类似。

\hypertarget{ux5b9aux4e49-requesthandler}{%
\subsubsection{定义 RequestHandler}\label{ux5b9aux4e49-requesthandler}}

URL
处理函数不一定是一个\texttt{coroutine},因此我们用\texttt{RequestHandler()}来封装一个
URL 处理函数。

\texttt{RequestHandler}是一个类,由于定义了\texttt{\_\_call\_\_()}方法,因此可以将其实例视为函数。

\texttt{RequestHandler}目的就是从 URL
函数中分析其需要接收的参数,从\texttt{request}中获取必要的参数,调用 URL
函数,然后把结果转换为\texttt{web.Response}对象,这样,就完全符合\texttt{aiohttp}框架的要求:

\begin{pythoncode}
class RequestHandler(object):

    def __init__(self, app, fn):
        self._app = app
        self._func = fn
        ...

    @asyncio.coroutine
    def __call__(self, request):
        kw = ... 获取参数
        r = yield from self._func(**kw)
        return r
\end{pythoncode}

再编写一个\texttt{add\_route}函数,用来注册一个 URL 处理函数:

\begin{pythoncode}
def add_route(app, fn):
    method = getattr(fn, '__method__', None)
    path = getattr(fn, '__route__', None)
    if path is None or method is None:
        raise ValueError('@get or @post not defined in %s.' % str(fn))
    if not asyncio.iscoroutinefunction(fn) and not inspect.isgeneratorfunction(fn):
        fn = asyncio.coroutine(fn)
    logging.info('add route %s %s => %s(%s)' % (method, path, fn.__name__, ', '.join(inspect.signature(fn).parameters.keys())))
    app.router.add_route(method, path, RequestHandler(app, fn))
\end{pythoncode}

最后一步,把很多次\texttt{add\_route()}注册的调用:

\begin{pythoncode}
add_route(app, handles.index)
add_route(app, handles.blog)
add_route(app, handles.create_comment)
...
\end{pythoncode}

变成自动扫描:

\begin{pythoncode}
# 自动把handler模块的所有符合条件的函数注册了:
add_routes(app, 'handlers')
\end{pythoncode}

\texttt{add\_routes()}定义如下:

\begin{pythoncode}
def add_routes(app, module_name):
    n = module_name.rfind('.')
    if n == (-1):
        mod = __import__(module_name, globals(), locals())
    else:
        name = module_name[n+1:]
        mod = getattr(__import__(module_name[:n], globals(), locals(), [name]), name)
    for attr in dir(mod):
        if attr.startswith('_'):
            continue
        fn = getattr(mod, attr)
        if callable(fn):
            method = getattr(fn, '__method__', None)
            path = getattr(fn, '__route__', None)
            if method and path:
                add_route(app, fn)
\end{pythoncode}

最后,在\texttt{app.py}中加入\texttt{middleware}、\texttt{jinja2}模板和自注册的支持:

\begin{pythoncode}
app = web.Application(loop=loop, middlewares=[
    logger_factory, response_factory
])
init_jinja2(app, filters=dict(datetime=datetime_filter))
add_routes(app, 'handlers')
add_static(app)
\end{pythoncode}

\hypertarget{middleware}{%
\subsubsection{middleware}\label{middleware}}

\texttt{middleware}是一种拦截器,一个 URL
在被某个函数处理前,可以经过一系列的\texttt{middleware}的处理。

一个\texttt{middleware}可以改变 URL
的输入、输出,甚至可以决定不继续处理而直接返回。middleware
的用处就在于把通用的功能从每个 URL
处理函数中拿出来,集中放到一个地方。例如,一个记录 URL
日志的\texttt{logger}可以简单定义如下:

\begin{pythoncode}
@asyncio.coroutine
def logger_factory(app, handler):
    @asyncio.coroutine
    def logger(request):
        
        logging.info('Request: %s %s' % (request.method, request.path))
        
        return (yield from handler(request))
    return logger
\end{pythoncode}

而\texttt{response}这个\texttt{middleware}把返回值转换为\texttt{web.Response}对象再返回,以保证满足\texttt{aiohttp}的要求:

\begin{pythoncode}
@asyncio.coroutine
def response_factory(app, handler):
    @asyncio.coroutine
    def response(request):
        
        r = yield from handler(request)
        if isinstance(r, web.StreamResponse):
            return r
        if isinstance(r, bytes):
            resp = web.Response(body=r)
            resp.content_type = 'application/octet-stream'
            return resp
        if isinstance(r, str):
            resp = web.Response(body=r.encode('utf-8'))
            resp.content_type = 'text/html;charset=utf-8'
            return resp
        if isinstance(r, dict):
            ...
\end{pythoncode}

有了这些基础设施,我们就可以专注地往\texttt{handlers}模块不断添加 URL
处理函数了,可以极大地提高开发效率。

\hypertarget{ux53c2ux8003ux6e90ux7801}{%
\subsubsection{参考源码}\label{ux53c2ux8003ux6e90ux7801}}

\href{https://github.com/michaelliao/awesome-python3-webapp/tree/day-05}{day-05}

