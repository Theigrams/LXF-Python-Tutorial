\hypertarget{filter}{%
\subsection{filter}\label{filter}}

Python 内建的\texttt{filter()}函数用于过滤序列。

和\texttt{map()}类似,\texttt{filter()}也接收一个函数和一个序列。和\texttt{map()}不同的是,\texttt{filter()}把传入的函数依次作用于每个元素,然后根据返回值是\texttt{True}还是\texttt{False}决定保留还是丢弃该元素。

例如,在一个 list 中,删掉偶数,只保留奇数,可以这么写:

\begin{pythoncode}
def is_odd(n):
    return n % 2 == 1

list(filter(is_odd, [1, 2, 4, 5, 6, 9, 10, 15]))
\end{pythoncode}

把一个序列中的空字符串删掉,可以这么写:

\begin{pythoncode}
def not_empty(s):
    return s and s.strip()

list(filter(not_empty, ['A', '', 'B', None, 'C', '  ']))
\end{pythoncode}

可见用\texttt{filter()}这个高阶函数,关键在于正确实现一个 ``筛选''
函数。

注意到\texttt{filter()}函数返回的是一个\texttt{Iterator},也就是一个惰性序列,所以要强迫\texttt{filter()}完成计算结果,需要用\texttt{list()}函数获得所有结果并返回
list。

\hypertarget{ux7528-filter-ux6c42ux7d20ux6570}{%
\subsubsection{用 filter
求素数}\label{ux7528-filter-ux6c42ux7d20ux6570}}

计算\href{http://baike.baidu.com/view/10626.htm}{素数}的一个方法是\href{http://baike.baidu.com/view/3784258.htm}{埃氏筛法},它的算法理解起来非常简单:

首先,列出从\texttt{2}开始的所有自然数,构造一个序列:

2, 3, 4, 5, 6, 7, 8, 9, 10, 11, 12, 13, 14, 15, 16, 17, 18, 19, 20,
\ldots{}

取序列的第一个数\texttt{2},它一定是素数,然后用\texttt{2}把序列的\texttt{2}的倍数筛掉:

3, \textsubscript{4}, 5, \textsubscript{6}, 7, \textsubscript{8}, 9,
\textsubscript{10}, 11, \textsubscript{12}, 13, \textsubscript{14}, 15,
\textsubscript{16}, 17, \textsubscript{18}, 19, \textsubscript{20},
\ldots{}

取新序列的第一个数\texttt{3},它一定是素数,然后用\texttt{3}把序列的\texttt{3}的倍数筛掉:

5, \textsubscript{6}, 7, \textsubscript{8}, \textsubscript{9},
\textsubscript{10}, 11, \textsubscript{12}, 13, \textsubscript{14},
\textsubscript{15}, \textsubscript{16}, 17, \textsubscript{18}, 19,
\textsubscript{20}, \ldots{}

取新序列的第一个数\texttt{5},然后用\texttt{5}把序列的\texttt{5}的倍数筛掉:

7, \textsubscript{8}, \textsubscript{9}, \textsubscript{10}, 11,
\textsubscript{12}, 13, \textsubscript{14}, \textsubscript{15},
\textsubscript{16}, 17, \textsubscript{18}, 19, \textsubscript{20},
\ldots{}

不断筛下去,就可以得到所有的素数。

用 Python 来实现这个算法,可以先构造一个从\texttt{3}开始的奇数序列:

\begin{pythoncode}
def _odd_iter():
    n = 1
    while True:
        n = n + 2
        yield n
\end{pythoncode}

注意这是一个生成器,并且是一个无限序列。

然后定义一个筛选函数:

\begin{pythoncode}
def _not_divisible(n):
    return lambda x: x % n > 0
\end{pythoncode}

最后,定义一个生成器,不断返回下一个素数:

\begin{pythoncode}
def primes():
    yield 2
    it = _odd_iter() 
    while True:
        n = next(it) 
        yield n
        it = filter(_not_divisible(n), it) 
\end{pythoncode}

这个生成器先返回第一个素数\texttt{2},然后,利用\texttt{filter()}不断产生筛选后的新的序列。

由于\texttt{primes()}也是一个无限序列,所以调用时需要设置一个退出循环的条件:

\begin{pythoncode}
for n in primes():
    if n < 1000:
        print(n)
    else:
        break
\end{pythoncode}

注意到\texttt{Iterator}是惰性计算的序列,所以我们可以用 Python 表示
``全体自然数'',``全体素数'' 这样的序列,而代码非常简洁。

\hypertarget{ux7ec3ux4e60}{%
\subsubsection{练习}\label{ux7ec3ux4e60}}

回数是指从左向右读和从右向左读都是一样的数,例如\texttt{12321},\texttt{909}。请利用\texttt{filter()}筛选出回数:

\begin{pythoncode}
# -*- coding: utf-8 -*-
\end{pythoncode}

\begin{pythoncode}
# 测试:
output = filter(is_palindrome, range(1, 1000))
print('1~1000:', list(output))
if list(filter(is_palindrome, range(1, 200))) == [1, 2, 3, 4, 5, 6, 7, 8, 9, 11, 22, 33, 44, 55, 66, 77, 88, 99, 101, 111, 121, 131, 141, 151, 161, 171, 181, 191]:
    print('测试成功!')
else:
    print('测试失败!')
\end{pythoncode}

\hypertarget{ux5c0fux7ed3}{%
\subsubsection{小结}\label{ux5c0fux7ed3}}

\texttt{filter()}的作用是从一个序列中筛出符合条件的元素。由于\texttt{filter()}使用了惰性计算,所以只有在取\texttt{filter()}结果的时候,才会真正筛选并每次返回下一个筛出的元素。

\hypertarget{ux53c2ux8003ux6e90ux7801}{%
\subsubsection{参考源码}\label{ux53c2ux8003ux6e90ux7801}}

\href{https://github.com/michaelliao/learn-python3/blob/master/samples/functional/do_filter.py}{do\_filter.py}

\href{https://github.com/michaelliao/learn-python3/blob/master/samples/functional/prime_numbers.py}{prime\_numbers.py}

