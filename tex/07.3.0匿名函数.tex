\hypertarget{ux533fux540dux51fdux6570}{%
\subsection{匿名函数}\label{ux533fux540dux51fdux6570}}

当我们在传入函数时,有些时候,不需要显式地定义函数,直接传入匿名函数更方便。

在 Python
中,对匿名函数提供了有限支持。还是以\texttt{map()}函数为例,计算 f(x)=x2
时,除了定义一个\texttt{f(x)}的函数外,还可以直接传入匿名函数:

\begin{pythoncode}
>>> list(map(lambda x: x * x, [1, 2, 3, 4, 5, 6, 7, 8, 9]))
[1, 4, 9, 16, 25, 36, 49, 64, 81]
\end{pythoncode}

通过对比可以看出,匿名函数\texttt{lambda\ x:\ x\ *\ x}实际上就是:

\begin{pythoncode}
def f(x):
    return x * x
\end{pythoncode}

关键字\texttt{lambda}表示匿名函数,冒号前面的\texttt{x}表示函数参数。

匿名函数有个限制,就是只能有一个表达式,不用写\texttt{return},返回值就是该表达式的结果。

用匿名函数有个好处,因为函数没有名字,不必担心函数名冲突。此外,匿名函数也是一个函数对象,也可以把匿名函数赋值给一个变量,再利用变量来调用该函数:

\begin{pythoncode}
>>> f = lambda x: x * x
>>> f
<function <lambda> at 0x101c6ef28>
>>> f(5)
25
\end{pythoncode}

同样,也可以把匿名函数作为返回值返回,比如:

\begin{pythoncode}
def build(x, y):
    return lambda: x * x + y * y
\end{pythoncode}

\hypertarget{ux7ec3ux4e60}{%
\subsubsection{练习}\label{ux7ec3ux4e60}}

请用匿名函数改造下面的代码:

\begin{pythoncode}
# -*- coding: utf-8 -*-
\end{pythoncode}

\begin{pythoncode}
print(L)
\end{pythoncode}

\hypertarget{ux5c0fux7ed3}{%
\subsubsection{小结}\label{ux5c0fux7ed3}}

Python 对匿名函数的支持有限,只有一些简单的情况下可以使用匿名函数。

