\hypertarget{udp-ux7f16ux7a0b}{%
\subsection{UDP 编程}\label{udp-ux7f16ux7a0b}}

TCP 是建立可靠连接,并且通信双方都可以以流的形式发送数据。相对 TCP,UDP
则是面向无连接的协议。

使用 UDP 协议时,不需要建立连接,只需要知道对方的 IP
地址和端口号,就可以直接发数据包。但是,能不能到达就不知道了。

虽然用 UDP 传输数据不可靠,但它的优点是和 TCP
比,速度快,对于不要求可靠到达的数据,就可以使用 UDP 协议。

我们来看看如何通过 UDP 协议传输数据。和 TCP 类似,使用 UDP
的通信双方也分为客户端和服务器。服务器首先需要绑定端口:

\begin{pythoncode}
s = socket.socket(socket.AF_INET, socket.SOCK_DGRAM)

s.bind(('127.0.0.1', 9999))
\end{pythoncode}

创建 Socket 时,\texttt{SOCK\_DGRAM}指定了这个 Socket 的类型是
UDP。绑定端口和 TCP
一样,但是不需要调用\texttt{listen()}方法,而是直接接收来自任何客户端的数据:

\begin{pythoncode}
print('Bind UDP on 9999...')
while True:
    
    data, addr = s.recvfrom(1024)
    print('Received from %s:%s.' % addr)
    s.sendto(b'Hello, %s!' % data, addr)
\end{pythoncode}

\texttt{recvfrom()}方法返回数据和客户端的地址与端口,这样,服务器收到数据后,直接调用\texttt{sendto()}就可以把数据用
UDP 发给客户端。

注意这里省掉了多线程,因为这个例子很简单。

客户端使用 UDP 时,首先仍然创建基于 UDP 的
Socket,然后,不需要调用\texttt{connect()},直接通过\texttt{sendto()}给服务器发数据:

\begin{pythoncode}
s = socket.socket(socket.AF_INET, socket.SOCK_DGRAM)
for data in [b'Michael', b'Tracy', b'Sarah']:
    
    s.sendto(data, ('127.0.0.1', 9999))
    
    print(s.recv(1024).decode('utf-8'))
s.close()
\end{pythoncode}

从服务器接收数据仍然调用\texttt{recv()}方法。

仍然用两个命令行分别启动服务器和客户端测试,结果如下:

\begin{pythoncode}
┌────────────────────────────────────────────────────────┐
│Command Prompt                                    - □ x │
├────────────────────────────────────────────────────────┤
│$ python udp_server.py                                  │
│Bind UDP on 9999...                                     │
│Received from 127.0.0.1:63823...                        │
│Received from 127.0.0.1:63823...                        │
│Received from 127.0.0.1:63823...                        │
│       ┌────────────────────────────────────────────────┴───────┐
│       │Command Prompt                                    - □ x │
│       ├────────────────────────────────────────────────────────┤
│       │$ python udp_client.py                                  │
│       │Welcome!                                                │
│       │Hello, Michael!                                         │
└───────┤Hello, Tracy!                                           │
        │Hello, Sarah!                                           │
        │$                                                       │
        │                                                        │
        │                                                        │
        └────────────────────────────────────────────────────────┘
\end{pythoncode}

\hypertarget{ux5c0fux7ed3}{%
\subsubsection{小结}\label{ux5c0fux7ed3}}

UDP 的使用与 TCP 类似,但是不需要建立连接。此外,服务器绑定 UDP 端口和
TCP 端口互不冲突,也就是说,UDP 的 9999 端口与 TCP 的 9999
端口可以各自绑定。

\hypertarget{ux53c2ux8003ux6e90ux7801}{%
\subsubsection{参考源码}\label{ux53c2ux8003ux6e90ux7801}}

\href{https://github.com/michaelliao/learn-python3/blob/master/samples/socket/udp_server.py}{udp\_server.py}

\href{https://github.com/michaelliao/learn-python3/blob/master/samples/socket/udp_client.py}{udp\_client.py}

