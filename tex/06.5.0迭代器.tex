\hypertarget{ux8fedux4ee3ux5668}{%
\subsection{迭代器}\label{ux8fedux4ee3ux5668}}

我们已经知道,可以直接作用于\texttt{for}循环的数据类型有以下几种:

一类是集合数据类型,如\texttt{list}、\texttt{tuple}、\texttt{dict}、\texttt{set}、\texttt{str}等;

一类是\texttt{generator},包括生成器和带\texttt{yield}的 generator
function。

这些可以直接作用于\texttt{for}循环的对象统称为可迭代对象:\texttt{Iterable}。

可以使用\texttt{isinstance()}判断一个对象是否是\texttt{Iterable}对象:

\begin{pythoncode}
>>> from collections.abc import Iterable
>>> isinstance([], Iterable)
True
>>> isinstance({}, Iterable)
True
>>> isinstance('abc', Iterable)
True
>>> isinstance((x for x in range(10)), Iterable)
True
>>> isinstance(100, Iterable)
False
\end{pythoncode}

而生成器不但可以作用于\texttt{for}循环,还可以被\texttt{next()}函数不断调用并返回下一个值,直到最后抛出\texttt{StopIteration}错误表示无法继续返回下一个值了。

可以被\texttt{next()}函数调用并不断返回下一个值的对象称为迭代器:\texttt{Iterator}。

可以使用\texttt{isinstance()}判断一个对象是否是\texttt{Iterator}对象:

\begin{pythoncode}
>>> from collections.abc import Iterator
>>> isinstance((x for x in range(10)), Iterator)
True
>>> isinstance([], Iterator)
False
>>> isinstance({}, Iterator)
False
>>> isinstance('abc', Iterator)
False
\end{pythoncode}

生成器都是\texttt{Iterator}对象,但\texttt{list}、\texttt{dict}、\texttt{str}虽然是\texttt{Iterable},却不是\texttt{Iterator}。

把\texttt{list}、\texttt{dict}、\texttt{str}等\texttt{Iterable}变成\texttt{Iterator}可以使用\texttt{iter()}函数:

\begin{pythoncode}
>>> isinstance(iter([]), Iterator)
True
>>> isinstance(iter('abc'), Iterator)
True
\end{pythoncode}

你可能会问,为什么\texttt{list}、\texttt{dict}、\texttt{str}等数据类型不是\texttt{Iterator}?

这是因为 Python 的\texttt{Iterator}对象表示的是一个数据流,Iterator
对象可以被\texttt{next()}函数调用并不断返回下一个数据,直到没有数据时抛出\texttt{StopIteration}错误。可以把这个数据流看做是一个有序序列,但我们却不能提前知道序列的长度,只能不断通过\texttt{next()}函数实现按需计算下一个数据,所以\texttt{Iterator}的计算是惰性的,只有在需要返回下一个数据时它才会计算。

\texttt{Iterator}甚至可以表示一个无限大的数据流,例如全体自然数。而使用
list 是永远不可能存储全体自然数的。

\hypertarget{ux5c0fux7ed3}{%
\subsubsection{小结}\label{ux5c0fux7ed3}}

凡是可作用于\texttt{for}循环的对象都是\texttt{Iterable}类型;

凡是可作用于\texttt{next()}函数的对象都是\texttt{Iterator}类型,它们表示一个惰性计算的序列;

集合数据类型如\texttt{list}、\texttt{dict}、\texttt{str}等是\texttt{Iterable}但不是\texttt{Iterator},不过可以通过\texttt{iter()}函数获得一个\texttt{Iterator}对象。

Python
的\texttt{for}循环本质上就是通过不断调用\texttt{next()}函数实现的,例如:

\begin{pythoncode}
for x in [1, 2, 3, 4, 5]:
    pass
\end{pythoncode}

实际上完全等价于:

\begin{pythoncode}
it = iter([1, 2, 3, 4, 5])

while True:
    try:
        
        x = next(it)
    except StopIteration:
        
        break
\end{pythoncode}

\hypertarget{ux53c2ux8003ux6e90ux7801}{%
\subsubsection{参考源码}\label{ux53c2ux8003ux6e90ux7801}}

\href{https://github.com/michaelliao/learn-python3/blob/master/samples/advance/do_iter.py}{do\_iter.py}

