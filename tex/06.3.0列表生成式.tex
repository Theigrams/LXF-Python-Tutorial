\hypertarget{ux5217ux8868ux751fux6210ux5f0f}{%
\subsection{列表生成式}\label{ux5217ux8868ux751fux6210ux5f0f}}

列表生成式即 List Comprehensions,是 Python
内置的非常简单却强大的可以用来创建 list 的生成式。

举个例子,要生成 list
\texttt{{[}1,\ 2,\ 3,\ 4,\ 5,\ 6,\ 7,\ 8,\ 9,\ 10{]}}可以用\texttt{list(range(1,\ 11))}:

\begin{pythoncode}
>>> list(range(1, 11))
[1, 2, 3, 4, 5, 6, 7, 8, 9, 10]
\end{pythoncode}

但如果要生成\texttt{{[}1x1,\ 2x2,\ 3x3,\ ...,\ 10x10{]}}怎么做?方法一是循环:

\begin{pythoncode}
>>> L = []
>>> for x in range(1, 11):
...    L.append(x * x)
...
>>> L
[1, 4, 9, 16, 25, 36, 49, 64, 81, 100]
\end{pythoncode}

但是循环太繁琐,而列表生成式则可以用一行语句代替循环生成上面的 list:

\begin{pythoncode}
>>> [x * x for x in range(1, 11)]
[1, 4, 9, 16, 25, 36, 49, 64, 81, 100]
\end{pythoncode}

写列表生成式时,把要生成的元素\texttt{x\ *\ x}放到前面,后面跟\texttt{for}循环,就可以把
list 创建出来,十分有用,多写几次,很快就可以熟悉这种语法。

for 循环后面还可以加上 if 判断,这样我们就可以筛选出仅偶数的平方:

\begin{pythoncode}
>>> [x * x for x in range(1, 11) if x % 2 == 0]
[4, 16, 36, 64, 100]
\end{pythoncode}

还可以使用两层循环,可以生成全排列:

\begin{pythoncode}
>>> [m + n for m in 'ABC' for n in 'XYZ']
['AX', 'AY', 'AZ', 'BX', 'BY', 'BZ', 'CX', 'CY', 'CZ']
\end{pythoncode}

三层和三层以上的循环就很少用到了。

运用列表生成式,可以写出非常简洁的代码。例如,列出当前目录下的所有文件和目录名,可以通过一行代码实现:

\begin{pythoncode}
>>> import os 
>>> [d for d in os.listdir('.')] 
['.emacs.d', '.ssh', '.Trash', 'Adlm', 'Applications', 'Desktop', 'Documents', 'Downloads', 'Library', 'Movies', 'Music', 'Pictures', 'Public', 'VirtualBox VMs', 'Workspace', 'XCode']
\end{pythoncode}

\texttt{for}循环其实可以同时使用两个甚至多个变量,比如\texttt{dict}的\texttt{items()}可以同时迭代
key 和 value:

\begin{pythoncode}
>>> d = {'x': 'A', 'y': 'B', 'z': 'C' }
>>> for k, v in d.items():
...     print(k, '=', v)
...
y = B
x = A
z = C
\end{pythoncode}

因此,列表生成式也可以使用两个变量来生成 list:

\begin{pythoncode}
>>> d = {'x': 'A', 'y': 'B', 'z': 'C' }
>>> [k + '=' + v for k, v in d.items()]
['y=B', 'x=A', 'z=C']
\end{pythoncode}

最后把一个 list 中所有的字符串变成小写:

\begin{pythoncode}
>>> L = ['Hello', 'World', 'IBM', 'Apple']
>>> [s.lower() for s in L]
['hello', 'world', 'ibm', 'apple']
\end{pythoncode}

\hypertarget{if-else}{%
\subsubsection{if \ldots{} else}\label{if-else}}

使用列表生成式的时候,有些童鞋经常搞不清楚\texttt{if...else}的用法。

例如,以下代码正常输出偶数:

\begin{pythoncode}
>>> [x for x in range(1, 11) if x % 2 == 0]
[2, 4, 6, 8, 10]
\end{pythoncode}

但是,我们不能在最后的\texttt{if}加上\texttt{else}:

\begin{pythoncode}
>>> [x for x in range(1, 11) if x % 2 == 0 else 0]
  File "<stdin>", line 1
    [x for x in range(1, 11) if x % 2 == 0 else 0]
                                              ^
SyntaxError: invalid syntax
\end{pythoncode}

这是因为跟在\texttt{for}后面的\texttt{if}是一个筛选条件,不能带\texttt{else},否则如何筛选?

另一些童鞋发现把\texttt{if}写在\texttt{for}前面必须加\texttt{else},否则报错:

\begin{pythoncode}
>>> [x if x % 2 == 0 for x in range(1, 11)]
  File "<stdin>", line 1
    [x if x % 2 == 0 for x in range(1, 11)]
                       ^
SyntaxError: invalid syntax
\end{pythoncode}

这是因为\texttt{for}前面的部分是一个表达式,它必须根据\texttt{x}计算出一个结果。因此,考察表达式:\texttt{x\ if\ x\ \%\ 2\ ==\ 0},它无法根据\texttt{x}计算出结果,因为缺少\texttt{else},必须加上\texttt{else}:

\begin{pythoncode}
>>> [x if x % 2 == 0 else -x for x in range(1, 11)]
[-1, 2, -3, 4, -5, 6, -7, 8, -9, 10]
\end{pythoncode}

上述\texttt{for}前面的表达式\texttt{x\ if\ x\ \%\ 2\ ==\ 0\ else\ -x}才能根据\texttt{x}计算出确定的结果。

可见,在一个列表生成式中,\texttt{for}前面的\texttt{if\ ...\ else}是表达式,而\texttt{for}后面的\texttt{if}是过滤条件,不能带\texttt{else}。

\hypertarget{ux7ec3ux4e60}{%
\subsubsection{练习}\label{ux7ec3ux4e60}}

如果 list
中既包含字符串,又包含整数,由于非字符串类型没有\texttt{lower()}方法,所以列表生成式会报错:

\begin{pythoncode}
>>> L = ['Hello', 'World', 18, 'Apple', None]
>>> [s.lower() for s in L]
Traceback (most recent call last):
  File "<stdin>", line 1, in <module>
  File "<stdin>", line 1, in <listcomp>
AttributeError: 'int' object has no attribute 'lower'
\end{pythoncode}

使用内建的\texttt{isinstance}函数可以判断一个变量是不是字符串:

\begin{pythoncode}
>>> x = 'abc'
>>> y = 123
>>> isinstance(x, str)
True
>>> isinstance(y, str)
False
\end{pythoncode}

请修改列表生成式,通过添加\texttt{if}语句保证列表生成式能正确地执行:

\begin{pythoncode}
# -*- coding: utf-8 -*-
L1 = ['Hello', 'World', 18, 'Apple', None]
\end{pythoncode}

\begin{pythoncode}
# 测试:
print(L2)
if L2 == ['hello', 'world', 'apple']:
    print('测试通过!')
else:
    print('测试失败!')
\end{pythoncode}

\hypertarget{ux5c0fux7ed3}{%
\subsubsection{小结}\label{ux5c0fux7ed3}}

运用列表生成式,可以快速生成 list,可以通过一个 list 推导出另一个
list,而代码却十分简洁。

\hypertarget{ux53c2ux8003ux6e90ux7801}{%
\subsubsection{参考源码}\label{ux53c2ux8003ux6e90ux7801}}

\href{https://github.com/michaelliao/learn-python3/blob/master/samples/advance/do_listcompr.py}{do\_listcompr.py}

