\hypertarget{ux83b7ux53d6ux5bf9ux8c61ux4fe1ux606f}{%
\subsection{获取对象信息}\label{ux83b7ux53d6ux5bf9ux8c61ux4fe1ux606f}}

当我们拿到一个对象的引用时,如何知道这个对象是什么类型、有哪些方法呢?

\hypertarget{ux4f7fux7528-type}{%
\subsubsection{使用 type()}\label{ux4f7fux7528-type}}

首先,我们来判断对象类型,使用\texttt{type()}函数:

基本类型都可以用\texttt{type()}判断:

\begin{pythoncode}
>>> type(123)
<class 'int'>
>>> type('str')
<class 'str'>
>>> type(None)
<type(None) 'NoneType'>
\end{pythoncode}

如果一个变量指向函数或者类,也可以用\texttt{type()}判断:

\begin{pythoncode}
>>> type(abs)
<class 'builtin_function_or_method'>
>>> type(a)
<class '__main__.Animal'>
\end{pythoncode}

但是\texttt{type()}函数返回的是什么类型呢?它返回对应的 Class
类型。如果我们要在\texttt{if}语句中判断,就需要比较两个变量的 type
类型是否相同:

\begin{pythoncode}
>>> type(123)==type(456)
True
>>> type(123)==int
True
>>> type('abc')==type('123')
True
>>> type('abc')==str
True
>>> type('abc')==type(123)
False
\end{pythoncode}

判断基本数据类型可以直接写\texttt{int},\texttt{str}等,但如果要判断一个对象是否是函数怎么办?可以使用\texttt{types}模块中定义的常量:

\begin{pythoncode}
>>> import types
>>> def fn():
...     pass
...
>>> type(fn)==types.FunctionType
True
>>> type(abs)==types.BuiltinFunctionType
True
>>> type(lambda x: x)==types.LambdaType
True
>>> type((x for x in range(10)))==types.GeneratorType
True
\end{pythoncode}

\hypertarget{ux4f7fux7528-isinstance}{%
\subsubsection{使用 isinstance()}\label{ux4f7fux7528-isinstance}}

对于 class 的继承关系来说,使用\texttt{type()}就很不方便。我们要判断
class 的类型,可以使用\texttt{isinstance()}函数。

我们回顾上次的例子,如果继承关系是:

\begin{pythoncode}
object -> Animal -> Dog -> Husky
\end{pythoncode}

那么,\texttt{isinstance()}就可以告诉我们,一个对象是否是某种类型。先创建
3 种类型的对象:

\begin{pythoncode}
>>> a = Animal()
>>> d = Dog()
>>> h = Husky()
\end{pythoncode}

然后,判断:

\begin{pythoncode}
>>> isinstance(h, Husky)
True
\end{pythoncode}

没有问题,因为\texttt{h}变量指向的就是 Husky 对象。

再判断:

\begin{pythoncode}
>>> isinstance(h, Dog)
True
\end{pythoncode}

\texttt{h}虽然自身是 Husky 类型,但由于 Husky 是从 Dog
继承下来的,所以,\texttt{h}也还是 Dog
类型。换句话说,\texttt{isinstance()}判断的是一个对象是否是该类型本身,或者位于该类型的父继承链上。

因此,我们可以确信,\texttt{h}还是 Animal 类型:

\begin{pythoncode}
>>> isinstance(h, Animal)
True
\end{pythoncode}

同理,实际类型是 Dog 的\texttt{d}也是 Animal 类型:

\begin{pythoncode}
>>> isinstance(d, Dog) and isinstance(d, Animal)
True
\end{pythoncode}

但是,\texttt{d}不是 Husky 类型:

\begin{pythoncode}
>>> isinstance(d, Husky)
False
\end{pythoncode}

能用\texttt{type()}判断的基本类型也可以用\texttt{isinstance()}判断:

\begin{pythoncode}
>>> isinstance('a', str)
True
>>> isinstance(123, int)
True
>>> isinstance(b'a', bytes)
True
\end{pythoncode}

并且还可以判断一个变量是否是某些类型中的一种,比如下面的代码就可以判断是否是
list 或者 tuple:

\begin{pythoncode}
>>> isinstance([1, 2, 3], (list, tuple))
True
>>> isinstance((1, 2, 3), (list, tuple))
True
\end{pythoncode}

总是优先使用 isinstance() 判断类型,可以将指定类型及其子类
``一网打尽''。

\hypertarget{ux4f7fux7528-dir}{%
\subsubsection{使用 dir()}\label{ux4f7fux7528-dir}}

如果要获得一个对象的所有属性和方法,可以使用\texttt{dir()}函数,它返回一个包含字符串的
list,比如,获得一个 str 对象的所有属性和方法:

\begin{pythoncode}
>>> dir('ABC')
['__add__', '__class__',..., '__subclasshook__', 'capitalize', 'casefold',..., 'zfill']
\end{pythoncode}

类似\texttt{\_\_xxx\_\_}的属性和方法在 Python
中都是有特殊用途的,比如\texttt{\_\_len\_\_}方法返回长度。在 Python
中,如果你调用\texttt{len()}函数试图获取一个对象的长度,实际上,在\texttt{len()}函数内部,它自动去调用该对象的\texttt{\_\_len\_\_()}方法,所以,下面的代码是等价的:

\begin{pythoncode}
>>> len('ABC')
3
>>> 'ABC'.__len__()
3
\end{pythoncode}

我们自己写的类,如果也想用\texttt{len(myObj)}的话,就自己写一个\texttt{\_\_len\_\_()}方法:

\begin{pythoncode}
>>> class MyDog(object):
...     def __len__(self):
...         return 100
...
>>> dog = MyDog()
>>> len(dog)
100
\end{pythoncode}

剩下的都是普通属性或方法,比如\texttt{lower()}返回小写的字符串:

\begin{pythoncode}
>>> 'ABC'.lower()
'abc'
\end{pythoncode}

仅仅把属性和方法列出来是不够的,配合\texttt{getattr()}、\texttt{setattr()}以及\texttt{hasattr()},我们可以直接操作一个对象的状态:

\begin{pythoncode}
>>> class MyObject(object):
...     def __init__(self):
...         self.x = 9
...     def power(self):
...         return self.x * self.x
...
>>> obj = MyObject()
\end{pythoncode}

紧接着,可以测试该对象的属性:

\begin{pythoncode}
>>> hasattr(obj, 'x') 
True
>>> obj.x
9
>>> hasattr(obj, 'y') 
False
>>> setattr(obj, 'y', 19) 
>>> hasattr(obj, 'y') 
True
>>> getattr(obj, 'y') 
19
>>> obj.y 
19
\end{pythoncode}

如果试图获取不存在的属性,会抛出 AttributeError 的错误:

\begin{pythoncode}
>>> getattr(obj, 'z') # 获取属性'z'
Traceback (most recent call last):
  File "<stdin>", line 1, in <module>
AttributeError: 'MyObject' object has no attribute 'z'
\end{pythoncode}

可以传入一个 default 参数,如果属性不存在,就返回默认值:

\begin{pythoncode}
>>> getattr(obj, 'z', 404) 
404
\end{pythoncode}

也可以获得对象的方法:

\begin{pythoncode}
>>> hasattr(obj, 'power') 
True
>>> getattr(obj, 'power') 
<bound method MyObject.power of <__main__.MyObject object at 0x10077a6a0>>
>>> fn = getattr(obj, 'power') 
>>> fn 
<bound method MyObject.power of <__main__.MyObject object at 0x10077a6a0>>
>>> fn() 
81
\end{pythoncode}

\hypertarget{ux5c0fux7ed3}{%
\subsubsection{小结}\label{ux5c0fux7ed3}}

通过内置的一系列函数,我们可以对任意一个 Python
对象进行剖析,拿到其内部的数据。要注意的是,只有在不知道对象信息的时候,我们才会去获取对象信息。如果可以直接写:

\begin{pythoncode}
sum = obj.x + obj.y
\end{pythoncode}

就不要写:

\begin{pythoncode}
sum = getattr(obj, 'x') + getattr(obj, 'y')
\end{pythoncode}

一个正确的用法的例子如下:

\begin{pythoncode}
def readImage(fp):
    if hasattr(fp, 'read'):
        return readData(fp)
    return None
\end{pythoncode}

假设我们希望从文件流 fp 中读取图像,我们首先要判断该 fp 对象是否存在
read
方法,如果存在,则该对象是一个流,如果不存在,则无法读取。\texttt{hasattr()}就派上了用场。

请注意,在 Python
这类动态语言中,根据鸭子类型,有\texttt{read()}方法,不代表该 fp
对象就是一个文件流,它也可能是网络流,也可能是内存中的一个字节流,但只要\texttt{read()}方法返回的是有效的图像数据,就不影响读取图像的功能。

\hypertarget{ux53c2ux8003ux6e90ux7801}{%
\subsubsection{参考源码}\label{ux53c2ux8003ux6e90ux7801}}

\href{https://github.com/michaelliao/learn-python3/blob/master/samples/oop_basic/get_type.py}{get\_type.py}

\href{https://github.com/michaelliao/learn-python3/blob/master/samples/oop_basic/attrs.py}{attrs.py}

