\hypertarget{ux534fux7a0b}{%
\subsection{协程}\label{ux534fux7a0b}}

在学习异步 IO 模型前,我们先来了解协程。

协程,又称微线程,纤程。英文名 Coroutine。

协程的概念很早就提出来了,但直到最近几年才在某些语言(如
Lua)中得到广泛应用。

子程序,或者称为函数,在所有语言中都是层级调用,比如 A 调用 B,B
在执行过程中又调用了 C,C 执行完毕返回,B 执行完毕返回,最后是 A
执行完毕。

所以子程序调用是通过栈实现的,一个线程就是执行一个子程序。

子程序调用总是一个入口,一次返回,调用顺序是明确的。而协程的调用和子程序不同。

协程看上去也是子程序,但执行过程中,在子程序内部可中断,然后转而执行别的子程序,在适当的时候再返回来接着执行。

注意,在一个子程序中中断,去执行其他子程序,不是函数调用,有点类似 CPU
的中断。比如子程序 A、B:

\begin{pythoncode}
def A():
    print('1')
    print('2')
    print('3')

def B():
    print('x')
    print('y')
    print('z')
\end{pythoncode}

假设由协程执行,在执行 A 的过程中,可以随时中断,去执行 B,B
也可能在执行过程中中断再去执行 A,结果可能是:

\begin{pythoncode}
1
2
x
y
3
z
\end{pythoncode}

但是在 A 中是没有调用 B 的,所以协程的调用比函数调用理解起来要难一些。

看起来 A、B
的执行有点像多线程,但协程的特点在于是一个线程执行,那和多线程比,协程有何优势?

最大的优势就是协程极高的执行效率。因为子程序切换不是线程切换,而是由程序自身控制,因此,没有线程切换的开销,和多线程比,线程数量越多,协程的性能优势就越明显。

第二大优势就是不需要多线程的锁机制,因为只有一个线程,也不存在同时写变量冲突,在协程中控制共享资源不加锁,只需要判断状态就好了,所以执行效率比多线程高很多。

因为协程是一个线程执行,那怎么利用多核 CPU 呢?最简单的方法是多进程 +
协程,既充分利用多核,又充分发挥协程的高效率,可获得极高的性能。

Python 对协程的支持是通过 generator 实现的。

在 generator
中,我们不但可以通过\texttt{for}循环来迭代,还可以不断调用\texttt{next()}函数获取由\texttt{yield}语句返回的下一个值。

但是 Python
的\texttt{yield}不但可以返回一个值,它还可以接收调用者发出的参数。

来看例子:

传统的生产者 -
消费者模型是一个线程写消息,一个线程取消息,通过锁机制控制队列和等待,但一不小心就可能死锁。

如果改用协程,生产者生产消息后,直接通过\texttt{yield}跳转到消费者开始执行,待消费者执行完毕后,切换回生产者继续生产,效率极高:

\begin{pythoncode}
def consumer():
    r = ''
    while True:
        n = yield r
        if not n:
            return
        print('[CONSUMER] Consuming %s...' % n)
        r = '200 OK'

def produce(c):
    c.send(None)
    n = 0
    while n < 5:
        n = n + 1
        print('[PRODUCER] Producing %s...' % n)
        r = c.send(n)
        print('[PRODUCER] Consumer return: %s' % r)
    c.close()

c = consumer()
produce(c)
\end{pythoncode}

执行结果:

\begin{pythoncode}
[PRODUCER] Producing 1...
[CONSUMER] Consuming 1...
[PRODUCER] Consumer return: 200 OK
[PRODUCER] Producing 2...
[CONSUMER] Consuming 2...
[PRODUCER] Consumer return: 200 OK
[PRODUCER] Producing 3...
[CONSUMER] Consuming 3...
[PRODUCER] Consumer return: 200 OK
[PRODUCER] Producing 4...
[CONSUMER] Consuming 4...
[PRODUCER] Consumer return: 200 OK
[PRODUCER] Producing 5...
[CONSUMER] Consuming 5...
[PRODUCER] Consumer return: 200 OK
\end{pythoncode}

注意到\texttt{consumer}函数是一个\texttt{generator},把一个\texttt{consumer}传入\texttt{produce}后:

\begin{enumerate}
\def\labelenumi{\arabic{enumi}.}
\item
  首先调用\texttt{c.send(None)}启动生成器;
\item
  然后,一旦生产了东西,通过\texttt{c.send(n)}切换到\texttt{consumer}执行;
\item
  \texttt{consumer}通过\texttt{yield}拿到消息,处理,又通过\texttt{yield}把结果传回;
\item
  \texttt{produce}拿到\texttt{consumer}处理的结果,继续生产下一条消息;
\item
  \texttt{produce}决定不生产了,通过\texttt{c.close()}关闭\texttt{consumer},整个过程结束。
\end{enumerate}

整个流程无锁,由一个线程执行,\texttt{produce}和\texttt{consumer}协作完成任务,所以称为
``协程'',而非线程的抢占式多任务。

最后套用 Donald Knuth 的一句话总结协程的特点:

``子程序就是协程的一种特例。''

\hypertarget{ux53c2ux8003ux6e90ux7801}{%
\subsubsection{参考源码}\label{ux53c2ux8003ux6e90ux7801}}

\href{https://github.com/michaelliao/learn-python3/blob/master/samples/async/coroutine.py}{coroutine.py}

