\hypertarget{ux5b9eux4f8bux5c5eux6027ux548cux7c7bux5c5eux6027}{%
\subsection{实例属性和类属性}\label{ux5b9eux4f8bux5c5eux6027ux548cux7c7bux5c5eux6027}}

由于 Python 是动态语言,根据类创建的实例可以任意绑定属性。

给实例绑定属性的方法是通过实例变量,或者通过\texttt{self}变量:

\begin{pythoncode}
class Student(object):
    def __init__(self, name):
        self.name = name

s = Student('Bob')
s.score = 90
\end{pythoncode}

但是,如果\texttt{Student}类本身需要绑定一个属性呢?可以直接在 class
中定义属性,这种属性是类属性,归\texttt{Student}类所有:

\begin{pythoncode}
class Student(object):
    name = 'Student'
\end{pythoncode}

当我们定义了一个类属性后,这个属性虽然归类所有,但类的所有实例都可以访问到。来测试一下:

\begin{pythoncode}
>>> class Student(object):
...     name = 'Student'
...
>>> s = Student() 
>>> print(s.name) 
Student
>>> print(Student.name) 
Student
>>> s.name = 'Michael' 
>>> print(s.name) 
Michael
>>> print(Student.name) 
Student
>>> del s.name 
>>> print(s.name) 
Student
\end{pythoncode}

从上面的例子可以看出,在编写程序的时候,千万不要对实例属性和类属性使用相同的名字,因为相同名称的实例属性将屏蔽掉类属性,但是当你删除实例属性后,再使用相同的名称,访问到的将是类属性。

\hypertarget{ux7ec3ux4e60}{%
\subsubsection{练习}\label{ux7ec3ux4e60}}

为了统计学生人数,可以给 Student
类增加一个类属性,每创建一个实例,该属性自动增加:

\begin{pythoncode}
# -*- coding: utf-8 -*-
\end{pythoncode}

\begin{pythoncode}
# 测试:
if Student.count != 0:
    print('测试失败!')
else:
    bart = Student('Bart')
    if Student.count != 1:
        print('测试失败!')
    else:
        lisa = Student('Bart')
        if Student.count != 2:
            print('测试失败!')
        else:
            print('Students:', Student.count)
            print('测试通过!')
\end{pythoncode}

\hypertarget{ux5c0fux7ed3}{%
\subsubsection{小结}\label{ux5c0fux7ed3}}

实例属性属于各个实例所有,互不干扰;

类属性属于类所有,所有实例共享一个属性;

不要对实例属性和类属性使用相同的名字,否则将产生难以发现的错误。

