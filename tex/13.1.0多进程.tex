\hypertarget{ux591aux8fdbux7a0b}{%
\subsection{多进程}\label{ux591aux8fdbux7a0b}}

要让 Python
程序实现多进程(multiprocessing),我们先了解操作系统的相关知识。

Unix/Linux
操作系统提供了一个\texttt{fork()}系统调用,它非常特殊。普通的函数调用,调用一次,返回一次,但是\texttt{fork()}调用一次,返回两次,因为操作系统自动把当前进程(称为父进程)复制了一份(称为子进程),然后,分别在父进程和子进程内返回。

子进程永远返回\texttt{0},而父进程返回子进程的
ID。这样做的理由是,一个父进程可以 fork
出很多子进程,所以,父进程要记下每个子进程的
ID,而子进程只需要调用\texttt{getppid()}就可以拿到父进程的 ID。

Python
的\texttt{os}模块封装了常见的系统调用,其中就包括\texttt{fork},可以在
Python 程序中轻松创建子进程:

\begin{pythoncode}
import os

print('Process (%s) start...' % os.getpid())

pid = os.fork()
if pid == 0:
    print('I am child process (%s) and my parent is %s.' % (os.getpid(), os.getppid()))
else:
    print('I (%s) just created a child process (%s).' % (os.getpid(), pid))
\end{pythoncode}

运行结果如下:

\begin{pythoncode}
Process (876) start...
I (876) just created a child process (877).
I am child process (877) and my parent is 876.
\end{pythoncode}

由于 Windows 没有\texttt{fork}调用,上面的代码在 Windows 上无法运行。而
Mac 系统是基于 BSD(Unix 的一种)内核,所以,在 Mac
下运行是没有问题的,推荐大家用 Mac 学 Python!

有了\texttt{fork}调用,一个进程在接到新任务时就可以复制出一个子进程来处理新任务,常见的
Apache 服务器就是由父进程监听端口,每当有新的 http 请求时,就 fork
出子进程来处理新的 http 请求。

\hypertarget{multiprocessing}{%
\subsubsection{multiprocessing}\label{multiprocessing}}

如果你打算编写多进程的服务程序,Unix/Linux 无疑是正确的选择。由于
Windows 没有\texttt{fork}调用,难道在 Windows 上无法用 Python
编写多进程的程序?

由于 Python
是跨平台的,自然也应该提供一个跨平台的多进程支持。\texttt{multiprocessing}模块就是跨平台版本的多进程模块。

\texttt{multiprocessing}模块提供了一个\texttt{Process}类来代表一个进程对象,下面的例子演示了启动一个子进程并等待其结束:

\begin{pythoncode}
from multiprocessing import Process
import os
def run_proc(name):
    print('Run child process %s (%s)...' % (name, os.getpid()))

if __name__=='__main__':
    print('Parent process %s.' % os.getpid())
    p = Process(target=run_proc, args=('test',))
    print('Child process will start.')
    p.start()
    p.join()
    print('Child process end.')
\end{pythoncode}

执行结果如下:

\begin{pythoncode}
Parent process 928.
Child process will start.
Run child process test (929)...
Process end.
\end{pythoncode}

创建子进程时,只需要传入一个执行函数和函数的参数,创建一个\texttt{Process}实例,用\texttt{start()}方法启动,这样创建进程比\texttt{fork()}还要简单。

\texttt{join()}方法可以等待子进程结束后再继续往下运行,通常用于进程间的同步。

\hypertarget{pool}{%
\subsubsection{Pool}\label{pool}}

如果要启动大量的子进程,可以用进程池的方式批量创建子进程:

\begin{pythoncode}
from multiprocessing import Pool
import os, time, random

def long_time_task(name):
    print('Run task %s (%s)...' % (name, os.getpid()))
    start = time.time()
    time.sleep(random.random() * 3)
    end = time.time()
    print('Task %s runs %0.2f seconds.' % (name, (end - start)))

if __name__=='__main__':
    print('Parent process %s.' % os.getpid())
    p = Pool(4)
    for i in range(5):
        p.apply_async(long_time_task, args=(i,))
    print('Waiting for all subprocesses done...')
    p.close()
    p.join()
    print('All subprocesses done.')
\end{pythoncode}

执行结果如下:

\begin{pythoncode}
Parent process 669.
Waiting for all subprocesses done...
Run task 0 (671)...
Run task 1 (672)...
Run task 2 (673)...
Run task 3 (674)...
Task 2 runs 0.14 seconds.
Run task 4 (673)...
Task 1 runs 0.27 seconds.
Task 3 runs 0.86 seconds.
Task 0 runs 1.41 seconds.
Task 4 runs 1.91 seconds.
All subprocesses done.
\end{pythoncode}

代码解读:

对\texttt{Pool}对象调用\texttt{join()}方法会等待所有子进程执行完毕,调用\texttt{join()}之前必须先调用\texttt{close()},调用\texttt{close()}之后就不能继续添加新的\texttt{Process}了。

请注意输出的结果,task
\texttt{0},\texttt{1},\texttt{2},\texttt{3}是立刻执行的,而 task
\texttt{4}要等待前面某个 task
完成后才执行,这是因为\texttt{Pool}的默认大小在我的电脑上是
4,因此,最多同时执行 4
个进程。这是\texttt{Pool}有意设计的限制,并不是操作系统的限制。如果改成:

\begin{pythoncode}
p = Pool(5)
\end{pythoncode}

就可以同时跑 5 个进程。

由于\texttt{Pool}的默认大小是 CPU 的核数,如果你不幸拥有 8 核
CPU,你要提交至少 9 个子进程才能看到上面的等待效果。

\hypertarget{ux5b50ux8fdbux7a0b}{%
\subsubsection{子进程}\label{ux5b50ux8fdbux7a0b}}

很多时候,子进程并不是自身,而是一个外部进程。我们创建了子进程后,还需要控制子进程的输入和输出。

\texttt{subprocess}模块可以让我们非常方便地启动一个子进程,然后控制其输入和输出。

下面的例子演示了如何在 Python
代码中运行命令\texttt{nslookup\ www.python.org},这和命令行直接运行的效果是一样的:

\begin{pythoncode}
import subprocess

print('$ nslookup www.python.org')
r = subprocess.call(['nslookup', 'www.python.org'])
print('Exit code:', r)
\end{pythoncode}

运行结果:

\begin{pythoncode}
$ nslookup www.python.org
Server:     192.168.19.4
Address:    192.168.19.4

Non-authoritative answer:
www.python.org  canonical name = python.map.fastly.net.
Name:   python.map.fastly.net
Address: 199.27.79.223

Exit code: 0
\end{pythoncode}

如果子进程还需要输入,则可以通过\texttt{communicate()}方法输入:

\begin{pythoncode}
import subprocess

print('$ nslookup')
p = subprocess.Popen(['nslookup'], stdin=subprocess.PIPE, stdout=subprocess.PIPE, stderr=subprocess.PIPE)
output, err = p.communicate(b'set q=mx\npython.org\nexit\n')
print(output.decode('utf-8'))
print('Exit code:', p.returncode)
\end{pythoncode}

上面的代码相当于在命令行执行命令\texttt{nslookup},然后手动输入:

\begin{pythoncode}
set q=mx
python.org
exit
\end{pythoncode}

运行结果如下:

\begin{pythoncode}
$ nslookup
Server:     192.168.19.4
Address:    192.168.19.4

Non-authoritative answer:
python.org  mail exchanger = 50 mail.python.org.

Authoritative answers can be found from:
mail.python.org internet address = 82.94.164.166
mail.python.org has AAAA address 2001:888:2000:d::a6
Exit code: 0
\end{pythoncode}

\hypertarget{ux8fdbux7a0bux95f4ux901aux4fe1}{%
\subsubsection{进程间通信}\label{ux8fdbux7a0bux95f4ux901aux4fe1}}

\texttt{Process}之间肯定是需要通信的,操作系统提供了很多机制来实现进程间的通信。Python
的\texttt{multiprocessing}模块包装了底层的机制,提供了\texttt{Queue}、\texttt{Pipes}等多种方式来交换数据。

我们以\texttt{Queue}为例,在父进程中创建两个子进程,一个往\texttt{Queue}里写数据,一个从\texttt{Queue}里读数据:

\begin{pythoncode}
from multiprocessing import Process, Queue
import os, time, random
def write(q):
    print('Process to write: %s' % os.getpid())
    for value in ['A', 'B', 'C']:
        print('Put %s to queue...' % value)
        q.put(value)
        time.sleep(random.random())
def read(q):
    print('Process to read: %s' % os.getpid())
    while True:
        value = q.get(True)
        print('Get %s from queue.' % value)

if __name__=='__main__':
    
    q = Queue()
    pw = Process(target=write, args=(q,))
    pr = Process(target=read, args=(q,))
    
    pw.start()
    
    pr.start()
    
    pw.join()
    
    pr.terminate()
\end{pythoncode}

运行结果如下:

\begin{pythoncode}
Process to write: 50563
Put A to queue...
Process to read: 50564
Get A from queue.
Put B to queue...
Get B from queue.
Put C to queue...
Get C from queue.
\end{pythoncode}

在 Unix/Linux
下,\texttt{multiprocessing}模块封装了\texttt{fork()}调用,使我们不需要关注\texttt{fork()}的细节。由于
Windows 没有\texttt{fork}调用,因此,\texttt{multiprocessing}需要
``模拟'' 出\texttt{fork}的效果,父进程所有 Python 对象都必须通过 pickle
序列化再传到子进程去,所以,如果\texttt{multiprocessing}在 Windows
下调用失败了,要先考虑是不是 pickle 失败了。

\hypertarget{ux5c0fux7ed3}{%
\subsubsection{小结}\label{ux5c0fux7ed3}}

在 Unix/Linux 下,可以使用\texttt{fork()}调用实现多进程。

要实现跨平台的多进程,可以使用\texttt{multiprocessing}模块。

进程间通信是通过\texttt{Queue}、\texttt{Pipes}等实现的。

\hypertarget{ux53c2ux8003ux6e90ux7801}{%
\subsubsection{参考源码}\label{ux53c2ux8003ux6e90ux7801}}

\href{https://github.com/michaelliao/learn-python3/blob/master/samples/multitask/do_folk.py}{do\_folk.py}

\href{https://github.com/michaelliao/learn-python3/blob/master/samples/multitask/multi_processing.py}{multi\_processing.py}

\href{https://github.com/michaelliao/learn-python3/blob/master/samples/multitask/pooled_processing.py}{pooled\_processing.py}

\href{https://github.com/michaelliao/learn-python3/blob/master/samples/multitask/do_subprocess.py}{do\_subprocess.py}

\href{https://github.com/michaelliao/learn-python3/blob/master/samples/multitask/do_queue.py}{do\_queue.py}

