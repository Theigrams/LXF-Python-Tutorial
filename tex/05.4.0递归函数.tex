\hypertarget{ux9012ux5f52ux51fdux6570}{%
\subsection{递归函数}\label{ux9012ux5f52ux51fdux6570}}

在函数内部,可以调用其他函数。如果一个函数在内部调用自身本身,这个函数就是递归函数。

举个例子,我们来计算阶乘\texttt{n!\ =\ 1\ x\ 2\ x\ 3\ x\ ...\ x\ n},用函数\texttt{fact(n)}表示,可以看出:

\(fact(n)=n!=1\times2\times3\times\cdot\cdot\cdot\times(n-1)\times n=(n-1)!\times n=fact(n-1)\times n\)

所以,\texttt{fact(n)}可以表示为\texttt{n\ x\ fact(n-1)},只有 n=1
时需要特殊处理。

于是,\texttt{fact(n)}用递归的方式写出来就是:

\begin{pythoncode}
def fact(n):
    if n==1:
        return 1
    return n * fact(n - 1)
\end{pythoncode}

上面就是一个递归函数。可以试试:

\begin{pythoncode}
>>> fact(1)
1
>>> fact(5)
120
>>> fact(100)
93326215443944152681699238856266700490715968264381621468592963895217599993229915608941463976156518286253697920827223758251185210916864000000000000000000000000
\end{pythoncode}

如果我们计算\texttt{fact(5)},可以根据函数定义看到计算过程如下:

\begin{pythoncode}
===> fact(5)
===> 5 * fact(4)
===> 5 * (4 * fact(3))
===> 5 * (4 * (3 * fact(2)))
===> 5 * (4 * (3 * (2 * fact(1))))
===> 5 * (4 * (3 * (2 * 1)))
===> 5 * (4 * (3 * 2))
===> 5 * (4 * 6)
===> 5 * 24
===> 120
\end{pythoncode}

递归函数的优点是定义简单,逻辑清晰。理论上,所有的递归函数都可以写成循环的方式,但循环的逻辑不如递归清晰。

使用递归函数需要注意防止栈溢出。在计算机中,函数调用是通过栈(stack)这种数据结构实现的,每当进入一个函数调用,栈就会加一层栈帧,每当函数返回,栈就会减一层栈帧。由于栈的大小不是无限的,所以,递归调用的次数过多,会导致栈溢出。可以试试\texttt{fact(1000)}:

\begin{pythoncode}
>>> fact(1000)
Traceback (most recent call last):
  File "<stdin>", line 1, in <module>
  File "<stdin>", line 4, in fact
  ...
  File "<stdin>", line 4, in fact
RuntimeError: maximum recursion depth exceeded in comparison
\end{pythoncode}

解决递归调用栈溢出的方法是通过\textbf{尾递归}优化,事实上尾递归和循环的效果是一样的,所以,把循环看成是一种特殊的尾递归函数也是可以的。

尾递归是指,在函数返回的时候,调用自身本身,并且,return
语句不能包含表达式。这样,编译器或者解释器就可以把尾递归做优化,使递归本身无论调用多少次,都只占用一个栈帧,不会出现栈溢出的情况。

上面的\texttt{fact(n)}函数由于\texttt{return\ n\ *\ fact(n\ -\ 1)}引入了乘法表达式,所以就不是尾递归了。要改成尾递归方式,需要多一点代码,主要是要把每一步的乘积传入到递归函数中:

\begin{pythoncode}
def fact(n):
    return fact_iter(n, 1)

def fact_iter(num, product):
    if num == 1:
        return product
    return fact_iter(num - 1, num * product)
\end{pythoncode}

可以看到,\texttt{return\ fact\_iter(num\ -\ 1,\ num\ *\ product)}仅返回递归函数本身,\texttt{num\ -\ 1}和\texttt{num\ *\ product}在函数调用前就会被计算,不影响函数调用。

\texttt{fact(5)}对应的\texttt{fact\_iter(5,\ 1)}的调用如下:

\begin{pythoncode}
===> fact_iter(5, 1)
===> fact_iter(4, 5)
===> fact_iter(3, 20)
===> fact_iter(2, 60)
===> fact_iter(1, 120)
===> 120
\end{pythoncode}

尾递归调用时,如果做了优化,栈不会增长,因此,无论多少次调用也不会导致栈溢出。

遗憾的是,大多数编程语言没有针对尾递归做优化,Python
解释器也没有做优化,所以,即使把上面的\texttt{fact(n)}函数改成尾递归方式,也会导致栈溢出。

\hypertarget{ux5c0fux7ed3}{%
\subsubsection{小结}\label{ux5c0fux7ed3}}

使用递归函数的优点是逻辑简单清晰,缺点是过深的调用会导致栈溢出。

针对尾递归优化的语言可以通过尾递归防止栈溢出。尾递归事实上和循环是等价的,没有循环语句的编程语言只能通过尾递归实现循环。

Python
标准的解释器没有针对尾递归做优化,任何递归函数都存在栈溢出的问题。

\hypertarget{ux7ec3ux4e60}{%
\subsubsection{练习}\label{ux7ec3ux4e60}}

\href{http://baike.baidu.com/view/191666.htm}{汉诺塔}的移动可以用递归函数非常简单地实现。

请编写\texttt{move(n,\ a,\ b,\ c)}函数,它接收参数\texttt{n},表示 3
个柱子 A、B、C 中第 1 个柱子 A 的盘子数量,然后打印出把所有盘子从 A 借助
B 移动到 C 的方法,例如:

\begin{pythoncode}
# -*- coding: utf-8 -*-
def move(n, a, b, c):
\end{pythoncode}

\begin{pythoncode}
# 期待输出:
# A --> C
# A --> B
# C --> B
# A --> C
# B --> A
# B --> C
# A --> C
move(3, 'A', 'B', 'C')
\end{pythoncode}

\hypertarget{ux53c2ux8003ux6e90ux7801}{%
\subsubsection{参考源码}\label{ux53c2ux8003ux6e90ux7801}}

\href{https://github.com/michaelliao/learn-python3/blob/master/samples/function/recur.py}{recur.py}

