\hypertarget{ux6a21ux5757}{%
\subsection{模块}\label{ux6a21ux5757}}

在计算机程序的开发过程中,随着程序代码越写越多,在一个文件里代码就会越来越长,越来越不容易维护。

为了编写可维护的代码,我们把很多函数分组,分别放到不同的文件里,这样,每个文件包含的代码就相对较少,很多编程语言都采用这种组织代码的方式。在
Python 中,一个. py 文件就称之为一个模块(Module)。

使用模块有什么好处?

最大的好处是大大提高了代码的可维护性。其次,编写代码不必从零开始。当一个模块编写完毕,就可以被其他地方引用。我们在编写程序的时候,也经常引用其他模块,包括
Python 内置的模块和来自第三方的模块。

使用模块还可以避免函数名和变量名冲突。相同名字的函数和变量完全可以分别存在不同的模块中,因此,我们自己在编写模块时,不必考虑名字会与其他模块冲突。但是也要注意,尽量不要与内置函数名字冲突。点\href{http://docs.python.org/3/library/functions.html}{这里}查看
Python 的所有内置函数。

你也许还想到,如果不同的人编写的模块名相同怎么办?为了避免模块名冲突,Python
又引入了按目录来组织模块的方法,称为包(Package)。

举个例子,一个\texttt{abc.py}的文件就是一个名字叫\texttt{abc}的模块,一个\texttt{xyz.py}的文件就是一个名字叫\texttt{xyz}的模块。

现在,假设我们的\texttt{abc}和\texttt{xyz}这两个模块名字与其他模块冲突了,于是我们可以通过包来组织模块,避免冲突。方法是选择一个顶层包名,比如\texttt{mycompany},按照如下目录存放:

\begin{pythoncode}
mycompany
├─ __init__.py
├─ abc.py
└─ xyz.py
\end{pythoncode}

引入了包以后,只要顶层的包名不与别人冲突,那所有模块都不会与别人冲突。现在,\texttt{abc.py}模块的名字就变成了\texttt{mycompany.abc},类似的,\texttt{xyz.py}的模块名变成了\texttt{mycompany.xyz}。

请注意,每一个包目录下面都会有一个\texttt{\_\_init\_\_.py}的文件,这个文件是必须存在的,否则,Python
就把这个目录当成普通目录,而不是一个包。\texttt{\_\_init\_\_.py}可以是空文件,也可以有
Python
代码,因为\texttt{\_\_init\_\_.py}本身就是一个模块,而它的模块名就是\texttt{mycompany}。

类似的,可以有多级目录,组成多级层次的包结构。比如如下的目录结构:

\begin{pythoncode}
mycompany
 ├─ web
 │  ├─ __init__.py
 │  ├─ utils.py
 │  └─ www.py
 ├─ __init__.py
 ├─ abc.py
 └─ utils.py
\end{pythoncode}

文件\texttt{www.py}的模块名就是\texttt{mycompany.web.www},两个文件\texttt{utils.py}的模块名分别是\texttt{mycompany.utils}和\texttt{mycompany.web.utils}。

自己创建模块时要注意命名,不能和 Python
自带的模块名称冲突。例如,系统自带了 sys 模块,自己的模块就不可命名为
sys.py,否则将无法导入系统自带的 sys 模块。

\texttt{mycompany.web}也是一个模块,请指出该模块对应的. py 文件。

\hypertarget{ux603bux7ed3}{%
\subsubsection{总结}\label{ux603bux7ed3}}

模块是一组 Python 代码的集合,可以使用其他模块,也可以被其他模块使用。

创建自己的模块时,要注意:

\begin{itemize}
\item
  模块名要遵循 Python 变量命名规范,不要使用中文、特殊字符;
\item
  模块名不要和系统模块名冲突,最好先查看系统是否已存在该模块,检查方法是在
  Python 交互环境执行\texttt{import\ abc},若成功则说明系统存在此模块。
\end{itemize}

