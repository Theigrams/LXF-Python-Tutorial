\hypertarget{ux504fux51fdux6570}{%
\subsection{偏函数}\label{ux504fux51fdux6570}}

Python
的\texttt{functools}模块提供了很多有用的功能,其中一个就是偏函数(Partial
function)。要注意,这里的偏函数和数学意义上的偏函数不一样。

在介绍函数参数的时候,我们讲到,通过设定参数的默认值,可以降低函数调用的难度。而偏函数也可以做到这一点。举例如下:

\texttt{int()}函数可以把字符串转换为整数,当仅传入字符串时,\texttt{int()}函数默认按十进制转换:

\begin{pythoncode}
>>> int('12345')
12345
\end{pythoncode}

但\texttt{int()}函数还提供额外的\texttt{base}参数,默认值为\texttt{10}。如果传入\texttt{base}参数,就可以做
N 进制的转换:

\begin{pythoncode}
>>> int('12345', base=8)
5349
>>> int('12345', 16)
74565
\end{pythoncode}

假设要转换大量的二进制字符串,每次都传入\texttt{int(x,\ base=2)}非常麻烦,于是,我们想到,可以定义一个\texttt{int2()}的函数,默认把\texttt{base=2}传进去:

\begin{pythoncode}
def int2(x, base=2):
    return int(x, base)
\end{pythoncode}

这样,我们转换二进制就非常方便了:

\begin{pythoncode}
>>> int2('1000000')
64
>>> int2('1010101')
85
\end{pythoncode}

\texttt{functools.partial}就是帮助我们创建一个偏函数的,不需要我们自己定义\texttt{int2()},可以直接使用下面的代码创建一个新的函数\texttt{int2}:

\begin{pythoncode}
>>> import functools
>>> int2 = functools.partial(int, base=2)
>>> int2('1000000')
64
>>> int2('1010101')
85
\end{pythoncode}

所以,简单总结\texttt{functools.partial}的作用就是,把一个函数的某些参数给固定住(也就是设置默认值),返回一个新的函数,调用这个新函数会更简单。

注意到上面的新的\texttt{int2}函数,仅仅是把\texttt{base}参数重新设定默认值为\texttt{2},但也可以在函数调用时传入其他值:

\begin{pythoncode}
>>> int2('1000000', base=10)
1000000
\end{pythoncode}

最后,创建偏函数时,实际上可以接收函数对象、\texttt{*args}和\texttt{**kw}这
3 个参数,当传入:

\begin{pythoncode}
int2 = functools.partial(int, base=2)
\end{pythoncode}

实际上固定了 int() 函数的关键字参数\texttt{base},也就是:

\begin{pythoncode}
int2('10010')
\end{pythoncode}

相当于:

\begin{pythoncode}
kw = { 'base': 2 }
int('10010', **kw)
\end{pythoncode}

当传入:

\begin{pythoncode}
max2 = functools.partial(max, 10)
\end{pythoncode}

实际上会把\texttt{10}作为\texttt{*args}的一部分自动加到左边,也就是:

\begin{pythoncode}
max2(5, 6, 7)
\end{pythoncode}

相当于:

\begin{pythoncode}
args = (10, 5, 6, 7)
max(*args)
\end{pythoncode}

结果为\texttt{10}。

\hypertarget{ux5c0fux7ed3}{%
\subsubsection{小结}\label{ux5c0fux7ed3}}

当函数的参数个数太多,需要简化时,使用\texttt{functools.partial}可以创建一个新的函数,这个新函数可以固定住原函数的部分参数,从而在调用时更简单。

\hypertarget{ux53c2ux8003ux6e90ux7801}{%
\subsubsection{参考源码}\label{ux53c2ux8003ux6e90ux7801}}

\href{https://github.com/michaelliao/learn-python3/blob/master/samples/functional/do_partial.py}{do\_partial.py}

