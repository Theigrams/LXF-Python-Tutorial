\hypertarget{map-reduce}{%
\subsection{map reduce}\label{map-reduce}}

Python 内建了\texttt{map()}和\texttt{reduce()}函数。

如果你读过 Google 的那篇大名鼎鼎的论文
``\href{http://research.google.com/archive/mapreduce.html}{MapReduce:
Simplified Data Processing on Large Clusters}'',你就能大概明白
map/reduce 的概念。

我们先看
map。\texttt{map()}函数接收两个参数,一个是函数,一个是\texttt{Iterable},\texttt{map}将传入的函数依次作用到序列的每个元素,并把结果作为新的\texttt{Iterator}返回。

举例说明,比如我们有一个函数
\texttt{f(x)=x\^{}2},要把这个函数作用在一个 list
\texttt{{[}1,\ 2,\ 3,\ 4,\ 5,\ 6,\ 7,\ 8,\ 9{]}}上,就可以用\texttt{map()}实现如下:

\begin{pythoncode}
            f(x) = x * x

                  │
                  │
  ┌───┬───┬───┬───┼───┬───┬───┬───┐
  │   │   │   │   │   │   │   │   │
  ▼   ▼   ▼   ▼   ▼   ▼   ▼   ▼   ▼

[ 1   2   3   4   5   6   7   8   9 ]

  │   │   │   │   │   │   │   │   │
  │   │   │   │   │   │   │   │   │
  ▼   ▼   ▼   ▼   ▼   ▼   ▼   ▼   ▼

[ 1   4   9  16  25  36  49  64  81 ]
\end{pythoncode}

现在,我们用 Python 代码实现:

\begin{pythoncode}
>>> def f(x):
...     return x * x
...
>>> r = map(f, [1, 2, 3, 4, 5, 6, 7, 8, 9])
>>> list(r)
[1, 4, 9, 16, 25, 36, 49, 64, 81]
\end{pythoncode}

\texttt{map()}传入的第一个参数是\texttt{f},即函数对象本身。由于结果\texttt{r}是一个\texttt{Iterator},\texttt{Iterator}是惰性序列,因此通过\texttt{list()}函数让它把整个序列都计算出来并返回一个
list。

你可能会想,不需要\texttt{map()}函数,写一个循环,也可以计算出结果:

\begin{pythoncode}
L = []
for n in [1, 2, 3, 4, 5, 6, 7, 8, 9]:
    L.append(f(n))
print(L)
\end{pythoncode}

的确可以,但是,从上面的循环代码,能一眼看明白 ``把 f(x) 作用在 list
的每一个元素并把结果生成一个新的 list''吗?

所以,\texttt{map()}作为高阶函数,事实上它把运算规则抽象了,因此,我们不但可以计算简单的
\texttt{f(x)=x\^{}2},还可以计算任意复杂的函数,比如,把这个 list
所有数字转为字符串:

\begin{pythoncode}
>>> list(map(str, [1, 2, 3, 4, 5, 6, 7, 8, 9]))
['1', '2', '3', '4', '5', '6', '7', '8', '9']
\end{pythoncode}

只需要一行代码。

再看\texttt{reduce}的用法。\texttt{reduce}把一个函数作用在一个序列\texttt{{[}x1,\ x2,\ x3,\ ...{]}}上,这个函数必须接收两个参数,\texttt{reduce}把结果继续和序列的下一个元素做累积计算,其效果就是:

\begin{pythoncode}
reduce(f, [x1, x2, x3, x4]) = f(f(f(x1, x2), x3), x4)
\end{pythoncode}

比方说对一个序列求和,就可以用\texttt{reduce}实现:

\begin{pythoncode}
>>> from functools import reduce
>>> def add(x, y):
...     return x + y
...
>>> reduce(add, [1, 3, 5, 7, 9])
25
\end{pythoncode}

当然求和运算可以直接用 Python
内建函数\texttt{sum()},没必要动用\texttt{reduce}。

但是如果要把序列\texttt{{[}1,\ 3,\ 5,\ 7,\ 9{]}}变换成整数\texttt{13579},\texttt{reduce}就可以派上用场:

\begin{pythoncode}
>>> from functools import reduce
>>> def fn(x, y):
...     return x * 10 + y
...
>>> reduce(fn, [1, 3, 5, 7, 9])
13579
\end{pythoncode}

这个例子本身没多大用处,但是,如果考虑到字符串\texttt{str}也是一个序列,对上面的例子稍加改动,配合\texttt{map()},我们就可以写出把\texttt{str}转换为\texttt{int}的函数:

\begin{pythoncode}
>>> from functools import reduce
>>> def fn(x, y):
...     return x * 10 + y
...
>>> def char2num(s):
...     digits = {'0': 0, '1': 1, '2': 2, '3': 3, '4': 4, '5': 5, '6': 6, '7': 7, '8': 8, '9': 9}
...     return digits[s]
...
>>> reduce(fn, map(char2num, '13579'))
13579
\end{pythoncode}

整理成一个\texttt{str2int}的函数就是:

\begin{pythoncode}
from functools import reduce

DIGITS = {'0': 0, '1': 1, '2': 2, '3': 3, '4': 4, '5': 5, '6': 6, '7': 7, '8': 8, '9': 9}

def str2int(s):
    def fn(x, y):
        return x * 10 + y
    def char2num(s):
        return DIGITS[s]
    return reduce(fn, map(char2num, s))
\end{pythoncode}

还可以用 lambda 函数进一步简化成:

\begin{pythoncode}
from functools import reduce

DIGITS = {'0': 0, '1': 1, '2': 2, '3': 3, '4': 4, '5': 5, '6': 6, '7': 7, '8': 8, '9': 9}

def char2num(s):
    return DIGITS[s]

def str2int(s):
    return reduce(lambda x, y: x * 10 + y, map(char2num, s))
\end{pythoncode}

也就是说,假设 Python
没有提供\texttt{int()}函数,你完全可以自己写一个把字符串转化为整数的函数,而且只需要几行代码!

lambda 函数的用法在后面介绍。

\hypertarget{ux7ec3ux4e60}{%
\subsubsection{练习}\label{ux7ec3ux4e60}}

利用\texttt{map()}函数,把用户输入的不规范的英文名字,变为首字母大写,其他小写的规范名字。输入:\texttt{{[}\textquotesingle{}adam\textquotesingle{},\ \textquotesingle{}LISA\textquotesingle{},\ \textquotesingle{}barT\textquotesingle{}{]}},输出:\texttt{{[}\textquotesingle{}Adam\textquotesingle{},\ \textquotesingle{}Lisa\textquotesingle{},\ \textquotesingle{}Bart\textquotesingle{}{]}}:

\begin{pythoncode}
# -*- coding: utf-8 -*-
\end{pythoncode}

\begin{pythoncode}
# 测试:
L1 = ['adam', 'LISA', 'barT']
L2 = list(map(normalize, L1))
print(L2)
\end{pythoncode}

Python 提供的\texttt{sum()}函数可以接受一个 list
并求和,请编写一个\texttt{prod()}函数,可以接受一个 list
并利用\texttt{reduce()}求积:

\begin{pythoncode}
# -*- coding: utf-8 -*-
from functools import reduce
\end{pythoncode}

\begin{pythoncode}
print('3 * 5 * 7 * 9 =', prod([3, 5, 7, 9]))
if prod([3, 5, 7, 9]) == 945:
    print('测试成功!')
else:
    print('测试失败!')
\end{pythoncode}

利用\texttt{map}和\texttt{reduce}编写一个\texttt{str2float}函数,把字符串\texttt{\textquotesingle{}123.456\textquotesingle{}}转换成浮点数\texttt{123.456}:

\begin{pythoncode}
# -*- coding: utf-8 -*-
from functools import reduce

def str2float(s):
\end{pythoncode}

\begin{pythoncode}
print('str2float(\'123.456\') =', str2float('123.456'))
if abs(str2float('123.456') - 123.456) < 0.00001:
    print('测试成功!')
else:
    print('测试失败!')
\end{pythoncode}

\hypertarget{ux53c2ux8003ux4ee3ux7801}{%
\subsubsection{参考代码}\label{ux53c2ux8003ux4ee3ux7801}}

\href{https://github.com/michaelliao/learn-python3/blob/master/samples/functional/do_map.py}{do\_map.py}

\href{https://github.com/michaelliao/learn-python3/blob/master/samples/functional/do_reduce.py}{do\_reduce.py}

