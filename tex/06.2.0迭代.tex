\hypertarget{ux8fedux4ee3}{%
\subsection{迭代}\label{ux8fedux4ee3}}

如果给定一个 list 或 tuple,我们可以通过\texttt{for}循环来遍历这个 list
或 tuple,这种遍历我们称为迭代(Iteration)。

在 Python 中,迭代是通过\texttt{for\ ...\ in}来完成的,而很多语言比如 C
语言,迭代 list 是通过下标完成的,比如 Java 代码:

\begin{pythoncode}
for (i=0; i<list.length; i++) {
    n = list[i];
}
\end{pythoncode}

可以看出,Python 的\texttt{for}循环抽象程度要高于 C
的\texttt{for}循环,因为 Python 的\texttt{for}循环不仅可以用在 list 或
tuple 上,还可以作用在其他可迭代对象上。

list
这种数据类型虽然有下标,但很多其他数据类型是没有下标的,但是,只要是可迭代对象,无论有无下标,都可以迭代,比如
dict 就可以迭代:

\begin{pythoncode}
>>> d = {'a': 1, 'b': 2, 'c': 3}
>>> for key in d:
...     print(key)
...
a
c
b
\end{pythoncode}

因为 dict 的存储不是按照 list
的方式顺序排列,所以,迭代出的结果顺序很可能不一样。

默认情况下,dict 迭代的是 key。如果要迭代
value,可以用\texttt{for\ value\ in\ d.values()},如果要同时迭代 key 和
value,可以用\texttt{for\ k,\ v\ in\ d.items()}。

由于字符串也是可迭代对象,因此,也可以作用于\texttt{for}循环:

\begin{pythoncode}
>>> for ch in 'ABC':
...     print(ch)
...
A
B
C
\end{pythoncode}

所以,当我们使用\texttt{for}循环时,只要作用于一个可迭代对象,\texttt{for}循环就可以正常运行,而我们不太关心该对象究竟是
list 还是其他数据类型。

那么,如何判断一个对象是可迭代对象呢?方法是通过 collections 模块的
Iterable 类型判断:

\begin{pythoncode}
>>> from collections import Iterable
>>> isinstance('abc', Iterable) 
True
>>> isinstance([1,2,3], Iterable) 
True
>>> isinstance(123, Iterable) 
False
\end{pythoncode}

最后一个小问题,如果要对 list 实现类似 Java 那样的下标循环怎么办?Python
内置的\texttt{enumerate}函数可以把一个 list 变成索引 -
元素对,这样就可以在\texttt{for}循环中同时迭代索引和元素本身:

\begin{pythoncode}
>>> for i, value in enumerate(['A', 'B', 'C']):
...     print(i, value)
...
0 A
1 B
2 C
\end{pythoncode}

上面的\texttt{for}循环里,同时引用了两个变量,在 Python
里是很常见的,比如下面的代码:

\begin{pythoncode}
>>> for x, y in [(1, 1), (2, 4), (3, 9)]:
...     print(x, y)
...
1 1
2 4
3 9
\end{pythoncode}

\hypertarget{ux7ec3ux4e60}{%
\subsubsection{练习}\label{ux7ec3ux4e60}}

请使用迭代查找一个 list 中最小和最大值,并返回一个 tuple:

\begin{pythoncode}
# -*- coding: utf-8 -*-
def findMinAndMax(L):
\end{pythoncode}

\begin{pythoncode}
# 测试
if findMinAndMax([]) != (None, None):
    print('测试失败!')
elif findMinAndMax([7]) != (7, 7):
    print('测试失败!')
elif findMinAndMax([7, 1]) != (1, 7):
    print('测试失败!')
elif findMinAndMax([7, 1, 3, 9, 5]) != (1, 9):
    print('测试失败!')
else:
    print('测试成功!')
\end{pythoncode}

\hypertarget{ux5c0fux7ed3}{%
\subsubsection{小结}\label{ux5c0fux7ed3}}

任何可迭代对象都可以作用于\texttt{for}循环,包括我们自定义的数据类型,只要符合迭代条件,就可以使用\texttt{for}循环。

\hypertarget{ux53c2ux8003ux6e90ux7801}{%
\subsubsection{参考源码}\label{ux53c2ux8003ux6e90ux7801}}

\href{https://github.com/michaelliao/learn-python3/blob/master/samples/advance/do_iter.py}{do\_iter.py}

