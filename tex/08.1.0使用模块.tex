\hypertarget{ux4f7fux7528ux6a21ux5757}{%
\subsection{使用模块}\label{ux4f7fux7528ux6a21ux5757}}

Python
本身就内置了很多非常有用的模块,只要安装完毕,这些模块就可以立刻使用。

我们以内建的\texttt{sys}模块为例,编写一个\texttt{hello}的模块:

\begin{pythoncode}
' a test module '

__author__ = 'Michael Liao'

import sys

def test():
    args = sys.argv
    if len(args)==1:
        print('Hello, world!')
    elif len(args)==2:
        print('Hello, %s!' % args[1])
    else:
        print('Too many arguments!')

if __name__=='__main__':
    test()
\end{pythoncode}

第 1 行和第 2 行是标准注释,第 1
行注释可以让这个\texttt{hello.py}文件直接在 Unix/Linux/Mac 上运行,第 2
行注释表示. py 文件本身使用标准 UTF-8 编码;

第 4
行是一个字符串,表示模块的文档注释,任何模块代码的第一个字符串都被视为模块的文档注释;

第 6
行使用\texttt{\_\_author\_\_}变量把作者写进去,这样当你公开源代码后别人就可以瞻仰你的大名;

以上就是 Python
模块的标准文件模板,当然也可以全部删掉不写,但是,按标准办事肯定没错。

后面开始就是真正的代码部分。

你可能注意到了,使用\texttt{sys}模块的第一步,就是导入该模块:

\begin{pythoncode}
import sys
\end{pythoncode}

导入\texttt{sys}模块后,我们就有了变量\texttt{sys}指向该模块,利用\texttt{sys}这个变量,就可以访问\texttt{sys}模块的所有功能。

\texttt{sys}模块有一个\texttt{argv}变量,用 list
存储了命令行的所有参数。\texttt{argv}至少有一个元素,因为第一个参数永远是该.
py 文件的名称,例如:

运行\texttt{python3\ hello.py}获得的\texttt{sys.argv}就是\texttt{{[}\textquotesingle{}hello.py\textquotesingle{}{]}};

运行\texttt{python3\ hello.py\ Michael}获得的\texttt{sys.argv}就是\texttt{{[}\textquotesingle{}hello.py\textquotesingle{},\ \textquotesingle{}Michael\textquotesingle{}{]}}。

最后,注意到这两行代码:

\begin{pythoncode}
if __name__=='__main__':
    test()
\end{pythoncode}

当我们在命令行运行\texttt{hello}模块文件时,Python
解释器把一个特殊变量\texttt{\_\_name\_\_}置为\texttt{\_\_main\_\_},而如果在其他地方导入该\texttt{hello}模块时,\texttt{if}判断将失败,因此,这种\texttt{if}测试可以让一个模块通过命令行运行时执行一些额外的代码,最常见的就是运行测试。

我们可以用命令行运行\texttt{hello.py}看看效果:

\begin{pythoncode}
$ python3 hello.py
Hello, world!
$ python hello.py Michael
Hello, Michael!
\end{pythoncode}

如果启动 Python 交互环境,再导入\texttt{hello}模块:

\begin{pythoncode}
$ python3
Python 3.4.3 (v3.4.3:9b73f1c3e601, Feb 23 2015, 02:52:03) 
[GCC 4.2.1 (Apple Inc. build 5666) (dot 3)] on darwin
Type "help", "copyright", "credits" or "license" for more information.
>>> import hello
>>>
\end{pythoncode}

导入时,没有打印\texttt{Hello,\ word!},因为没有执行\texttt{test()}函数。

调用\texttt{hello.test()}时,才能打印出\texttt{Hello,\ word!}:

\begin{pythoncode}
>>> hello.test()
Hello, world!
\end{pythoncode}

\hypertarget{ux4f5cux7528ux57df}{%
\subsubsection{作用域}\label{ux4f5cux7528ux57df}}

在一个模块中,我们可能会定义很多函数和变量,但有的函数和变量我们希望给别人使用,有的函数和变量我们希望仅仅在模块内部使用。在
Python 中,是通过\texttt{\_}前缀来实现的。

正常的函数和变量名是公开的(public),可以被直接引用,比如:\texttt{abc},\texttt{x123},\texttt{PI}等;

类似\texttt{\_\_xxx\_\_}这样的变量是特殊变量,可以被直接引用,但是有特殊用途,比如上面的\texttt{\_\_author\_\_},\texttt{\_\_name\_\_}就是特殊变量,\texttt{hello}模块定义的文档注释也可以用特殊变量\texttt{\_\_doc\_\_}访问,我们自己的变量一般不要用这种变量名;

类似\texttt{\_xxx}和\texttt{\_\_xxx}这样的函数或变量就是非公开的(private),不应该被直接引用,比如\texttt{\_abc},\texttt{\_\_abc}等;

之所以我们说,private 函数和变量 ``不应该'' 被直接引用,而不是 ``不能''
被直接引用,是因为 Python 并没有一种方法可以完全限制访问 private
函数或变量,但是,从编程习惯上不应该引用 private 函数或变量。

private 函数或变量不应该被别人引用,那它们有什么用呢?请看例子:

\begin{pythoncode}
def _private_1(name):
    return 'Hello, %s' % name

def _private_2(name):
    return 'Hi, %s' % name

def greeting(name):
    if len(name) > 3:
        return _private_1(name)
    else:
        return _private_2(name)
\end{pythoncode}

我们在模块里公开\texttt{greeting()}函数,而把内部逻辑用 private
函数隐藏起来了,这样,调用\texttt{greeting()}函数不用关心内部的 private
函数细节,这也是一种非常有用的代码封装和抽象的方法,即:

外部不需要引用的函数全部定义成 private,只有外部需要引用的函数才定义为
public。

