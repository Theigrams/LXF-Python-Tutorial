\hypertarget{ux9519ux8befux8c03ux8bd5ux548cux6d4bux8bd5}{%
\subsection{错误、调试和测试}\label{ux9519ux8befux8c03ux8bd5ux548cux6d4bux8bd5}}

在程序运行过程中,总会遇到各种各样的错误。

有的错误是程序编写有问题造成的,比如本来应该输出整数结果输出了字符串,这种错误我们通常称之为
bug,bug 是必须修复的。

有的错误是用户输入造成的,比如让用户输入 email
地址,结果得到一个空字符串,这种错误可以通过检查用户输入来做相应的处理。

还有一类错误是完全无法在程序运行过程中预测的,比如写入文件的时候,磁盘满了,写不进去了,或者从网络抓取数据,网络突然断掉了。这类错误也称为异常,在程序中通常是必须处理的,否则,程序会因为各种问题终止并退出。

Python 内置了一套异常处理机制,来帮助我们进行错误处理。

此外,我们也需要跟踪程序的执行,查看变量的值是否正确,这个过程称为调试。Python
的 pdb 可以让我们以单步方式执行代码。

最后,编写测试也很重要。有了良好的测试,就可以在程序修改后反复运行,确保程序输出符合我们编写的测试。

