\hypertarget{ux6587ux4ef6ux8bfbux5199}{%
\subsection{文件读写}\label{ux6587ux4ef6ux8bfbux5199}}

读写文件是最常见的 IO 操作。Python 内置了读写文件的函数,用法和 C
是兼容的。

读写文件前,我们先必须了解一下,在磁盘上读写文件的功能都是由操作系统提供的,现代操作系统不允许普通的程序直接操作磁盘,所以,读写文件就是请求操作系统打开一个文件对象(通常称为文件描述符),然后,通过操作系统提供的接口从这个文件对象中读取数据(读文件),或者把数据写入这个文件对象(写文件)。

\hypertarget{ux8bfbux6587ux4ef6}{%
\subsubsection{读文件}\label{ux8bfbux6587ux4ef6}}

要以读文件的模式打开一个文件对象,使用 Python
内置的\texttt{open()}函数,传入文件名和标示符:

\begin{pythoncode}
>>> f = open('/Users/michael/test.txt', 'r')
\end{pythoncode}

标示符'r'表示读,这样,我们就成功地打开了一个文件。

如果文件不存在,\texttt{open()}函数就会抛出一个\texttt{IOError}的错误,并且给出错误码和详细的信息告诉你文件不存在:

\begin{pythoncode}
>>> f=open('/Users/michael/notfound.txt', 'r')
Traceback (most recent call last):
  File "<stdin>", line 1, in <module>
FileNotFoundError: [Errno 2] No such file or directory: '/Users/michael/notfound.txt'
\end{pythoncode}

如果文件打开成功,接下来,调用\texttt{read()}方法可以一次读取文件的全部内容,Python
把内容读到内存,用一个\texttt{str}对象表示:

\begin{pythoncode}
>>> f.read()
'Hello, world!'
\end{pythoncode}

最后一步是调用\texttt{close()}方法关闭文件。文件使用完毕后必须关闭,因为文件对象会占用操作系统的资源,并且操作系统同一时间能打开的文件数量也是有限的:

\begin{pythoncode}
>>> f.close()
\end{pythoncode}

由于文件读写时都有可能产生\texttt{IOError},一旦出错,后面的\texttt{f.close()}就不会调用。所以,为了保证无论是否出错都能正确地关闭文件,我们可以使用\texttt{try\ ...\ finally}来实现:

\begin{pythoncode}
try:
    f = open('/path/to/file', 'r')
    print(f.read())
finally:
    if f:
        f.close()
\end{pythoncode}

但是每次都这么写实在太繁琐,所以,Python
引入了\texttt{with}语句来自动帮我们调用\texttt{close()}方法:

\begin{pythoncode}
with open('/path/to/file', 'r') as f:
    print(f.read())
\end{pythoncode}

这和前面的\texttt{try\ ...\ finally}是一样的,但是代码更佳简洁,并且不必调用\texttt{f.close()}方法。

调用\texttt{read()}会一次性读取文件的全部内容,如果文件有
10G,内存就爆了,所以,要保险起见,可以反复调用\texttt{read(size)}方法,每次最多读取
size
个字节的内容。另外,调用\texttt{readline()}可以每次读取一行内容,调用\texttt{readlines()}一次读取所有内容并按行返回\texttt{list}。因此,要根据需要决定怎么调用。

如果文件很小,\texttt{read()}一次性读取最方便;如果不能确定文件大小,反复调用\texttt{read(size)}比较保险;如果是配置文件,调用\texttt{readlines()}最方便:

\begin{pythoncode}
for line in f.readlines():
    print(line.strip()) 
\end{pythoncode}

\hypertarget{file-like-object}{%
\subsubsection{file-like Object}\label{file-like-object}}

像\texttt{open()}函数返回的这种有个\texttt{read()}方法的对象,在 Python
中统称为 file-like Object。除了 file
外,还可以是内存的字节流,网络流,自定义流等等。file-like Object
不要求从特定类继承,只要写个\texttt{read()}方法就行。

\texttt{StringIO}就是在内存中创建的 file-like Object,常用作临时缓冲。

\hypertarget{ux4e8cux8fdbux5236ux6587ux4ef6}{%
\subsubsection{二进制文件}\label{ux4e8cux8fdbux5236ux6587ux4ef6}}

前面讲的默认都是读取文本文件,并且是 UTF-8
编码的文本文件。要读取二进制文件,比如图片、视频等等,用\texttt{\textquotesingle{}rb\textquotesingle{}}模式打开文件即可:

\begin{pythoncode}
>>> f = open('/Users/michael/test.jpg', 'rb')
>>> f.read()
b'\xff\xd8\xff\xe1\x00\x18Exif\x00\x00...' 
\end{pythoncode}

\hypertarget{ux5b57ux7b26ux7f16ux7801}{%
\subsubsection{字符编码}\label{ux5b57ux7b26ux7f16ux7801}}

要读取非 UTF-8
编码的文本文件,需要给\texttt{open()}函数传入\texttt{encoding}参数,例如,读取
GBK 编码的文件:

\begin{pythoncode}
>>> f = open('/Users/michael/gbk.txt', 'r', encoding='gbk')
>>> f.read()
'测试'
\end{pythoncode}

遇到有些编码不规范的文件,你可能会遇到\texttt{UnicodeDecodeError},因为在文本文件中可能夹杂了一些非法编码的字符。遇到这种情况,\texttt{open()}函数还接收一个\texttt{errors}参数,表示如果遇到编码错误后如何处理。最简单的方式是直接忽略:

\begin{pythoncode}
>>> f = open('/Users/michael/gbk.txt', 'r', encoding='gbk', errors='ignore')
\end{pythoncode}

\hypertarget{ux5199ux6587ux4ef6}{%
\subsubsection{写文件}\label{ux5199ux6587ux4ef6}}

写文件和读文件是一样的,唯一区别是调用\texttt{open()}函数时,传入标识符\texttt{\textquotesingle{}w\textquotesingle{}}或者\texttt{\textquotesingle{}wb\textquotesingle{}}表示写文本文件或写二进制文件:

\begin{pythoncode}
>>> f = open('/Users/michael/test.txt', 'w')
>>> f.write('Hello, world!')
>>> f.close()
\end{pythoncode}

你可以反复调用\texttt{write()}来写入文件,但是务必要调用\texttt{f.close()}来关闭文件。当我们写文件时,操作系统往往不会立刻把数据写入磁盘,而是放到内存缓存起来,空闲的时候再慢慢写入。只有调用\texttt{close()}方法时,操作系统才保证把没有写入的数据全部写入磁盘。忘记调用\texttt{close()}的后果是数据可能只写了一部分到磁盘,剩下的丢失了。所以,还是用\texttt{with}语句来得保险:

\begin{pythoncode}
with open('/Users/michael/test.txt', 'w') as f:
    f.write('Hello, world!')
\end{pythoncode}

要写入特定编码的文本文件,请给\texttt{open()}函数传入\texttt{encoding}参数,将字符串自动转换成指定编码。

细心的童鞋会发现,以\texttt{\textquotesingle{}w\textquotesingle{}}模式写入文件时,如果文件已存在,会直接覆盖(相当于删掉后新写入一个文件)。如果我们希望追加到文件末尾怎么办?可以传入\texttt{\textquotesingle{}a\textquotesingle{}}以追加(append)模式写入。

所有模式的定义及含义可以参考 Python
的\href{https://docs.python.org/3/library/functions.html\#open}{官方文档}。

\hypertarget{ux7ec3ux4e60}{%
\subsubsection{练习}\label{ux7ec3ux4e60}}

请将本地一个文本文件读为一个 str 并打印出来:

\begin{pythoncode}
# -*- coding: utf-8 -*-
\end{pythoncode}

\begin{pythoncode}
# 运行代码观察结果
\end{pythoncode}

\hypertarget{ux5c0fux7ed3}{%
\subsubsection{小结}\label{ux5c0fux7ed3}}

在 Python
中,文件读写是通过\texttt{open()}函数打开的文件对象完成的。使用\texttt{with}语句操作文件
IO 是个好习惯。

\hypertarget{ux53c2ux8003ux6e90ux7801}{%
\subsubsection{参考源码}\label{ux53c2ux8003ux6e90ux7801}}

\href{https://github.com/michaelliao/learn-python3/tree/master/samples/io/with_file.py}{with\_file.py}

