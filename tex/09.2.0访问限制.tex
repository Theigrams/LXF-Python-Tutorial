\hypertarget{ux8bbfux95eeux9650ux5236}{%
\subsection{访问限制}\label{ux8bbfux95eeux9650ux5236}}

在 Class
内部,可以有属性和方法,而外部代码可以通过直接调用实例变量的方法来操作数据,这样,就隐藏了内部的复杂逻辑。

但是,从前面 Student
类的定义来看,外部代码还是可以自由地修改一个实例的\texttt{name}、\texttt{score}属性:

\begin{pythoncode}
>>> bart = Student('Bart Simpson', 59)
>>> bart.score
59
>>> bart.score = 99
>>> bart.score
99
\end{pythoncode}

如果要让内部属性不被外部访问,可以把属性的名称前加上两个下划线\texttt{\_\_},在
Python
中,实例的变量名如果以\texttt{\_\_}开头,就变成了一个私有变量(private),只有内部可以访问,外部不能访问,所以,我们把
Student 类改一改:

\begin{pythoncode}
class Student(object):

    def __init__(self, name, score):
        self.__name = name
        self.__score = score

    def print_score(self):
        print('%s: %s' % (self.__name, self.__score))
\end{pythoncode}

改完后,对于外部代码来说,没什么变动,但是已经无法从外部访问\texttt{实例变量.\_\_name}和\texttt{实例变量.\_\_score}了:

\begin{pythoncode}
>>> bart = Student('Bart Simpson', 59)
>>> bart.__name
Traceback (most recent call last):
  File "<stdin>", line 1, in <module>
AttributeError: 'Student' object has no attribute '__name'
\end{pythoncode}

这样就确保了外部代码不能随意修改对象内部的状态,这样通过访问限制的保护,代码更加健壮。

但是如果外部代码要获取 name 和 score 怎么办?可以给 Student
类增加\texttt{get\_name}和\texttt{get\_score}这样的方法:

\begin{pythoncode}
class Student(object):
    ...

    def get_name(self):
        return self.__name

    def get_score(self):
        return self.__score
\end{pythoncode}

如果又要允许外部代码修改 score 怎么办?可以再给 Student
类增加\texttt{set\_score}方法:

\begin{pythoncode}
class Student(object):
    ...

    def set_score(self, score):
        self.__score = score
\end{pythoncode}

你也许会问,原先那种直接通过\texttt{bart.score\ =\ 99}也可以修改啊,为什么要定义一个方法大费周折?因为在方法中,可以对参数做检查,避免传入无效的参数:

\begin{pythoncode}
class Student(object):
    ...

    def set_score(self, score):
        if 0 <= score <= 100:
            self.__score = score
        else:
            raise ValueError('bad score')
\end{pythoncode}

需要注意的是,在 Python
中,变量名类似\texttt{\_\_xxx\_\_}的,也就是以双下划线开头,并且以双下划线结尾的,是特殊变量,特殊变量是可以直接访问的,不是
private
变量,所以,不能用\texttt{\_\_name\_\_}、\texttt{\_\_score\_\_}这样的变量名。

有些时候,你会看到以一个下划线开头的实例变量名,比如\texttt{\_name},这样的实例变量外部是可以访问的,但是,按照约定俗成的规定,当你看到这样的变量时,意思就是,``虽然我可以被访问,但是,请把我视为私有变量,不要随意访问''。

双下划线开头的实例变量是不是一定不能从外部访问呢?其实也不是。不能直接访问\texttt{\_\_name}是因为
Python
解释器对外把\texttt{\_\_name}变量改成了\texttt{\_Student\_\_name},所以,仍然可以通过\texttt{\_Student\_\_name}来访问\texttt{\_\_name}变量:

\begin{pythoncode}
>>> bart._Student__name
'Bart Simpson'
\end{pythoncode}

但是强烈建议你不要这么干,因为不同版本的 Python
解释器可能会把\texttt{\_\_name}改成不同的变量名。

总的来说就是,Python 本身没有任何机制阻止你干坏事,一切全靠自觉。

最后注意下面的这种\_错误写法\_:

\begin{pythoncode}
>>> bart = Student('Bart Simpson', 59)
>>> bart.get_name()
'Bart Simpson'
>>> bart.__name = 'New Name' 
>>> bart.__name
'New Name'
\end{pythoncode}

表面上看,外部代码 ``成功''
地设置了\texttt{\_\_name}变量,但实际上这个\texttt{\_\_name}变量和 class
内部的\texttt{\_\_name}变量\_不是\_一个变量!内部的\texttt{\_\_name}变量已经被
Python
解释器自动改成了\texttt{\_Student\_\_name},而外部代码给\texttt{bart}新增了一个\texttt{\_\_name}变量。不信试试:

\begin{pythoncode}
>>> bart.get_name() 
'Bart Simpson'
\end{pythoncode}

\hypertarget{ux7ec3ux4e60}{%
\subsubsection{练习}\label{ux7ec3ux4e60}}

请把下面的\texttt{Student}对象的\texttt{gender}字段对外隐藏起来,用\texttt{get\_gender()}和\texttt{set\_gender()}代替,并检查参数有效性:

\begin{pythoncode}
# -*- coding: utf-8 -*-
\end{pythoncode}

\begin{pythoncode}
# 测试:
bart = Student('Bart', 'male')
if bart.get_gender() != 'male':
    print('测试失败!')
else:
    bart.set_gender('female')
    if bart.get_gender() != 'female':
        print('测试失败!')
    else:
        print('测试成功!')
\end{pythoncode}

\hypertarget{ux53c2ux8003ux6e90ux7801}{%
\subsubsection{参考源码}\label{ux53c2ux8003ux6e90ux7801}}

\href{https://github.com/michaelliao/learn-python3/blob/master/samples/oop_basic/protected_student.py}{protected\_student.py}

