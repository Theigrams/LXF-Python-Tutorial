\hypertarget{async-await}{%
\subsection{async await}\label{async-await}}

用\texttt{asyncio}提供的\texttt{@asyncio.coroutine}可以把一个 generator
标记为 coroutine 类型,然后在 coroutine
内部用\texttt{yield\ from}调用另一个 coroutine 实现异步操作。

为了简化并更好地标识异步 IO,从 Python 3.5
开始引入了新的语法\texttt{async}和\texttt{await},可以让 coroutine
的代码更简洁易读。

请注意,\texttt{async}和\texttt{await}是针对 coroutine
的新语法,要使用新的语法,只需要做两步简单的替换:

\begin{enumerate}
\def\labelenumi{\arabic{enumi}.}
\item
  把\texttt{@asyncio.coroutine}替换为\texttt{async};
\item
  把\texttt{yield\ from}替换为\texttt{await}。
\end{enumerate}

让我们对比一下上一节的代码:

\begin{pythoncode}
@asyncio.coroutine
def hello():
    print("Hello world!")
    r = yield from asyncio.sleep(1)
    print("Hello again!")
\end{pythoncode}

用新语法重新编写如下:

\begin{pythoncode}
async def hello():
    print("Hello world!")
    r = await asyncio.sleep(1)
    print("Hello again!")
\end{pythoncode}

剩下的代码保持不变。

\hypertarget{ux5c0fux7ed3}{%
\subsubsection{小结}\label{ux5c0fux7ed3}}

Python 从 3.5
版本开始为\texttt{asyncio}提供了\texttt{async}和\texttt{await}的新语法;

注意新语法只能用在 Python 3.5 以及后续版本,如果使用 3.4
版本,则仍需使用上一节的方案。

\hypertarget{ux7ec3ux4e60}{%
\subsubsection{练习}\label{ux7ec3ux4e60}}

将上一节的异步获取 sina、sohu 和 163 的网站首页源码用新语法重写并运行。

\hypertarget{ux53c2ux8003ux6e90ux7801}{%
\subsubsection{参考源码}\label{ux53c2ux8003ux6e90ux7801}}

\href{https://github.com/michaelliao/learn-python3/blob/master/samples/async/async_hello2.py}{async\_hello2.py}

\href{https://github.com/michaelliao/learn-python3/blob/master/samples/async/async_wget2.py}{async\_wget2.py}

