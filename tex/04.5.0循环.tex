\hypertarget{ux5faaux73af}{%
\subsection{循环}\label{ux5faaux73af}}

\hypertarget{ux5faaux73af-1}{%
\subsubsection{循环}\label{ux5faaux73af-1}}

要计算 1+2+3,我们可以直接写表达式:

\begin{pythoncode}
>>> 1 + 2 + 3
6
\end{pythoncode}

要计算 1+2+3+\ldots+10,勉强也能写出来。

但是,要计算 1+2+3+\ldots+10000,直接写表达式就不可能了。

为了让计算机能计算成千上万次的重复运算,我们就需要循环语句。

Python 的循环有两种,一种是 for\ldots in 循环,依次把 list 或 tuple
中的每个元素迭代出来,看例子:

\begin{pythoncode}
names = ['Michael', 'Bob', 'Tracy']
for name in names:
    print(name)
\end{pythoncode}

执行这段代码,会依次打印\texttt{names}的每一个元素:

\begin{pythoncode}
Michael
Bob
Tracy
\end{pythoncode}

所以\texttt{for\ x\ in\ ...}循环就是把每个元素代入变量\texttt{x},然后执行缩进块的语句。

再比如我们想计算 1-10 的整数之和,可以用一个\texttt{sum}变量做累加:

\begin{pythoncode}
sum = 0
for x in [1, 2, 3, 4, 5, 6, 7, 8, 9, 10]:
    sum = sum + x
print(sum)
\end{pythoncode}

如果要计算 1-100 的整数之和,从 1 写到 100 有点困难,幸好 Python
提供一个\texttt{range()}函数,可以生成一个整数序列,再通过\texttt{list()}函数可以转换为
list。比如\texttt{range(5)}生成的序列是从 0 开始小于 5 的整数:

\begin{pythoncode}
>>> list(range(5))
[0, 1, 2, 3, 4]
\end{pythoncode}

\texttt{range(101)}就可以生成 0-100 的整数序列,计算如下:

\begin{pythoncode}
# -*- coding: utf-8 -*-
\end{pythoncode}

请自行运行上述代码,看看结果是不是当年高斯同学心算出的 5050。

第二种循环是 while
循环,只要条件满足,就不断循环,条件不满足时退出循环。比如我们要计算 100
以内所有奇数之和,可以用 while 循环实现:

\begin{pythoncode}
sum = 0
n = 99
while n > 0:
    sum = sum + n
    n = n - 2
print(sum)
\end{pythoncode}

在循环内部变量\texttt{n}不断自减,直到变为\texttt{-1}时,不再满足 while
条件,循环退出。

\hypertarget{ux7ec3ux4e60}{%
\subsubsection{练习}\label{ux7ec3ux4e60}}

请利用循环依次对 list 中的每个名字打印出\texttt{Hello,\ xxx!}:

\begin{pythoncode}
# -*- coding: utf-8 -*-
\end{pythoncode}

\hypertarget{break}{%
\subsubsection{break}\label{break}}

在循环中,\texttt{break}语句可以提前退出循环。例如,本来要循环打印
1~100 的数字:

\begin{pythoncode}
n = 1
while n <= 100:
    print(n)
    n = n + 1
print('END')
\end{pythoncode}

上面的代码可以打印出 1\textasciitilde100。

如果要提前结束循环,可以用\texttt{break}语句:

\begin{pythoncode}
n = 1
while n <= 100:
    if n > 10: 
        break 
    print(n)
    n = n + 1
print('END')
\end{pythoncode}

执行上面的代码可以看到,打印出 1\textasciitilde10
后,紧接着打印\texttt{END},程序结束。

可见\texttt{break}的作用是提前结束循环。

\hypertarget{continue}{%
\subsubsection{continue}\label{continue}}

在循环过程中,也可以通过\texttt{continue}语句,跳过当前的这次循环,直接开始下一次循环。

\begin{pythoncode}
n = 0
while n < 10:
    n = n + 1
    print(n)
\end{pythoncode}

上面的程序可以打印出
1~10。但是,如果我们想只打印奇数,可以用\texttt{continue}语句跳过某些循环:

\begin{pythoncode}
n = 0
while n < 10:
    n = n + 1
    if n % 2 == 0: 
        continue 
    print(n)
\end{pythoncode}

执行上面的代码可以看到,打印的不再是 1~10,而是 1,3,5,7,9。

可见\texttt{continue}的作用是提前结束本轮循环,并直接开始下一轮循环。

\hypertarget{ux5c0fux7ed3}{%
\subsubsection{小结}\label{ux5c0fux7ed3}}

循环是让计算机做重复任务的有效的方法。

\texttt{break}语句可以在循环过程中直接退出循环,而\texttt{continue}语句可以提前结束本轮循环,并直接开始下一轮循环。这两个语句通常都\_必须\_配合\texttt{if}语句使用。

\emph{要特别注意},不要滥用\texttt{break}和\texttt{continue}语句。\texttt{break}和\texttt{continue}会造成代码执行逻辑分叉过多,容易出错。大多数循环并不需要用到\texttt{break}和\texttt{continue}语句,上面的两个例子,都可以通过改写循环条件或者修改循环逻辑,去掉\texttt{break}和\texttt{continue}语句。

有些时候,如果代码写得有问题,会让程序陷入
``死循环'',也就是永远循环下去。这时可以用\texttt{Ctrl+C}退出程序,或者强制结束
Python 进程。

请试写一个死循环程序。

\hypertarget{ux53c2ux8003ux6e90ux7801}{%
\subsubsection{参考源码}\label{ux53c2ux8003ux6e90ux7801}}

\href{https://github.com/michaelliao/learn-python3/blob/master/samples/basic/do_for.py}{do\_for.py}

\href{https://github.com/michaelliao/learn-python3/blob/master/samples/basic/do_while.py}{do\_while.py}

