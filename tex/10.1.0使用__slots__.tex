\hypertarget{ux4f7fux7528__slots__}{%
\subsection{使用\_\_slots\_\_}\label{ux4f7fux7528__slots__}}

正常情况下,当我们定义了一个 class,创建了一个 class
的实例后,我们可以给该实例绑定任何属性和方法,这就是动态语言的灵活性。先定义
class:

\begin{pythoncode}
class Student(object):
    pass
\end{pythoncode}

然后,尝试给实例绑定一个属性:

\begin{pythoncode}
>>> s = Student()
>>> s.name = 'Michael' 
>>> print(s.name)
Michael
\end{pythoncode}

还可以尝试给实例绑定一个方法:

\begin{pythoncode}
>>> def set_age(self, age): 
...     self.age = age
...
>>> from types import MethodType
>>> s.set_age = MethodType(set_age, s) 
>>> s.set_age(25) 
>>> s.age 
25
\end{pythoncode}

但是,给一个实例绑定的方法,对另一个实例是不起作用的:

\begin{pythoncode}
>>> s2 = Student() # 创建新的实例
>>> s2.set_age(25) # 尝试调用方法
Traceback (most recent call last):
  File "<stdin>", line 1, in <module>
AttributeError: 'Student' object has no attribute 'set_age'
\end{pythoncode}

为了给所有实例都绑定方法,可以给 class 绑定方法:

\begin{pythoncode}
>>> def set_score(self, score):
...     self.score = score
...
>>> Student.set_score = set_score
\end{pythoncode}

给 class 绑定方法后,所有实例均可调用:

\begin{pythoncode}
>>> s.set_score(100)
>>> s.score
100
>>> s2.set_score(99)
>>> s2.score
99
\end{pythoncode}

通常情况下,上面的\texttt{set\_score}方法可以直接定义在 class
中,但动态绑定允许我们在程序运行的过程中动态给 class
加上功能,这在静态语言中很难实现。

\hypertarget{ux4f7fux7528__slots__-1}{%
\subsubsection{使用\_\_slots\_\_}\label{ux4f7fux7528__slots__-1}}

但是,如果我们想要限制实例的属性怎么办?比如,只允许对 Student
实例添加\texttt{name}和\texttt{age}属性。

为了达到限制的目的,Python 允许在定义 class
的时候,定义一个特殊的\texttt{\_\_slots\_\_}变量,来限制该 class
实例能添加的属性:

\begin{pythoncode}
class Student(object):
    __slots__ = ('name', 'age') 
\end{pythoncode}

然后,我们试试:

\begin{pythoncode}
>>> s = Student() 
>>> s.name = 'Michael' 
>>> s.age = 25 
>>> s.score = 99 
Traceback (most recent call last):
  File "<stdin>", line 1, in <module>
AttributeError: 'Student' object has no attribute 'score'
\end{pythoncode}

由于\texttt{\textquotesingle{}score\textquotesingle{}}没有被放到\texttt{\_\_slots\_\_}中,所以不能绑定\texttt{score}属性,试图绑定\texttt{score}将得到\texttt{AttributeError}的错误。

使用\texttt{\_\_slots\_\_}要注意,\texttt{\_\_slots\_\_}定义的属性仅对当前类实例起作用,对继承的子类是不起作用的:

\begin{pythoncode}
>>> class GraduateStudent(Student):
...     pass
...
>>> g = GraduateStudent()
>>> g.score = 9999
\end{pythoncode}

除非在子类中也定义\texttt{\_\_slots\_\_},这样,子类实例允许定义的属性就是自身的\texttt{\_\_slots\_\_}加上父类的\texttt{\_\_slots\_\_}。

\hypertarget{ux53c2ux8003ux6e90ux7801}{%
\subsubsection{参考源码}\label{ux53c2ux8003ux6e90ux7801}}

\href{https://github.com/michaelliao/learn-python3/blob/master/samples/oop_advance/use_slots.py}{use\_slots.py}

