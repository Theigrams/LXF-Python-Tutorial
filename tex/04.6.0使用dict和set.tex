\hypertarget{ux4f7fux7528-dict-ux548c-set}{%
\subsection{使用 dict 和 set}\label{ux4f7fux7528-dict-ux548c-set}}

\hypertarget{dict}{%
\subsubsection{dict}\label{dict}}

Python 内置了字典:dict 的支持,dict 全称 dictionary,在其他语言中也称为
map,使用键 - 值(key-value)存储,具有极快的查找速度。

举个例子,假设要根据同学的名字查找对应的成绩,如果用 list 实现,需要两个
list:

\begin{pythoncode}
names = ['Michael', 'Bob', 'Tracy']
scores = [95, 75, 85]
\end{pythoncode}

给定一个名字,要查找对应的成绩,就先要在 names 中找到对应的位置,再从
scores 取出对应的成绩,list 越长,耗时越长。

如果用 dict 实现,只需要一个 ``名字''-``成绩''
的对照表,直接根据名字查找成绩,无论这个表有多大,查找速度都不会变慢。用
Python 写一个 dict 如下:

\begin{pythoncode}
>>> d = {'Michael': 95, 'Bob': 75, 'Tracy': 85}
>>> d['Michael']
95
\end{pythoncode}

为什么 dict 查找速度这么快?因为 dict
的实现原理和查字典是一样的。假设字典包含了 1
万个汉字,我们要查某一个字,一个办法是把字典从第一页往后翻,直到找到我们想要的字为止,这种方法就是在
list 中查找元素的方法,list 越大,查找越慢。

第二种方法是先在字典的索引表里(比如部首表)查这个字对应的页码,然后直接翻到该页,找到这个字。无论找哪个字,这种查找速度都非常快,不会随着字典大小的增加而变慢。

dict
就是第二种实现方式,给定一个名字,比如\texttt{\textquotesingle{}Michael\textquotesingle{}},dict
在内部就可以直接计算出\texttt{Michael}对应的存放成绩的
``页码'',也就是\texttt{95}这个数字存放的内存地址,直接取出来,所以速度非常快。

你可以猜到,这种 key-value 存储方式,在放进去的时候,必须根据 key 算出
value 的存放位置,这样,取的时候才能根据 key 直接拿到 value。

把数据放入 dict 的方法,除了初始化时指定外,还可以通过 key 放入:

\begin{pythoncode}
>>> d['Adam'] = 67
>>> d['Adam']
67
\end{pythoncode}

由于一个 key 只能对应一个 value,所以,多次对一个 key 放入
value,后面的值会把前面的值冲掉:

\begin{pythoncode}
>>> d['Jack'] = 90
>>> d['Jack']
90
>>> d['Jack'] = 88
>>> d['Jack']
88
\end{pythoncode}

如果 key 不存在,dict 就会报错:

\begin{pythoncode}
>>> d['Thomas']
Traceback (most recent call last):
  File "<stdin>", line 1, in <module>
KeyError: 'Thomas'
\end{pythoncode}

要避免 key 不存在的错误,有两种办法,一是通过\texttt{in}判断 key
是否存在:

\begin{pythoncode}
>>> 'Thomas' in d
False
\end{pythoncode}

二是通过 dict 提供的\texttt{get()}方法,如果 key
不存在,可以返回\texttt{None},或者自己指定的 value:

\begin{pythoncode}
>>> d.get('Thomas')
>>> d.get('Thomas', -1)
-1
\end{pythoncode}

注意:返回\texttt{None}的时候 Python 的交互环境不显示结果。

要删除一个 key,用\texttt{pop(key)}方法,对应的 value 也会从 dict
中删除:

\begin{pythoncode}
>>> d.pop('Bob')
75
>>> d
{'Michael': 95, 'Tracy': 85}
\end{pythoncode}

请务必注意,dict 内部存放的顺序和 key 放入的顺序是没有关系的。

和 list 比较,dict 有以下几个特点:

\begin{enumerate}
\def\labelenumi{\arabic{enumi}.}
\item
  查找和插入的速度极快,不会随着 key 的增加而变慢;
\item
  需要占用大量的内存,内存浪费多。
\end{enumerate}

而 list 相反:

\begin{enumerate}
\def\labelenumi{\arabic{enumi}.}
\item
  查找和插入的时间随着元素的增加而增加;
\item
  占用空间小,浪费内存很少。
\end{enumerate}

所以,dict 是用空间来换取时间的一种方法。

dict 可以用在需要高速查找的很多地方,在 Python
代码中几乎无处不在,正确使用 dict 非常重要,需要牢记的第一条就是 dict 的
key 必须是\textbf{不可变对象}。

这是因为 dict 根据 key 来计算 value 的存储位置,如果每次计算相同的 key
得出的结果不同,那 dict 内部就完全混乱了。这个通过 key
计算位置的算法称为哈希算法(Hash)。

要保证 hash 的正确性,作为 key 的对象就不能变。在 Python
中,字符串、整数等都是不可变的,因此,可以放心地作为 key。而 list
是可变的,就不能作为 key:

\begin{pythoncode}
>>> key = [1, 2, 3]
>>> d[key] = 'a list'
Traceback (most recent call last):
  File "<stdin>", line 1, in <module>
TypeError: unhashable type: 'list'
\end{pythoncode}

\hypertarget{set}{%
\subsubsection{set}\label{set}}

set 和 dict 类似,也是一组 key 的集合,但不存储 value。由于 key
不能重复,所以,在 set 中,没有重复的 key。

要创建一个 set,需要提供一个 list 作为输入集合:

\begin{pythoncode}
>>> s = set([1, 2, 3])
>>> s
{1, 2, 3}
\end{pythoncode}

注意,传入的参数\texttt{{[}1,\ 2,\ 3{]}}是一个
list,而显示的\texttt{\{1,\ 2,\ 3\}}只是告诉你这个 set 内部有 1,2,3 这
3 个元素,显示的顺序也不表示 set 是有序的。。

重复元素在 set 中自动被过滤:

\begin{pythoncode}
>>> s = set([1, 1, 2, 2, 3, 3])
>>> s
{1, 2, 3}
\end{pythoncode}

通过\texttt{add(key)}方法可以添加元素到 set
中,可以重复添加,但不会有效果:

\begin{pythoncode}
>>> s.add(4)
>>> s
{1, 2, 3, 4}
>>> s.add(4)
>>> s
{1, 2, 3, 4}
\end{pythoncode}

通过\texttt{remove(key)}方法可以删除元素:

\begin{pythoncode}
>>> s.remove(4)
>>> s
{1, 2, 3}
\end{pythoncode}

set 可以看成数学意义上的无序和无重复元素的集合,因此,两个 set
可以做数学意义上的交集、并集等操作:

\begin{pythoncode}
>>> s1 = set([1, 2, 3])
>>> s2 = set([2, 3, 4])
>>> s1 & s2
{2, 3}
>>> s1 | s2
{1, 2, 3, 4}
\end{pythoncode}

set 和 dict 的唯一区别仅在于没有存储对应的 value,但是,set 的原理和
dict
一样,所以,同样不可以放入可变对象,因为无法判断两个可变对象是否相等,也就无法保证
set 内部 ``不会有重复元素''。试试把 list 放入 set,看看是否会报错。

\hypertarget{ux518dux8baeux4e0dux53efux53d8ux5bf9ux8c61}{%
\subsubsection{再议不可变对象}\label{ux518dux8baeux4e0dux53efux53d8ux5bf9ux8c61}}

上面我们讲了,str 是不变对象,而 list 是可变对象。

对于可变对象,比如 list,对 list 进行操作,list
内部的内容是会变化的,比如:

\begin{pythoncode}
>>> a = ['c', 'b', 'a']
>>> a.sort()
>>> a
['a', 'b', 'c']
\end{pythoncode}

而对于不可变对象,比如 str,对 str 进行操作呢:

\begin{pythoncode}
>>> a = 'abc'
>>> a.replace('a', 'A')
'Abc'
>>> a
'abc'
\end{pythoncode}

虽然字符串有个\texttt{replace()}方法,也确实变出了\texttt{\textquotesingle{}Abc\textquotesingle{}},但变量\texttt{a}最后仍是\texttt{\textquotesingle{}abc\textquotesingle{}},应该怎么理解呢?

我们先把代码改成下面这样:

\begin{pythoncode}
>>> a = 'abc'
>>> b = a.replace('a', 'A')
>>> b
'Abc'
>>> a
'abc'
\end{pythoncode}

要始终牢记的是,\texttt{a}是变量,而\texttt{\textquotesingle{}abc\textquotesingle{}}才是字符串对象!有些时候,我们经常说,对象\texttt{a}的内容是\texttt{\textquotesingle{}abc\textquotesingle{}},但其实是指,\texttt{a}本身是一个变量,它指向的对象的内容才是\texttt{\textquotesingle{}abc\textquotesingle{}}:

\begin{pythoncode}
┌───┐                  ┌───────┐
│ a │─────────────────>│ 'abc' │
└───┘                  └───────┘
\end{pythoncode}

当我们调用\texttt{a.replace(\textquotesingle{}a\textquotesingle{},\ \textquotesingle{}A\textquotesingle{})}时,实际上调用方法\texttt{replace}是作用在字符串对象\texttt{\textquotesingle{}abc\textquotesingle{}}上的,而这个方法虽然名字叫\texttt{replace},但却没有改变字符串\texttt{\textquotesingle{}abc\textquotesingle{}}的内容。相反,\texttt{replace}方法创建了一个新字符串\texttt{\textquotesingle{}Abc\textquotesingle{}}并返回,如果我们用变量\texttt{b}指向该新字符串,就容易理解了,变量\texttt{a}仍指向原有的字符串\texttt{\textquotesingle{}abc\textquotesingle{}},但变量\texttt{b}却指向新字符串\texttt{\textquotesingle{}Abc\textquotesingle{}}了:

\begin{pythoncode}
┌───┐                  ┌───────┐
│ a │─────────────────>│ 'abc' │
└───┘                  └───────┘
┌───┐                  ┌───────┐
│ b │─────────────────>│ 'Abc' │
└───┘                  └───────┘
\end{pythoncode}

所以,对于不变对象来说,调用对象自身的任意方法,也不会改变该对象自身的内容。相反,这些方法会创建新的对象并返回,这样,就保证了不可变对象本身永远是不可变的。

\hypertarget{ux5c0fux7ed3}{%
\subsubsection{小结}\label{ux5c0fux7ed3}}

使用 key-value 存储结构的 dict 在 Python 中非常有用,选择不可变对象作为
key 很重要,最常用的 key 是字符串。

tuple
虽然是不变对象,但试试把\texttt{(1,\ 2,\ 3)}和\texttt{(1,\ {[}2,\ 3{]})}放入
dict 或 set 中,并解释结果。

\hypertarget{ux53c2ux8003ux6e90ux7801}{%
\subsubsection{参考源码}\label{ux53c2ux8003ux6e90ux7801}}

\href{https://github.com/michaelliao/learn-python3/blob/master/samples/basic/the_dict.py}{the\_dict.py}

\href{https://github.com/michaelliao/learn-python3/blob/master/samples/basic/the_set.py}{the\_set.py}

