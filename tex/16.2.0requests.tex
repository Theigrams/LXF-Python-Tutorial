\hypertarget{requests}{%
\subsection{requests}\label{requests}}

我们已经讲解了 Python 内置的 urllib
模块,用于访问网络资源。但是,它用起来比较麻烦,而且,缺少很多实用的高级功能。

更好的方案是使用 requests。它是一个 Python 第三方库,处理 URL
资源特别方便。

\hypertarget{ux5b89ux88c5-requests}{%
\subsubsection{安装 requests}\label{ux5b89ux88c5-requests}}

如果安装了 Anaconda,requests 就已经可用了。否则,需要在命令行下通过 pip
安装:

\begin{pythoncode}
$ pip install requests
\end{pythoncode}

如果遇到 Permission denied 安装失败,请加上 sudo 重试。

\hypertarget{ux4f7fux7528-requests}{%
\subsubsection{使用 requests}\label{ux4f7fux7528-requests}}

要通过 GET 访问一个页面,只需要几行代码:

\begin{pythoncode}
>>> import requests
>>> r = requests.get('https://www.douban.com/') # 豆瓣首页
>>> r.status_code
200
>>> r.text
r.text
'<!DOCTYPE HTML>\n<html>\n<head>\n<meta 提供图书、电影、音乐唱片的推荐、评论和...'
\end{pythoncode}

对于带参数的 URL,传入一个 dict 作为\texttt{params}参数:

\begin{pythoncode}
>>> r = requests.get('https://www.douban.com/search', params={'q': 'python', 'cat': '1001'})
>>> r.url # 实际请求的URL
'https://www.douban.com/search?q=python&cat=1001'
\end{pythoncode}

requests 自动检测编码,可以使用\texttt{encoding}属性查看:

\begin{pythoncode}
>>> r.encoding
'utf-8'
\end{pythoncode}

无论响应是文本还是二进制内容,我们都可以用\texttt{content}属性获得\texttt{bytes}对象:

\begin{pythoncode}
>>> r.content
b'<!DOCTYPE html>\n<html>\n<head>\n<meta http-equiv="Content-Type" content="text/html; charset=utf-8">\n...'
\end{pythoncode}

requests 的方便之处还在于,对于特定类型的响应,例如 JSON,可以直接获取:

\begin{pythoncode}
>>> r = requests.get('https://query.yahooapis.com/v1/public/yql?q=select%20*%20from%20weather.forecast%20where%20woeid%20%3D%202151330&format=json')
>>> r.json()
{'query': {'count': 1, 'created': '2017-11-17T07:14:12Z', ...
\end{pythoncode}

需要传入 HTTP Header 时,我们传入一个 dict 作为\texttt{headers}参数:

\begin{pythoncode}
>>> r = requests.get('https://www.douban.com/', headers={'User-Agent': 'Mozilla/5.0 (iPhone; CPU iPhone OS 11_0 like Mac OS X) AppleWebKit'})
>>> r.text
'<!DOCTYPE html>\n<html>\n<head>\n<meta charset="UTF-8">\n <title>豆瓣(手机版)</title>...'
\end{pythoncode}

要发送 POST
请求,只需要把\texttt{get()}方法变成\texttt{post()},然后传入\texttt{data}参数作为
POST 请求的数据:

\begin{pythoncode}
>>> r = requests.post('https://accounts.douban.com/login', data={'form_email': 'abc@example.com', 'form_password': '123456'})
\end{pythoncode}

requests 默认使用\texttt{application/x-www-form-urlencoded}对 POST
数据编码。如果要传递 JSON 数据,可以直接传入 json 参数:

\begin{pythoncode}
params = {'key': 'value'}
r = requests.post(url, json=params) # 内部自动序列化为JSON
\end{pythoncode}

类似的,上传文件需要更复杂的编码格式,但是 requests
把它简化成\texttt{files}参数:

\begin{pythoncode}
>>> upload_files = {'file': open('report.xls', 'rb')}
>>> r = requests.post(url, files=upload_files)
\end{pythoncode}

在读取文件时,注意务必使用\texttt{\textquotesingle{}rb\textquotesingle{}}即二进制模式读取,这样获取的\texttt{bytes}长度才是文件的长度。

把\texttt{post()}方法替换为\texttt{put()},\texttt{delete()}等,就可以以
PUT 或 DELETE 方式请求资源。

除了能轻松获取响应内容外,requests 对获取 HTTP
响应的其他信息也非常简单。例如,获取响应头:

\begin{pythoncode}
>>> r.headers
{Content-Type': 'text/html; charset=utf-8', 'Transfer-Encoding': 'chunked', 'Content-Encoding': 'gzip', ...}
>>> r.headers['Content-Type']
'text/html; charset=utf-8'
\end{pythoncode}

requests 对 Cookie 做了特殊处理,使得我们不必解析 Cookie
就可以轻松获取指定的 Cookie:

\begin{pythoncode}
>>> r.cookies['ts']
'example_cookie_12345'
\end{pythoncode}

要在请求中传入 Cookie,只需准备一个 dict 传入\texttt{cookies}参数:

\begin{pythoncode}
>>> cs = {'token': '12345', 'status': 'working'}
>>> r = requests.get(url, cookies=cs)
\end{pythoncode}

最后,要指定超时,传入以秒为单位的 timeout 参数:

\begin{pythoncode}
>>> r = requests.get(url, timeout=2.5) # 2.5秒后超时
\end{pythoncode}

\hypertarget{ux5c0fux7ed3}{%
\subsubsection{小结}\label{ux5c0fux7ed3}}

用 requests 获取 URL 资源,就是这么简单!

