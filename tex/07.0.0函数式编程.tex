\hypertarget{ux51fdux6570ux5f0fux7f16ux7a0b}{%
\subsection{函数式编程}\label{ux51fdux6570ux5f0fux7f16ux7a0b}}

函数是 Python
内建支持的一种封装,我们通过把大段代码拆成函数,通过一层一层的函数调用,就可以把复杂任务分解成简单的任务,这种分解可以称之为面向过程的程序设计。函数就是面向过程的程序设计的基本单元。

而函数式编程(请注意多了一个 ``式'' 字)------Functional
Programming,虽然也可以归结到面向过程的程序设计,但其思想更接近数学计算。

我们首先要搞明白计算机(Computer)和计算(Compute)的概念。

在计算机的层次上,CPU
执行的是加减乘除的指令代码,以及各种条件判断和跳转指令,所以,汇编语言是最贴近计算机的语言。

而计算则指数学意义上的计算,越是抽象的计算,离计算机硬件越远。

对应到编程语言,就是越低级的语言,越贴近计算机,抽象程度低,执行效率高,比如
C 语言;越高级的语言,越贴近计算,抽象程度高,执行效率低,比如 Lisp
语言。

函数式编程就是一种抽象程度很高的编程范式,纯粹的函数式编程语言编写的函数没有变量,因此,任意一个函数,只要输入是确定的,输出就是确定的,这种纯函数我们称之为没有副作用。而允许使用变量的程序设计语言,由于函数内部的变量状态不确定,同样的输入,可能得到不同的输出,因此,这种函数是有副作用的。

函数式编程的一个特点就是,允许把函数本身作为参数传入另一个函数,还允许返回一个函数!

Python 对函数式编程提供部分支持。由于 Python 允许使用变量,因此,Python
不是纯函数式编程语言。

