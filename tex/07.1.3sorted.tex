\hypertarget{sorted}{%
\subsection{sorted}\label{sorted}}

\hypertarget{ux6392ux5e8fux7b97ux6cd5}{%
\subsubsection{排序算法}\label{ux6392ux5e8fux7b97ux6cd5}}

排序也是在程序中经常用到的算法。无论使用冒泡排序还是快速排序,排序的核心是比较两个元素的大小。如果是数字,我们可以直接比较,但如果是字符串或者两个
dict
呢?直接比较数学上的大小是没有意义的,因此,比较的过程必须通过函数抽象出来。

Python 内置的\texttt{sorted()}函数就可以对 list 进行排序:

\begin{pythoncode}
>>> sorted([36, 5, -12, 9, -21])
[-21, -12, 5, 9, 36]
\end{pythoncode}

此外,\texttt{sorted()}函数也是一个高阶函数,它还可以接收一个\texttt{key}函数来实现自定义的排序,例如按绝对值大小排序:

\begin{pythoncode}
>>> sorted([36, 5, -12, 9, -21], key=abs)
[5, 9, -12, -21, 36]
\end{pythoncode}

key 指定的函数将作用于 list 的每一个元素上,并根据 key
函数返回的结果进行排序。对比原始的 list 和经过\texttt{key=abs}处理过的
list:

\begin{pythoncode}
list = [36, 5, -12, 9, -21]

keys = [36, 5,  12, 9,  21]
\end{pythoncode}

然后\texttt{sorted()}函数按照 keys 进行排序,并按照对应关系返回 list
相应的元素:

\begin{pythoncode}
keys排序结果 => [5, 9,  12,  21, 36]
                |  |    |    |   |
最终结果     => [5, 9, -12, -21, 36]
\end{pythoncode}

我们再看一个字符串排序的例子:

\begin{pythoncode}
>>> sorted(['bob', 'about', 'Zoo', 'Credit'])
['Credit', 'Zoo', 'about', 'bob']
\end{pythoncode}

默认情况下,对字符串排序,是按照 ASCII
的大小比较的,由于\texttt{\textquotesingle{}Z\textquotesingle{}\ \textless{}\ \textquotesingle{}a\textquotesingle{}},结果,大写字母\texttt{Z}会排在小写字母\texttt{a}的前面。

现在,我们提出排序应该忽略大小写,按照字母序排序。要实现这个算法,不必对现有代码大加改动,只要我们能用一个
key
函数把字符串映射为忽略大小写排序即可。忽略大小写来比较两个字符串,实际上就是先把字符串都变成大写(或者都变成小写),再比较。

这样,我们给\texttt{sorted}传入 key 函数,即可实现忽略大小写的排序:

\begin{pythoncode}
>>> sorted(['bob', 'about', 'Zoo', 'Credit'], key=str.lower)
['about', 'bob', 'Credit', 'Zoo']
\end{pythoncode}

要进行反向排序,不必改动 key
函数,可以传入第三个参数\texttt{reverse=True}:

\begin{pythoncode}
>>> sorted(['bob', 'about', 'Zoo', 'Credit'], key=str.lower, reverse=True)
['Zoo', 'Credit', 'bob', 'about']
\end{pythoncode}

从上述例子可以看出,高阶函数的抽象能力是非常强大的,而且,核心代码可以保持得非常简洁。

\hypertarget{ux5c0fux7ed3}{%
\subsubsection{小结}\label{ux5c0fux7ed3}}

\texttt{sorted()}也是一个高阶函数。用\texttt{sorted()}排序的关键在于实现一个映射函数。

\hypertarget{ux7ec3ux4e60}{%
\subsubsection{练习}\label{ux7ec3ux4e60}}

假设我们用一组 tuple 表示学生名字和成绩:

\begin{pythoncode}
L = [('Bob', 75), ('Adam', 92), ('Bart', 66), ('Lisa', 88)]
\end{pythoncode}

请用\texttt{sorted()}对上述列表分别按名字排序:

\begin{pythoncode}
# -*- coding: utf-8 -*-

L = [('Bob', 75), ('Adam', 92), ('Bart', 66), ('Lisa', 88)]
\end{pythoncode}

\begin{pythoncode}
L2 = sorted(L, key=by_name)
print(L2)
\end{pythoncode}

再按成绩从高到低排序:

\begin{pythoncode}
# -*- coding: utf-8 -*-

L = [('Bob', 75), ('Adam', 92), ('Bart', 66), ('Lisa', 88)]
\end{pythoncode}

\begin{pythoncode}
L2 = sorted(L, key=by_score)
print(L2)
\end{pythoncode}

\hypertarget{ux53c2ux8003ux6e90ux7801}{%
\subsubsection{参考源码}\label{ux53c2ux8003ux6e90ux7801}}

\href{https://github.com/michaelliao/learn-python3/blob/master/samples/functional/do_sorted.py}{do\_sorted.py}

