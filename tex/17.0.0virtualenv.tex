\hypertarget{virtualenv}{%
\subsection{virtualenv}\label{virtualenv}}

在开发 Python 应用程序的时候,系统安装的 Python3
只有一个版本:3.4。所有第三方的包都会被\texttt{pip}安装到 Python3
的\texttt{site-packages}目录下。

如果我们要同时开发多个应用程序,那这些应用程序都会共用一个
Python,就是安装在系统的 Python 3。如果应用 A 需要 jinja 2.7,而应用 B
需要 jinja 2.6 怎么办?

这种情况下,每个应用可能需要各自拥有一套 ``独立'' 的 Python
运行环境。virtualenv 就是用来为一个应用创建一套 ``隔离'' 的 Python
运行环境。

首先,我们用\texttt{pip}安装 virtualenv:

\begin{pythoncode}
$ pip3 install virtualenv
\end{pythoncode}

然后,假定我们要开发一个新的项目,需要一套独立的 Python
运行环境,可以这么做:

第一步,创建目录:

\begin{pythoncode}
Mac:~ michael$ mkdir myproject
Mac:~ michael$ cd myproject/
Mac:myproject michael$
\end{pythoncode}

第二步,创建一个独立的 Python 运行环境,命名为\texttt{venv}:

\begin{pythoncode}
Mac:myproject michael$ virtualenv --no-site-packages venv
Using base prefix '/usr/local/.../Python.framework/Versions/3.4'
New python executable in venv/bin/python3.4
Also creating executable in venv/bin/python
Installing setuptools, pip, wheel...done.
\end{pythoncode}

命令\texttt{virtualenv}就可以创建一个独立的 Python
运行环境,我们还加上了参数\texttt{-\/-no-site-packages},这样,已经安装到系统
Python
环境中的所有第三方包都不会复制过来,这样,我们就得到了一个不带任何第三方包的
``干净'' 的 Python 运行环境。

新建的 Python
环境被放到当前目录下的\texttt{venv}目录。有了\texttt{venv}这个 Python
环境,可以用\texttt{source}进入该环境:

\begin{pythoncode}
Mac:myproject michael$ source venv/bin/activate
(venv)Mac:myproject michael$
\end{pythoncode}

注意到命令提示符变了,有个\texttt{(venv)}前缀,表示当前环境是一个名为\texttt{venv}的
Python 环境。

下面正常安装各种第三方包,并运行\texttt{python}命令:

\begin{pythoncode}
(venv)Mac:myproject michael$ pip install jinja2
...
Successfully installed jinja2-2.7.3 markupsafe-0.23
(venv)Mac:myproject michael$ python myapp.py
...
\end{pythoncode}

在\texttt{venv}环境下,用\texttt{pip}安装的包都被安装到\texttt{venv}这个环境下,系统
Python
环境不受任何影响。也就是说,\texttt{venv}环境是专门针对\texttt{myproject}这个应用创建的。

退出当前的\texttt{venv}环境,使用\texttt{deactivate}命令:

\begin{pythoncode}
(venv)Mac:myproject michael$ deactivate 
Mac:myproject michael$ 
\end{pythoncode}

此时就回到了正常的环境,现在\texttt{pip}或\texttt{python}均是在系统
Python 环境下执行。

完全可以针对每个应用创建独立的 Python 运行环境,这样就可以对每个应用的
Python 环境进行隔离。

virtualenv 是如何创建 ``独立'' 的 Python
运行环境的呢?原理很简单,就是把系统 Python 复制一份到 virtualenv
的环境,用命令\texttt{source\ venv/bin/activate}进入一个 virtualenv
环境时,virtualenv
会修改相关环境变量,让命令\texttt{python}和\texttt{pip}均指向当前的
virtualenv 环境。

\hypertarget{ux5c0fux7ed3}{%
\subsubsection{小结}\label{ux5c0fux7ed3}}

virtualenv 为应用提供了隔离的 Python
运行环境,解决了不同应用间多版本的冲突问题。

