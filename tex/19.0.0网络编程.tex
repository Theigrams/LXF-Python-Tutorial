\hypertarget{ux7f51ux7edcux7f16ux7a0b}{%
\subsection{网络编程}\label{ux7f51ux7edcux7f16ux7a0b}}

自从互联网诞生以来,现在基本上所有的程序都是网络程序,很少有单机版的程序了。

计算机网络就是把各个计算机连接到一起,让网络中的计算机可以互相通信。网络编程就是如何在程序中实现两台计算机的通信。

举个例子,当你使用浏览器访问新浪网时,你的计算机就和新浪的某台服务器通过互联网连接起来了,然后,新浪的服务器把网页内容作为数据通过互联网传输到你的电脑上。

由于你的电脑上可能不止浏览器,还有QQ、Skype、Dropbox、邮件客户端等,不同的程序连接的别的计算机也会不同,所以,更确切地说,网络通信是两台计算机上的两个进程之间的通信。比如,浏览器进程和新浪服务器上的某个Web服务进程在通信,而QQ进程是和腾讯的某个服务器上的某个进程在通信。

网络编程对所有开发语言都是一样的,Python也不例外。用Python进行网络编程,就是在Python程序本身这个进程内,连接别的服务器进程的通信端口进行通信。

本章我们将详细介绍Python网络编程的概念和最主要的两种网络类型的编程。

