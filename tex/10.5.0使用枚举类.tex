\hypertarget{ux4f7fux7528ux679aux4e3eux7c7b}{%
\subsection{使用枚举类}\label{ux4f7fux7528ux679aux4e3eux7c7b}}

当我们需要定义常量时,一个办法是用大写变量通过整数来定义,例如月份:

\begin{pythoncode}
JAN = 1
FEB = 2
MAR = 3
...
NOV = 11
DEC = 12
\end{pythoncode}

好处是简单,缺点是类型是\texttt{int},并且仍然是变量。

更好的方法是为这样的枚举类型定义一个 class 类型,然后,每个常量都是
class 的一个唯一实例。Python 提供了\texttt{Enum}类来实现这个功能:

\begin{pythoncode}
from enum import Enum

Month = Enum('Month', ('Jan', 'Feb', 'Mar', 'Apr', 'May', 'Jun', 'Jul', 'Aug', 'Sep', 'Oct', 'Nov', 'Dec'))
\end{pythoncode}

这样我们就获得了\texttt{Month}类型的枚举类,可以直接使用\texttt{Month.Jan}来引用一个常量,或者枚举它的所有成员:

\begin{pythoncode}
for name, member in Month.__members__.items():
    print(name, '=>', member, ',', member.value)
\end{pythoncode}

\texttt{value}属性则是自动赋给成员的\texttt{int}常量,默认从\texttt{1}开始计数。

如果需要更精确地控制枚举类型,可以从\texttt{Enum}派生出自定义类:

\begin{pythoncode}
from enum import Enum, unique

@unique
class Weekday(Enum):
    Sun = 0 
    Mon = 1
    Tue = 2
    Wed = 3
    Thu = 4
    Fri = 5
    Sat = 6
\end{pythoncode}

\texttt{@unique}装饰器可以帮助我们检查保证没有重复值。

访问这些枚举类型可以有若干种方法:

\begin{pythoncode}
>>> day1 = Weekday.Mon
>>> print(day1)
Weekday.Mon
>>> print(Weekday.Tue)
Weekday.Tue
>>> print(Weekday['Tue'])
Weekday.Tue
>>> print(Weekday.Tue.value)
2
>>> print(day1 == Weekday.Mon)
True
>>> print(day1 == Weekday.Tue)
False
>>> print(Weekday(1))
Weekday.Mon
>>> print(day1 == Weekday(1))
True
>>> Weekday(7)
Traceback (most recent call last):
  ...
ValueError: 7 is not a valid Weekday
>>> for name, member in Weekday.__members__.items():
...     print(name, '=>', member)
...
Sun => Weekday.Sun
Mon => Weekday.Mon
Tue => Weekday.Tue
Wed => Weekday.Wed
Thu => Weekday.Thu
Fri => Weekday.Fri
Sat => Weekday.Sat
\end{pythoncode}

可见,既可以用成员名称引用枚举常量,又可以直接根据 value
的值获得枚举常量。

\hypertarget{ux7ec3ux4e60}{%
\subsubsection{练习}\label{ux7ec3ux4e60}}

把\texttt{Student}的\texttt{gender}属性改造为枚举类型,可以避免使用字符串:

\begin{pythoncode}
# -*- coding: utf-8 -*-
from enum import Enum, unique
\end{pythoncode}

\begin{pythoncode}
# 测试:
bart = Student('Bart', Gender.Male)
if bart.gender == Gender.Male:
    print('测试通过!')
else:
    print('测试失败!')
\end{pythoncode}

\hypertarget{ux5c0fux7ed3}{%
\subsubsection{小结}\label{ux5c0fux7ed3}}

\texttt{Enum}可以把一组相关常量定义在一个 class 中,且 class
不可变,而且成员可以直接比较。

\hypertarget{ux53c2ux8003ux6e90ux7801}{%
\subsubsection{参考源码}\label{ux53c2ux8003ux6e90ux7801}}

\href{https://github.com/michaelliao/learn-python3/blob/master/samples/oop_advance/use_enum.py}{use\_enum.py}

