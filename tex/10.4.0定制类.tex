\hypertarget{ux5b9aux5236ux7c7b}{%
\subsection{定制类}\label{ux5b9aux5236ux7c7b}}

看到类似\texttt{\_\_slots\_\_}这种形如\texttt{\_\_xxx\_\_}的变量或者函数名就要注意,这些在
Python 中是有特殊用途的。

\texttt{\_\_slots\_\_}我们已经知道怎么用了,\texttt{\_\_len\_\_()}方法我们也知道是为了能让
class 作用于\texttt{len()}函数。

除此之外,Python 的 class
中还有许多这样有特殊用途的函数,可以帮助我们定制类。

\hypertarget{str}{%
\subsubsection{\texorpdfstring{\textbf{str}}{str}}\label{str}}

我们先定义一个\texttt{Student}类,打印一个实例:

\begin{pythoncode}
>>> class Student(object):
...     def __init__(self, name):
...         self.name = name
...
>>> print(Student('Michael'))
<__main__.Student object at 0x109afb190>
\end{pythoncode}

打印出一堆\texttt{\textless{}\_\_main\_\_.Student\ object\ at\ 0x109afb190\textgreater{}},不好看。

怎么才能打印得好看呢?只需要定义好\texttt{\_\_str\_\_()}方法,返回一个好看的字符串就可以了:

\begin{pythoncode}
>>> class Student(object):
...     def __init__(self, name):
...         self.name = name
...     def __str__(self):
...         return 'Student object (name: %s)' % self.name
...
>>> print(Student('Michael'))
Student object (name: Michael)
\end{pythoncode}

这样打印出来的实例,不但好看,而且容易看出实例内部重要的数据。

但是细心的朋友会发现直接敲变量不用\texttt{print},打印出来的实例还是不好看:

\begin{pythoncode}
>>> s = Student('Michael')
>>> s
<__main__.Student object at 0x109afb310>
\end{pythoncode}

这是因为直接显示变量调用的不是\texttt{\_\_str\_\_()},而是\texttt{\_\_repr\_\_()},两者的区别是\texttt{\_\_str\_\_()}返回用户看到的字符串,而\texttt{\_\_repr\_\_()}返回程序开发者看到的字符串,也就是说,\texttt{\_\_repr\_\_()}是为调试服务的。

解决办法是再定义一个\texttt{\_\_repr\_\_()}。但是通常\texttt{\_\_str\_\_()}和\texttt{\_\_repr\_\_()}代码都是一样的,所以,有个偷懒的写法:

\begin{pythoncode}
class Student(object):
    def __init__(self, name):
        self.name = name
    def __str__(self):
        return 'Student object (name=%s)' % self.name
    __repr__ = __str__
\end{pythoncode}

\hypertarget{iter}{%
\subsubsection{\texorpdfstring{\textbf{iter}}{iter}}\label{iter}}

如果一个类想被用于\texttt{for\ ...\ in}循环,类似 list 或 tuple
那样,就必须实现一个\texttt{\_\_iter\_\_()}方法,该方法返回一个迭代对象,然后,Python
的 for
循环就会不断调用该迭代对象的\texttt{\_\_next\_\_()}方法拿到循环的下一个值,直到遇到\texttt{StopIteration}错误时退出循环。

我们以斐波那契数列为例,写一个 Fib 类,可以作用于 for 循环:

\begin{pythoncode}
class Fib(object):
    def __init__(self):
        self.a, self.b = 0, 1 

    def __iter__(self):
        return self 

    def __next__(self):
        self.a, self.b = self.b, self.a + self.b 
        if self.a > 100000: 
            raise StopIteration()
        return self.a 
\end{pythoncode}

现在,试试把 Fib 实例作用于 for 循环:

\begin{pythoncode}
>>> for n in Fib():
...     print(n)
...
1
1
2
3
5
...
46368
75025
\end{pythoncode}

\hypertarget{getitem}{%
\subsubsection{\texorpdfstring{\textbf{getitem}}{getitem}}\label{getitem}}

Fib 实例虽然能作用于 for 循环,看起来和 list 有点像,但是,把它当成 list
来使用还是不行,比如,取第 5 个元素:

\begin{pythoncode}
>>> Fib()[5]
Traceback (most recent call last):
  File "<stdin>", line 1, in <module>
TypeError: 'Fib' object does not support indexing
\end{pythoncode}

要表现得像 list
那样按照下标取出元素,需要实现\texttt{\_\_getitem\_\_()}方法:

\begin{pythoncode}
class Fib(object):
    def __getitem__(self, n):
        a, b = 1, 1
        for x in range(n):
            a, b = b, a + b
        return a
\end{pythoncode}

现在,就可以按下标访问数列的任意一项了:

\begin{pythoncode}
>>> f = Fib()
>>> f[0]
1
>>> f[1]
1
>>> f[2]
2
>>> f[3]
3
>>> f[10]
89
>>> f[100]
573147844013817084101
\end{pythoncode}

但是 list 有个神奇的切片方法:

\begin{pythoncode}
>>> list(range(100))[5:10]
[5, 6, 7, 8, 9]
\end{pythoncode}

对于 Fib 却报错。原因是\texttt{\_\_getitem\_\_()}传入的参数可能是一个
int,也可能是一个切片对象\texttt{slice},所以要做判断:

\begin{pythoncode}
class Fib(object):
    def __getitem__(self, n):
        if isinstance(n, int): 
            a, b = 1, 1
            for x in range(n):
                a, b = b, a + b
            return a
        if isinstance(n, slice): 
            start = n.start
            stop = n.stop
            if start is None:
                start = 0
            a, b = 1, 1
            L = []
            for x in range(stop):
                if x >= start:
                    L.append(a)
                a, b = b, a + b
            return L
\end{pythoncode}

现在试试 Fib 的切片:

\begin{pythoncode}
>>> f = Fib()
>>> f[0:5]
[1, 1, 2, 3, 5]
>>> f[:10]
[1, 1, 2, 3, 5, 8, 13, 21, 34, 55]
\end{pythoncode}

但是没有对 step 参数作处理:

\begin{pythoncode}
>>> f[:10:2]
[1, 1, 2, 3, 5, 8, 13, 21, 34, 55, 89]
\end{pythoncode}

也没有对负数作处理,所以,要正确实现一个\texttt{\_\_getitem\_\_()}还是有很多工作要做的。

此外,如果把对象看成\texttt{dict},\texttt{\_\_getitem\_\_()}的参数也可能是一个可以作
key 的 object,例如\texttt{str}。

与之对应的是\texttt{\_\_setitem\_\_()}方法,把对象视作 list 或 dict
来对集合赋值。最后,还有一个\texttt{\_\_delitem\_\_()}方法,用于删除某个元素。

总之,通过上面的方法,我们自己定义的类表现得和 Python 自带的
list、tuple、dict 没什么区别,这完全归功于动态语言的
``鸭子类型'',不需要强制继承某个接口。

\hypertarget{getattr}{%
\subsubsection{\texorpdfstring{\textbf{getattr}}{getattr}}\label{getattr}}

正常情况下,当我们调用类的方法或属性时,如果不存在,就会报错。比如定义\texttt{Student}类:

\begin{pythoncode}
class Student(object):
    
    def __init__(self):
        self.name = 'Michael'
\end{pythoncode}

调用\texttt{name}属性,没问题,但是,调用不存在的\texttt{score}属性,就有问题了:

\begin{pythoncode}
>>> s = Student()
>>> print(s.name)
Michael
>>> print(s.score)
Traceback (most recent call last):
  ...
AttributeError: 'Student' object has no attribute 'score'
\end{pythoncode}

错误信息很清楚地告诉我们,没有找到\texttt{score}这个 attribute。

要避免这个错误,除了可以加上一个\texttt{score}属性外,Python
还有另一个机制,那就是写一个\texttt{\_\_getattr\_\_()}方法,动态返回一个属性。修改如下:

\begin{pythoncode}
class Student(object):

    def __init__(self):
        self.name = 'Michael'

    def __getattr__(self, attr):
        if attr=='score':
            return 99
\end{pythoncode}

当调用不存在的属性时,比如\texttt{score},Python
解释器会试图调用\texttt{\_\_getattr\_\_(self,\ \textquotesingle{}score\textquotesingle{})}来尝试获得属性,这样,我们就有机会返回\texttt{score}的值:

\begin{pythoncode}
>>> s = Student()
>>> s.name
'Michael'
>>> s.score
99
\end{pythoncode}

返回函数也是完全可以的:

\begin{pythoncode}
class Student(object):

    def __getattr__(self, attr):
        if attr=='age':
            return lambda: 25
\end{pythoncode}

只是调用方式要变为:

\begin{pythoncode}
>>> s.age()
25
\end{pythoncode}

注意,只有在没有找到属性的情况下,才调用\texttt{\_\_getattr\_\_},已有的属性,比如\texttt{name},不会在\texttt{\_\_getattr\_\_}中查找。

此外,注意到任意调用如\texttt{s.abc}都会返回\texttt{None},这是因为我们定义的\texttt{\_\_getattr\_\_}默认返回就是\texttt{None}。要让
class
只响应特定的几个属性,我们就要按照约定,抛出\texttt{AttributeError}的错误:

\begin{pythoncode}
class Student(object):

    def __getattr__(self, attr):
        if attr=='age':
            return lambda: 25
        raise AttributeError('\'Student\' object has no attribute \'%s\'' % attr)
\end{pythoncode}

这实际上可以把一个类的所有属性和方法调用全部动态化处理了,不需要任何特殊手段。

这种完全动态调用的特性有什么实际作用呢?作用就是,可以针对完全动态的情况作调用。

举个例子:

现在很多网站都搞 REST API,比如新浪微博、豆瓣啥的,调用 API 的 URL
类似:

\begin{itemize}
\item
  http://api.server/user/friends
\item
  http://api.server/user/timeline/list
\end{itemize}

如果要写 SDK,给每个 URL 对应的 API 都写一个方法,那得累死,而且,API
一旦改动,SDK 也要改。

利用完全动态的\texttt{\_\_getattr\_\_},我们可以写出一个链式调用:

\begin{pythoncode}
class Chain(object):

    def __init__(self, path=''):
        self._path = path

    def __getattr__(self, path):
        return Chain('%s/%s' % (self._path, path))

    def __str__(self):
        return self._path

    __repr__ = __str__
\end{pythoncode}

试试:

\begin{pythoncode}
>>> Chain().status.user.timeline.list
'/status/user/timeline/list'
\end{pythoncode}

这样,无论 API 怎么变,SDK 都可以根据 URL 实现完全动态的调用,而且,不随
API 的增加而改变!

还有些 REST API 会把参数放到 URL 中,比如 GitHub 的 API:

\begin{pythoncode}
GET /users/:user/repos
\end{pythoncode}

调用时,需要把\texttt{:user}替换为实际用户名。如果我们能写出这样的链式调用:

\begin{pythoncode}
Chain().users('michael').repos
\end{pythoncode}

就可以非常方便地调用 API 了。有兴趣的童鞋可以试试写出来。

\hypertarget{call}{%
\subsubsection{\texorpdfstring{\textbf{call}}{call}}\label{call}}

一个对象实例可以有自己的属性和方法,当我们调用实例方法时,我们用\texttt{instance.method()}来调用。能不能直接在实例本身上调用呢?在
Python 中,答案是肯定的。

任何类,只需要定义一个\texttt{\_\_call\_\_()}方法,就可以直接对实例进行调用。请看示例:

\begin{pythoncode}
class Student(object):
    def __init__(self, name):
        self.name = name

    def __call__(self):
        print('My name is %s.' % self.name)
\end{pythoncode}

调用方式如下:

\begin{pythoncode}
>>> s = Student('Michael')
>>> s() 
My name is Michael.
\end{pythoncode}

\texttt{\_\_call\_\_()}还可以定义参数。对实例进行直接调用就好比对一个函数进行调用一样,所以你完全可以把对象看成函数,把函数看成对象,因为这两者之间本来就没啥根本的区别。

如果你把对象看成函数,那么函数本身其实也可以在运行期动态创建出来,因为类的实例都是运行期创建出来的,这么一来,我们就模糊了对象和函数的界限。

那么,怎么判断一个变量是对象还是函数呢?其实,更多的时候,我们需要判断一个对象是否能被调用,能被调用的对象就是一个\texttt{Callable}对象,比如函数和我们上面定义的带有\texttt{\_\_call\_\_()}的类实例:

\begin{pythoncode}
>>> callable(Student())
True
>>> callable(max)
True
>>> callable([1, 2, 3])
False
>>> callable(None)
False
>>> callable('str')
False
\end{pythoncode}

通过\texttt{callable()}函数,我们就可以判断一个对象是否是 ``可调用''
对象。

\hypertarget{ux5c0fux7ed3}{%
\subsubsection{小结}\label{ux5c0fux7ed3}}

Python 的 class 允许定义许多定制方法,可以让我们非常方便地生成特定的类。

本节介绍的是最常用的几个定制方法,还有很多可定制的方法,请参考
\href{http://docs.python.org/3/reference/datamodel.html\#special-method-names}{Python
的官方文档}。

\hypertarget{ux53c2ux8003ux6e90ux7801}{%
\subsubsection{参考源码}\label{ux53c2ux8003ux6e90ux7801}}

\href{https://github.com/michaelliao/learn-python3/blob/master/samples/oop_advance/special_str.py}{special\_str.py}

\href{https://github.com/michaelliao/learn-python3/blob/master/samples/oop_advance/special_iter.py}{special\_iter.py}

\href{https://github.com/michaelliao/learn-python3/blob/master/samples/oop_advance/special_getitem.py}{special\_getitem.py}

\href{https://github.com/michaelliao/learn-python3/blob/master/samples/oop_advance/special_getattr.py}{special\_getattr.py}

\href{https://github.com/michaelliao/learn-python3/blob/master/samples/oop_advance/special_call.py}{special\_call.py}

