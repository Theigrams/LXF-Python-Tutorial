\hypertarget{ux8fdbux7a0b-vs.-ux7ebfux7a0b}{%
\subsection{进程 vs.~线程}\label{ux8fdbux7a0b-vs.-ux7ebfux7a0b}}

我们介绍了多进程和多线程,这是实现多任务最常用的两种方式。现在,我们来讨论一下这两种方式的优缺点。

首先,要实现多任务,通常我们会设计 Master-Worker 模式,Master
负责分配任务,Worker 负责执行任务,因此,多任务环境下,通常是一个
Master,多个 Worker。

如果用多进程实现 Master-Worker,主进程就是 Master,其他进程就是 Worker。

如果用多线程实现 Master-Worker,主线程就是 Master,其他线程就是 Worker。

多进程模式最大的优点就是稳定性高,因为一个子进程崩溃了,不会影响主进程和其他子进程。(当然主进程挂了所有进程就全挂了,但是
Master 进程只负责分配任务,挂掉的概率低)著名的 Apache
最早就是采用多进程模式。

多进程模式的缺点是创建进程的代价大,在 Unix/Linux
系统下,用\texttt{fork}调用还行,在 Windows
下创建进程开销巨大。另外,操作系统能同时运行的进程数也是有限的,在内存和
CPU 的限制下,如果有几千个进程同时运行,操作系统连调度都会成问题。

多线程模式通常比多进程快一点,但是也快不到哪去,而且,多线程模式致命的缺点就是任何一个线程挂掉都可能直接造成整个进程崩溃,因为所有线程共享进程的内存。在
Windows
上,如果一个线程执行的代码出了问题,你经常可以看到这样的提示:``该程序执行了非法操作,即将关闭'',其实往往是某个线程出了问题,但是操作系统会强制结束整个进程。

在 Windows 下,多线程的效率比多进程要高,所以微软的 IIS
服务器默认采用多线程模式。由于多线程存在稳定性的问题,IIS 的稳定性就不如
Apache。为了缓解这个问题,IIS 和 Apache 现在又有多进程 +
多线程的混合模式,真是把问题越搞越复杂。

\hypertarget{ux7ebfux7a0bux5207ux6362}{%
\subsubsection{线程切换}\label{ux7ebfux7a0bux5207ux6362}}

无论是多进程还是多线程,只要数量一多,效率肯定上不去,为什么呢?

我们打个比方,假设你不幸正在准备中考,每天晚上需要做语文、数学、英语、物理、化学这
5 科的作业,每项作业耗时 1 小时。

如果你先花 1 小时做语文作业,做完了,再花 1
小时做数学作业,这样,依次全部做完,一共花 5
小时,这种方式称为单任务模型,或者批处理任务模型。

假设你打算切换到多任务模型,可以先做 1 分钟语文,再切换到数学作业,做 1
分钟,再切换到英语,以此类推,只要切换速度足够快,这种方式就和单核 CPU
执行多任务是一样的了,以幼儿园小朋友的眼光来看,你就正在同时写 5
科作业。

但是,切换作业是有代价的,比如从语文切到数学,要先收拾桌子上的语文书本、钢笔(这叫保存现场),然后,打开数学课本、找出圆规直尺(这叫准备新环境),才能开始做数学作业。操作系统在切换进程或者线程时也是一样的,它需要先保存当前执行的现场环境(CPU
寄存器状态、内存页等),然后,把新任务的执行环境准备好(恢复上次的寄存器状态,切换内存页等),才能开始执行。这个切换过程虽然很快,但是也需要耗费时间。如果有几千个任务同时进行,操作系统可能就主要忙着切换任务,根本没有多少时间去执行任务了,这种情况最常见的就是硬盘狂响,点窗口无反应,系统处于假死状态。

所以,多任务一旦多到一个限度,就会消耗掉系统所有的资源,结果效率急剧下降,所有任务都做不好。

\hypertarget{ux8ba1ux7b97ux5bc6ux96c6ux578b-vs.-io-ux5bc6ux96c6ux578b}{%
\subsubsection{计算密集型 vs.~IO
密集型}\label{ux8ba1ux7b97ux5bc6ux96c6ux578b-vs.-io-ux5bc6ux96c6ux578b}}

是否采用多任务的第二个考虑是任务的类型。我们可以把任务分为计算密集型和
IO 密集型。

计算密集型任务的特点是要进行大量的计算,消耗 CPU
资源,比如计算圆周率、对视频进行高清解码等等,全靠 CPU
的运算能力。这种计算密集型任务虽然也可以用多任务完成,但是任务越多,花在任务切换的时间就越多,CPU
执行任务的效率就越低,所以,要最高效地利用
CPU,计算密集型任务同时进行的数量应当等于 CPU 的核心数。

计算密集型任务由于主要消耗 CPU 资源,因此,代码运行效率至关重要。Python
这样的脚本语言运行效率很低,完全不适合计算密集型任务。对于计算密集型任务,最好用
C 语言编写。

第二种任务的类型是 IO 密集型,涉及到网络、磁盘 IO 的任务都是 IO
密集型任务,这类任务的特点是 CPU 消耗很少,任务的大部分时间都在等待 IO
操作完成(因为 IO 的速度远远低于 CPU 和内存的速度)。对于 IO
密集型任务,任务越多,CPU 效率越高,但也有一个限度。常见的大部分任务都是
IO 密集型任务,比如 Web 应用。

IO 密集型任务执行期间,99\% 的时间都花在 IO 上,花在 CPU
上的时间很少,因此,用运行速度极快的 C 语言替换用 Python
这样运行速度极低的脚本语言,完全无法提升运行效率。对于 IO
密集型任务,最合适的语言就是开发效率最高(代码量最少)的语言,脚本语言是首选,C
语言最差。

\hypertarget{ux5f02ux6b65-io}{%
\subsubsection{异步 IO}\label{ux5f02ux6b65-io}}

考虑到 CPU 和 IO
之间巨大的速度差异,一个任务在执行的过程中大部分时间都在等待 IO
操作,单进程单线程模型会导致别的任务无法并行执行,因此,我们才需要多进程模型或者多线程模型来支持多任务并发执行。

现代操作系统对 IO 操作已经做了巨大的改进,最大的特点就是支持异步
IO。如果充分利用操作系统提供的异步 IO
支持,就可以用单进程单线程模型来执行多任务,这种全新的模型称为事件驱动模型,Nginx
就是支持异步 IO 的 Web 服务器,它在单核 CPU
上采用单进程模型就可以高效地支持多任务。在多核 CPU
上,可以运行多个进程(数量与 CPU 核心数相同),充分利用多核
CPU。由于系统总的进程数量十分有限,因此操作系统调度非常高效。用异步 IO
编程模型来实现多任务是一个主要的趋势。

对应到 Python
语言,单线程的异步编程模型称为协程,有了协程的支持,就可以基于事件驱动编写高效的多任务程序。我们会在后面讨论如何编写协程。

