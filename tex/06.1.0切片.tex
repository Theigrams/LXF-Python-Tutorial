\hypertarget{ux5207ux7247}{%
\subsection{切片}\label{ux5207ux7247}}

取一个 list 或 tuple 的部分元素是非常常见的操作。比如,一个 list 如下:

\begin{pythoncode}
>>> L = ['Michael', 'Sarah', 'Tracy', 'Bob', 'Jack']
\end{pythoncode}

取前 3 个元素,应该怎么做?

笨办法:

\begin{pythoncode}
>>> [L[0], L[1], L[2]]
['Michael', 'Sarah', 'Tracy']
\end{pythoncode}

之所以是笨办法是因为扩展一下,取前 N 个元素就没辙了。

取前 N 个元素,也就是索引为 0-(N-1) 的元素,可以用循环:

\begin{pythoncode}
>>> r = []
>>> n = 3
>>> for i in range(n):
...     r.append(L[i])
... 
>>> r
['Michael', 'Sarah', 'Tracy']
\end{pythoncode}

对这种经常取指定索引范围的操作,用循环十分繁琐,因此,Python
提供了切片(Slice)操作符,能大大简化这种操作。

对应上面的问题,取前 3 个元素,用一行代码就可以完成切片:

\begin{pythoncode}
>>> L[0:3]
['Michael', 'Sarah', 'Tracy']
\end{pythoncode}

\texttt{L{[}0:3{]}}表示,从索引\texttt{0}开始取,直到索引\texttt{3}为止,但不包括索引\texttt{3}。即索引\texttt{0},\texttt{1},\texttt{2},正好是
3 个元素。

如果第一个索引是\texttt{0},还可以省略:

\begin{pythoncode}
>>> L[:3]
['Michael', 'Sarah', 'Tracy']
\end{pythoncode}

也可以从索引 1 开始,取出 2 个元素出来:

\begin{pythoncode}
>>> L[1:3]
['Sarah', 'Tracy']
\end{pythoncode}

类似的,既然 Python
支持\texttt{L{[}-1{]}}取倒数第一个元素,那么它同样支持倒数切片,试试:

\begin{pythoncode}
>>> L[-2:]
['Bob', 'Jack']
>>> L[-2:-1]
['Bob']
\end{pythoncode}

记住倒数第一个元素的索引是\texttt{-1}。

切片操作十分有用。我们先创建一个 0-99 的数列:

\begin{pythoncode}
>>> L = list(range(100))
>>> L
[0, 1, 2, 3, ..., 99]
\end{pythoncode}

可以通过切片轻松取出某一段数列。比如前 10 个数:

\begin{pythoncode}
>>> L[:10]
[0, 1, 2, 3, 4, 5, 6, 7, 8, 9]
\end{pythoncode}

后 10 个数:

\begin{pythoncode}
>>> L[-10:]
[90, 91, 92, 93, 94, 95, 96, 97, 98, 99]
\end{pythoncode}

前 11-20 个数:

\begin{pythoncode}
>>> L[10:20]
[10, 11, 12, 13, 14, 15, 16, 17, 18, 19]
\end{pythoncode}

前 10 个数,每两个取一个:

\begin{pythoncode}
>>> L[:10:2]
[0, 2, 4, 6, 8]
\end{pythoncode}

所有数,每 5 个取一个:

\begin{pythoncode}
>>> L[::5]
[0, 5, 10, 15, 20, 25, 30, 35, 40, 45, 50, 55, 60, 65, 70, 75, 80, 85, 90, 95]
\end{pythoncode}

甚至什么都不写,只写\texttt{{[}:{]}}就可以原样复制一个 list:

\begin{pythoncode}
>>> L[:]
[0, 1, 2, 3, ..., 99]
\end{pythoncode}

tuple 也是一种 list,唯一区别是 tuple 不可变。因此,tuple
也可以用切片操作,只是操作的结果仍是 tuple:

\begin{pythoncode}
>>> (0, 1, 2, 3, 4, 5)[:3]
(0, 1, 2)
\end{pythoncode}

字符串\texttt{\textquotesingle{}xxx\textquotesingle{}}也可以看成是一种
list,每个元素就是一个字符。因此,字符串也可以用切片操作,只是操作结果仍是字符串:

\begin{pythoncode}
>>> 'ABCDEFG'[:3]
'ABC'
>>> 'ABCDEFG'[::2]
'ACEG'
\end{pythoncode}

在很多编程语言中,针对字符串提供了很多各种截取函数(例如,substring),其实目的就是对字符串切片。Python
没有针对字符串的截取函数,只需要切片一个操作就可以完成,非常简单。

\hypertarget{ux7ec3ux4e60}{%
\subsubsection{练习}\label{ux7ec3ux4e60}}

利用切片操作,实现一个 trim() 函数,去除字符串首尾的空格,注意不要调用
str 的\texttt{strip()}方法:

\begin{pythoncode}
# -*- coding: utf-8 -*-
def trim(s):
\end{pythoncode}

\begin{pythoncode}
# 测试:
if trim('hello  ') != 'hello':
    print('测试失败!')
elif trim('  hello') != 'hello':
    print('测试失败!')
elif trim('  hello  ') != 'hello':
    print('测试失败!')
elif trim('  hello  world  ') != 'hello  world':
    print('测试失败!')
elif trim('') != '':
    print('测试失败!')
elif trim('    ') != '':
    print('测试失败!')
else:
    print('测试成功!')
\end{pythoncode}

\hypertarget{ux5c0fux7ed3}{%
\subsubsection{小结}\label{ux5c0fux7ed3}}

有了切片操作,很多地方循环就不再需要了。Python
的切片非常灵活,一行代码就可以实现很多行循环才能完成的操作。

\hypertarget{ux53c2ux8003ux6e90ux7801}{%
\subsubsection{参考源码}\label{ux53c2ux8003ux6e90ux7801}}

\href{https://github.com/michaelliao/learn-python3/blob/master/samples/advance/do_slice.py}{do\_slice.py}

