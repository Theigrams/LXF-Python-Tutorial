\hypertarget{ux7c7bux548cux5b9eux4f8b}{%
\subsection{类和实例}\label{ux7c7bux548cux5b9eux4f8b}}

面向对象最重要的概念就是类(Class)和实例(Instance),必须牢记类是抽象的模板,比如
Student 类,而实例是根据类创建出来的一个个具体的
``对象'',每个对象都拥有相同的方法,但各自的数据可能不同。

仍以 Student 类为例,在 Python 中,定义类是通过\texttt{class}关键字:

\begin{pythoncode}
class Student(object):
    pass
\end{pythoncode}

\texttt{class}后面紧接着是类名,即\texttt{Student},类名通常是大写开头的单词,紧接着是\texttt{(object)},表示该类是从哪个类继承下来的,继承的概念我们后面再讲,通常,如果没有合适的继承类,就使用\texttt{object}类,这是所有类最终都会继承的类。

定义好了\texttt{Student}类,就可以根据\texttt{Student}类创建出\texttt{Student}的实例,创建实例是通过类名
+() 实现的:

\begin{pythoncode}
>>> bart = Student()
>>> bart
<__main__.Student object at 0x10a67a590>
>>> Student
<class '__main__.Student'>
\end{pythoncode}

可以看到,变量\texttt{bart}指向的就是一个\texttt{Student}的实例,后面的\texttt{0x10a67a590}是内存地址,每个
object 的地址都不一样,而\texttt{Student}本身则是一个类。

可以自由地给一个实例变量绑定属性,比如,给实例\texttt{bart}绑定一个\texttt{name}属性:

\begin{pythoncode}
>>> bart.name = 'Bart Simpson'
>>> bart.name
'Bart Simpson'
\end{pythoncode}

由于类可以起到模板的作用,因此,可以在创建实例的时候,把一些我们认为必须绑定的属性强制填写进去。通过定义一个特殊的\texttt{\_\_init\_\_}方法,在创建实例的时候,就把\texttt{name},\texttt{score}等属性绑上去:

\begin{pythoncode}
class Student(object):

    def __init__(self, name, score):
        self.name = name
        self.score = score
\end{pythoncode}

注意:特殊方法 ``\textbf{init}'' 前后分别有两个下划线!!!

注意到\texttt{\_\_init\_\_}方法的第一个参数永远是\texttt{self},表示创建的实例本身,因此,在\texttt{\_\_init\_\_}方法内部,就可以把各种属性绑定到\texttt{self},因为\texttt{self}就指向创建的实例本身。

有了\texttt{\_\_init\_\_}方法,在创建实例的时候,就不能传入空的参数了,必须传入与\texttt{\_\_init\_\_}方法匹配的参数,但\texttt{self}不需要传,Python
解释器自己会把实例变量传进去:

\begin{pythoncode}
>>> bart = Student('Bart Simpson', 59)
>>> bart.name
'Bart Simpson'
>>> bart.score
59
\end{pythoncode}

和普通的函数相比,在类中定义的函数只有一点不同,就是第一个参数永远是实例变量\texttt{self},并且,调用时,不用传递该参数。除此之外,类的方法和普通函数没有什么区别,所以,你仍然可以用默认参数、可变参数、关键字参数和命名关键字参数。

\hypertarget{ux6570ux636eux5c01ux88c5}{%
\subsubsection{数据封装}\label{ux6570ux636eux5c01ux88c5}}

面向对象编程的一个重要特点就是数据封装。在上面的\texttt{Student}类中,每个实例就拥有各自的\texttt{name}和\texttt{score}这些数据。我们可以通过函数来访问这些数据,比如打印一个学生的成绩:

\begin{pythoncode}
>>> def print_score(std):
...     print('%s: %s' % (std.name, std.score))
...
>>> print_score(bart)
Bart Simpson: 59
\end{pythoncode}

但是,既然\texttt{Student}实例本身就拥有这些数据,要访问这些数据,就没有必要从外面的函数去访问,可以直接在\texttt{Student}类的内部定义访问数据的函数,这样,就把
``数据''
给封装起来了。这些封装数据的函数是和\texttt{Student}类本身是关联起来的,我们称之为类的方法:

\begin{pythoncode}
class Student(object):

    def __init__(self, name, score):
        self.name = name
        self.score = score

    def print_score(self):
        print('%s: %s' % (self.name, self.score))
\end{pythoncode}

要定义一个方法,除了第一个参数是\texttt{self}外,其他和普通函数一样。要调用一个方法,只需要在实例变量上直接调用,除了\texttt{self}不用传递,其他参数正常传入:

\begin{pythoncode}
>>> bart.print_score()
Bart Simpson: 59
\end{pythoncode}

这样一来,我们从外部看\texttt{Student}类,就只需要知道,创建实例需要给出\texttt{name}和\texttt{score},而如何打印,都是在\texttt{Student}类的内部定义的,这些数据和逻辑被
``封装'' 起来了,调用很容易,但却不用知道内部实现的细节。

封装的另一个好处是可以给\texttt{Student}类增加新的方法,比如\texttt{get\_grade}:

\begin{pythoncode}
class Student(object):
    ...

    def get_grade(self):
        if self.score >= 90:
            return 'A'
        elif self.score >= 60:
            return 'B'
        else:
            return 'C'
\end{pythoncode}

同样的,\texttt{get\_grade}方法可以直接在实例变量上调用,不需要知道内部实现细节:

\begin{pythoncode}
# -*- coding: utf-8 -*-
\end{pythoncode}

\begin{pythoncode}
lisa = Student('Lisa', 99)
bart = Student('Bart', 59)
print(lisa.name, lisa.get_grade())
print(bart.name, bart.get_grade())
\end{pythoncode}

\hypertarget{ux5c0fux7ed3}{%
\subsubsection{小结}\label{ux5c0fux7ed3}}

类是创建实例的模板,而实例则是一个一个具体的对象,各个实例拥有的数据都互相独立,互不影响;

方法就是与实例绑定的函数,和普通函数不同,方法可以直接访问实例的数据;

通过在实例上调用方法,我们就直接操作了对象内部的数据,但无需知道方法内部的实现细节。

和静态语言不同,Python
允许对实例变量绑定任何数据,也就是说,对于两个实例变量,虽然它们都是同一个类的不同实例,但拥有的变量名称都可能不同:

\begin{pythoncode}
>>> bart = Student('Bart Simpson', 59)
>>> lisa = Student('Lisa Simpson', 87)
>>> bart.age = 8
>>> bart.age
8
>>> lisa.age
Traceback (most recent call last):
  File "<stdin>", line 1, in <module>
AttributeError: 'Student' object has no attribute 'age'
\end{pythoncode}

\hypertarget{ux53c2ux8003ux6e90ux7801}{%
\subsubsection{参考源码}\label{ux53c2ux8003ux6e90ux7801}}

\href{https://github.com/michaelliao/learn-python3/blob/master/samples/oop_basic/student.py}{student.py}

