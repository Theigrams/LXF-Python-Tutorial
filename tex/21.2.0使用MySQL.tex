\hypertarget{ux4f7fux7528-mysql}{%
\subsection{使用 MySQL}\label{ux4f7fux7528-mysql}}

MySQL 是 Web 世界中使用最广泛的数据库服务器。SQLite
的特点是轻量级、可嵌入,但不能承受高并发访问,适合桌面和移动应用。而
MySQL
是为服务器端设计的数据库,能承受高并发访问,同时占用的内存也远远大于
SQLite。

此外,MySQL 内部有多种数据库引擎,最常用的引擎是支持数据库事务的
InnoDB。

\hypertarget{ux5b89ux88c5-mysql}{%
\subsubsection{安装 MySQL}\label{ux5b89ux88c5-mysql}}

可以直接从 MySQL 官方网站下载最新的
\href{http://dev.mysql.com/downloads/mysql/5.6.html}{Community Server
5.6.x} 版本。MySQL 是跨平台的,选择对应的平台下载安装文件,安装即可。

安装时,MySQL
会提示输入\texttt{root}用户的口令,请务必记清楚。如果怕记不住,就把口令设置为\texttt{password}。

在 Windows 上,安装时请选择\texttt{UTF-8}编码,以便正确地处理中文。

在 Mac 或 Linux 上,需要编辑 MySQL
的配置文件,把数据库默认的编码全部改为 UTF-8。MySQL
的配置文件默认存放在\texttt{/etc/my.cnf}或者\texttt{/etc/mysql/my.cnf}:

\begin{pythoncode}
[client]
default-character-set = utf8

[mysqld]
default-storage-engine = INNODB
character-set-server = utf8
collation-server = utf8_general_ci
\end{pythoncode}

重启 MySQL 后,可以通过 MySQL 的客户端命令行检查编码:

\begin{pythoncode}
$ mysql -u root -p
Enter password: 
Welcome to the MySQL monitor...
...

mysql> show variables like '%char%';
+
| Variable_name            | Value                                                  |
+
| character_set_client     | utf8                                                   |
| character_set_connection | utf8                                                   |
| character_set_database   | utf8                                                   |
| character_set_filesystem | binary                                                 |
| character_set_results    | utf8                                                   |
| character_set_server     | utf8                                                   |
| character_set_system     | utf8                                                   |
| character_sets_dir       | /usr/local/mysql-5.1.65-osx10.6-x86_64/share/charsets/ |
+
8 rows in set (0.00 sec)
\end{pythoncode}

看到\texttt{utf8}字样就表示编码设置正确。

\emph{注}:如果 MySQL
的版本≥5.5.3,可以把编码设置为\texttt{utf8mb4},\texttt{utf8mb4}和\texttt{utf8}完全兼容,但它支持最新的
Unicode 标准,可以显示 emoji 字符。

\hypertarget{ux5b89ux88c5-mysql-ux9a71ux52a8}{%
\subsubsection{安装 MySQL 驱动}\label{ux5b89ux88c5-mysql-ux9a71ux52a8}}

由于 MySQL 服务器以独立的进程运行,并通过网络对外服务,所以,需要支持
Python 的 MySQL 驱动来连接到 MySQL 服务器。MySQL 官方提供了
mysql-connector-python 驱动,但是安装的时候需要给 pip
命令加上参数\texttt{-\/-allow-external}:

\begin{pythoncode}
$ pip install mysql-connector-python --allow-external mysql-connector-python
\end{pythoncode}

如果上面的命令安装失败,可以试试另一个驱动:

\begin{pythoncode}
$ pip install mysql-connector
\end{pythoncode}

我们演示如何连接到 MySQL 服务器的 test 数据库:

\begin{pythoncode}
>>> import mysql.connector

>>> conn = mysql.connector.connect(user='root', password='password', database='test')
>>> cursor = conn.cursor()

>>> cursor.execute('create table user (id varchar(20) primary key, name varchar(20))')

>>> cursor.execute('insert into user (id, name) values (%s, %s)', ['1', 'Michael'])
>>> cursor.rowcount
1

>>> conn.commit()
>>> cursor.close()

>>> cursor = conn.cursor()
>>> cursor.execute('select * from user where id = %s', ('1',))
>>> values = cursor.fetchall()
>>> values
[('1', 'Michael')]

>>> cursor.close()
True
>>> conn.close()
\end{pythoncode}

由于 Python 的 DB-API 定义都是通用的,所以,操作 MySQL 的数据库代码和
SQLite 类似。

\hypertarget{ux5c0fux7ed3}{%
\subsubsection{小结}\label{ux5c0fux7ed3}}

\begin{itemize}
\item
  执行 INSERT 等操作后要调用\texttt{commit()}提交事务;
\item
  MySQL 的 SQL 占位符是\texttt{\%s}。
\end{itemize}

\hypertarget{ux53c2ux8003ux6e90ux7801}{%
\subsubsection{参考源码}\label{ux53c2ux8003ux6e90ux7801}}

\href{https://github.com/michaelliao/learn-python3/blob/master/samples/db/do_mysql.py}{do\_mysql.py}

