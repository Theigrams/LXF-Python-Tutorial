\hypertarget{ux5e8fux5217ux5316}{%
\subsection{序列化}\label{ux5e8fux5217ux5316}}

在程序运行的过程中,所有的变量都是在内存中,比如,定义一个 dict:

\begin{pythoncode}
d = dict(name='Bob', age=20, score=88)
\end{pythoncode}

可以随时修改变量,比如把\texttt{name}改成\texttt{\textquotesingle{}Bill\textquotesingle{}},但是一旦程序结束,变量所占用的内存就被操作系统全部回收。如果没有把修改后的\texttt{\textquotesingle{}Bill\textquotesingle{}}存储到磁盘上,下次重新运行程序,变量又被初始化为\texttt{\textquotesingle{}Bob\textquotesingle{}}。

我们把变量从内存中变成可存储或传输的过程称之为序列化,在 Python 中叫
pickling,在其他语言中也被称之为 serialization,marshalling,flattening
等等,都是一个意思。

序列化之后,就可以把序列化后的内容写入磁盘,或者通过网络传输到别的机器上。

反过来,把变量内容从序列化的对象重新读到内存里称之为反序列化,即
unpickling。

Python 提供了\texttt{pickle}模块来实现序列化。

首先,我们尝试把一个对象序列化并写入文件:

\begin{pythoncode}
>>> import pickle
>>> d = dict(name='Bob', age=20, score=88)
>>> pickle.dumps(d)
b'\x80\x03}q\x00(X\x03\x00\x00\x00ageq\x01K\x14X\x05\x00\x00\x00scoreq\x02KXX\x04\x00\x00\x00nameq\x03X\x03\x00\x00\x00Bobq\x04u.'
\end{pythoncode}

\texttt{pickle.dumps()}方法把任意对象序列化成一个\texttt{bytes},然后,就可以把这个\texttt{bytes}写入文件。或者用另一个方法\texttt{pickle.dump()}直接把对象序列化后写入一个
file-like Object:

\begin{pythoncode}
>>> f = open('dump.txt', 'wb')
>>> pickle.dump(d, f)
>>> f.close()
\end{pythoncode}

看看写入的\texttt{dump.txt}文件,一堆乱七八糟的内容,这些都是 Python
保存的对象内部信息。

当我们要把对象从磁盘读到内存时,可以先把内容读到一个\texttt{bytes},然后用\texttt{pickle.loads()}方法反序列化出对象,也可以直接用\texttt{pickle.load()}方法从一个\texttt{file-like\ Object}中直接反序列化出对象。我们打开另一个
Python 命令行来反序列化刚才保存的对象:

\begin{pythoncode}
>>> f = open('dump.txt', 'rb')
>>> d = pickle.load(f)
>>> f.close()
>>> d
{'age': 20, 'score': 88, 'name': 'Bob'}
\end{pythoncode}

变量的内容又回来了!

当然,这个变量和原来的变量是完全不相干的对象,它们只是内容相同而已。

Pickle 的问题和所有其他编程语言特有的序列化问题一样,就是它只能用于
Python,并且可能不同版本的 Python 彼此都不兼容,因此,只能用 Pickle
保存那些不重要的数据,不能成功地反序列化也没关系。

\hypertarget{json}{%
\subsubsection{JSON}\label{json}}

如果我们要在不同的编程语言之间传递对象,就必须把对象序列化为标准格式,比如
XML,但更好的方法是序列化为 JSON,因为 JSON
表示出来就是一个字符串,可以被所有语言读取,也可以方便地存储到磁盘或者通过网络传输。JSON
不仅是标准格式,并且比 XML 更快,而且可以直接在 Web
页面中读取,非常方便。

JSON 表示的对象就是标准的 JavaScript 语言的对象,JSON 和 Python
内置的数据类型对应如下:

\begin{longtable}[]{@{}ll@{}}
\toprule
JSON类型 & Python类型 \\ \addlinespace
\midrule
\endhead
\{\} & dict \\ \addlinespace
{[}{]} & list \\ \addlinespace
``string'' & str \\ \addlinespace
1234.56 & int或float \\ \addlinespace
true/false & True/False \\ \addlinespace
null & None \\ \addlinespace
\bottomrule
\end{longtable}

Python 内置的\texttt{json}模块提供了非常完善的 Python 对象到 JSON
格式的转换。我们先看看如何把 Python 对象变成一个 JSON:

\begin{pythoncode}
>>> import json
>>> d = dict(name='Bob', age=20, score=88)
>>> json.dumps(d)
'{"age": 20, "score": 88, "name": "Bob"}'
\end{pythoncode}

\texttt{dumps()}方法返回一个\texttt{str},内容就是标准的
JSON。类似的,\texttt{dump()}方法可以直接把 JSON
写入一个\texttt{file-like\ Object}。

要把 JSON 反序列化为 Python
对象,用\texttt{loads()}或者对应的\texttt{load()}方法,前者把 JSON
的字符串反序列化,后者从\texttt{file-like\ Object}中读取字符串并反序列化:

\begin{pythoncode}
>>> json_str = '{"age": 20, "score": 88, "name": "Bob"}'
>>> json.loads(json_str)
{'age': 20, 'score': 88, 'name': 'Bob'}
\end{pythoncode}

由于 JSON 标准规定 JSON 编码是 UTF-8,所以我们总是能正确地在 Python
的\texttt{str}与 JSON 的字符串之间转换。

\hypertarget{json-ux8fdbux9636}{%
\subsubsection{JSON 进阶}\label{json-ux8fdbux9636}}

Python 的\texttt{dict}对象可以直接序列化为 JSON
的\texttt{\{\}},不过,很多时候,我们更喜欢用\texttt{class}表示对象,比如定义\texttt{Student}类,然后序列化:

\begin{pythoncode}
import json

class Student(object):
    def __init__(self, name, age, score):
        self.name = name
        self.age = age
        self.score = score

s = Student('Bob', 20, 88)
print(json.dumps(s))
\end{pythoncode}

运行代码,毫不留情地得到一个\texttt{TypeError}:

\begin{pythoncode}
Traceback (most recent call last):
  ...
TypeError: <__main__.Student object at 0x10603cc50> is not JSON serializable
\end{pythoncode}

错误的原因是\texttt{Student}对象不是一个可序列化为 JSON 的对象。

如果连\texttt{class}的实例对象都无法序列化为 JSON,这肯定不合理!

别急,我们仔细看看\texttt{dumps()}方法的参数列表,可以发现,除了第一个必须的\texttt{obj}参数外,\texttt{dumps()}方法还提供了一大堆的可选参数:

\url{https://docs.python.org/3/library/json.html\#json.dumps}

这些可选参数就是让我们来定制 JSON
序列化。前面的代码之所以无法把\texttt{Student}类实例序列化为
JSON,是因为默认情况下,\texttt{dumps()}方法不知道如何将\texttt{Student}实例变为一个
JSON 的\texttt{\{\}}对象。

可选参数\texttt{default}就是把任意一个对象变成一个可序列为 JSON
的对象,我们只需要为\texttt{Student}专门写一个转换函数,再把函数传进去即可:

\begin{pythoncode}
def student2dict(std):
    return {
        'name': std.name,
        'age': std.age,
        'score': std.score
    }
\end{pythoncode}

这样,\texttt{Student}实例首先被\texttt{student2dict()}函数转换成\texttt{dict},然后再被顺利序列化为
JSON:

\begin{pythoncode}
>>> print(json.dumps(s, default=student2dict))
{"age": 20, "name": "Bob", "score": 88}
\end{pythoncode}

不过,下次如果遇到一个\texttt{Teacher}类的实例,照样无法序列化为
JSON。我们可以偷个懒,把任意\texttt{class}的实例变为\texttt{dict}:

\begin{pythoncode}
print(json.dumps(s, default=lambda obj: obj.__dict__))
\end{pythoncode}

因为通常\texttt{class}的实例都有一个\texttt{\_\_dict\_\_}属性,它就是一个\texttt{dict},用来存储实例变量。也有少数例外,比如定义了\texttt{\_\_slots\_\_}的
class。

同样的道理,如果我们要把 JSON
反序列化为一个\texttt{Student}对象实例,\texttt{loads()}方法首先转换出一个\texttt{dict}对象,然后,我们传入的\texttt{object\_hook}函数负责把\texttt{dict}转换为\texttt{Student}实例:

\begin{pythoncode}
def dict2student(d):
    return Student(d['name'], d['age'], d['score'])
\end{pythoncode}

运行结果如下:

\begin{pythoncode}
>>> json_str = '{"age": 20, "score": 88, "name": "Bob"}'
>>> print(json.loads(json_str, object_hook=dict2student))
<__main__.Student object at 0x10cd3c190>
\end{pythoncode}

打印出的是反序列化的\texttt{Student}实例对象。

\hypertarget{ux7ec3ux4e60}{%
\subsubsection{练习}\label{ux7ec3ux4e60}}

对中文进行 JSON
序列化时,\texttt{json.dumps()}提供了一个\texttt{ensure\_ascii}参数,观察该参数对结果的影响:

\begin{pythoncode}
# -*- coding: utf-8 -*-

import json
\end{pythoncode}

\begin{pythoncode}
print(s)
\end{pythoncode}

\hypertarget{ux5c0fux7ed3}{%
\subsubsection{小结}\label{ux5c0fux7ed3}}

Python
语言特定的序列化模块是\texttt{pickle},但如果要把序列化搞得更通用、更符合
Web 标准,就可以使用\texttt{json}模块。

\texttt{json}模块的\texttt{dumps()}和\texttt{loads()}函数是定义得非常好的接口的典范。当我们使用时,只需要传入一个必须的参数。但是,当默认的序列化或反序列机制不满足我们的要求时,我们又可以传入更多的参数来定制序列化或反序列化的规则,既做到了接口简单易用,又做到了充分的扩展性和灵活性。

\hypertarget{ux53c2ux8003ux6e90ux7801}{%
\subsubsection{参考源码}\label{ux53c2ux8003ux6e90ux7801}}

\href{https://github.com/michaelliao/learn-python3/blob/master/samples/io/use_pickle.py}{use\_pickle.py}

\href{https://github.com/michaelliao/learn-python3/blob/master/samples/io/use_json.py}{use\_json.py}

