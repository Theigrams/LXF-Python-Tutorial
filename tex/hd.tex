\documentclass[a4paper]{ctexart}
\usepackage{amsmath,amsfonts,amssymb,amsthm,mathrsfs,bm}
\usepackage{unicode-math}
\usepackage{fontspec}
\setmonofont{CMU Typewriter Text}
% \setmonofont[Scale = 0.8]{Source Code Pro}
\usepackage{longtable}
\usepackage{booktabs}
\usepackage{setspace}		%行间距
\usepackage{geometry}		%页边距
%\geometry{left=3.18cm,right=3.18cm,top=2.54cm,bottom=2.54cm}
\geometry{left=2cm,right=2cm,top=2.54cm,bottom=2.54cm}
\usepackage{graphicx}		%插图 \includegraphics
\PassOptionsToPackage{unicode}{hyperref}
\usepackage[colorlinks=true,
pdfborder=001,     
citecolor=blue,
linkcolor=red,
anchorcolor=green,
urlcolor=blue,
bookmarksopen=true,bookmarksnumbered=true]{hyperref}

\usepackage{tcolorbox}
\tcbuselibrary{most}
\usepackage{minted}
% 取消minted环境中特殊字符的红框
\AtBeginEnvironment{minted}{%
  \renewcommand{\fcolorbox}[4][]{#4}}

\usepackage{tabularx}
\usepackage{xcolor}
\usepackage{soul}
\definecolor{Light}{RGB}{251,234,236}
\sethlcolor{Light}

% 设置 \texttt 红色样式
\let\oldtexttt\texttt 
% 备选色为 [green!30!red]
\renewcommand{\texttt}[1]{\colorbox{Light}{\ttfamily \textcolor{red}{#1}}}



% 设置行号颜色
\renewcommand{\theFancyVerbLine}{\sffamily
	\textcolor[rgb]{0.5,0.5,1.0}{\scriptsize
		\oldstylenums{\arabic{FancyVerbLine}}}}



\newminted{python}{texcl=true,mathescape,linenos,breaklines,breakanywhere,frame=single,baselinestretch=1}

\newminted{pycon}{texcl=true,mathescape,linenos,breaklines,breakanywhere,style=colorful,frame=single,fontsize=\small}
