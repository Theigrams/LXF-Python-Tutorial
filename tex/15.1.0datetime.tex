\hypertarget{datetime}{%
\subsection{datetime}\label{datetime}}

datetime 是 Python 处理日期和时间的标准库。

\hypertarget{ux83b7ux53d6ux5f53ux524dux65e5ux671fux548cux65f6ux95f4}{%
\subsubsection{获取当前日期和时间}\label{ux83b7ux53d6ux5f53ux524dux65e5ux671fux548cux65f6ux95f4}}

我们先看如何获取当前日期和时间:

\begin{pythoncode}
>>> from datetime import datetime
>>> now = datetime.now() 
>>> print(now)
2015-05-18 16:28:07.198690
>>> print(type(now))
<class 'datetime.datetime'>
\end{pythoncode}

注意到\texttt{datetime}是模块,\texttt{datetime}模块还包含一个\texttt{datetime}类,通过\texttt{from\ datetime\ import\ datetime}导入的才是\texttt{datetime}这个类。

如果仅导入\texttt{import\ datetime},则必须引用全名\texttt{datetime.datetime}。

\texttt{datetime.now()}返回当前日期和时间,其类型是\texttt{datetime}。

\hypertarget{ux83b7ux53d6ux6307ux5b9aux65e5ux671fux548cux65f6ux95f4}{%
\subsubsection{获取指定日期和时间}\label{ux83b7ux53d6ux6307ux5b9aux65e5ux671fux548cux65f6ux95f4}}

要指定某个日期和时间,我们直接用参数构造一个\texttt{datetime}:

\begin{pythoncode}
>>> from datetime import datetime
>>> dt = datetime(2015, 4, 19, 12, 20) 
>>> print(dt)
2015-04-19 12:20:00
\end{pythoncode}

\hypertarget{datetime-ux8f6cux6362ux4e3a-timestamp}{%
\subsubsection{datetime 转换为
timestamp}\label{datetime-ux8f6cux6362ux4e3a-timestamp}}

在计算机中,时间实际上是用数字表示的。我们把 1970 年 1 月 1 日 00:00:00
UTC+00:00 时区的时刻称为 epoch time,记为\texttt{0}(1970 年以前的时间
timestamp 为负数),当前时间就是相对于 epoch time 的秒数,称为
timestamp。

你可以认为:

\begin{pythoncode}
timestamp = 0 = 1970-1-1 00:00:00 UTC+0:00
\end{pythoncode}

对应的北京时间是:

\begin{pythoncode}
timestamp = 0 = 1970-1-1 08:00:00 UTC+8:00
\end{pythoncode}

可见 timestamp 的值与时区毫无关系,因为 timestamp 一旦确定,其 UTC
时间就确定了,转换到任意时区的时间也是完全确定的,这就是为什么计算机存储的当前时间是以
timestamp 表示的,因为全球各地的计算机在任意时刻的 timestamp
都是完全相同的(假定时间已校准)。

把一个\texttt{datetime}类型转换为 timestamp
只需要简单调用\texttt{timestamp()}方法:

\begin{pythoncode}
>>> from datetime import datetime
>>> dt = datetime(2015, 4, 19, 12, 20) 
>>> dt.timestamp() 
1429417200.0
\end{pythoncode}

注意 Python 的 timestamp 是一个浮点数,整数位表示秒。

某些编程语言(如 Java 和 JavaScript)的 timestamp
使用整数表示毫秒数,这种情况下只需要把 timestamp 除以 1000 就得到 Python
的浮点表示方法。

\hypertarget{timestamp-ux8f6cux6362ux4e3a-datetime}{%
\subsubsection{timestamp 转换为
datetime}\label{timestamp-ux8f6cux6362ux4e3a-datetime}}

要把 timestamp
转换为\texttt{datetime},使用\texttt{datetime}提供的\texttt{fromtimestamp()}方法:

\begin{pythoncode}
>>> from datetime import datetime
>>> t = 1429417200.0
>>> print(datetime.fromtimestamp(t))
2015-04-19 12:20:00
\end{pythoncode}

注意到 timestamp 是一个浮点数,它没有时区的概念,而 datetime
是有时区的。上述转换是在 timestamp 和本地时间做转换。

本地时间是指当前操作系统设定的时区。例如北京时区是东 8 区,则本地时间:

\begin{pythoncode}
2015-04-19 12:20:00
\end{pythoncode}

实际上就是 UTC+8:00 时区的时间:

\begin{pythoncode}
2015-04-19 12:20:00 UTC+8:00
\end{pythoncode}

而此刻的格林威治标准时间与北京时间差了 8 小时,也就是 UTC+0:00
时区的时间应该是:

\begin{pythoncode}
2015-04-19 04:20:00 UTC+0:00
\end{pythoncode}

timestamp 也可以直接被转换到 UTC 标准时区的时间:

\begin{pythoncode}
>>> from datetime import datetime
>>> t = 1429417200.0
>>> print(datetime.fromtimestamp(t)) 
2015-04-19 12:20:00
>>> print(datetime.utcfromtimestamp(t)) 
2015-04-19 04:20:00
\end{pythoncode}

\hypertarget{str-ux8f6cux6362ux4e3a-datetime}{%
\subsubsection{str 转换为
datetime}\label{str-ux8f6cux6362ux4e3a-datetime}}

很多时候,用户输入的日期和时间是字符串,要处理日期和时间,首先必须把 str
转换为
datetime。转换方法是通过\texttt{datetime.strptime()}实现,需要一个日期和时间的格式化字符串:

\begin{pythoncode}
>>> from datetime import datetime
>>> cday = datetime.strptime('2015-6-1 18:19:59', '%Y-%m-%d %H:%M:%S')
>>> print(cday)
2015-06-01 18:19:59
\end{pythoncode}

字符串\texttt{\textquotesingle{}\%Y-\%m-\%d\ \%H:\%M:\%S\textquotesingle{}}规定了日期和时间部分的格式。详细的说明请参考
\href{https://docs.python.org/3/library/datetime.html\#strftime-strptime-behavior}{Python
文档}。

注意转换后的 datetime 是没有时区信息的。

\hypertarget{datetime-ux8f6cux6362ux4e3a-str}{%
\subsubsection{datetime 转换为
str}\label{datetime-ux8f6cux6362ux4e3a-str}}

如果已经有了 datetime 对象,要把它格式化为字符串显示给用户,就需要转换为
str,转换方法是通过\texttt{strftime()}实现的,同样需要一个日期和时间的格式化字符串:

\begin{pythoncode}
>>> from datetime import datetime
>>> now = datetime.now()
>>> print(now.strftime('%a, %b %d %H:%M'))
Mon, May 05 16:28
\end{pythoncode}

\hypertarget{datetime-ux52a0ux51cf}{%
\subsubsection{datetime 加减}\label{datetime-ux52a0ux51cf}}

对日期和时间进行加减实际上就是把 datetime 往后或往前计算,得到新的
datetime。加减可以直接用\texttt{+}和\texttt{-}运算符,不过需要导入\texttt{timedelta}这个类:

\begin{pythoncode}
>>> from datetime import datetime, timedelta
>>> now = datetime.now()
>>> now
datetime.datetime(2015, 5, 18, 16, 57, 3, 540997)
>>> now + timedelta(hours=10)
datetime.datetime(2015, 5, 19, 2, 57, 3, 540997)
>>> now - timedelta(days=1)
datetime.datetime(2015, 5, 17, 16, 57, 3, 540997)
>>> now + timedelta(days=2, hours=12)
datetime.datetime(2015, 5, 21, 4, 57, 3, 540997)
\end{pythoncode}

可见,使用\texttt{timedelta}你可以很容易地算出前几天和后几天的时刻。

\hypertarget{ux672cux5730ux65f6ux95f4ux8f6cux6362ux4e3a-utc-ux65f6ux95f4}{%
\subsubsection{本地时间转换为 UTC
时间}\label{ux672cux5730ux65f6ux95f4ux8f6cux6362ux4e3a-utc-ux65f6ux95f4}}

本地时间是指系统设定时区的时间,例如北京时间是 UTC+8:00 时区的时间,而
UTC 时间指 UTC+0:00 时区的时间。

一个\texttt{datetime}类型有一个时区属性\texttt{tzinfo},但是默认为\texttt{None},所以无法区分这个\texttt{datetime}到底是哪个时区,除非强行给\texttt{datetime}设置一个时区:

\begin{pythoncode}
>>> from datetime import datetime, timedelta, timezone
>>> tz_utc_8 = timezone(timedelta(hours=8)) 
>>> now = datetime.now()
>>> now
datetime.datetime(2015, 5, 18, 17, 2, 10, 871012)
>>> dt = now.replace(tzinfo=tz_utc_8) 
>>> dt
datetime.datetime(2015, 5, 18, 17, 2, 10, 871012, tzinfo=datetime.timezone(datetime.timedelta(0, 28800)))
\end{pythoncode}

如果系统时区恰好是
UTC+8:00,那么上述代码就是正确的,否则,不能强制设置为 UTC+8:00 时区。

\hypertarget{ux65f6ux533aux8f6cux6362}{%
\subsubsection{时区转换}\label{ux65f6ux533aux8f6cux6362}}

我们可以先通过\texttt{utcnow()}拿到当前的 UTC
时间,再转换为任意时区的时间:

\begin{pythoncode}
>>> utc_dt = datetime.utcnow().replace(tzinfo=timezone.utc)
>>> print(utc_dt)
2015-05-18 09:05:12.377316+00:00

>>> bj_dt = utc_dt.astimezone(timezone(timedelta(hours=8)))
>>> print(bj_dt)
2015-05-18 17:05:12.377316+08:00

>>> tokyo_dt = utc_dt.astimezone(timezone(timedelta(hours=9)))
>>> print(tokyo_dt)
2015-05-18 18:05:12.377316+09:00

>>> tokyo_dt2 = bj_dt.astimezone(timezone(timedelta(hours=9)))
>>> print(tokyo_dt2)
2015-05-18 18:05:12.377316+09:00
\end{pythoncode}

时区转换的关键在于,拿到一个\texttt{datetime}时,要获知其正确的时区,然后强制设置时区,作为基准时间。

利用带时区的\texttt{datetime},通过\texttt{astimezone()}方法,可以转换到任意时区。

注:不是必须从 UTC+0:00
时区转换到其他时区,任何带时区的\texttt{datetime}都可以正确转换,例如上述\texttt{bj\_dt}到\texttt{tokyo\_dt}的转换。

\hypertarget{ux5c0fux7ed3}{%
\subsubsection{小结}\label{ux5c0fux7ed3}}

\texttt{datetime}表示的时间需要时区信息才能确定一个特定的时间,否则只能视为本地时间。

如果要存储\texttt{datetime},最佳方法是将其转换为 timestamp 再存储,因为
timestamp 的值与时区完全无关。

\hypertarget{ux7ec3ux4e60}{%
\subsubsection{练习}\label{ux7ec3ux4e60}}

假设你获取了用户输入的日期和时间如\texttt{2015-1-21\ 9:01:30},以及一个时区信息如\texttt{UTC+5:00},均是\texttt{str},请编写一个函数将其转换为
timestamp:

\begin{pythoncode}
# -*- coding:utf-8 -*-

import re
from datetime import datetime, timezone, timedelta
\end{pythoncode}

\begin{pythoncode}
# 测试:
t1 = to_timestamp('2015-6-1 08:10:30', 'UTC+7:00')
assert t1 == 1433121030.0, t1

t2 = to_timestamp('2015-5-31 16:10:30', 'UTC-09:00')
assert t2 == 1433121030.0, t2

print('ok')
\end{pythoncode}

\hypertarget{ux53c2ux8003ux6e90ux7801}{%
\subsubsection{参考源码}\label{ux53c2ux8003ux6e90ux7801}}

\href{https://github.com/michaelliao/learn-python3/blob/master/samples/commonlib/use_datetime.py}{use\_datetime.py}

