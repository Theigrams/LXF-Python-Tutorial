\hypertarget{ux591aux91cdux7ee7ux627f}{%
\subsection{多重继承}\label{ux591aux91cdux7ee7ux627f}}

继承是面向对象编程的一个重要的方式,因为通过继承,子类就可以扩展父类的功能。

回忆一下\texttt{Animal}类层次的设计,假设我们要实现以下 4 种动物:

\begin{itemize}
\item
  Dog - 狗狗;
\item
  Bat - 蝙蝠;
\item
  Parrot - 鹦鹉;
\item
  Ostrich - 鸵鸟。
\end{itemize}

如果按照哺乳动物和鸟类归类,我们可以设计出这样的类的层次:

\begin{pythoncode}
                ┌───────────────┐
                │    Animal     │
                └───────────────┘
                        │
           ┌────────────┴────────────┐
           │                         │
           ▼                         ▼
    ┌─────────────┐           ┌─────────────┐
    │   Mammal    │           │    Bird     │
    └─────────────┘           └─────────────┘
           │                         │
     ┌─────┴──────┐            ┌─────┴──────┐
     │            │            │            │
     ▼            ▼            ▼            ▼
┌─────────┐  ┌─────────┐  ┌─────────┐  ┌─────────┐
│   Dog   │  │   Bat   │  │ Parrot  │  │ Ostrich │
└─────────┘  └─────────┘  └─────────┘  └─────────┘
\end{pythoncode}

但是如果按照 ``能跑'' 和``能飞''来归类,我们就应该设计出这样的类的层次:

\begin{pythoncode}
                ┌───────────────┐
                │    Animal     │
                └───────────────┘
                        │
           ┌────────────┴────────────┐
           │                         │
           ▼                         ▼
    ┌─────────────┐           ┌─────────────┐
    │  Runnable   │           │   Flyable   │
    └─────────────┘           └─────────────┘
           │                         │
     ┌─────┴──────┐            ┌─────┴──────┐
     │            │            │            │
     ▼            ▼            ▼            ▼
┌─────────┐  ┌─────────┐  ┌─────────┐  ┌─────────┐
│   Dog   │  │ Ostrich │  │ Parrot  │  │   Bat   │
└─────────┘  └─────────┘  └─────────┘  └─────────┘
\end{pythoncode}

如果要把上面的两种分类都包含进来,我们就得设计更多的层次:

\begin{itemize}
\item
  哺乳类:能跑的哺乳类,能飞的哺乳类;
\item
  鸟类:能跑的鸟类,能飞的鸟类。
\end{itemize}

这么一来,类的层次就复杂了:

\begin{pythoncode}
                ┌───────────────┐
                │    Animal     │
                └───────────────┘
                        │
           ┌────────────┴────────────┐
           │                         │
           ▼                         ▼
    ┌─────────────┐           ┌─────────────┐
    │   Mammal    │           │    Bird     │
    └─────────────┘           └─────────────┘
           │                         │
     ┌─────┴──────┐            ┌─────┴──────┐
     │            │            │            │
     ▼            ▼            ▼            ▼
┌─────────┐  ┌─────────┐  ┌─────────┐  ┌─────────┐
│  MRun   │  │  MFly   │  │  BRun   │  │  BFly   │
└─────────┘  └─────────┘  └─────────┘  └─────────┘
     │            │            │            │
     │            │            │            │
     ▼            ▼            ▼            ▼
┌─────────┐  ┌─────────┐  ┌─────────┐  ┌─────────┐
│   Dog   │  │   Bat   │  │ Ostrich │  │ Parrot  │
└─────────┘  └─────────┘  └─────────┘  └─────────┘
\end{pythoncode}

如果要再增加 ``宠物类''
和``非宠物类'',这么搞下去,类的数量会呈指数增长,很明显这样设计是不行的。

正确的做法是采用多重继承。首先,主要的类层次仍按照哺乳类和鸟类设计:

\begin{pythoncode}
class Animal(object):
    pass
class Mammal(Animal):
    pass

class Bird(Animal):
    pass
class Dog(Mammal):
    pass

class Bat(Mammal):
    pass

class Parrot(Bird):
    pass

class Ostrich(Bird):
    pass
\end{pythoncode}

现在,我们要给动物再加上\texttt{Runnable}和\texttt{Flyable}的功能,只需要先定义好\texttt{Runnable}和\texttt{Flyable}的类:

\begin{pythoncode}
class Runnable(object):
    def run(self):
        print('Running...')

class Flyable(object):
    def fly(self):
        print('Flying...')
\end{pythoncode}

对于需要\texttt{Runnable}功能的动物,就多继承一个\texttt{Runnable},例如\texttt{Dog}:

\begin{pythoncode}
class Dog(Mammal, Runnable):
    pass
\end{pythoncode}

对于需要\texttt{Flyable}功能的动物,就多继承一个\texttt{Flyable},例如\texttt{Bat}:

\begin{pythoncode}
class Bat(Mammal, Flyable):
    pass
\end{pythoncode}

通过多重继承,一个子类就可以同时获得多个父类的所有功能。

\hypertarget{mixin}{%
\subsubsection{MixIn}\label{mixin}}

在设计类的继承关系时,通常,主线都是单一继承下来的,例如,\texttt{Ostrich}继承自\texttt{Bird}。但是,如果需要
``混入''
额外的功能,通过多重继承就可以实现,比如,让\texttt{Ostrich}除了继承自\texttt{Bird}外,再同时继承\texttt{Runnable}。这种设计通常称之为
MixIn。

为了更好地看出继承关系,我们把\texttt{Runnable}和\texttt{Flyable}改为\texttt{RunnableMixIn}和\texttt{FlyableMixIn}。类似的,你还可以定义出肉食动物\texttt{CarnivorousMixIn}和植食动物\texttt{HerbivoresMixIn},让某个动物同时拥有好几个
MixIn:

\begin{pythoncode}
class Dog(Mammal, RunnableMixIn, CarnivorousMixIn):
    pass
\end{pythoncode}

MixIn
的目的就是给一个类增加多个功能,这样,在设计类的时候,我们优先考虑通过多重继承来组合多个
MixIn 的功能,而不是设计多层次的复杂的继承关系。

Python 自带的很多库也使用了 MixIn。举个例子,Python
自带了\texttt{TCPServer}和\texttt{UDPServer}这两类网络服务,而要同时服务多个用户就必须使用多进程或多线程模型,这两种模型由\texttt{ForkingMixIn}和\texttt{ThreadingMixIn}提供。通过组合,我们就可以创造出合适的服务来。

比如,编写一个多进程模式的 TCP 服务,定义如下:

\begin{pythoncode}
class MyTCPServer(TCPServer, ForkingMixIn):
    pass
\end{pythoncode}

编写一个多线程模式的 UDP 服务,定义如下:

\begin{pythoncode}
class MyUDPServer(UDPServer, ThreadingMixIn):
    pass
\end{pythoncode}

如果你打算搞一个更先进的协程模型,可以编写一个\texttt{CoroutineMixIn}:

\begin{pythoncode}
class MyTCPServer(TCPServer, CoroutineMixIn):
    pass
\end{pythoncode}

这样一来,我们不需要复杂而庞大的继承链,只要选择组合不同的类的功能,就可以快速构造出所需的子类。

\hypertarget{ux5c0fux7ed3}{%
\subsubsection{小结}\label{ux5c0fux7ed3}}

由于 Python 允许使用多重继承,因此,MixIn 就是一种常见的设计。

只允许单一继承的语言(如 Java)不能使用 MixIn 的设计。

